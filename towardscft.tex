\documentclass[a4paper, 12pt,oneside,openany]{book}
\usepackage{import}
\subimport{../MacTex/condiments/}{preamble.tex}
\subimport{../MacTex/condiments/}{letterfonts.tex}
\subimport{../MacTex/condiments/}{macros.tex}
\usepackage{tikz-cd} 
\usepackage[all]{xy}
\usepackage[margin=0.7in, foot=.25in, bottom=1.5in]{geometry}
\usepackage{fancyhdr}

\usepackage{lmodern}
\usepackage{xcolor}
\usepackage[Bjornstrup]{fncychap}
\usetikzlibrary{calc}
\renewcommand{\headrulewidth}{0pt}
\setlength{\headsep}{0.7in}

\definecolor{line}{HTML}{292524}
\definecolor{fillh}{HTML}{f5f5f4}

\renewcommand{\headrulewidth}{0pt}
\newcommand \hdheight{1in}

\pagestyle{empty}
\renewcommand{\chaptermark}[1]{\markboth{#1}{}}
\fancyhf{}
\fancyhead[E]{%
\begin{tikzpicture}[overlay, remember picture]%
    \fill[fillh] (current page.north west) rectangle ($(current page.north east)+(0,-\hdheight)$);
    \draw[line] ([yshift=-\hdheight]current page.north west) -- ([yshift=-\hdheight]current page.north east);
    \node[anchor=south west, text width=1cm] (evenpagenum) at ($(current page.north west)+(.5\hdheight,-\hdheight)$) {\thepage};
    \node[anchor=south west, text width=1.5cm, text=darkgray, font=\fontsize{2cm}{5.5cm}\selectfont] (chapter) at (evenpagenum.south east) {\thechapter};
    \node[anchor=south west] at (chapter.south east) {\bfseries\leftmark};
\end{tikzpicture}
}%
\fancyhead[O]{%
\begin{tikzpicture}[overlay, remember picture]%
    \fill[fillh] (current page.north west) rectangle ($(current page.north east)+(0,-\hdheight)$);
    \draw[line] ([yshift=-\hdheight]current page.north west) -- ([yshift=-\hdheight]current page.north east);
    \node[align=right, anchor=south east, text width=1cm] (evenpagenum) at ($(current page.north east)+(-.5\hdheight,-\hdheight)$) {\thepage};
    \node[align=right, anchor=south east, text width=1.5cm, text=darkgray, font=\fontsize{2cm}{5.5cm}\selectfont] (chapter) at (evenpagenum.south west) {\thechapter};
    \node[align=right, anchor=south east] at (chapter.south west) {\bfseries\leftmark};
\end{tikzpicture}
}
\fancyfoot[CE]{}
\fancyfoot[CO]{}
\setlength{\headheight}{22pt}
\setlength{\textheight}{700pt}
  % a nice math font
\usepackage{newpxtext,newpxmath}   % a nice text font


\title{\textbf{\Huge{Towards Class Field Theory}}\\ \vspace{0.5cm} \large{(version 1.0)}}

\author{\Large{Ethan Choi}}
\date{\Large{March 24th, 2024}\\}
\begin{document}


\maketitle
\dominitoc[n]

\tableofcontents

\clearpage
\pagestyle{fancy}
\chapter{Introduction}

Class field theory aims to characterise abelian Galois extensions over local or global fields using properties inherent to the field. I believe that every number theorist should acquaint themselves with at least the basics of class field theory, just as an algebraist would with Galois theory. Formally, we aim to study field extensions of the form $$\bbQ \subset K \subset L \subset K^{ab} \subset \bar{K} \subset \bar{\bbQ}.$$

Many undergraduate or graduate students who are interested in number theory are dissuaded from studying class field theory because of the many prerequisites that come before it. Some of these pre-requisites come from algebraic number theory, the study of number fields using arithmetic means. Some other prerequisites come from topological groups, but they will not be taught here. We do briefly mention homological algebra in the chapters on class field theory, but do not get bogged down in "abstract nonsense". The chief rationale for using cohomology is that it makes working with class field theory much more compact.

In algebraic number theory, we study the behaviour of ideals, field extensions, unique factorisation, the structure of Galois groups, cyclotomic polynomials, and more. Most of this text will be about algebraic number theory but do contain some analytical elements, especially in Chapter 5. 

This text focuses on acquainting the number theorist with knowledge in algebraic number theory to undertake more sophisticated topics in number theory. This text is not a substitute to a textbook. I'd describe it as a more polished stack of lecture notes.

The prerequisites for this text are Galois theory and point-set topology at an undergraduate level. I've tried to avoid mentioning too many results from algebraic geometry or commutative algebra. Some knowledge of basic category theory (categories, functors, limits, short exact sequences etc.) would be helpful but the homological algebra in Chapter 7 is built entirely from scratch.

\section{Structure}

In Chapter 1, we first give some basic definitions in algebraic number theory. We then study the failure of unique factorisation through ramification theory and ideal class groups.

Chapter 2 studies key results in the \emph{geometry of numbers}. These results, which were mostly suggested and proven by Minkowski, are shown and proven. Chapter 3 extends upon this and looks at Dirichlet's unit theorem.

Chapter 4 provides an introduction to cyclotomic fields, where the law of quadratic reciprocity is proven. Chapter 5 focuses on analysis. After an introduction to Dirichlet series and the Dirichlet zeta function, we prove the class number formula.

Chapter 6 has two major components. These are the $p$-adic numbers, and the adeles and ideles. We also discuss results on ramification theory.

Chapter 7 gives an overview of class field theory. After using the Kronecker-Weber theorem as a prototypical example, we explore the main theorems of local class field theory via Artin and Tate's cohomological route. En route, we study profinite groups and group cohomology to make proving the main theorems more direct.

Chapter 8 makes the link from local to global class field theory. We prove global Artin reciprocity via the ideles. Then, we talk about some corollaries of the first and second inequalities, as well as results following Artin reciprocity. Finally, we recap the major results in this text.

This version of \emph{Towards Class Field Theory} will be edited and amended when appropriate. Please send any comments, edits, or misprints to \url{ethanchoi422@gmail.com} with a description of the amendment including page number. 

(version 1.0: March 24th, 2024)
\newpage

\chapter{Basic definitions}
\minitoc
\section{Number fields and their invariants}
\defn{Number field}{
    A number field is a finite extension of the rational numbers $\mathbb{Q}$ such that $[K:\mathbb{Q}]$ is finite. 
}

From a number field, we can define its \emph{ring of integers.}

\defn{Ring of integers}{
    Let $K$ be a number field. The ring of integers $\mathcal{O}_K \subset K$ is the set of roots of some monic polynomial in $\mathbb{Z}[X].$
}

\lemma{Gauss's Lemma (on polynomials)}{
	A polynomial $f(x)$ is irreducible in $\bbQ(X)$ if and only if it is irreducible in $\bbZ[X]$. 
	
	Put otherwise, a polynomial $f(x)$ can be factored as $f(x)=g(x)h(x)$ with $\deg(g), \deg(h) < \deg(f)$ in $\bbQ(X)$ if and only if such a factorisation can be attained in $\bbZ{[X]}.$

}

By Gauss's Lemma, we have that the following are equal:
\begin{enumerate}
    \item $\alpha \in \mathcal{O}_K$
    \item The minimal $\mathbb{Q}$-polynomial of $\alpha$ has integer coefficients.
\end{enumerate}

We have the following result regarding quadratic fields.

\ex{$\mathcal{O}_K$ for quadratic fields}{
    Let $K=\mathbb{Q}(\sqrt{d})$.
    
    $\mathcal{O}_K =\begin{cases}
    \mathbb{Z}[\sqrt{d}], & \text{if}\ d \not\equiv 1 \Mod 4.\\
    \mathbb{Z}[\frac{1+\sqrt{d}}{2}], & \text{if}\ d \equiv 1 \Mod 4.\label{quadfield}
  \end{cases}$
}

\textbf{Proof:} Consider the element $\alpha=a+b\sqrt{d} \in K.$ The minimal polynomial of $\alpha$ is given by $x^2-2ax+a^2-db^2.$ (The coefficients of the minimal polynomial will make more sense when we study trace and norm).

Thus, $\alpha$ being an algebraic integer implies that $2a$ and $a^2-db^2$ are integers. In any case, this suggests that $2b$ is also an integer, and thus $(2a)^2-d(2b)^2=4a^2-4db^2 \in \mathbb{Z}$, so $d(2b)^2 \in \mathbb{Z}.$

Since $a$ and $b$ are even, we can represent all algebraic integers as $(u/2)+(v/2)d$ by $a=u/2, b=v/2$, where $\frac{u^2-dv^2}{4} \in \mathbb{Z}.$ Since $u^2-dv^2 \equiv 0 \Mod 4$, $u^2$ and $dv^2$ have the same parity. (odd squares $\equiv 1 \Mod 4$, even squares $\equiv 0 \Mod 4.$)

However, the condition that $u^2$ and $dv^2$ are odd squares only arises when $d \equiv 1 \Mod 4.$ When both squares are even (d $\not\equiv 1 \Mod 4$), $u/2$ and $v/2 \in \mathbb{Z}$, and thus $\alpha = a+b\sqrt{d}. \qed$

Let $L/K$ be a field extension. Consider a $K$-basis of an $n$-dimensional vector space $V$; denote that by $\{e_1, \dots, e_n\}.$ Write $\phi(e_{ij}) = \sum\limits_{i=1}^{n} a_{ij} e_i$, with all $a_{ij} \in K.$ We can represent $\phi$ as a matrix given by $\phi: V\to V$. This matrix is given with respect to the basis as $[\phi] = \{a_{ij}\}.$

The linear operator $\phi$ is "natural"; it comes from the fact that $A=\{a_{ij}\}$, combined with the standard basis of $K^n$ (denoted $\{e_1, \dots, e_n\}$), can produce $Ae_j=\sum\limits_{i=1}^{n} a_{ij} e_i$.

\defn{Norm and Trace}{
   Define the \emph{norm} and \emph{trace} of a number field as the determinant and trace of $[\phi]$ respectively. 
   
   Similarly, define the \emph{characteristic polynomial} as the determinant of $XI-[\phi]$, as you would do so in linear algebra. \\

   In other words, $$N_{L/K} = \det(\{[\phi]\}),$$ and $$T_{L/K} = \text{tr}(\{[\phi]\}).$$
}


\ex{$\mathbb{Q}(i)$}{
    Consider the map $\phi: \mathbb{C} \to \mathbb{C}$ given by $\phi(z) = \bar{z}.$ We use the basis $\{1, i\}$; by $\phi(1)=1, \phi(i)=-i$, we get $[\phi]$ = 
    $\begin{pmatrix}
        1 & 0 \\
        0 & -1 
    \end{pmatrix}$.

    Thus, we can extract the norm and trace of $[\phi]$ $\text{Tr}(\phi)=-1,\ N(\phi)=0$.

    Let $K=\mathbb{Q}, L=\mathbb{Q}(i)$. We consider the general case for a Gaussian integer - take $\alpha = a+bi.$ By multiplying $\alpha$ by our basis, we have $\alpha \cdot 1 = a+bi$, and $\alpha \cdot i = ai-b.$ Thus, $[\phi]$ = 
    $\begin{pmatrix}
        a & -b \\
        b & a 
    \end{pmatrix}$, with $N_{L/K}(\alpha)=\det(\phi)=a^2+b^2, T_{L/K}(\alpha)=\text{Tr}(\phi)=2a.$
}

\ex{$\mathbb{Q}(\sqrt[3]{2})$}{
Consider $\mathbb{Q}(\sqrt[3]{2})$ with the basis $\{1, \sqrt[3]{2}, \sqrt[3]{4}\}.$ Then set $\alpha=a+b\sqrt[3]{2}+c\sqrt[3]{4}$, with $a, b, c \in \mathbb{Q}.$ By multiplying $\alpha$ by basis elements, we get
\begin{align*}
    \alpha \cdot 1 &= a+b\sqrt[3]{2}+c\sqrt[3]{4}\\
    \alpha \cdot \sqrt[3]{2} &= 2c + a\sqrt[3]{2} + b\sqrt[3]{4}\\
    \alpha \cdot \sqrt[3]{4} &= 2b + 2c\sqrt[3]{2} + a\sqrt[3]{4}
\end{align*}

Thus, $[\phi]=\begin{pmatrix}
    a & 2c & 2b \\
    b & a & 2c \\
    c & b & a
\end{pmatrix}$. \\

$N_{L/K}(\alpha)=a^3+2b^3+4c^3-6abc, T_{L/K}(\alpha)=3a.$
    
}

Consider number field extensions of the form $\gamma$. We can apply a similar theory as the ones studied in the previous examples.

\ex{$\mathbb{Q}(\gamma)$}{
    Let $\gamma$ be a root of $x^3-x-1.$ Then, $\gamma^3=\gamma+1.$ We apply the basis $\{1, \gamma, \gamma^2.\}$ Letting $\alpha = a+b\gamma+c\gamma^2$,
    
    \begin{align*}
    \alpha \cdot 1 &= a+b\gamma+c\gamma^2\\
    \alpha \cdot \sqrt[3]{2} &= a\gamma+b\gamma^2+c(\gamma+1)=c+(a+c)\gamma+b\gamma^2\\
    \alpha \cdot \sqrt[3]{4} &= a\gamma^2 + b(\gamma+1) + c\gamma(\gamma+1)=b+(b+c)\gamma+(a+c)\gamma^2
    \end{align*}

    Thus, $[\phi]=\begin{pmatrix}
    a & c & b \\
    b & a+c & b+c \\
    c & b & a+c
\end{pmatrix}$ \\

$N_{L/K}(\alpha)=a^3+b^3+c^3-ab^2+ac^2-bc^2+2a^2c-3abc, T_{L/K}(\alpha)=3a+2c.$

}

Trace is linear. This means that we can write $\text{Tr}(a+b\gamma+c\gamma^2)$ as $a\text{Tr}(1)+b\text{Tr}(\gamma)+c\text{Tr}(\gamma^2).$

Norm, meanwhile, is multiplicative. So given an extension of fields $L/K$, we have that $N_{L/K}(\alpha\beta) = N_{L/K}(\alpha)N_{L/K}(\beta).$

It was previously established that $T_{\mathbb{Q}(\gamma)/\mathbb{Q}}(\alpha)=3a+2c.$ Notice that 

$\text{Tr}(1)= \text{Tr}\begin{pmatrix}
    1 & 0 & 0 \\
    0 & 1 & 0 \\
    0 & 0 & 1
\end{pmatrix} = 3.$

$\text{Tr}(\gamma)= \text{Tr}\begin{pmatrix}
    0 & 0 & 1 \\
    1 & 0 & 1 \\
    0 & 1 & 0
\end{pmatrix} = 0.$

$\text{Tr}(\gamma)= \text{Tr}\begin{pmatrix}
    0 & 1 & 0 \\
    0 & 1 & 1 \\
    1 & 0 & 1
\end{pmatrix} = 2.$

Thus, $T_{\mathbb{Q}(\gamma)/\mathbb{Q}}=3a+2c.$

The notion of discriminant is very common in algebra. In number theory, it can be viewed as a geometric invariant attached to a field extension. It describes the geometric structure of the ring of integers of the said field extension. 

After a discussion of ramification theory in future sections, we can continue by establishing the notions of \emph{discriminant} and \emph{different} ideal respectively. 

In the \emph{AKLB} \label{AKLB} setup ($A$ being a Dedekind domain, $K=\text{Frac}(A)$, $L/K$ a finite separable extension and $B$ the integral closure of $A$ in $L$), the discriminant ideal is a fractional deal of $A$, and the different ideal is a fractional ideal of $B$. Both the discriminant and different are algebraic invariants attached to a number field.

Letting $\mfp \in \text{Spec}(A)$ and $\mfq \in \text{Spec}(B)$ where $\text{Spec}(A)$ is the set of all ideals of $A$, we write that $\mfq$ \emph{lies above $\mfp$} if $\mfq \cap K =\mfp$.

\defn{Discriminant}{
    Let $K$ be a number field and $\mathcal{O}_K$ be its ring of integers. Let $(e_1, \dots, e_n)$ be the $\mathbb{Z}$-basis of a Module that generates $\mathcal{O}_K.$ The discriminant, denoted $\disc(K)$, is defined as: \label{discriminant}
    $$\disc(K) = \det\text{Tr}(e_ie_j),$$ or
    $$\det \begin{pmatrix}
        \text{Tr}_{K/\mathbb{Q}}(e_1e_1) & \dots & \text{Tr}_{K/\mathbb{Q}}(e_1e_n) \\
        \vdots & \vdots & \vdots \\
        \text{Tr}_{K/\mathbb{Q}}(e_ne_1) & \dots & \text{Tr}_{K/\mathbb{Q}}(e_ne_n) \\
    \end{pmatrix}$$
    An equivalent formulation of the discriminant is:
    $$\disc(K) = (\det\sigma_i(e_j))^2,$$ or $$\det \begin{pmatrix}
        \sigma_1(e_1)& \dots & \sigma_1(e_n) \\
        \vdots & \ddots & \vdots \\
        \sigma_n(e_1) & \dots & \sigma_n(e_n) \\
    \end{pmatrix}^2$$
}

These two definitions are equal as one can take $A=\sigma_i(e_j)$, and get

$$A^{\top}A=\left(\sum\limits_{j} \sigma_j(e_i) \sigma_j(e_k)\right)=\text{Tr}(e_ie_k),$$ and so taking $\disc(K) = (\det A)^2 = \det \text{Tr}(e_ie_k)$, we get our desired result.

We refer to the prototypical example of a quadratic field in our following calculations.

\ex{Discriminant of a quadratic field}{
    Let $K=\mathbb{Q}(\sqrt{d})$, where $d$ is square-free. Then, \\

    $\disc(K) =\begin{cases}
    4d, & \text{if}\ d \not\equiv 1 \Mod 4.\\
    d, & \text{if}\ d \equiv 1 \Mod 4.
    \end{cases}$
}

\textbf{Proof:} We know that we can characterise all algebraic integers of $\mathbb{Q}(\sqrt{d})$ as $u/2+v/2\sqrt{d}$, where $u, v \in \mathbb{Z}.$ 

Let $m \not\equiv 1 \Mod 4$. By a previous result on quadratic fields, note that the trace of $a+b\sqrt{d} = 2a$. The field discriminant is thus

$\det\begin{pmatrix}
    2 & 0 \\
    0 & 2d   
\end{pmatrix}=4d.$

Let $d \equiv 1 \Mod 4$. Consider the $\mathbb{Z}$-basis given by $\{1, \frac{1+\sqrt{d}}{2}.\}$. Thus, the field discriminant is thus

$\det\begin{pmatrix}
    Tr_{K/\mathbb{Q}}(1\cdot 1) & Tr_{K/\mathbb{Q}}(1\cdot \frac{1+\sqrt{d}}{2}) \\
    Tr_{K/\mathbb{Q}}(\frac{1+\sqrt{d}}{2} \cdot 1) & Tr_{K/\mathbb{Q}}(\frac{1+\sqrt{d}}{2}\cdot \frac{1+\sqrt{d}}{2})
\end{pmatrix}=\det\begin{pmatrix}
    2 & 1 \\ 1 & \frac{1+d}{2}
\end{pmatrix}=d.$

We can check this with the alternate definition of field discriminant. Note that the two embeddings of $a+b\sqrt{d}$ are $\sigma_1(a+b\sqrt{d}) = a+b\sqrt{d}$, and $\sigma_2(a+b\sqrt{d})=a-b\sqrt{d}$. Thus, the discriminant for the $d \neq 1 \Mod 4$ case is given by 

$\left(\det\begin{pmatrix}
    1 & \sqrt{d} \\ 1 & -\sqrt{d}
\end{pmatrix}\right)^2=(-2\sqrt{d})^2=4d$. 

The discriminant for $d \equiv 1 \Mod 4$ is left as an exercise.

Before we begin a deeper dive into the theory of discriminants, we first define the concept of an \textbf{order.}

\defn{Order}{
    Let $K$ be a number field. An order of $K$ is a subring $\mathcal{O} \subset \mathcal{O}_K$ which is isomorphic to $\mathbb{Z}^n$, where $n=[K:\mathbb{Q}].$
}

\defn{Picard group}{
	Let $\mcO$ be an order, its group of invertible ideals be denoted $J(\mcO)$, and the group which contains fractional principal ideals be denoted $P(\mcO)=\alpha \mcO$, where $\alpha \in K^\times$. The Picard group $Pic(\mcO)$ is defined as $$Pic(\mcO)=J(\mcO)/P(\mcO).$$
}

\prop{}{
    Let $\alpha \in \mathcal{O}_K$ be such that $K=\mathbb{Q}(\alpha).$ This means that $\mathbb{Z}(\alpha)$ is an order of $K$. Letting $p_{\alpha}$ being the minimal polynomial of $\alpha$ over $\mathbb{Q}$, we have that $\disc(\mathbb{Z}[\alpha])=\disc(p_{\alpha}(x)).$
}

\textbf{Proof:} Take a $\mathbb{Z}$-basis $\{1, \alpha, \dots, \alpha^{n-1}\}$ of the order $\mathbb{Z}[\alpha]$. Then the discriminant of this order is given by $\disc(1, \alpha, \dots, \alpha^{n-1})$. This is equal to 

$$\left(\det\begin{pmatrix}
    1 & \alpha_1 & \alpha_1^2 & \dots & \alpha_1^{n-1}\\
    1 & \alpha_2 & \alpha_2^2 & \dots & \alpha_2^{n-1}\\
    \vdots & \vdots & \vdots & \vdots & \vdots \\
    1 & \alpha_n & \alpha_n^2 & \dots & \alpha_n^{n-1}
\end{pmatrix}\right)^2$$

This is a Vandermonde matrix, with determinant $\prod\limits_{i<j}(\alpha_j-\alpha_i)$, and the result follows since the discriminant of $p_{\alpha}$ is $\left(\prod\limits_{i<j}(\alpha_j-\alpha_i)\right)^2.$ \qed

We can represent $\disc(p_{\alpha})$ in a more elegant way. 

\lemma{Discriminant calculation}{
    $\disc(p_{\alpha})=(-1)^{\binom{n}{2}} N_{K/\mathbb{Q}}(f'(\alpha))$, 
    where $n=[K:\mathbb{Q}]$.    
}

By this lemma, we can easily compute discriminants of polynomials given information about its $\mathbb{C}$-embeddings. 

\textbf{Proof:} Let $\sigma_1, \dots, \sigma_n$ be complex embeddings $K \hookrightarrow \mathbb{C}.$ We begin from the previous lemma, which showed that $\disc(p_{\alpha})$ is $\left(\prod\limits_{i<j}(\alpha_j-\alpha_i)\right)^2.$ 

\begin{align*}
    \disc(f)&=\left(\prod\limits_{i<j}(\alpha_j-\alpha_i)\right)^2\\
    &= (-1)^{\binom{n}{2}} \prod\limits_j \prod\limits_{i\neq j} (\sigma_j(\alpha)-\sigma_i(\alpha))
\end{align*}

Taking $f$ as $\prod_i(x-\sigma_i(\alpha))$, we get that $f'=\sum_j \prod_{i\neq j} (x-\sigma_i(\alpha))$ Thus, $$f'(\sigma_j(\alpha)) = \prod\limits_{i\neq j} (\sigma_j(\alpha)-\sigma_i(\alpha)),$$ suggesting that $\disc(f) = (-1)^{\binom{n}{2}} \prod\limits_j f'(\sigma_j(\alpha))$. And since $f \in \mathbb{Z}(X)$ and $\sigma_j$ fixes the integers $f'(\sigma_j(\alpha))=\sigma_j(f'(\alpha)).$ Thus,

\begin{align*}
    \disc(f)&=(-1)^{\binom{n}{2}} \prod\limits_j \sigma_j(f'(\alpha))\\
    &= (-1)^{\binom{n}{2}} N_{K/\mathbb{Q}}(f'(\alpha)),
\end{align*} as desired. \qed


\section{Ideal class groups}

\textbf{Ideal class groups} measure how far a Dedekind domain is from attaining unique factorisation. Recall that if $I$ is a fractional ideal in a Dedekind domain $R$, we can uniquely factorise $I$ in $R$ as $I=P_1^{n_1}\dots P_k^{n_k}$, for $n_i$ integers. We want more than unique factorisation of \emph{ideals} - we aim for unique factorisation of \emph{elements}. 

We first prove an important result regarding Dedekind domains to gain some context behind the connection between Dedekind domains and PIDs. Recall that \textbf{a Dedekind domain is a UFD if and only if it is a PID.} We will be working with PIDs in the following lemmas. 

\prop{}{
    Let $I$ be a non-zero ideal of a Dedekind domain $A.$ Then, there exists another ideal $I'$ such that $II'$ can be generated by a single element $(a).$ In other words, $II'$ is a principal ideal. Take $J$ to be a non-zero ideal of $A$; we can make $I'$ coprime to $J.$ \label{lemma1.5}
}

We do not prove this result here. Instead, we assume this result to prove the following lemma:

\lemma{}{
    Let $I$ be a non-zero ideal of a $A$, and let $a$ be a non-zero element of $I.$ Then, $I$ can be generated by two elements, one of such is $a.$ 

    (This lemma shows the semblance between Dedekind domains and PIDs.)
}

\textbf{Proof:} Take $a \in I$. This means that $(a) \subset I,$ so $I \mid (a).$ Taking $(a)=IJ$, it is clear that there exists an ideal $I'$ such that $II'$ is principal, and that $I'$ and $J$ are coprime. Let $II'$ be generated by $(b).$ 

Then, $\gcd((a), (b)) = \gcd(II', IJ) = I$, since $I'J=(1)=A.$ \qed

Now, we define the ideal class group.

\defn{}{
    Let $I(A)$ be the group of all \emph{fractional ideals} of a a Dedekind domain $A$, and let $P(A)$ denote the group of all \emph{principal fractional ideals}. The ideal class group is defined as the quotient group $$C(A)=I(A)/P(A).$$
}

The \textbf{class number} of $C(A)$ is the order of $C(A)$, given that the ideal class gorup is finite. In fact, the class number is \textbf{finite} - and we use $h(A)$ to denote the class number. This result is proven following the proof of \emph{Minkowski's bound}. An understanding of \emph{Minkowski theory} can be used to evaluate the ideal class numbers of number fields.

\ex{}{
    The class numbers of $\mathbb{Q}(\sqrt{-d})$, for positive, square-free $d$, is 1 if and only if $$d=1, 2, 3, 7, 11, 19, 43, 67, 163.$$
}

By definition, $C(A)$ is trivial if and only if $A$ is a PID.

\thrm{Finite class group}{
    The ideal class group of a number field is finite.
}

\textbf{Proof:} We first prove a result by Kronecker, which then directly implies this result. 

\thrm{Kronecker's Bound}{
    The ideal classes of $C(K)$ are represented by fractional ideals $\mathfrak{a}_i$ such that $\mathcal{O}_K \subset \mathfrak{a}_i$ for all $i$, and $[\mathfrak{a}:\mathcal{O}_K] \leq C$, where $$C=\prod\limits_{\sigma: K \to \mathbb{C}}\sum\limits_{i=1}^n |\sigma(e_i)|,$$ for a $\mathbb{Z}$-basis $\{e_1, \dots, e_n\}$ of $\mathcal{O}_K.$
}

Given that every fractional ideal class can be represented by an ideal $\mathfrak{a} \subset \mathcal{O}_K$, pick a non-zero $\alpha \in \mathfrak{a}$ such that $$|N_{K/\mathbb{Q}}(\alpha)| \leq \mathcal{C}.$$ What is $\mathcal{C}?$ Well, 

\begin{align*}
    |N_{K/\bbQ}(\alpha)|&= \prod\limits_{j=1}^{n} |\sigma_j(x)|\\
    &= \prod\limits_{j=1}^n \sum\limits_{i=1}^{n} e_i \sigma_j(e_i) \\
    &\leq \prod\limits_{j=1}^n \sum\limits_{i=1}^{n} |e_i| |\sigma_j(e_i)| \\
    &\leq (\max(e_i))^n \prod\limits_{j=1}^{n} \sum\limits_{i=1}^{n} \sigma_j(e_i) \\
\end{align*}

where the product ranging from $j=1$ to $n$ represents the $n$ $\mathbb{C}$-embeddings $\sigma:K\to \mathbb{C}.$ 

We have that $(\max(e_i))^n \prod\limits_{j=1}^n \sum\limits_{i=1}^{n} \sigma_j(e_i).$ lies between the $n$-th powers of two consecutive integers. Let the integers be $k^n$ and $(k+1)^n$ respectively. Then set $\sum\limits_{i=1}^n a_ie_i$ for $a_i \in \mathbb{Z}, 0 \leq a_i \leq k$. By setting $\sum\limits_{i=1}^n a_ie_i=\sum\limits_{i=1}^n a'_ie_i\Mod{\mathfrak{a}}$ and setting $c_i = a_i-a'_i$, we have that $\alpha=\sum\limits_{i=1}^n c_ie_i \in \mathfrak{a}$, and $|c_i| \leq k.$ 

Set $C=\prod\limits_{j=1}^n \sum\limits_{i=1}^{n} \sigma_j(e_i).$ Then,

\begin{align*}
    |N_{K/\mathbb{Q}}(\alpha)| &\leq (\max(e_i))^n C \\
    &\leq k^nC \\
    &\leq C[\mathcal{O}_K:\mathfrak{a}].
\end{align*}

Since the ideal classes are represented by fractional ideals satisfying $\mathcal{O}_K \in \mathfrak{a}$ and $[\mathfrak{a}:\mathcal{O}_K] \leq C,$ it remains for us to show that $[\mathcal{O}_K:\mathfrak{a}^{-1}]=[\mathfrak{a}:\mathcal{O}_K]$, and \textbf{thus the ideal classes of $\mathcal{O}_K$ are represented by ideal classes of norm at most $C.$}

Write $\mathfrak{a}^{-1}=\mathfrak{b}$, which is an ideal in $\mathcal{O}_K.$ Additionally, write $\mathfrak{a}=\frac{1}{a}\mathfrak{c}.$ Then, $(a)=\mathfrak{bc}$, and so $N((a))=N(\mathfrak{bc}).$ Additionally we have $[\mathfrak{a}:\mathcal{O}_K]=[\frac{1}{a}\mathfrak{c}:\mathcal{O}_K]=[\mathfrak{c}:a\mathcal{O}_K]=\frac{N((a))}{N(\mathfrak{c})}=N(\mathfrak{b}) = [\mathcal{O}_K: \mathfrak{b}] = [\mathcal{O}_K:\mathfrak{a}^{-1}].$ Thus we can "invert" $[\mathcal{O}_K:\mathfrak{a}^{-1}]$ and $[\mathfrak{a}:\mathcal{O}_K]$. In words, the fractional ideals containing $\mathcal{O}_K$ with norm at most $C$ coincide with the ideals contained in $\mathcal{O}_K$ with norm at most $C.$ \label{inversion}
 
\section{Ramification theory}

There exists a canonical method of factorising a prime ideal $(p) = p\mathcal{O}_K$ in a number field $K$ as $$p\mathcal{O}_K=\mathfrak{P}_1^{e_1}\mathfrak{P}_2^{e_2}\dots\mathfrak{P}_g^{e_g}.$$ The method goes as follows:

\begin{enumerate}
    \item Choose $p \in \mathbb{Z}$ to factorise in $p\mathcal{O}_K.$
    \item Let $f$ be the minimal polynomial of $\theta$, with $\theta$ satisfying $\mathcal{O}_K = \mathbb{Z}[\theta]$
    \item Reduce $f \Mod p$. Denote $\bar{f} = f \Mod p$, and factorise $\bar{f}$ in $\mathbb{F}_p$ as $$\bar{f}(X) = \prod\limits_{i=1}^g \phi_i(X)^{e_i}.$$
    \item Lift each $\phi_i$ to a polynomial $f_i \in \mathbb{Z}[X].$
    \item Evaluate $f_i$ in $\theta$, and compute $\mathfrak{p}_i=(p, f_i(\theta)).$
    \item We have that $p\mathcal{O}_K$ factorises as $(\mathfrak{p}, f_i(\theta)).$
    \item Finally, we conclude that $p\mathcal{O}_K$ factorises as $\mathfrak{p}_i^{e_1}\dots \mathfrak{p}_g^{e_g}$, or that $p\mathcal{O}_K$ admits a unique factorisation of ideals.
\end{enumerate}

\ex{$\mathbb{Q}(\sqrt[3]{2})$}{
    For example, take $K=\mathbb{Q}(\sqrt[3]{2})$, with $\mathcal{O}_K = \mathbb{Z}[\sqrt[3]{2}].$ We aim to factorise $(5) \in \mathcal{O}_K.$

    The minimal polynomial of $\sqrt[3]{2}$ is $x^3-2.$ Factorise $x^3-2 \Mod 5$ as 
    \begin{align*}
        x^3-2 &= (x-3)(x^2+3x+4) \\
        &= (x+2)(x^2-2x-1) \Mod 5
    \end{align*}

    Note that $f$ factors into two polynomials $\phi_1(x)\phi_2(x).$ Thus, $5\mathcal{O}_K=\mathfrak{p}_1\mathfrak{p}_2$, where $\mathfrak{p}_1=(5, \sqrt[3]{2}+2)$, and $\mathfrak{p}_2 = (5, \sqrt[3]{4}-2\sqrt[3]{2}-1).$
}

\ex{$\mathbb{Q}(i)$}{
    Let $K=\mathbb{Q}(i)$ and thus $\mathcal{O}_K=\mathbb{Z}[i].$ We can factorise $(2)$ as such:

    We know that the minimal polynomial of $i$ is $f=x^2+1$. Now,
    \begin{align*}
        f &= x^2+1 \\
        &\equiv x^2-1 \Mod 2 \\
        &\equiv (x+1)(x-1) \Mod 2 \\
        &\equiv (x-1)^2 \Mod 2
    \end{align*}
    
    Thus we have that $2\mathcal{O}_K=(2, i-1)^2.$ Note that we can represent \begin{align*}
        (2, i-1)^2&= (2, i+1)^2\\
        &= ((i-1)(i+1), i+1)^2 \\
        &= \mathfrak{p}^2
    \end{align*}, where $\mathfrak{p}=i+1.$
}

\defn{Inertial degree - $f$}{
    Denote the \emph{inertial degree}, or $f_{\mathfrak{p}}$, as the dimension of the $\mathbb{F}_p$ vector space $\mathcal{O}/\mathfrak{p}$, or $$f_{\mathfrak{p}}=\dim_{\mathbb{F}_p} (\mathcal{O}/\mathfrak{p}).$$ We commonly abuse notation by writing $f$ as $f_{\mathfrak{p}}$ or $f(\mathfrak{p}/p).$
}

\defn{Ramification notation}{
   Similarly, in the ideal factorisation given by $p\mathcal{O}_K=\mathfrak{P}_1^{e_1}\mathfrak{P}_2^{e_2}\dots\mathfrak{P}_g^{e_g}$, we write: $$e_i=e(\mathfrak{P}_i/p),$$ and $$g_i=g(\mathfrak{P}_i/p)$$
}

We classify these prime ideals as follows:

\defn{Ramification vocabulary}{
    Consider the canonical factorisation of $p\mathcal{O}_K$. A prime ideal is \textbf{ramified} if any of the $e_i$ are larger than $1.$ Otherwise, if $e_i=1$ for all $e_i$, we say that the ideal is \textbf{unramified}. We call $e_i$ the \textbf{ramification index} of $\mathfrak{P}_i$ over $(p).$

    A prime ideal is \textbf{inert} if $p\mathcal{O}_K$ is a prime ideal. In other words, we have $e=g=1$, but $f=n$, where $n=[K:\mathbb{Q}].$

    A prime ideal is \textbf{split} if we have $e_i=f=1$; there are no restrictions on $g.$
    
    A prime ideal is \emph{totally ramified} if any of the $e_i=[L:K]$, suggesting that the ramification index is maximal.
}

We now present a theorem relating ramification numbers and inertial degrees with the degree of an extension.

\thrm{}{
    Let $m$ be the degree of $L$ over $K$, and write $\mathfrak{P}_1, \dots, \mathfrak{P}_g$ as the prime ideals dividing $\mathfrak{p}.$ Then, $$\sum\limits_{i=1}^g e_if_i=m.$$ If $L$ is Galois over $K$, the ramification numbers are equal, and so are the residue class degrees $g_i$. Thus, $$efg=m.$$
}

\textbf{Proof:} Consider the \emph{AKLB setup}. We show that both sides are equal to $[B:\mathfrak{p}B:A/\mathfrak{p}].$

Consider the first equality. We note that
    $$B/\mathfrak{p}B = B / \prod\limits_i {\mathfrak{P}_i}^{e_i}.$$
and by the Chinese Remainder Theorem, $=\prod\limits_i B/{\mathfrak{P}_i}^{e_i}.$

Recall that the \emph{inertia degree} is the degree of extension of fields $B/\mathfrak{P}_i \supset A/\mathfrak{P}$. Consider that for each $r_i$, $\mathfrak{P}_i^{r_i} / \mathfrak{P}_i^{r_i+1}$ is a $B/\mathfrak{P}_i$-Module, and thus has dimension 1 as a $B/\mathfrak{P}_i$-vector space since there are no ideals between $\mathfrak{P}_i^{r_i}$ and $\mathfrak{P}_i^{r_i+1}.$ As an $A/\mathfrak{p}$-vector space, it has dimension $f_i.$ 

Thus, $B/\mathfrak{P}_i$, and all $\mathfrak{P}_i^{r_i}/\mathfrak{P}_i^{r_i+1}$ have dimension $f_i$ over $A/\mathfrak{p}.$ Since the chain $B \supset \mathfrak{P}_i \supset \mathfrak{P}_i^2 \supset \dots \supset \mathfrak{P}_i^{e_i}$ has $e_i$ terms, the dimension of $B/\mathfrak{P}_i^{e_i}$ is $e_if_i.$

By the structure theorem for finitely generated Modules over a PID, $B$ is free of finite rank over $A.$ Thus, every $A$-basis for $K$ yields a $B$-basis for $L$.  Thus, $B$ has rank $n=[L:K]$ over $A$;  thus, $B \simeq A^n.$  Thus, $B/\mathfrak{p}B \simeq (A/\mathfrak{p})^n$ is a free  $A/\mathfrak{p}$-Module of dimension $N. \qed$ 

Consider the quadratic field $\mathbb{Z}[i].$

\ex{Ramification in $\mathbb{Z}[i]$}{
    The ideal $(2)$ factors in $\mathbb{Z}[i]$ as $(1+i)^2$, and thus the ramification index of $(2)$ is 2. The ideal $(3)$ is inert in $\mathbb{Z}[i]$. The ideal $(5)$ factors as $(2+i)(2-i)$, and is split. 
    
    In fact, for every prime $p\equiv 1 \Mod 4,$ we can represent $p=N(a+bi)=a^2+b^2$ as a sum of two squares. Thus, $(p)$ splits for all $p\equiv 1 \Mod 4.$
}

Let's consider the general case of $K=\mathbb{Q}(\sqrt{d}).$ 

\begin{enumerate}
    \item If $p\mathcal{O}_K = \mathfrak{p}^2$, $p$ is ramified. In this case, we have $g=2, e=1, f=1.$
    \item If $p\mathcal{O}_K = \mathfrak{p}_1\mathfrak{p}_2$, $p$ is split. In this case, we have $g=1, e=1, f=2.$
    \item If $p\mathcal{O}_K=\mathfrak{p}$, $p$ is inert. In this case, we have $g=1, e=2, f=1.$
\end{enumerate}

We prove an important result linking ramified primes and discriminant.

\prop{}{
    Let $p$ be ramified. This means that $p \mid \disc(K)$, where $\disc(K)$ denotes the discriminant of $K.$
}

\textbf{Proof:} Since $p$ is ramified, by definition of a ramified prime, let $\mathfrak{p} \mid p\mathcal{O}_K$ be such that $p^2 \mid p\mathcal{O}_K.$ Let $\alpha_1, \dots, \alpha_n$ be a $\mathbb{Z}$-basis of $\mathcal{O}$. 

We can set $p\mathcal{O}_K=pI$, with $I$ being divisible by all the primes above $p$. Write $\alpha=\alpha_1 b_1+\dots+\alpha_n b_n.$ We have that $\alpha$ is an element in $I$ and $\alpha \in p\mathcal{O}_K.$ There exists a $b_i$ satisfying $p \nmid b_i$. Label that $b_1.$

By the definition of discriminant, we have that $$\disc(K)=\det \begin{pmatrix}
        \sigma_1(\alpha_1)& \dots & \sigma_1(\alpha_n) \\
        \vdots & \ddots & \vdots \\
        \sigma_n(\alpha_1) & \dots & \sigma_n(\alpha_n) \\
    \end{pmatrix}^2$$

Replace $\alpha_1$ with $\alpha.$ We thus have that 

\begin{align*}
    & \det \begin{pmatrix}
        \sigma_1(\alpha)& \dots & \sigma_1(\alpha_n) \\
        \vdots & \ddots & \vdots \\
        \sigma_n(\alpha) & \dots & \sigma_n(\alpha_n) \\
    \end{pmatrix}^2 \\
    &= \det \left(\begin{pmatrix}
        \sigma_1(\alpha_1)& \dots & \sigma_1(\alpha_n) \\
        \vdots & \ddots & \vdots \\
        \sigma_n(\alpha_1) & \dots & \sigma_n(\alpha_n) \\
    \end{pmatrix}
    \begin{pmatrix}
        b_1 & 0 & \dots & 0 \\
        b_2 & 1 & \dots & 0 \\
        \vdots & & \ddots & \\
        b_n & \dots & \dots & 1
    \end{pmatrix}
    \right) \\
    &= \disc(K) b_1^2.
\end{align*}

Let $\disc = \disc(K) b_1^2.$ It remains to prove that $p \mid D$, since $p \nmid b_1.$ Since we don't know where $p$ arises in $\disc(K)$, construct $\disc$ with all its conjugates above $p.$

\textbf{Want:} $\disc = p\mathbb{Z}$; by which the result follows easily. 

Let $L$ be the Galois closure of $K$, and thus all the automorphisms of $\alpha$ are in L. Similar to how $\alpha$ belongs to all primes $\mathcal{O}_K$ above $p$, we know that all $\alpha \in K \subset L$ belongs to all primes of $\mathcal{O}_L$ above $p.$ Denote that set by $P.$ 

We know that $\mathfrak{P} \cap \mathcal{O}_K$ is a prime ideal of $\mathcal{O}_K$ above $p.$ Fix a prime $\mathfrak{P} \in P$ above $p$ in $\mathcal{O}_L.$ We have that $\sigma_i(\mathfrak{P})$ is a prime ideal of $\mathcal{O}_L$, and since $\sigma_i(\mathfrak{P})$ is also in $L.$ 

Since $\sigma_i(\mathfrak{P})$ is prime, $p=\sigma_i(p) \in \sigma_i(\mathfrak{P})$, thus implying that $\sigma_i(\alpha) \in \mathfrak{P}$ for all $\sigma_i.$ Thus, the leftmost column of 
$$\disc = \det \begin{pmatrix}
        \sigma_1(\alpha)& \dots & \sigma_1(\alpha_n) \\
        \vdots & \ddots & \vdots \\
        \sigma_n(\alpha) & \dots & \sigma_n(\alpha_n) \\
\end{pmatrix}^2$$ is all in $\mathfrak{P}$, implying that $\disc \in \mathfrak{P}.$ Since $\disc$ is also an integer, this shows that $\disc \in \mathfrak{P} \cap \mathbb{Z} = p\mathbb{Z}. \qed$ 

Now we know that $p$ divides the discriminant of $K.$ 

\coro{Ramification in number ring towers}{
    Consider a tower of number rings $R_1 \subset R_2 \subset \dots \subset R_n.$ For any $R_{i-1} \subset R_i$, only finitely many primes in $R_{i-1}$ ramify in $R_i.$
}

\textbf{Proof:} Let $P$ be a prime of $R_{i-1}$ that ramifies in $R_i.$ Then $P \cap \mathbb{Z} = p\mathbb{Z}$ ramifies in $R_i.$ The choices of $p$ are finite - since $p \mid \disc(K)$, and each $p$ lies underneath finitely many primes of $R_{i-1}.$ Thus, the choices of $P$ are finite. $\qed$

Let $A$ and $B$ (AKLB setup) be such that $A \subset B$, and set $P \cap \mathbb{Z} = p\mathbb{Z}$. $p$ can be ramified in $B$ without $P$ being ramified in $B$. There is a set of specific criterion that determines whether a specific prime $Q \in B$ is ramified over $A.$ That is, setting $P=Q \cap A$, $e(Q|P)>1.$

The following section is a study of the theory of lattices and differents.

\section{The different ideal}

It was previously proven that the \textbf{prime factors} of the discriminant ramify in $\mathcal{O}_K.$ We can express this as saying $e(\mathfrak{p}|p)>1$ if and only if $p \mid \disc(K).$  

However, which prime ideals ramify in $\mathcal{O}_K?$ More specifically, which $\mathfrak{p} \in \mathcal{O}_K$ satisfy $e(\mathfrak{p}|p)>1?$ We also only know that the multiplicity of a prime is positive for ramified primes. Can we do better?

\defn{Lattice}{
    The lattices in $\mathbb{R}^n$ are defined by discrete (additive) subgroups $\Lambda$ such that $\mathbb{R}^n/\Lambda$ is compact. Or, given a basis $(e_1, \dots, e_n) \in \mathbb{R}^n$, a lattice is the $\mathbb{Z}$-span of said basis.
}

For any lattice $M \subset \mathbb{R}^n$, we associate with it a dual lattice given by $L^{\vee}=\{w \in \mathbb{R}^n: w \cdot L \subset \mathbb{Z}.\}$. However, for a lattice in a number field, we use the following definition:

\defn{Dual lattice in number field}{
   Let $L$ be a lattice in $K$. Its dual lattice is given by$$L^{\vee}=\alpha \in K: \text{Tr}_{K/\mathbb{Q}}(\alpha L) \subset \mathbb{Z}.$$
}

This is done by replacing $\mathbb{R}^n$ and its dot product $(v, w) \mapsto v \cdot w$ with a number field $K$ and the \emph{trace pairing} $(x, y) \mapsto \text{Tr}_{K/\mathbb{Q}}(xy).$

We can illustrate this with an example. 

\ex{}{
    Let $K=\mathbb{Q}(i)$, and $L=\mathbb{Z}[i].$ For all $a+bi \in \mathbb{Q}(i)$, $a+bi \in L^{\vee}$ when $\text{Tr}_{\mathbb{Q}(i)/\mathbb{Q}}(a+bi) \in \mathbb{Z}$, and $\text{Tr}_{\mathbb{Q}(i)/\mathbb{Q}}(a+bi)i=-b+ai \in \mathbb{Z}$.  
    
    Thus, we have that $2a \in \mathbb{Z}$ and $-2b\in\mathbb{Z}$, so we get that $(e_1, e_2)=(2, -2)$. An appropriate choice of dual lattice is $(e_1^{\vee}, e_2^{\vee})=(\frac{1}{2}, \frac{1}{2})$
    
    The dual lattice $\mathbb{Z}[i]^{\vee}=\frac{1}{2}\mathbb{Z}+\frac{1}{2}\mathbb{Z}i=\frac{1}{2}\mathbb{Z}[i].$
}

Is $L^{\vee}$ always a lattice? We affirm this question in the following proposition.

\prop{}{
    Let $K$ be a number field and $L \subset K$ be a lattice with $\mathbb{Z}$-basis $e_1, \dots, e_n.$ Then, $L^{\vee}=\bigoplus\limits_{i=1}^n \mathbb{Z} e_i^{\vee}$, where $\{e_i^{\vee}\}$  is the dual basis to $\{e_i\}$ w.r.t the trace pairing on $K/\mathbb{Q}.$ Thus, $L^{\vee}$ is a lattice.
}

\textbf{Proof:} Take an element $\alpha \in K$, and write it as $\alpha = \sum\limits_{i=1}^n c_i e_i^{\vee}.$ Then, $\alpha\cdot e_i = c_i$, meaning that all the coefficients $c_i$ are integers. $L^{\vee}$ is thus spanned by the $e_i^{\vee}$s.  $\qed$

An important property of lattices is that $L^{\vee\vee}=L.$ This is analogous to how the double dual of a vector space $V$, denoted $V^{**}$, is isomorphic to $V$ - since the dual basis of a dual basis (commonly known as the \emph{double dual}) is the original basis.

\ex{Lattice of $\mathbb{Z}[\sqrt{d}]$}{
   We studied the dual lattice of the Gaussian integers. Now, consider a general quadratic field $K=\mathbb{Q}(\sqrt{d}).$

    Consider the lattices $\mathbb{Z}+\mathbb{Z}\sqrt{d}$ and $\mathbb{Z}+\mathbb{Z}\frac{1+\sqrt{d}}{2}$, in which the second case only applies for $d\equiv 1 \Mod 4.$ 
    
    Referring to the trace product $\text{Tr}_{\mathbb{Q}(i)/\mathbb{Q}}(a+b\sqrt{d})$ and $\text{Tr}_{\mathbb{Q}(i)/\mathbb{Q}}((a+b\sqrt{d})\sqrt{d})=bd+a\sqrt{d}$, we get that $2a \in \mathbb{Z}$ and $2bd \in \mathbb{Z}$. 
    
    Thus, the dual basis of $\{1, \sqrt{d}\}$ relative to the trace product is $\{\frac{1}{2}, \frac{1}{2\sqrt{d}}\}$, so $(\mathbb{Z}+\mathbb{Z}\sqrt{d})^{\vee} = \frac{1}{2}\mathbb{Z}+\frac{1}{2\sqrt{d}}\mathbb{Z}=\frac{1}{2\sqrt{d}}(\mathbb{Z}+\mathbb{Z}\sqrt{d}).$

    Referring to $\text{Tr}_{\mathbb{Q}(i)/\mathbb{Q}}(a+b(\frac{1+\sqrt{d}}{2}))$, we have that the trace of $(a+b(\frac{1+\sqrt{d}}{2})) \in \mathbb{Z}$ when $2a+b \in \mathbb{Z}$, or the trace of $(a+b(\frac{1+\sqrt{d}}{2}))(\frac{1+\sqrt{d}}{2})=a(\frac{1+\sqrt{d}}{2})+b(\frac{1+2\sqrt{d}+d}{4})\in\mathbb{Z}$. 
    
    The trace of the second sum is $(a+\frac{b}{2}+\frac{bd}{2})=\frac{2a+b+bd}{2}$, which $\in \mathbb{Z}.$ We therefore have that the dual basis is $\{-\frac{1}{2}, \frac{1}{\sqrt{d}}\}$, corresponding to $a=-\frac{1}{2}, b=\frac{1}{\sqrt{d}}.$ A check shows that $\mathbb{Z}+\mathbb{Z}(\frac{1+\sqrt{d}}{2})=\mathbb{Z}(-\frac{1}{2})+\mathbb{Z}(\frac{1}{\sqrt{d}})=\frac{1}{\sqrt{d}}(\mathbb{Z} (\frac{-1+\sqrt{d}}{2})+\mathbb{Z})$, which gives our desired result.        
}

Thus we have the following results:

\begin{enumerate}
    \item $(\mathbb{Z}+\mathbb{Z}\sqrt{d})^{\vee}= \frac{1}{2\sqrt{d}}(\mathbb{Z}+\mathbb{Z}\sqrt{d})$ 
    \item $(\mathbb{Z}+\mathbb{Z}(\frac{1+\sqrt{d}}{2}))^{\vee}=\frac{1}{\sqrt{d}}(\mathbb{Z}+\mathbb{Z}(\frac{1+\sqrt{d}}{2}))$
\end{enumerate}

A lattice worth studying is the ring of integers $\mathcal{O}_K.$ Its dual lattice is given by $$\mathcal{O}_K^{\vee} = \{\alpha \in K: \text{Tr}_{K/\mathbb{Q}}(\alpha \mathcal{O}_K) \subset \mathbb{Z}.\}$$

Since the trace of algebraic integers $\in \mathbb{Z}$, and $\mathcal{O}_K$ is a ring, we have that $$\mathcal{O}_K \subset \mathcal{O}_K^{\vee}$$

The results regarding the dual lattice of quadratic fields show that the dual lattice contains the original lattice. $\mathcal{O}_K^{\vee}$, a fractional ideal achieves a similar effect - in which the dual lattice contains $\mathcal{O}_K^{\vee}$. 

Consider a fractional ideal $\mathfrak{a} \in K$. $\mathfrak{a}^{\vee}$ is a fractional ideal satisfying $$\mathfrak{a}^{\vee}=\mathfrak{a}^{-1} \mathcal{O}_K^{\vee}.$$
\textbf{Proof:} Consider $\mathfrak{a}^{\vee} = \{\alpha \in K: \text{Tr}_{K/\mathbb{Q}}(\alpha\mathfrak{a}) \subset \mathbb{Z}\}$ . We show that $\mathfrak{a}^{\vee}$ is a fractional ideal. Firstly, it is finitely generated as a $\mathbb{Z}$-Module. Secondly, for any $x \in \mathcal{O}_K$ and $\alpha \in \mathfrak{a}^{\vee}$, $\text{Tr}_{K/\mathbb{Q}}((x\alpha)\mathfrak{a})=\text{Tr}_{K/\mathbb{Q}}(\alpha(x\mathfrak{a})) \subset \text{Tr}_{K/\mathbb{Q}}(\alpha \mathfrak{a})$, which $\in \mathbb{Z}$.

Pick arbitrary $\alpha \in \mathfrak{a}^{\vee}.$ For any $\beta \in \mathfrak{a}$, we have that $\text{Tr}_{K/\mathbb{Q}}(\alpha\beta\mathcal{O}_K) \in \mathbb{Z}$ since $\beta\mathcal{O}_K \subset \mathfrak{a}$. This means that $\alpha\beta \in \mathcal{O}_K$, or $\mathfrak{a}^{\vee}\mathfrak{a} \subset \mathcal{O}_K^{\vee}.$

We reverse this inclusion to conclude that $\mathfrak{a}^{\vee} \subset \mathfrak{a}\mathcal{O}_K^{\vee}. \qed$

\coro{}{
    The dual lattice $\mathcal{O}_K^{\vee}$ is the largest fractional ideal in $K$ with the trace of all its elements $\in \mathbb{Z}.$ 
}

\textbf{Proof:} We have that $\mathfrak{a}=\mathfrak{a}\mathcal{O}_K$, where $\mathfrak{a}$ is a fractional ideal. Thus, $\text{Tr}_{K/\mathbb{Q}}(\mathfrak{a}) \subset \mathbb{Z} \iff \text{Tr}_{K/\mathbb{Q}}(\mathfrak{a}\mathcal{O}_K) \subset \mathbb{Z}$, thus implying that $\mathfrak{a}\subset\mathcal{O}_K. \qed$ 

It is important to note that $\mathcal{O}_K^{\vee}$ does not denote the set of all elements in $K$ with trace $\in \mathbb{Z}$. It is merely the largest fractional ideal where all elements have integral trace. Consider the quadratic field $\mathbb{Q}(\sqrt{2})$ for example. The set of all eleents with integral trace are $\frac{1}{2}\mathbb{Z}+\mathbb{Q}\sqrt{2}$, which is an additive group but not a fractional ideal. However, $\frac{1}{2}\mathbb{Z}+\mathbb{Q}\sqrt{2}$ contains $\frac{1}{2\sqrt{2}}\mathbb{Z}+\frac{1}{2\sqrt{2}}\mathbb{Z}\sqrt{2}=\frac{1}{2\sqrt{2}}\mathbb{Z}[\sqrt{2}].$ 

We now introduce the notion of a different ideal. We know that $\mathcal{O}_K \subset \mathcal{O}_K^{\vee}.$ Since $\mathcal{O}_K^{\vee}$ is a fractional ideal, the inverse of $\mathcal{O}_K^{\vee}$ is a integral ideal which is contained inside $\mathcal{O}_K$.

\defn{Different ideal}{
    The different ideal of $K$ is defined as $$\mathcal{D}_K=(\mathcal{O}_K^{\vee})^{-1} = \{x \in K: x \mathcal{O}_K^{\vee} \subset \mathcal{O}_K\}$$
}

\ex{Examples of different ideals}{

    $$\mathcal{D}_{\mathbb{Q}(i)} = 2\mathbb{Z}[i]$$
    
and this is because $\mathbb{Z}[i]^{\vee} = \frac{1}{2}\mathbb{Z}[i].$ Additionally, we have that $$\mathcal{D}_{\mathbb{Q}(\sqrt{d})}=\begin{cases} 2\sqrt{d}, & \text{if}\ d \not\equiv 1\Mod 4 \\ \sqrt{d}, & \text{if}\ d \equiv 1 \Mod 4 \end{cases}$$
    
}

In fact, if $\mathcal{O}_K = \mathbb{Z}[\alpha]$, then $\mathcal{D}_K = f'(\alpha)$, where $f(X)$ is the minimal polynomial of $\alpha$ in $\mathbb{Z}[X].$

\textbf{Proof:} We first present this theorem. The result directly follows from this theorem.

\thrm{}{
    Let $K=\mathbb{Q}(\alpha)$ and $f(X)$ be the minimal polynomial of $\alpha$ in $\mathbb{Q}[X].$ Expressing $f(X)=(X-\alpha)(c_0(\alpha)+c_1(\alpha)T+\dots+c_{n-1}(\alpha)T^{n-1})$, with all the $c_i(\alpha) \in K$, we have that the dual basis to $\{1, \alpha, \dots, \alpha^{n-1}\}$ w.r.t the trace pairing is $\{\frac{c_0(\alpha)}{f'(\alpha)}, \frac{c_1(\alpha)}{f'(\alpha)}, \dots, \frac{c_{n-1}(\alpha)}{f'(\alpha)}\}.$

    Thus, $(\mathbb{Z}+\mathbb{Z}\alpha+\dots+\mathbb{Z}\alpha^{n-1})^{\vee} = \frac{1}{f'(\alpha)}(\mathbb{Z}+\mathbb{Z}\alpha+\dots+\mathbb{Z}\alpha^{n-1}).$
}

\textbf{Proof:} Denote $\alpha_1, \dots, \alpha_n$ as the conjugates of $\alpha$. Note that $$\sum\limits_{i=1}^n \frac{1}{f'(\alpha_i)} \cdot \frac{f(X)}{X-\alpha_i}=1.$$ 

This identity is commonly accredited to Euler. From this, we know that $$\sum\limits_{i=1}^n \frac{\alpha_i^k}{f'(\alpha_i)} \cdot \frac{f(X)}{X-\alpha_i}=X^k$$ 
and since $\frac{f(X)}{X-\alpha}=(c_0(\alpha)+c_1(\alpha)T+\dots+c_{n-1}(\alpha)T^{n-1})$,  $$\sum\limits_{i=1}^n \frac{\alpha_i^k}{f'(\alpha_i)} \cdot c_j(a_i)=\delta_{jk}.$$ 
We know that the left side is $\text{Tr}_{K/\mathbb{Q}} (\frac{\alpha^k c_j(\alpha)}{f'(\alpha)})$. Thus, $\{\frac{c_j(\alpha)}{f'(\alpha)}\}$ is dual to $\{\alpha^j\}$. By the definition of dual lattice, we have that $$(\mathbb{Z}+\mathbb{Z}\alpha+\dots+\mathbb{Z}\alpha^{n-1})^{\vee} = \frac{1}{f'(\alpha)}(\mathbb{Z}c_0(\alpha)+\mathbb{Z}c_1(\alpha)+\dots+\mathbb{Z}c_{n-1}(\alpha)).$$

It remains to show that $\mathbb{Z}c_0(\alpha)+\mathbb{Z}c_1(\alpha)+\dots+\mathbb{Z}c_{n-1}(\alpha)=\mathbb{Z}+\mathbb{Z}\alpha+\dots+\mathbb{Z}\alpha^{n-1}.$ We intend to find a closed form for $c_j(\alpha)$, the coefficient of $X^j$ in $\frac{f(X)}{X-\alpha}.$ Let $f(X)=a_0+a_1X+\dots+a_{n-1}X^{n-1}+a_nX^n \in \mathbb{Z}[X].$ We know that since $f(\alpha)=0$, 

\begin{align*}
    \frac{f(X)}{X-\alpha}&= \frac{f(X)-f(\alpha)}{X-\alpha} \\
    &= \sum\limits_{i=1}^n a_i \frac{X^i-\alpha^i}{X-\alpha} \\
    &= \sum\limits_{i=1}^n a_i \sum\limits_{j=0}^{i-1} \alpha^{i-1-j} X^j \\
    &= \sum\limits_{j=0}^{n-1} \sum\limits_{i=j+1}^n a_i\alpha^{i-1-j}X^j,
\end{align*}

thus establishing the result that $c_j(\alpha)=a_i \alpha^{i-1-j}$. A mapping from $\{1, \alpha, \dots, \alpha^{n-1}\}$ to $\{c_{n-1}(\alpha), \dots, c_1(\alpha), c_0(\alpha)\}$ yields an invertible map since its matrix is triangular with 1's on the diagonal. Thus, $\mathbb{Z}c_0(\alpha)+\mathbb{Z}c_1(\alpha)+\dots+\mathbb{Z}c_{n-1}(\alpha)$ and $\mathbb{Z}+\mathbb{Z}\alpha+\dots+\mathbb{Z}\alpha^{n-1}$ have the same span, suggesting equivalence. $\qed$

\prop{}{
    For every number field $K$, $N(\mathcal{D}_K) = |\disc(K)|.$
}

\textbf{Proof:} Let $e_1, \dots, e_n$ be a basis for $\mathcal{O}_K$. Represent $\mathcal{O}_K=\bigoplus\limits_{i=1}^n \mathbb{Z} e_i.$ From $\mathcal{O}_K = \mathcal{D}_K^{-1}$, we have that $$N(\mathcal{D}_K)=[\mathcal{O}_K:\mathcal{D}_K]=[\mathcal{D}_K^{-1}:\mathcal{O}_K]=[\mathcal{O}_K^{\vee}:\mathcal{O}_K],$$

where $$[\mathcal{O}_K:\mathcal{D}_K]=[\mathcal{D}_K^{-1}:\mathcal{O}_K]$$ because of the inversion of ideals. Then, represent a $\mathbb{Z}$-basis of $\mathcal{O}_K^{\vee}$ in matrix form; label the matrix $M_2$ and the basis $e_i^{\vee}.$ Letting the $\mathbb{Z}$-basis of $\mathcal{O}_K$ be $e_i$, we label the matrix represented by the basis be $M_1$.  As we know that $e_j=\sum\limits_{i=1}^n a_{ij} e_i^{\vee}$, we note that $a_{ij}=\text{Tr}_{K/\mathbb{Q}} (e_ie_j)$, and thus the determinant of the matrix $(a_{ij})$ is $\det(\text{Tr}_{K/\mathbb{Q}} (e_ie_j))$, which equals $|\disc(K)|$, as required.

We will end this section on different ideals with a theorem of Dedekind.

\thrm{Dedekind}{
    The prime ideal factors of $\mathcal{D}_K$ are the primes in $K$ that ramify over $\mathbb{Q}.$
}

This is a result with a non-trivial proof. However, the following result is a corollary of the fact that $N(\mathcal{D}_K)=\disc(K).$

\thrm{}{
    The prime ideal factors of $\mathcal{D}_K$ are the primes in $Q$ that ramify in $K.$
}

\textbf{Proof:} 

$(\Rightarrow)$ We follow from the fact that $N(\mathcal{D}_K) = \disc(K).$ If p is a prime dividing $\disc(K)$, then $\mathcal{D}_K$ has a prime ideal factor which norm is $p^k$ for some $k$. Thus, $(p)$ is divisible by some factor of $\mathcal{D}_K$, meaning that $p$ ramifies in $K.$

$(\Leftarrow)$ If $p$ ramifies in $K$, then $\mathcal{D}_K$ is divisible by a prime ideal factor of $(p)$, so $N(\mathcal{D}_K)$ is divisible by some factor of $(p),$ and is thus divisible by $p. \qed$   

\chapter{Minkowski theory} \label{ch2}
\minitoc
\section{Minkowski's Bound}
\thrm{Minkowski's Bound}{
    Let $K$ be a number field, where $[K:\mathbb{Q}]=n.$ Each ideal class contains an integral ideal $\mathfrak{a}$ satisfying $$N(\mathfrak{a})\leq\frac{n!}{n^n}\left(\frac{4}{\pi}\right)^{r_2} \sqrt{|\disc(K)|}$$
    where $r_2$ denotes the number of complex embeddings of $K$, and $n=r_1+2r_2$ where $r_1$ is the number of real embeddings of $K$.
}

\defn{Minkowski's Constant}{
    Define Minkowski's Constant, or $m_K$, as $$\frac{n!}{n^n}\left(\frac{4}{\pi}\right)^{r_2} \sqrt{|\disc(K)|}.$$

    In other words, $m_K$ is defined such that $N(\mathfrak{a}) \leq m_K.$
}

Consider $C$, a bounded domain in $\mathbb{R}^n$. Set a "shift" function $$\psi(x) = \sum\limits_{\gamma \in \mathbb{Z}^n} \varphi(x+\gamma) $$

Consider the quotient space $\mathbb{R}^n / \mathbb{Z}^n.$ We have, by a change of summation and integration,

\begin{align*}
    \int_{\mathbb{R}^n / \mathbb{Z}^n} \psi(x) &= \sum\limits_{\gamma \in \mathbb{Z}^n }\int_{\mathbb{R}^n / \mathbb{Z}^n}\psi(x+\gamma) \ dx \\
    &= \sum\limits_{\gamma \in \mathbb{Z}^n} \int_{\gamma + \mathbb{R}^n / \mathbb{Z}^n} \psi(x) \ dx \ \text{(in which the $\gamma$ "cover" all the quotiented $\mathbb{Z}^n$)} \\ &= \int_{\mathbb{R}^n} \psi(x) \ dx \\ &= \text{vol}(C) > 1 \ \text{(since there exists a non-trivial $x \in C \implies -x \in C.$)}
\end{align*}

This also implies that $\psi(x) \geq 2.$ Consider the symmetric domain $\frac{1}{2}C$, and let $C$ have volume $>2^n.$

By definition, there exists two points $\frac{1}{2}P$ and $\frac{1}{2}Q$ such that their difference is a lattice point - because $$\text{vol}\left(\frac{1}{2}C\right) = \frac{\text{vol}(C)}{2^n}>1$$

which is a non-trivial lattice point. 

The lower bound on $2^n$ is strict - consider $-1 < x_i < 1$ for $i=1, \dots, n.$ The hypercube has volume $2^n$ - removing the lattice point $0$ yields a convex domain of size $2^n-1$ that does not contain a lattice point. 
\lemma{}{
    Consider the set $B_t = \Bigg\{\sum\limits_{i=1}^{r_1} |y_i| + 2 \sum\limits_{j=1}^{r_2} |z_j| \leq t\Bigg\}$, where $r_1$ denotes the number of copies of $\mathbb{R}$ and $r_2$ denotes the number of copies of $\mathbb{C}.$ The volume of $B_t$, denoted $V(r_1, r_2, t)$, is given by $$V(r_1, r_2, t) = \frac{2^{r_1-r_2}\pi^{r_2}t^n}{n!},$$ where $n=r_1+2r_2.$
}


\textbf{Proof:} We proceed with double induction on $r_1$ and $r_2$.

\textbf{Base case:} 

\tbf{Case 1}: Set $r_1 = 1, r_2 = 0$. This is equivalent to calculating the length of $[-t, t]$, which is $2t.$

\tbf{Case 2}: Set $r_1 = 0, r_2 =2$, and we are calculating the area of a disk with radius $\frac{t}{2}$. Thus, area is $\frac{\pi t^2}{4}$.

\textbf{Inductive step:} Assume that the formula holds for all $r_1, r_2, t$. 

\tbf{Case 1}: Consider $V(r_1+1, r_2, t)$. Note that this describes the volume of the set given by $$|y|+\sum\limits_{i=1}^{r_1} |y_i| + 2 \sum\limits_{j=1}^{r_2} |z_j| \leq t$$

or $$\sum\limits_{i=1}^{r_1} |y_i| + 2 \sum\limits_{j=1}^{r_2} |z_j| \leq t-|y|.$$

This creates the set $B_{t-|y|}.$ Change $|y|$ to $|y| + dy.$ This creates a box in $\mathbb{R}^{n+1}$ with $dy$ as one of the dimensions, evaluated as $B_{t-|y|}\ dy.$ Thus, $$V(r_1+1, r_2, t) = \int\limits_{-t}^t V(r_1, r_2, t-|y|) \ dy.$$ 

By the induction hypothesis, the R.H.S \begin{align*}
&=2\int\limits_{0}^{t}\frac{2^{r_1-r_2}\pi^{r_2}(t-y)^n}{n!} \ dy \\
&=\frac{2^{r_1+1-r_2}\pi^{r_2}t^{n+1}}{(n+1)!} \qed
\end{align*}

\tbf{Case 2}: We know that $V(r_1, r_2+1, t)$ is given by $$\sum\limits_{i=1}^{r_1} |y_i| + 2 \sum\limits_{j=1}^{r_2} |z_j|+2|z| \leq t$$

Similar in the first case, this suggests that $$V(r_1, r_2+1, t)=\int_{|z|\leq t/2} V(r_1, r_2, t-2|z|)\ d\mu(z)$$ where $\mu$ is the Lebesgue measure on $\mathbb{C}$.

By the induction hypothesis, the R.H.S $$=2\int\limits_{0}^{t}\frac{2^{r_1-r_2}\pi^{r_2}(t-|z|)^n}{n!} \ d\mu(z)$$

which by a change to polar coordinates, becomes $$\int\limits_{0}^{2\pi}\int\limits_{0}^{t/2} \frac{2^{r_1-r_2}\pi^{r_2}(t-2r)^n}{n!} r \ dr \ d\theta$$

This is equivalent to $$\frac{2^{r_1-r_2+1}\pi^{r_2}}{n!} \int\limits_{0}^{t/2}(t-2r)^n\ r\ dr$$

in which the integrand is equal to 
\begin{align*}
&\hspace{0.5cm} \frac{1}{2(n+1)}\int\limits_{0}^{t/2}(t-2r)^{n+1}\ dr \\
&= \frac{-(t-2r)^{n+2}}{2(n+1)2(n+2)}\Biggr|_0^{t/2} \\
&=\frac{t^{n+2}}{4(n+1)(n+2)}
\end{align*}

Thus, \begin{align*}V(r_1, r_2+1, t) &= \frac{2^{r_1-r_2+1}\pi^{r_2}}{n!}\cdot \frac{t^{n+2}}{4(n+1)(n+2)} \\
&=\frac{2^{r_1-(r_2+1)}\pi^{r_2+1}t^{n+2}}{(n+2)!}
\end{align*}

which completes the inductive step. We need $n+2$ instead of $n+1$ for this inductive step, because we evaluate here $r_1+2(r_2+1)=r_1+2r_2+2=n+2$, instead of $n+1.$ \\

\thrm{Minkowski's Convex-Body Theorem}{
    Let $C$ be a bounded, symmetric, convex domain in $\mathbb{R}^n$, and $a_1, \dots, a_n$ be linearly independent vectors. Let $A$ be the integer lattice which points are all weighted sums formed by the $\{a_i\}.$ If $$\text{vol}(C)>2^n|\det A|,$$ there exists rational integers that span $C$, or in other words, there exists non-zero $x_1, \dots, x_n$ such that $$x_1a_1+\dots+x_na_n \in C.$$\label{lemma:1.2}

    This also means that the set $S \cap (A \backslash \{0\})$ is non-empty.
}

\textbf{Proof:} Let $D$ be the set of all $x_1, \dots, x_n$ such that $x_1a_1+\dots+x_na_n \in C.$ Since $C$ is a symmetric domain, $D$ is also a symmetric domain. 

From $D=A^{-1}C$, we have that $$\text{vol}(D) = \text{vol}(C)(|\det(A)|)^{-1}$$

If $\text{vol}(D) > 2^n$, there exists a non-trivial lattice point such that $x_1a_1+\dots+x_na_n \in C.$ The condition that $\text{vol}(D) > 2^n$ implies that $\text{vol}(C) > 2^n |\det(A)|. \qed$

With the above two lemmas, we set forth to prove Minkowski's bound. 

We reconsider $B_t = \Bigg\{\sum\limits_{i=1}^{r_1} |y_i| + 2 \sum\limits_{j=1}^{r_2} |z_j| \leq t\Bigg\}$

Let $\mathfrak{b}$ be an ideal with basis $\omega_1, \dots, \omega_n.$ The canonical embedding $(r_1, r_2) \to \mathbb{R}^n$ given by $$K \to \mathbb{R}^{r_1} \oplus \mathbb{C}^{r_2} \cong \mathbb{R}^n$$ is given by $$\omega_i \mapsto \sigma_1(\omega_i), \dots, \sigma_{r_1}(\omega_i), \sigma_{r_1+1}(\omega_i), \dots, \sigma_{r_1+r_2}(\omega_i).$$

The generator matrix of $\mathfrak{b}$, or $A$, is given by 

$$\begin{pmatrix}
\sigma_1(\omega_1) \dots \sigma_{r_1}(\omega_1) & \text{Re}(\sigma_{r_1+1}(\omega_1))\text{Im}(\sigma_{r_1+1}(\omega_1)) \dots \text{Re}(\sigma_{r_1+r_2}(\omega_1))\text{Im}(\sigma_{r_1+r_2}(\omega_1)) \\

\vdots & \vdots \\

\sigma_1(\omega_1) \dots \sigma_{r_1}(\omega_1) & \text{Re}(\sigma_{r_1+1}(\omega_1))\text{Im}(\sigma_{r_1+1}(\omega_1)) \dots \text{Re}(\sigma_{r_1+r_2}(\omega_1))\text{Im}(\sigma_{r_1+r_2}(\omega_1))\\

\end{pmatrix}$$

This matrix has determinant $\det A = 2^{-r_2}|N(\mathfrak{b})|\sqrt{|\disc(K)|}$. Set $\alpha = x_1\omega_1+\dots+x_n\omega_n \in \mathfrak{b}.$ By invoking the AM-GM inequality, we get that

\begin{align*}
|N(\alpha)|&=\prod\limits_{k=1}^{r_1} |\sigma_k(\alpha)|\prod\limits_{k=1}^{r_2}|\sigma_k(\alpha)|^2 \\
&\leq n^{-n} \left(\sum\limits_{k=1}^{r_1} +|\sigma_k(\alpha)|\sum\limits_{k=1}^{r_2}|\sigma_k(\alpha)|^2\right)^n\\
&=\frac{t^n}{n^n}.
\end{align*}

Choose $t$ such that the volume of $B_t$ is equal to $2^n |\det(A)|.$ We have that $\dfrac{2^{r_1-r_2}\pi^{r_2}t^n}{n!}=2^n|\det(A)|$, or 

\begin{align*}
    \frac{t^n}{n!}&=\frac{2^n|\det(A)|}{2^{r_1-r_2}\pi^{-r_2}}\\
    &= \frac{2^{r_1+2r_2}}{2^{r_1}\pi^{r_2}} |N(\mathfrak{b})| \sqrt{|\disc(K)|} \\
    &=\left(\frac{4}{\pi}\right)^{r_2} |N({\mathfrak{b})|\sqrt{\disc(K)}}.
\end{align*}

Combining the fact that $|N(\alpha)|n^n=t^n$, we get 
\begin{equation} \label{Min1}
     |N(\alpha)| \leq \frac{n!}{n^n}\left(\frac{4}{\pi}\right)^{r_2} |N({\mathfrak{b})|\sqrt{|\disc(K)|}}.  
\end{equation}
    
The result starts to look promising. We conclude by setting $(\alpha)=\mathfrak{a}\mathfrak{b}$, which means that  

\begin{equation}
    |N(\mathfrak{a})| \leq \frac{n!}{n^n}\left(\frac{4}{\pi}\right)^{r_2}\sqrt{|\disc(K)|}. \qed
\end{equation} \newpage

\section{Results following Minkowski's Bound}

Knowing Minkowski's Bound, we set to prove some key results in algebraic number theory.

\thrm{Minkowski's Bound on Ideal Norms}{ 
    \ Let $K$ be a number field with integral ideal $I$. Then, $$N(I) \leq \frac{n!}{n^n}\left(\frac{4}{\pi}\right)^{r_2}\sqrt{|\disc(K)|}.$$ \label{section:MIdeal}
}

\textbf{Proof:} Let $J^{-1}$ be a fractional ideal in the ideal class. Then, $J$ is an integral ideal.

Choose an element $\alpha$ satisfying $|N(\alpha)| \leq m_K |N(J^{-1})|=|N(J)^{-1}|,$ as mentioned in \ref{Min1}

Thus, $|N(\alpha)N(J)| = |N(\alpha J)|=m_K,$ proving that $a \in J^{-1}.$ Thus, $aJ \subseteq J^{-1}J = \mathcal{O}_K,$ meaning that $I=aJ$ is an ideal of $\mathcal{O}_K$ contained in the ideal class $J.$ Furthermore, $N(I) \leq m_K$, as desired.

\coro{Finite class group}{ 
    The class group of any number field is finite. \label{finitecg}
}

\textbf{Proof:} There are a finite number of integral ideals for any number field with a given norm. Associate with each ideal class an integral ideal with Minkowski's bound. Since the norms are bounded above by a constant, this suggests that an infinite ideal class group would require the use of the same ideal in two different ideal classes, contradicting uniqueness.

\coro{Bound on $\disc(K)$}{
    Let $[K:\mathbb{Q}]=n.$ Then, $$|\disc(K)| \geq \left(\frac{\pi}{4}\right)^n\left(\frac{n^n}{n!}\right)^2.$$\label{section:dkbound}
}

\textbf{Proof:} Since the norm of an ideal must $\geq 1,$
\begin{align*}
    |N(\mathfrak{a})| &\leq \frac{n!}{n^n}\left(\frac{4}{\pi}\right)^{r_2}\sqrt{|\disc(K)|} \\
    \frac{1}{\sqrt{|\disc(K)|}} &\leq \frac{n!}{n^n}\left(\frac{4}{\pi}\right)^{r_2} \\
    |\disc(K)| &\geq \left(\frac{n^n}{n!}\right)^2\left(\frac{\pi}{4}\right)^{2r_2}\\
    &\geq\left(\frac{n^n}{n!}\right)^2\left(\frac{\pi}{4}\right)^n. \ \qed
\end{align*}

We aim to characterise a relationship between degree and discriminant, as follows:

\prop{}{
    $|\disc(K)| \to \infty$ as $n \to \infty.$ 
}
 \textbf{Proof:} This is because $\log(n!)=n\log n-n + O(\log n)$, implying
\begin{align*}
    \log(\disc(K))&\geq \log\left(\left(\frac{\pi}{4}\right)^n\left(\frac{n^n}{n!}\right)^2\right)\\
    &=n\log\frac{\pi}{4}+2\log\left(\frac{n\log n}{n \log n - n+O(\log n)}\right)\\
    &= \left(2+\log\frac{\pi}{4}\right)n+O(\log n). \ \qed
\end{align*}

Also, by an explicit form of Stirling's approximation $n! \leq e \sqrt{n} \left(\frac{n}{e}\right)^n$, 

\begin{equation}
    |\disc(K)| \geq \left(\frac{\pi}{4}\right)^n\left(\frac{n^n}{n!}\right)^2 \geq \frac{1}{e^2n}\left(\frac{\pi e^2}{4}\right)^n.
\end{equation}

Stirling's formula states that $$n!=\sqrt{2\pi n}\left(\frac{n}{e}\right)^n e^{\frac{k}{12n}}$$ for some $k\in(0, 1)$, attaining the inequality 

\begin{equation}
    |\disc(K)| \geq \left(\frac{\pi}{4}\right)^{n}\frac{1}{2\pi n} e^{2n-\frac{k}{6n}} \label{eqn1.4}
\end{equation}

in which the $e^{2n-\frac{k}{6n}}$ is bounded upwards by $e^{2n-\frac{1}{6n}}.$

We can improve the bound in the next corollary.

\coro{Better bound on $\disc(K)$}{ 
    $\disc(K)$ satisfies
    $$|\disc(K)| \geq \left(\frac{\pi}{3}\right)\left(\frac{3\pi}{4}\right)^{n-1} $$\label{coro:1.3}

    In other words, if $K \neq \mathbb{Q}, |\disc(K)|>1.$ Hence, any nontrivial extension of $K$ contains a ramified prime.
}

\textbf{Proof:} Set $a_n = \left(\frac{\pi}{4}\right)^n\left(\frac{n^n}{n!}\right)^2.$ We know that 

\begin{align*}
    \frac{a_{n+1}}{a_n}&=\left(\frac{\pi}{4}\right)^{n+1}\frac{(n+1)^{2n+2}}{(n+1)!} / \left(\frac{\pi}{4}\right)^n\left(\frac{n^n}{n!}\right)^2 \\
    &= \frac{\pi}{4} \cdot \left(1+\frac{1}{n}\right)^{2n}
\end{align*}

The first few terms of $\left(1+\frac{1}{n}\right)^{2n}=1+\frac{2n}{n}+\dots=3+\dots$, in which the $\dots$ are positive. Thus, we can bound $\frac{a_{n+1}}{a_n}$ below by $\frac{3\pi}{4}.$

By verifying that $a_2=(\frac{\pi}{4})^2\cdot 4,$ we get that 

\begin{align*}
    |\disc(K)|=a_{n} &= a_2 \left(\frac{a_3}{a_2}\right)  \dots \left(\frac{a_n}{a_{n-1}}\right) \\
    &\geq \frac{\pi^2}{4}\left(\frac{3\pi}{4}\right)^{n-2}\\
    &= \frac{\pi}{3}\left(\frac{3\pi}{4}\right)^{n-1}\qed
\end{align*}

The result that $|\disc(K)|>1$ follows from the fact that $a_2 > 1$, and that $a_{n+1} > a_n\ \forall n.$ Finally, by Dedekind's theorem, any nontrivial extension of $K$ contains a ramified prime. 

\thrm{Hermite}{Let $N$ be an integer. There are only finitely many number fields with $|\disc(K)| \leq N.$}

\textbf{Proof:} Fix $n=[K:\mathbb{Q}].$ We can find a non-zero element $\alpha \in \mathcal{O}_K$ such that $\alpha^{(1)}<\sqrt{|\disc(K)|},$ with $|\alpha^{(i)}| <1$ for $i=2, \dots, r.$ We invoke the following lemma:

\lemma{}{Let $K$ be a number field. There are only finitely many extensions $L/K$ of degree $n$ with given discriminant.}

\textbf{Proof:} Assume, WLOG, that $K\neq\mathbb{Q}$, and that $L/\mathbb{Q}$ is an extension of degree $m=n[K:\mathbb{Q}]$.

Consider the set $X$, with values $z_{\tau} \in K_{\mathbb{R}}$ such that $\Im(z_{\tau_0}) < C \sqrt{|\disc(K)|},\ |\Re(z_{\tau_0})|<1,\ |z_{\tau}|<1$ for $\tau \neq \tau_0, \bar{\tau_0}.$ 

The discriminant of $L/\mathbb{Q} = \disc_{K/\mathbb{Q}}^n\ N_{K/\mathbb{Q}}(\disc(K)).$ 

Since $\text{vol}(X) > 2^n \sqrt{|\disc(\mcO_K)|} = 2^n \text{vol}(\mathcal{O}_K)$, we find a primitive element of $K$ (denote that $\alpha$) which satisfies the conditions $$\Im(z_{\tau_0}\alpha) < C \sqrt{|\disc(K)|},\ |\Re(z_{\tau_0}\alpha)|<1,\ |\tau\alpha|<1 (\tau \neq \tau_0, \bar{\tau_0}).$$

Consider the conjugates $\tau(\alpha)$ of $\alpha$. The conditions on $\Im(\tau_0\alpha), \Re(\tau_0\alpha), |\tau\alpha|$ depend on $d$ and $n$, and are bounded once they are fixed. Thus, every field $K/\mathbb{Q}$ of finite degree is generated by $\alpha$ in the bounded region $X.$ 

Fixing $n$ and $d$, it is clear that there are finitely many fields with given degree and discriminant. We get that since bounded degree suggests bounded discriminant, there exist finitely many number fields with bounded discriminant. $\qed$

The famous \emph{Minkowski's Theorem} also follows from this.

\thrm{Minkowski's Theorem}{
    Let $K\neq \mathbb{Q}$ be a number field. $\disc(K) \neq \pm 1.$ 
}

\textbf{Proof:} Follows directly from the proof process above, in which $a_2=\frac{\pi^2}{4}>1$ and that $a_{n+1}>a_n.$ Since $a_n=\disc(K)$, the result follows.

\newpage

\section{Calculations with Minkowski's Bound}

Minkowski's Bound is crucial in calculating class numbers and class groups. The latter requires an in-depth theory on the factorisation of ideals, while the former admits a less tedious theory.

Recall that Minkowski's Bound states that for each number field, there exists an ideal class with norm bounded above by $$N(\mathfrak{a})\leq\frac{n!}{n^n}\left(\frac{4}{\pi}\right)^{r_2} \sqrt{|\disc(K)|}.$$
\ex{Trivial Minkowski bound}{
    By the Minkowski bound, the class group is trivial is when the bound is less than 2. For $\mathbb{Q}(\sqrt{d})$ with $d>0$, the bound $<2$ when $|\disc(K)|<16.$ For $\mathbb{Q}(\sqrt{-d})$, the bound $<2$ when $|\disc(K)|<\pi^2.$ 
}

Thus, here are some examples of quadratic fields with class number 1:

\textbf{Case 1:} $d>0$
\begin{enumerate}
    \item $\mathbb{Q}(\sqrt{2})$
    \item $\mathbb{Q}(\sqrt{3})$
    \item $\mathbb{Q}(\sqrt{5})$
    \item $\mathbb{Q}(\sqrt{13})$
\end{enumerate}

\textbf{Case 2:} $d<0$
\begin{enumerate}
    \item $\mathbb{Q}(\sqrt{-1})$
    \item $\mathbb{Q}(\sqrt{-2})$
    \item $\mathbb{Q}(\sqrt{-3})$
    \item $\mathbb{Q}(\sqrt{-7})$
\end{enumerate}

Evidently, there are more fields with class number 1. The imaginary quadratic fields with trivial class group are given by $\mathbb{Q}(\sqrt{-d})$, where $d$ are the 9 \emph{Heegner numbers} - $1, 2, 3, 7, 11, 19, 43, 67, 163.$

The study of class numbers of real quadratic fields is less simple. We don't know if the number of quadratic fields with trivial class group are finite. The \emph{Cohen-Lenstra heuristics} show that the density of primes that satisfy $C(\mathbb{Q}(\sqrt{-d})) \neq 1$ are approximately 76\%.

\ex{Class group of $\mathbb{Q}(\sqrt{82})$}{
   Let $K=\mathbb{Q}(\sqrt{82}).$ Then, $n=2, r_2=0, \disc(K)=4\cdot82=328$, and thus the Minkowski bound is $9.056$. We then consider the primes lying over $2, 3, 5, \text{and}\ 7$, and consider how $(p)$ factors over these primes.
    \begin{enumerate}
        \item Consider when we take$\Mod 2$. In this case, $x^2-82$ factors as $x^2$, and thus $(p)$ factorises as $\mathfrak{p}_2^2$
        \item Now, take $\Mod 3$. In this case, $x^2-82$ factors as $x^2-1=(x-1)(x+1)$, and thus $(p)$ factors as $\mathfrak{p}_3\mathfrak{p}_3^{-1}.$
        \item  Take $\Mod 5$ or $\Mod 7$. $x^2-82$ becomes $x^2-2$ or $x^2-5$ respectively, which are irreducible. Thus, $(p)$ is prime in both cases.
    \end{enumerate}

    Therefore, the class group is generated by $[\mathfrak{p}_2]$ and $[\mathfrak{p}_3].$ We try to find relations between the two generators.
    
    For $a, b \in \mathbb{Z}$, note that $N_{K/\mathbb{Q}}(a+b\sqrt{82})=a^2-82b^2.$  Try setting $a=10, b=1$; thus, we get that $N_{K/\mathbb{Q}}(10+\sqrt{82})=18=2\cdot 3^2.$ Since $10+\sqrt{82}$ is not divisble by $3$, it is divisible by one of $\mathfrak{p}_3$ or $\mathfrak{p}_3^{-1}.$
    
    Set $10+\sqrt{82}=\mathfrak{p}_2\mathfrak{p}_3^2.$ We have that $\mathfrak{p}_2 \sim \mathfrak{p}_3^{-2}$, suggesting that $[\mathfrak{p}_2]^2=1$, and that $[\mathfrak{p}_3]^2=[\mathfrak{p}_2].$ Thus, $[\mathfrak{p}_3]^2$ has order dividing $4.$ The group satisfying $a^2=1, b^2=a$ is isomorphic to $b^4=1$. 
    
    It only remains that we show that $\mathfrak{p}_2$ is non-principal to show that $K$ is generated by $\langle[\mathfrak{p}_3]\rangle$, which $\cong \mathbb{Z}/4\mathbb{Z}.$

    Let $\mathfrak{p}_2=(a+b\sqrt{82})$. Since $(2)$ is principal, $(2)=\mathfrak{p}_2^2=((a+b\sqrt{82})^2).$ Thus, we have that $$2=((a+b\sqrt{82}))^2u$$ , where $u$ is a unit. The unit group of $\mathbb{Z}[\sqrt{82}]$ is $\pm(9+\sqrt{82})$ since $9+\sqrt{82}$ has norm $-1$. The positive units of norm 1 are the integral powers of $(9+\sqrt{82})^2$ - meaning that must be able to solve $2=(a+b\sqrt{82})^2$ for $a, b \in \mathbb{Z}.$ But this is a contradiction as clearly, $\sqrt{2}$ does not lie in $\mathbb{Z}[\sqrt{82]}.$
}

\ex{$\mathbb{Q}(\sqrt{-30})$}{
    Let $K=\mathbb{Q}(\sqrt{-30}).$ We have that $n=2, r_2=1$ and $\disc(K) = -120$. Thus, the Minkowski bound is $6.97$, so the class group is generated by primes dividing $2, 3, \text{and}\ 5.$ Consider the factorisation of these primes into prime ideals.

    \begin{enumerate}
        \item $x^2+30 \Mod 2 = x^2$, and thus $(p)$ factorises as $\mathfrak{p}_2^2.$ 
        \item $x^2+30 \Mod 3 = x^2$, and thus $(p)$ factorises as $\mathfrak{p}_3^2$
        \item $x^2+30 \Mod 5 = x^2$, and thus $(p)$ factorises as $\mathfrak{p}_5^2$
    \end{enumerate}

    We note that $\mathfrak{p}_2, \mathfrak{p}_3$, and $\mathfrak{p}_5$ are non-pricipal, since $N_{K/\mathbb{Q}}(a+b\sqrt{-30})$ is never $2, 3,$ or $5.$ Thus, $\sqrt{-30}$ factorises as $\mathfrak{p}_2\mathfrak{p}_3\mathfrak{p}_5$, which $\sim 1.$  
    
    From $\mathfrak{p}_2\mathfrak{p}_3\mathfrak{p}_5 \sim 1$, we have $[\mathfrak{p}_2][\mathfrak{p}_3]=[\mathfrak{p}_5]^{-1}=[\mathfrak{p}_5]$ (both $[\mathfrak{p}_2]$ and $[\mathfrak{p}_3]$ have order 2), or that the class group of $\mathbb{Q}(\sqrt{-30})$ is generated by $\langle [\mathfrak{p}_2][\mathfrak{p}_3] \rangle = \mathbb{Z}/2\mathbb{Z} \times \mathbb{Z}/2\mathbb{Z} \cong K_4.$ 
}

\ex{Cubic field}{
    Let $K=\mathbb{Q}(\alpha)$, where $\alpha$ is a root of the polynomial $x^3-x-9.$ Note that the discriminant of $x^3-x-9$ is $-4(-1)^3-27(-9)^2=-2183 = -27 \cdot 59.$ Since $-2183$ is square-free, $\mathcal{O}_K=\mathbb{Z}[\alpha].$ We know that $r_2=1, n=3$, and thus the Minkowski bound is $13.21.$ Though tedious, we now factor $x^3-x-9 \Mod$ all primes from $2$ to $13.$

    \begin{enumerate}
        \item $x^3-x-9 \Mod 2 = x^3-x-1 \Mod 2$, which is irreducible.
        \item $x^3-x-9 \Mod 3 = x^3-x \Mod 3= x(x+1)(x-1)=x(x-1)(x-2)$, meaning that $(p)$ factorises as $\mathfrak{p}_3\mathfrak{p}_3^{'}\mathfrak{p}_3^{''}.$
        \item $x^3-x-9 \Mod 5 = x^3-6x-9 \Mod 5=(x-3)(x^2+3x+3) \Mod 5$, meaning that $(p)$ factorises as $\mathfrak{p}_5\mathfrak{p}_{25}.$
        \item $x^3-x-9 \Mod 7$ is irreducible.
        \item $x^3-x-9 \Mod 11 = x^3-56x-64 \Mod 11$, which factorises as $(x-8)(x^2+8x+8)$. This suggests that $(p)$ factorises as $\mathfrak{p}_{11}\mathfrak{p}_{121}.$ 
        \item $x^3-x-9 \Mod 13$ is irreducible.
    \end{enumerate}

    From this, it is clear that the class group of $\mathbb{Q}(\alpha)$ is generated by $[\mathfrak{p}_3], [\mathfrak{p}_3^{'}], [\mathfrak{p}_3^{''}], [\mathfrak{p}_5],$ and $[\mathfrak{p}_{11}].$ 

    We aim to get some relations between the generators of the class group. To do this, we consider the norm of ($a+\alpha$) for some small $a$. We know that $N_{K/\mathbb{Q}} (a+\alpha)$ is the constant term of the minimal polynomial of $(a+\alpha)$, which is $(x-a)^3+(x-a)-9.$ The polynomial has constant term $-(x^3-x+9)$, and thus $N_{K/\mathbb{Q}} (a+\alpha)=a^3-a+9$. Now, we explicitly calculate the norm of $(a+\alpha)$ for small $a.$

    \begin{enumerate}
        \item For $a=0, N_{K/\mathbb{Q}}(\alpha)= 9 = 3^2$
        \item For $a=-1, N_{K/\mathbb{Q}}(-1+\alpha)= 9 = 3^2$
        \item For $a=1, N_{K/\mathbb{Q}}(1+\alpha)= 9 = 3^2$
        \item For $a=2, N_{K/\mathbb{Q}}(2+\alpha)= 15 = 3 \cdot 5$
        \item For $a=-2, N_{K/\mathbb{Q}}(2+\alpha)= 3$
        \item For $a=3, N_{K/\mathbb{Q}}(2+\alpha)= 33 = 3 \cdot 11$
    \end{enumerate}

    Since the ideals $(\alpha), (1+\alpha)$, and $(-1+\alpha)$ all have norm $9$, we can represent $\alpha=\mathfrak{p}_3^2$, $(1+\alpha)=\mathfrak{p}_3^{'}$, and $(-1+\alpha)=\mathfrak{p}_3^{''}.$ We also know trivially that $(-2+\alpha)=(1+\alpha)=\mathfrak{p}_3.$ But since $(1+\alpha)=\mathfrak{p}_3^2=\mathfrak{p}_3$, we get that $\mathfrak{p}_3=[1].$
    
    From the other values of $a$ we explicitly calculated,
    
    $(2+\alpha) = (-1+\alpha) \Mod \mathfrak{p}_3 = \mathfrak{p}_3^{''} \mathfrak{p}_5$
    
    $(3+\alpha)=\alpha \Mod \mathfrak{p}_3=\mathfrak{p}_3\mathfrak{p}_{11}$. Thus, $[\mathfrak{p}_5]=[\mathfrak{p}_3^{''}]$, and $[\mathfrak{p}_{11}]=[\mathfrak{p}_3]$.
    
    We then consider that since $\mathfrak{p}_3\mathfrak{p}_3^{'}\mathfrak{p}_3^{''}=\mathfrak{p}_3(-1+\alpha)\mathfrak{p}_3^{''}=(3)$, $\mathfrak{p}_3\mathfrak{p}_3^{''}=[1],$ and thus the class group of $\mathbb{Q}(\alpha)$ is generated by $\mathfrak{p}_3.$ Since we also have that $[\mathfrak{p}_3^2]=[1]$, the class number divides 2. By proving that $\mathfrak{p}_3$ is a non-principal ideal (exercise), $h(K)=2. \qed$ 
}

\chapter{Dirichlet's unit theorem}

\section{Dirichlet's unit theorem}

Dirichlet's unit theorem is a major result in algebraic number theory. In particular, it has to do with the structure of the \emph{unit group}. Firstly, we introduce the definition of \textbf{rank}, and give a primer on the \textbf{group of units}. This gives us enough information to introduce the notion of a \textbf{regulator}, which measures the density of units in a number field. 

We recall that a finitely generated abelian group $G$ is generated by $G_{\text{tors}} \oplus \mathbb{Z}^r$, where $G_{\text{tors}}$ is the finite subgroup generated by torsion elements of $G$. These are the elements of finite order. This result follows from the structure theorem for finitely generated abelian groups, which states that an abelian group $A$ can be decomposed as $$G \cong C_{{p_1}^{e_1}} \times C_{{p_2}^{e_2}} \times \dots \times C_{{p_k}^{e_k}} \times \mathbb{Z}^r.$$

\defn{Rank of an abelian group}{
    In the decomposition $$G \cong C_{{p_1}^{e_1}} \times C_{{p_2}^{e_2}} \times \dots \times C_{{p_k}^{e_k}} \times \mathbb{Z}^r,$$ the $r$ in $\mathbb{Z}^r$ is the rank of an abelian group. Moreover, $r$ is \textbf{uniquely determined} by $G$, and we call $p_1^{e_1}, \dots, p_k^{e_k}$ the invariant factors of $G$.

    Moreover, there exists a minimal subset $\{e_1, \dots, e_n\}$ that generates $G$. It generates $G$ through $G=\mathbb{Z}e_1+\dots+\mathbb{Z}e_n.$
}

Writing $r$ as the number of real embeddings and $2s$ as the number of complex embeddings, we have that $$K \otimes \mathbb{R} = \mathbb{R}^r \times \mathbb{C}^s.$$ Dirichlet's unit theorem states that 

\thrm{Dirichlet's unit theorem}{
    Let $K$ be a number field with ring of integers $\mathcal{O}_K$, and denote its group of units as $\mathcal{O}_K^*$ (also denoted $U_K$) The group of units is finitely generated with rank equal to $r+s-1.$ Moreover, we know that the finite cyclic group formed by the roots of unity in $\mathcal{O}_K$, denoted $\mu(\mathcal{O}_K)$, is a free group on $r+s-1$ generators. 
    
    The unit theorem describes how $\mathcal{O}_K^*$ factorises. It factorises as $$\mathcal{O}_K^* = \mathbb{Z}^{r+s-1} \times \mu({\mathcal{O}_K}).$$

    Explicitly, we can represent every unit $u \in U_K$ as $u=z u_1^{n_1}\dots u_r^{n_r}$, where $u_i$ are algebraic integers and $z$ is a root of unity.
}

Here are some examples of the Unit theorem in action.

\ex{}{
    Let $K=\mathbb{Q}(\sqrt{-d})$ be an imaginary quadratic field. It has field signature $(r_1, r_2)=(0, 1).$ Therefore, its group of units is given by $$\mathcal{O}_K = \mathbb{Z}^{r_1+r_2-1} \times G = G.$$ 
}


In fact, the units are the fourth units of unity for $K=\mathbb{Q}(\sqrt{-1})$, the 6th roots of unity for $K=\mathbb{Q}(\sqrt{-3})$, and $\pm 1$ otherwise. In fact, these units are the solutions to the two equations:

$$
\begin{cases}
    m^2-nd^2=\pm 1 \\
    (2m+n)^2-d^2= \pm 4
\end{cases}
$$

, which follows from the fact that $\mathcal{O}_K = \{m+n\sqrt{d}\}$, or $\{m+n\frac{1+\sqrt{d}}{2}\}.$ There are obviously finitely many solutions for imaginary quadratic fields. Thus, $\mathcal{O}_K^* = \mu(\mathcal{O}_K)$, since $r_1+r_2-1=0.$ 

Trivially, real quadratic fields satisfy the property that $\mu(\mathcal{O}_K)=\pm 1$ since these are the only roots of unity in $\mathbb{R}.$ However, it is less easy to compute the unit group for real quadratic fields.

The theorem shows that there are infinitely many units for real quadratic fields. However, all these units can be represented as powers of a unit. Therefore, we can represent $U_K=\pm u^{\mathbb{Z}}$ for some $u.$

\defn{Fundamental unit}{
    Let $\varep_1, \dots, \varep_r$ be a set of generators for $\mcO_K^*/\mu(\mathcal{O})$. This set of generators are known as a system of \emph{fundamental units}, and they generate the unit group. 

    Thus, the multiplicative group of units $U_K \subset K$ is generated by $U_K = \{\zeta_K^{e_0} \varep_1^{e_1}\dots\varep_r^{e_r}.\}$, with all the $e_i \in \bbZ$ and $\zeta_K$ a maximal primitive $n$'th root of unity that resides in $K.$
}

\ex{}{
    Consider the quadratic field $\mathbb{Q}(\sqrt{2}).$ The solutions to $m^2-2n^2=\pm 1$ are $(m, n)=(1, 0), (1, 1)$ etc. The fundamental unit for this is thus $(1+\sqrt{2})$, and thus $\mathbb{Z}[\sqrt{2}]^* = \pm (1+\sqrt{2})^{\mathbb{Z}}.$

    For $K=\mathbb{Q}(\sqrt{5})$, $U_K=\pm(\frac{1+\sqrt{5}}{2})^{\mathbb{Z}}.$
}

We then proceed with a proof of Dirichlet's unit theorem. 

\textbf{Proof:} We're familiar with the canonical embedding $\sigma: K \to \mathbb{R}^r \times \mathbb{C}^s \cong \mathbb{R}^n.$

We first begin by defining the \emph{logarithmic embedding} $$\lambda: U_K \to \mathbb{R}^{r+s}$$ by the mapping $$\alpha \mapsto (\log |\sigma_{1}(\alpha)|, \dots, \log |\sigma_{r+s}(\alpha)|).$$

Note that the image of $\lambda$ lies in the hyperplane $$W=\{(x_1, \dots, x_{r+s}) \in \mathbb{R}^{r+s}: x_1+\dots+x_r+2x_{r+1}+2x_{r+s}=0\}.$$ This hyperplane is a vector space of dimension $r+s-1.$ 

We know this as by taking the product of all the embeddings of $K$ into $\mathbb{C}$, we get $$\prod\limits_{i=1}^r |\sigma_i(\alpha)| \cdot \prod\limits_{i=r+1}^s |\sigma_i(\alpha)|^2 = 1.,$$ and we can take logs on both sides. Thus, $\lambda(U_K)$ is a free $\mathbb{Z}$-Module of rank $s \leq r+s-1.$

Representing $U_K = \mu(\mathcal{O}_K) \times \mathbb{Z}^n$, we have proven that $n \leq r+s-1.$

The hard part of this proof is proving that $n \geq r+s-1$. This follows from \href{https://faculty.math.illinois.edu/~r-ash/Ant/AntChapter6.pdf}{Robert Ash.} 

Consider a linear form $f$, and let $V=\lambda(U_K).$ Proving that $n \geq r+s-1$ is equivalent to proving that any linear form $f$ that vanishes on $V$ also vanishes on $W$. If $f$ doesn't vanish on $W$, it doesn't vanish on $V$, meaning that we have $f(\lambda(u))\neq 0$ for some unit $u\in U_K.$ Thus, we aim to prove that $\mathbf{f(\lambda(u))\neq 0}.$

We proceed by invoking Minkowski's Convex-Body Theorem to the set $$\{(x_1, \dots, x_r, x_{r+1}, \dots, x_{r+s}) \in \mathbb{R}^r \times \mathbb{C}^s: |y_i| \leq a_i, |z_j| \leq a_{r+j}\}$$ where $i \in [1, r], j \in [1, s]$. 

Fix a positive real number b $\leq \frac{2^{n-r}}{(2\pi)^{s}} \sqrt{\disc(K)}.$ Given arbitrary real numbers $a_1, \dots, a_n$ ($n=r+s-1$), chose $a_{r+1}$ such that $$\prod\limits_{i=1}^{r}a_i \prod\limits_{j=r+1}^{r+s} a_j^2 = b.$$

Note that the set is compact, convex, and symmetric about the origin, and that it has voulme $\prod\limits_{i=1}^{r} 2a_i \prod\limits_{j=r+1}^{r+rs} \pi a_j^2 = 2^{r}\pi^{s}b \geq 2^{n-s}\sqrt{\disc{K}}.$

With $B$ being as defined in the AKLB setup, and $A=\sigma(B)$, we get that $S \cap (H \backslash \{0\}) \neq \emptyset.$ There is a nonzero algebraic integer $x=x_a$ with $a=(a_1, \dots, a_r)$ such that $\sigma(x_a) \in S$, and thus all the embeddings $|\sigma_i(x_a)| \leq a_i$ for $i = 1, \dots, n.$ (we set $a_{j+s}=a_j$)

Now, we aim to bound the norms of $x_a$ by $b$. To do this, we show the following:

\lemma{}{$$0\leq \log a_i - \log |\sigma_i(x_a)| \leq \log b$$}

\textbf{Proof:} Note that $$1 \leq |N(x_a)| = \prod\limits_{i=1}^n |\sigma_i(x_a)| \leq \prod\limits_{i=1}^r a_i \prod\limits_{j=r+1}^{r+s} a_j^2 = b.$$

But then $|\sigma_i(x_a)| = |N(x_a)| \prod\limits_{j\neq i} |\sigma_i(x_a) \leq a_i| \geq \prod\limits_{j\neq i} a_j^{-1} = a_ib^{-1}$ for all $i$. 

From $a_ib^{-1} \leq |\sigma_i(x_a) \geq a_i$, we take reciprocals and multiply by $a_i$ to achieve $1 \leq \frac{a_i}{\sigma_i(x_a)} \leq b$. We attain our desired result by taking logarithms. $\qed$

In the hyperplane $W$, we can take $y_i$ to be arbitrary such that we can solve for $x_n$ (where $n$ is the $r+1$'th real embedding). We can thus express $f(y_1, \dots, y_r, y_n)=c_1y_1+\dots+c_ry_r+c_ny_n$, with the result being non-trivial. Recall the logarithmic embedding defined as the mapping $$\lambda(\alpha) = (\log |\sigma_{1}(\alpha)|, \dots, \log |\sigma_{r+s}(\alpha)|).$$ We know that $f(\lambda(x_a))=\sum\limits_{i=1}^r c_i \log |\sigma_i(x_a)|$ by summing over the logarithms. By multiplying the inequality in Lemma 2.4 by $c_i$, we have that 

\begin{align*}
    &\ \ \ \sum\limits_{i=1}^r c_i(\log a_i - \log |\sigma_i(x_a)|) \\
    &= \sum\limits_{i=1}^r c_i (\log a_i-\log - f(\lambda(x_a)))
\end{align*}

But we also know that $\sum\limits_{i=1}^r c_i(\log a_i - \log |\sigma_i(x_a)|) \leq \sum\limits_{i=1}^r |c_i| \log b$. Choose a positive real number $t \geq \sum\limits_{i=1}^r |c_i| \log b$, and choose $a_{ih}$ for $i=1, \dots, r$ such that $\sum\limits_{i=1}^r \log a_{ih}=2th.$ 

Let $a(h)=(a_{1h}, \dots, a_{rh})$, and define $x_h$ as the corresponding algebraic integer $x_{a(h)}.$ Then we have that $|f(\lambda(x_h))-2th|<t$, or

$$(2h-1)t < f(\lambda(x_h)) < (2h+1)t.$$ These are pairwise disjoint intervals, which means that the $f(\lambda(x_h))$ are all distinct. Consider two distinct algebraic integers $x_h$ ad $x_k$. Since there are finitely many distinct ideals of the form $\mathcal{O}x_h$, we can construct $\mathcal{O}x_h = \mathcal{O}x_k$ with $h$ and $k$ being distinct. 

But then we can represent $x_h=ux_k$ as they are associates. Thus, $f(\lambda(x_h)) = f(\lambda(u))+f(\lambda(x_k)).$ But since $f(\lambda(x_h)) \neq f(\lambda(x_k))$, we have that $f(\lambda(u)) \neq 0. \qed$ 

\section{Cubic fields and the regulator}

Artin proved a key result regarding cubic fields.

\thrm{Artin}{
    Let $\mathcal{O}$ be an order in $K$, and let $r=1.$ If $v>1$ is a unit of $\mathcal{O}^*$, then $|\disc(\mathcal{O})| < 4v^3+24.$
}

The proof of this theorem is rather messy and confusing. As with the Dirichlet unit theorem, we consider some applications of the theorem. However, this theorem is more useful once we link it with the theory of fundamental units:

\coro{Artin's result for fundamental units}{
    Let $\varepsilon >1$ be the fundamental unit of $\mathcal{O}$, and let $\mathcal{O}$ be an order with $r=1.$ For $u>1$ a unit in $\mathcal{O}^*$, if $4u^{\frac{3}{m}}+24 \leq |\disc(\mathcal{O})|$ for an integer $m\geq 2$, then $u=\varepsilon^k$ for $1\leq k < m.$ 
    
    It thus follows that if $4u^{\frac{3}{2}} + 24 \leq |\disc(\mathcal{O})|$, then $u=\varepsilon.$
}

\textbf{Proof:} We can represent the unit group $\mcO^*$ as $\pm \varep^{\bbZ}$, and thus we can write $u=\varep^k$ for some $k\in\bbZ^+.$ Set $v=\varep$ and apply Artin's inequality. We hvae that $$|\disc({\mathcal{O}})| < 4\varep^3+24 = 4u^{\frac{3}{k}}+24.$$ 

But then $k<m$, or else if $k \geq m$, then $|\disc({\mcO})|<4u^{\frac{3}{k}}+24\leq4u^{\frac{3}{m}}+24 \leq |\disc({\mcO})|$, arriving at a contradiction. $\qed$

We'll now work on finding the fundamental units of $K_1=\bbQ(\sqrt[3]{2})$ and $K_2=\bbQ(\sqrt[3]{6}).$ Before this, we need to prove that $\mcO_{K_1}=\bbZ(\sqrt[3]{2})$, and that $\mcO_{K_2}=(\sqrt[3]{6}).$ Before that, however, we explain the notion of an \textbf{Eisenstein polynomial}, and a property of it.

\defn{Eisenstein polynomials}{
    Let $f \in \bbZ[X]$ be a monic polynomial. Represent $$f(x) = x^n+c_{n-1}x^{n-1}+\dots+c_1x+c_0.$$

    $f(x)$ is \textbf{Eisenstein} at $p$ if $c_i$ is divisible by $p$ for all $c_i$, and $c_0$ is not divisible by $p^2.$
}

\prop{}{
    Let $f$ be Eisenstein at $p$ with degree $n$, and let $\alpha \in \mcO_K$ be a root of $f$. Then, $p \nmid [\mcO_K:\bbZ[\alpha]].$
}

\textbf{Proof:} By way of contradiction, suggest that $p\mid [\mcO_K:\bbZ[\alpha]].$ Then the abelian group $\mcO_K / \bbZ[\alpha]]$ has an element of order $p.$ Let an element $\beta \in \mcO_K$ be such that $\beta \not\in \bbZ[\alpha]$ but $p\beta \in \bbZ[\alpha].$ 

Since $K=\bbQ(\alpha)$, there exists a basis $\{1, \alpha, \dots, \alpha^{n-1}\}$ for $K/\bbQ.$ Thus, we have that $$\beta = c_0+c_1\alpha+\dots+c_{n-1}\alpha^{n-1},$$ with all the $c_i \in \bbQ.$ Since we defined $\beta \not\in \bbZ[\alpha]$, there are some $c_i$ which are not in $\bbZ$; however, $pc_i \in \bbZ.$ 

But then this suggests that $c_i$ has a $p$ in its denominator. Multiply each of the $c_i$ by its LCM - this suggests that we can construct $\frac{a_0+a_1\alpha+\dots+a_{n-1}\alpha^{n-1}}{d}$, with all the $a_i\in \bbZ$ and $d=\lcm(c_0, c_1, \dots, c_n)$. This result is in $\mcO_K$, and thus $a_0+a_1\alpha+\dots+a_{n-1}\alpha^{n-1} \in d\mcO_K \subset p\mcO_K$, which is a contradiction as $a_i \in p\bbZ. \qed$ 

\prop{Cubic rings of integers}{
    With regards to the cases $\bbQ(\sqrt[3]{2})$ and $\bbQ(\sqrt[3]{6})$,
    \begin{enumerate}
        \item Let $\mcO$ be the ring of algebraic integers of $\bbQ(\sqrt[3]{2}).$ Thus, $\bbZ(\sqrt[3]{2}) \subset \mcO$, and $\disc(\bbZ[\sqrt[3]{2}])=[\mcO:Z[\sqrt[3]{2}]]^2 \disc(\mcO)$.

        The minimal polynomial of $\sqrt[3]{2}$ is $x^3-2$, which has discriminant $-2^2\cdot3^3=108.$ Thus, 2 and 3 are the only primes that can divide $[\mcO:Z[\sqrt[3]{2}]]$

        Note that $x^3-2$ is Eisenstein at $2$. Thus, 2 doesn't divide $[\mcO:\bbZ[\sqrt[3]{2}]]$ by the previous theorem. 

        Note that the minimal polynomial of $1+\sqrt[3]{2}$ is $(x-1)^3-2 = x^3-3x^2+3x-3$, which is Eisenstein at 3. This means that 3 doesn't divide $[\mcO:\bbZ[\sqrt[3]{2}]]$. Thus the index is 1, meaning that $\mcO=\bbZ[\sqrt[3]{2}].$

        \item The discriminant of $\bbZ(\sqrt[3]{6})$ is $-2^2\cdot3^5.$ We can apply similar logic to conclude that the index is 1, and thus $\mcO=\bbZ[\sqrt[3]{6}].$
    \end{enumerate}
}

\ex{$\mathbb{Q}(\sqrt[3]{2})$}{
    Let $K=\bbQ(\sqrt[3]{2})$. Then, $\mcO_K=\bbZ[\sqrt[3]{2}]$, with the discriminant being $\disc(x^3-2)=108.$ Represent

    $$1=(\sqrt[3]{2})^3-1 = (\sqrt[3]{2}-1)(\sqrt[3]{4}+\sqrt[3]{2}+1).$$ 

    Thus, we have a unit $1+\sqrt[3]{2}+\sqrt[3]{4}.$ However, $4u^{\frac{3}{2}}+24 \leq 108$, thus making $u$ the fundamenetal unit of $\mcO_K.$
}

\ex{$\mathbb{Q}(\sqrt[3]{6})$}{
    $\mcO_K = \bbZ[\sqrt[3]{6}].$ In this case, the discriminant is -972. We attempt to find a unit of of $\mcO_K.$ 

    Let's consider how $x^3-6$ decomposes $\Mod p$. 

    \begin{enumerate}
        \item When $p=2$, $x^3-6 \Mod 2=x^3$, and thus $(p)$ factorises as $\mfp_2^3.$
        \item When $p=3$, $x^3-6 \Mod 3=x^3$, and thus $(p)$ factorises as $\mfp_3^3.$
        \item When $p=5$, $x^3-6 \Mod 5=x^3-1=(x-1)(x^2+x+1)$, and thus $(p)$ factorises as $\mfp_5\mfp_{25}.$
        \item When $p=7$, $x^3-6 \Mod 7=(x-3)(x-5)(x-6)$, and thus $(p)$ factorises as $\mfp_7\mfp_7^{'}\mfp_7^{''}.$
    \end{enumerate}

    Consider $\mfp_2$, which has norm 2. Consider that the norm $N_{K/\bbQ}(\sqrt[3]{6}+c)=c^3+6$. Taking $c=-2$, we get that $N_{K/\bbQ}(\sqrt[3]{6}-2)=-2$, and thus the ideal $(\sqrt[3]{6}-2)$ has norm $2.$

    Representing $(2)$ as $\mfp_2^3$, we know that $\mfp_2^3=(\sqrt[3]{6}-2)^3$ are equal up to units. Since $|\frac{(\sqrt[3]{6}-2)^3}{2}|<1$, we consider $$u=|\frac{2}{(\sqrt[3]{6}-2)}|.$$ However, $4u^{\frac{3}{2}} \approx 23676 > 972$, and thus we need to do more work to prove that $u$ is the fundamental unit of $\mcO_K.$

    We present the rest of the proof as such. Let $\varep>1$ be the fundamental unit of $\bbZ[\sqrt[3]{6}].$ We want to show that $u=\varepsilon.$ Since $u=\varep^k$ by Dirichlet's unit theorem, we instead want to show that $k=1.$ Putting $v=\varep,$ (by Artin), we have that $$|\disc(\mcO_K)|<4u^{\frac{3}{k}}+24.$$

    For which $k$ is this inequality satisfied? We note that the inequality is only satisfied when $k=1, 2, 3.$ It remains for us to prove that the integer is not a square or cube, thus concluding that $u=\varep$ is the fundamental unit of $\sqrt[3]{6}.$

    Consider $\mcO_K/\mfp_5.$ We have that $\sqrt[3]{6}\equiv 1 \Mod \mfp_5$, and thus $u\equiv 2 \Mod \mfp_5$ by considering the fact that $u=|\frac{2}{(\sqrt[3]{6}-2)}|.$ Since the non-zero squares $\Mod 5$ are 1 and 4, $u$ is not a square in $\mcO_K/\mfp_5$, and thus is not a square in $\mcO_K$.

    Apply similar logic to $\mcO_K/\mfp_7.$ In $\bbZ/7\bbZ$, we have that $\sqrt[3]{6} \equiv 1 \Mod \mfp_7$, and that $u\equiv 5 \Mod \mfp_7.$ The nonzero cubes in $\mcO_K$ are 1 and 6. Thus, $u$ is not a cube in $\mcO_K.$ 
}

There is a brilliant theory regarding cyclotomic fields and their units. We will build up this theory in the next chapter. Before this, we present our last topic on Minkowski's geometric theory of numbers - the \emph{regulator}.

\defn{Regulator}{
    Let $K$ be a number field with degree $k=r+2s$, wth $s$ being half the number of comlplex embeddings $\sigma_j: K \hookrightarrow \bbC.$ Let $n=r+s-1$ be the rank of the group of units, and denote $\varep_1, \dots, \varep_n$ be the set of fundamental units for $U_K/\mu(\mathcal{O})$, or the unit group Modulo roots of unity. Let $$\lambda_i(\alpha)=\begin{cases}
        \log |\sigma_i(\alpha)| & 1\leq i\leq r \\
        \log |\sigma_i(\alpha)|^2 & r+1 \leq i \leq r+s
    \end{cases}$$

    Define the $(n+1) \times n$-dimensional \emph{regulator matrix} $R_K$ as

    $\begin{pmatrix}
        \lambda_1(\varep_1) & \dots & \lambda_1(\varep_n) \\
        \vdots & & \vdots \\
        \lambda_{n+1}(\varep_1) & \dots & \lambda_{n+1}(\varep_n) 
    \end{pmatrix}.$

    The regulator $R$ is the \textbf{determinant} of an arbitrary minor of this matrix, or $R=\|R_K\|.$ The regulator is independent up to which minor we take.
}

The regulator measures the density of units in an number field. If the regulator is small, there is a high density of units. 

We now turn our attention to cubic fields. For two units $u, v \in \mcO$, write $\reg(u, v)$ as the regulator of $u$ and $v.$ Similarly, define the regulator of an order $\mcO$ in $K$ as the regulator of a set of fundamental units $\reg(\varep_1, \dots, \varep_n)$ - we denote it as $\reg(\mcO).$

\prop{}{
    Let $u_1$ and $u_2$ be a pair of units in $\mcO^*$, and let $\varep_1$ and $\varep_2$ be fundamental units. 
    
    Write $u_1=\pm\varep_1^a\varep_2^b$, and $u_2=\pm\varep_1^c\varep_2^d.$ Then, $\reg(u_1, u_2)=|ad-bc|\reg(\varep_1, \varep_2).$
}

\textbf{Proof:} Consider $v_1=(x, y, z)$ and $v_2=(x', y', z')$ associated to the cubic units $\varep_1$ and $\varep_2$ respectively. Associate $w_1$ and $w_2$ with the units $u_1$ and $u_2$, and thus $w_1=av_1+bv_2$ and $w_2=cv_1+v_2.$. We then have that 

$$\begin{pmatrix} u_1 \\ u_2 \end{pmatrix} = \begin{pmatrix} av_1+bv_2 \\ cv_1+v_2 \end{pmatrix} = \begin{pmatrix} ax+bx' & ay+by' & az+bz' \\ cx+dx' & cy+dy' & cz+dz'\end{pmatrix}$$

Note that it does not matter which minor matrix we extract from the $3\times2$ matrix as above. In this matrix, $\reg(u_1, u_2) = |ad-bc| \reg(\varep_1, \varep_2).$

We can synthesise Artin's inequality with cubic orders as such:

\coro{}{
    $\reg{\mcO} > \frac{1}{3}\log(\frac{|\disc{\mcO}|-24}{4})$
}

\textbf{Proof:} Direct calculation. From $|\disc{\mcO}|<4v^3+24$, we have that $v=\sqrt[3]{\frac{|\disc{\mcO}|-24}{4}}$, and thus $\log v = \frac{1}{3}\log{\frac{|\disc{\mcO}|-24}{4}}.$ 

Let $v$ be the fundamental unit of $\mcO$ that $>1$. In this case, $\reg{\mcO}=\log v$, and thus $$\reg(\mcO) > \frac{1}{3}\log{\frac{|\disc{\mcO}|-24}{4}}. \qed$$

\ex{Trivial regulator}{
    Let $K$ be an imaginary quadratic field. Since the determinant of a $0 \times 0$ matrix is 1, the regulator of $K$ is 1.
}

\ex{Real quadratic fields}{
    Consider $K=\bbQ(\sqrt{5}).$ The fundamental unit of $K$ is $\frac{\sqrt{5}+1}{2}$, and thus its two embeddings into $\bbR$ are $\pm\frac{\sqrt{5}+1}{2}$
    
    Construct the $n\times(n+1)$ matrix $\begin{pmatrix}
        \log\frac{\sqrt{5}+1}{2} & \log\frac{-\sqrt{5}+1}{2}.
    \end{pmatrix}$

    The result thus follows by taking an arbitrary minor of the matrix. 
}

\ex{Unit rank 1}{
    Let $K$ have unit rank 1, and $u \in U_K.$ Then, $\reg(u)=|\log|u||.$ 

    For example, imaginary cubic fields have unit rank 1.
}

\newpage

\chapter{Cyclotomic Fields}
\minitoc

The theory of cyclotomic fields is very deep and fundamental within the fabric of algebraic number theory. 

In this chapter, we will extend our study of units to cyclotomic fields, noting our previous section on Dirichlet's unit theorem. 

\section{Introduction}

Let $\zeta \in K.$ We call $\zeta$ a \emph{primitive} n'th root of unity if $\zeta^n=1$, but $\zeta^d \neq 1$ for any $d<n.$ Thus, we have that $\zeta$ has order $n$ in $K^*.$

We study the n'th rooots of unity upon studying complex numbers. in $\bbC$, the n'th roots of unity are the numbers $e^{2\pi i m/n}$ for $0 \leq m \leq n-1.$ 

It is well-known that $\zeta^m$ is a primitive n'th root of unity if and only if $m$ is relatively prime to $n.$ Note that in fields of characteristic $p$, there are no p'th roots of unity apart from 1. Since $x^p=1$, $x^p-1=(x-1)^p=0$, and thus $x=1.$ Thus we have that the Frobenius endomorphism $\sigma_p: x \mapsto x^p$ is injective. Some texts denote this as the \emph{Frobenius element}. 

In this section, we define $\sigma_n$ as a primitive n'th root of unity. We define the group of them by $\mu_n.$ - so for example, $\mu_3 = \{1, 2, 4\}$ in $\bbF_7.$

\defn{Cyclotomic polynomial}{
    Let $K=\bbQ(\zeta_n)$, formed by adjoining a primitive n'th root of unity to $\bbQ.$ Then, $K$ is the splitting field of $x^n-1$, and is Galois over $\bbQ.$ 

    Let $G=\Gal(\bbQ(\mu_n)/\bbQ)$ be the group that permutes the set of primitive n'th roots of unity in $K.$ We know that there exists an isomorphic mapping $\Gal(K(\mu_n)/K) \to (\bbZ/n\bbZ)^*$, given by the mapping $\sigma \mapsto a_\sigma \Mod n.$
    
    We define the \emph{n'th cyclotomic polynomial} $\Phi_n(x)$ as $$\Phi_n(x) = \prod(x-\zeta^m)$$ for all $m\in (\bbZ/n\bbZ)^*.$
}

\defn{Cyclotomic units}{
    Suppose $r$ and $s$ are integers with $(p, rs)=1$. Then, $\frac{\zeta^r-1}{\zeta^s-1}$ is a unit of $\bbZ[\zeta].$
}

\textbf{Proof:} Write $r \equiv st \Mod p.$ Then, $\frac{\zeta^r-1}{\zeta^s-1}=\frac{\zeta^{st}-1}{\zeta^s-1}=1+\zeta^s+\dots+\zeta^{s(t-1)}$, which $\in \bbZ[\zeta].$

Additionally, $\frac{\zeta^s-1}{\zeta^r-1}=\frac{\zeta^s-1}{\zeta^{st}-1}\in\bbZ[\zeta]. \qed$

In fact, we can characterise the units of cyclotomic fields as such: Let $\zeta$ be a root of unity. Then, all $\frac{\zeta^k-1}{\zeta-1}$, where $k$ is coprime to the order of $\zeta$, are units. Additionally, if $\zeta$ and $\zeta^\prime$ are two primitive n'th roots, then $\frac{1-\zeta^\prime}{1-\zeta}$ is a unit.

\prop{Cyclotomic embedding}{

Let $\mathbb{Q}(\zeta_n)$ be a cyclotomic field extension. Then, the embedding $\Gal(\mathbb{Q}(\zeta_n)/\bbQ) \hookrightarrow  (\bbZ/n\bbZ)^*$ is an isomorphism. Evidently, the embedding is surjective in this case.

}

\tbf{Proof:} Note that the size of $\mathbb{Q}(\zeta_n)$ is at most $\mu(n) = |(\bbZ/n\bbZ)^*|.$ We are left to show that for all $(a, n)=1, a\in\bbZ$, that $\mu_n$ and $\mu_n^a$ have the same minimal polynomials in $\bbQ$.

Let $a=p_1p_2\dots p_r$ be a product of primes. I aim to show that for all $p\nmid n$, that the $p$'th power of every primitive $n$'th root have the same minimal polynomial over $\mathbb{Q}$. By such, $\mu_n$, $\mu_n^{p_i}$ for any collection of $p_i$ will have the same $\bbQ$-minimal polynomial. 

Let us assume, by way of contradiction, that $\zeta_n$ and $\zeta_n^p$ are not conjugate for some $p\nmid n$. Then let $f(x)$ be the minimal polynomial of $\zeta_n$ over $\bbQ$, and let $g(x)$ be the minimal polynomial of $\zeta_n^p$ over $\bbQ$. Note that they are both in $\bbZ[X]$ as they divide $T^n-1$, and every $n$'th root of unity is a root of $T^n-1$. 

Let $T^n-1=f(x)g(x)h(x)$ for some monic $h(x) \in \bbQ(X)$. We can reduce this equation Modulo $p$; as such, $$T^n-\bar{1} = \bar{f}(x)\bar{g}(x)\bar{h}(x).$$

Note that since $p$ doesn't divide $n$, $\bar{f}(x)$ and $\bar{g}(x)$ are relatively prime in $\bbF_p[X]$. Thus, the reductions $\Mod p$ are of the sae degree as $f(x)$ and $g(x)$. The reductions are non-constant.

We know that $g(\zeta_n^p)=0$, so $g(x^p)$ has $\zeta_n$ as a root. Write $g(x^p)=f(x)k(x)$ for some $k(x) \in \bbQ[X]$. Thus, $k(x)\in\bbZ[X]$ by Gauss's lemma. Then, we can reduce $\bar{g}(x^p)=\bar{g}(x)^p=\bar{f}(x)\bar{k}(x)$, and thus every irreducible factor of $\bar{f}(x)$ in $\bbF_p(X)$ is a factor of $\bar{g}(x)$. This contradicts the fact that the polynomials $\bar{f}(x)$ and $\bar{g}(x)$ are coprime. $\qed$

\prop{Ring of integers}{
    Denote $\zeta_n$ as a primitive n'th root of unity, and denote $K=\bbQ(\zeta_n).$ Then, $\mcO_{K}=\bbZ[\zeta_n].$ 
}

\textbf{Proof:} We first prove this for when $n$ is a prime power $p^r$. Afterwards, we prove the theorem for general $n$.

Planning to induct on $r$, we start with the base case $r=1$.

Let $K_1=\bbQ(\zeta_p).$ We aim to show that the ring of integers of $K_1$, denoted $\mcO_1$, is $\bbZ[\zeta_p]$.

We first prove that $(1-\zeta_p)\mcO_1 \cap \bbZ = p\bbZ.$ We do this by considering the norm $N(1-\zeta_p)=p$ (why?), and thus $(1-\zeta_p) \mid N(1-\zeta_p)=p$ in $\mcO_1$. By this, we have the inclusion $p\mcO_1 \cap \bbZ \subset (1-\zeta_p)\mcO_1 \cap \bbZ$. 

But then note that $p\mcO_1$ is contained in $p\bbZ$ and $p\bbZ$ is maximal; thus, the inclusions are actually equalities. This is because $1-\zeta_p$ would divide $1$ in $\mcO_1$, and we find a contradiction by taking norms.

Let $\alpha \in\mcO_1$ and denote its conjugates by $\{\alpha_j\}.$ We note that $$\Tr(\alpha(1-\zeta_p))=\sum\limits_{j=1}^{p-1} \alpha_j(1-\zeta_p^j),$$ which is included in $(1-\zeta_p)\mcO_1 \cap \bbZ = p\bbZ.$

This is because $1-\zeta_p$ divides $1-\zeta_p^j$ in $\mathcal{O}_1$ for $j=2, \dots, p-2$. We can write $\alpha=\sum\limits_{j=0}^{p-2} a_j \zeta_p^j$, with the $a_j \in \bbQ$. 

By the properties of trace, and also knowing that $\Tr \zeta_p^j=-1, \Tr 1 = p-1$, we know that $$\Tr(\alpha(1-\zeta_p))=\sum\limits_{j=0}^{p-2} a_j \Tr(\zeta_p^j-\zeta_p^{j+1})=pa_0.$$ This suggests that $a_0\in \bbZ$. 

Afterwards, we know that $\zeta_p^{-1}(\alpha-a_0) \in \mcO_1$. Follow a similar line of argument to show that $a_1$, and by extension $a_j \in \bbZ$, and thus $\alpha \in \bbZ[\zeta_p]$. Thus, $\mcO_1=\bbZ[\zeta_p]. \qed$

The inductive step proves that for the field $K_r=\bbQ(\zeta_{p^r})$, its ring of integers (denoted $\mcO_{K_r}$) is $\bbZ(\zeta_{p^r}).$ We assume that $K_{r-1}$ implies $\mcO_{K_{r-1}}$, and proceed with the inductive argument.

We will declare, but without proof, the following facts:

\begin{enumerate}
	\item The Galois group of $K_r$ over $K_{r-1}$ is a subset of $(\bbZ/p^r\bbZ)^*$. Explicitly, it is $\{1+ip^{r-1}+p^r\bbZ\}.$
	\item $\Tr_{K_r/K_{r-1}} \zeta_{p^r}=0$
	\item $\Tr_{K_r/K_{r-1}} 1 =p$
	\item $\Tr_{K_r/K_{r-1}} \zeta_{p^r}^k =\sum\limits_{i=0}^{p-1} \zeta_{p^r}^{(1+ip^{e-1})k}$. This is also $\zeta_{p^r}^k \sum\limits_{i=0}^{p-1}(\zeta_p^k)^i$.
\end{enumerate}

Consider an arbitrary element of $\mcO_r$. Denote it $\alpha=\sum\limits_{i=0}^{p-1}a_i\zeta_{p^e}^i$, with $a_i \in K_{r-1} $ for all $i=0, \dots, p-1$.

Then consider $\alpha \zeta_{p^r}^{-j}$. It has trace $pa_j$, for all $j=0, \dots, p-1$. This means that $pa_j$ lies in $\mcO_{r-1}$ for each $j$, meaning that $\mcO_r \subset \frac{1}{p} \mcO_{r-1}[\zeta_{p^r}].$ Knowing that $\bbZ[\zeta_{p^r}]=\bbZ[1-\zeta_{p^r}]$, write a general element of $\mcO_r$ as $$\alpha = \frac{1}{p}\sum\limits_{i=0}^{\phi(p^r)-1} b_i(1-\zeta_{p^r})^i,$$ with all the $b_i\in\bbZ$. 

To prove that $\mcO_{r-1} = \bbZ[\zeta_{p^{r-1}}]$ implies $\mcO_r = \bbZ[\zeta_{p^r}]$, we're left with the proof that $p\mid b_i$ for all $i$. This is left as an exercise, which begins by proving that $p \mid b_1$ and using the same line of argument for all $i$. $\qed$

Now, we proceed to prove the theorem for general $n$. Let $n=m_1m_2$, where $m_i$ are relatively prime. We have that $\zeta^{m_1}=e^{(2\pi i /n)(m_1)}=e^{2\pi i m_1/n} = e^{2\pi i /m_2}=\zeta_{m_2}$. By similar logic, $\zeta^{m_2}$ is a primitive $m_1$-th root of unity.

Thus, $\bbQ(\zeta_{m_1})$ and $\bbQ(\zeta_{m_2})$ are contained in $\bbQ(\zeta_n)$. Since $(m_1, m_2)=1$, we can find $r,s$ such that $rm_1+sm_2=1$ by the Euclidean algorithm. Thus, we have that $\zeta_n=\zeta_n^{rm_1+sm_2}=\zeta_{m_1}^r\zeta_{m_2}^s$.

Let $K=\bbQ(\zeta_{m_1})$ and $L=\bbQ(\zeta_{m_2})$ be number fields of degrees $m$ and $n$ respectively over $\bbQ$, and let $KL=\bbQ(\zeta_n)$ be their compositum. Let $R=\bbZ(\zeta_{m_1})$, $S=\bbZ(\zeta_{m_2})$, and $T$ denote the algebraic integers of $K$, $L$, and $KL$. Assuming that $K$ and $L$ are linearly disjoint ($[KL:\bbQ]=mn$). 

We apply (without proof) the fact that if $d$ is the GCD of $\disc(R)$ and $\disc(S)$, that $T \subseteq \frac{1}{d}RS$. This means that if $d=1$, then $T=RS$.  

Note that $\disc(\zeta_1)$ divides a power of $m_1$ and $\disc(\zeta_2)$ divides a power of $m_2$. Thus, the greatest common divisor of $\disc(R)$ and $\disc(S)$ is 1. It then follows that $T=RS$. Then realise that we can repeat the same argument for $\bbZ$, and thus $\bbZ[\zeta_n]=\bbZ[\zeta_{m_1}]\bbZ[\zeta_{m_2}]. \qed$ 


\section{Quadratic reciprocity}

This section of quadratic reciprocity is optional for those who have studied number theory. This section aims to define quadratic reciprocity and give insights towards other reciprocity theorems in number theory. 

\defn{Quadratic residue}{
	Let $p$ be a prime. An integer $a$ is a quadratic residue $\Mod p$ if it is congruent to a perfect square $\Mod p$, and otherwise is a quadratic non-residue $\Mod p$. 
}

\defn{Legendre symbol}{
	We take $a$ and $p$ to be defined as above. The Legendre symbol is a function of $a$ and $p$. Denoted $\legendre{a}{p}$, it is defined as:
	
	$$\legendre{a}{p}=
	\begin{cases} 1 & \text{if}\ a\ \text{is a quadratic residue} \Mod p, a \not\equiv 0 \Mod p \\ 
	-1 & \text{if}\ a\ \text{is a quadratic non-residue} \Mod p, \\
	0 & \text{if}\ a \equiv 0 \Mod p.
	\end{cases}$$
}

Explicitly, $\legendre{a}{p}\equiv a^{\frac{p-1}{2}} \Mod p$, with $\legendre{a}{p} \in \{-1, 0, 1\}.$ This is also known as \tbf{Euler's criterion}.

The Legendre symbol is hard to compute by hand. However, a few facts about it, including the law of quadratic reciprocity, eases computation.

\begin{enumerate}
	\item The top argument of the Legendre symbol is periodic $\Mod$ p. If $a\equiv b\Mod p$, then $\legendre{a}{p}=\legendre{b}{p}$.
	\item The Legendre symbol is multiplicative in its top argument, i.e. $\legendre{ab}{p}=\legendre{a}{p}\legendre{b}{p}.$
\end{enumerate}

\thrm{Law of Quadratic Reciprocity}{
	Let $p$ and $q$ be distinct odd primes. Then, $$\legendre{p}{q}\legendre{q}{p}=(-1)^{\frac{p-1}{2}\cdot\frac{q-1}{2}}.$$
}

When computing Legendre symbols with large values of $a$ and $p$, one follows the following procedure:

\begin{enumerate}
	\item By the periodicity of the Legendre symbol, replace $\legendre{a}{p}$ with $\legendre{a \Mod p}{p}.$
	\item Let $a\Mod p = b$. Factorise $b$ into a product of prime powers, and replace $\legendre{b}{p}$ with $\legendre{q_1}{p}^{r_1}\legendre{q_2}{p}^{r_2}\dots.$
	\item Using quadratic reciprocity, attain $\legendre{p}{q_1}^{r_1}\legendre{p}{q_2}^{r_2}\dots$, and repeat the first two steps. Keep reducing $\Mod p$ and applying quadratic reciprocity until each of the elements in the product of Legendre symbols can be computed with relative ease.
\end{enumerate}

Here are some corollaries of the law of quadratic reciprocity.

\coro{}{
	$\legendre{-1}{p}= \begin{cases} 1 & \text{if}\ p \equiv 1 \Mod 4 \\ -1 & \text{if}\ p \equiv 3 \Mod 4 \end{cases}$
}

\tbf{Proof:} This directly follows from Euler's criterion. Letting $a=-1$, we get that $\legendre{-1}{p}=(-1)^{\frac{p-1}{2}} \Mod p.$ We casework on $p \Mod 4.$

If $p = 1\Mod 4, p=4k+1$ for some $k$. Thus, $\legendre{-1}{p}=(-1)^{\frac{4k+1-1}{2}}=(-1)^{2k}=1$.

Similarly, if $p = 3\Mod 4$, then $p=4k+3$ for some $k$. Thus, $\legendre{-1}{p}=(-1)^{\frac{4k+3-1}{2}}=(-1)^{2k+1}=-1$

\coro{}{
	$\legendre{2}{p} = \begin{cases} 1 & \text{if}\ p\equiv 1, 7 \Mod 8 \\ -1 & \text{if}\ p\equiv 3, 5 \Mod 8 \end{cases}$
}

\lemma{Gauss's Lemma (on residues)}{
	Let $p$ be an odd prime, and $p\nmid a$. Consider the least positive residues $\Mod p$ of the numbers $a, 2a, 3a, \dots, \frac{p-1}{2}a$. If $n$ of these numbers are greater than $\frac{p-1}{2}$, then $\legendre{a}{p}=(-1)^n.$
}

The proof of this theorem lies in dividing the least residues of $a, 2a, 3a, \dots, \frac{p-1}{2}a$ up into those which are larger than $\frac{p-1}{2}$ and those which are not. Applying the Eulerian criterion suffices to give us our desired result. $\qed$

Returning to $\legendre{2}{p}$, consider the set of least residues $2, 4, \dots, p-1$. We know that $n$ of them are greater than $\frac{p-1}{2}$. Suppose that $2m$ is the greatest even integer greater than $\frac{p-1}{2}$. We partition $2, 4, \dots, p-1$ up into $2, 4, \dots, 2(m-1)$, which is less than $\frac{p-1}{2}$, and $2m, 2m+1, \dots, p-1$, which is greater than $\frac{p-1}{2}$. 

Note that $n=\frac{p-1}{2}-(m-1)$ since there are $m-1$ integers less than $\frac{p-1}{2}$. Since $2m$ is the smallest integer greater than $\frac{p-1}{2}$, $m$ is the smallest integer greater than $\frac{p-1}{4}$, and so $m-1$ is the smallest integer greater than $\frac{p-5}{4}.$

This suggests that $n=\frac{p-1}{2}-\frac{p-5}{4}=\frac{p+3}{4}.$ The rest of the proof is obvious. $\qed$

From this, we set out to prove the Law of Quadratic Reciprocity. The following is one of over 300 known proofs of this theorem.

\lemma{}{
	There exists a unique quadratic subfield of $\bbQ(\zeta_p)$. It is $\bbQ(\sqrt{p})$ for $p \equiv 1 \Mod 4$, and $\bbQ(\sqrt{-p})$ for $p \equiv 3 \Mod 4.$ 
	
	For brevity, denote the quadratic subfield as $\bbQ(\sqrt{\bar{p}p})$, where $\bar{p}=(-1)^{(p-1)/2}.$ 	
}

\tbf{Proof:} Let $\bbQ(\sqrt{d}) \subseteq \bbQ(\theta)$, with $d$ square-free. If $q$ is a prime that ramifies in $\bbQ(\sqrt{d})$, it also ramifies in $\bbQ(\theta)$. Since $p$ is the only prime that ramifies in $\bbQ(\sqrt{d})$, it is also the only prime divisor of the discriminant of $\bbQ(\sqrt{d})$. 

Note that the discriminant of $\bbQ(\sqrt{d})$ is either $d$ or $4d$ depending on $d \Mod 4$. This then suggests that $d = \pm p$, with the sign determined by $d\Mod4$ - indeed it is $p$ for $d\equiv 1 \Mod 4$ and $-p$ otherwise.  $\qed$

\lemma{}{
	Let $p$ and $q$ be distinct primes. $q$ splits as a product of two primes in $\bbQ(\sqrt{\bar{p}p})$ if and only if $\legendre{q}{p}=1$.
}

\tbf{Proof:} As defined in the lemma above, let $A$ denote the algebraic integers in $\bbQ(\sqrt{\bar{p}p})$, and let $B$ denote the algebraic integers in $\bbQ(\sqrt{d})$. By factorising $(q)$ into a product of prime ideals in $S$, we have that $qB=\mfp_1\cdot \mfp_r.$ 

We invoke the fact that if $r$ is even if and only if $qA$ splits as a product of two primes in $A$. For the forward direction, one considers that $qA=\mfP_1\mfP_2$ for distinct primes, in which there is an automorphism that sends $\sigma(\mfP_1)=\mfP_2$. Letting $\mfP_1R=\mfS_1\cdot\dots\cdot \mfP_r$, we get that $qB=\mfP_1\mfP_2=\mfS_1\cdot\dots\cdot \mfP_r\ \sigma(\mfS_1)\cdot\dots\cdot \sigma(\mfP_r)$, which has an even number of prime factors. The reverse direction is left as an exercise. $\qed$

We return to the proof. Let $e$, $f$, $g$ be the ramification index, relative degree, and the number of prime ideals dividing $R$ respectively. Thus we have $efg = fg = p-1 = [\bbQ(\theta):\bbQ]$, and thus $g$ is even if and only if $f$ divides $\frac{p-1}{2}$. It then follows that $$q^{(p-1)/2}\equiv 1 \Mod p,$$ by $[\bbQ(\theta):\bbQ]=p-1$.

Recall that we aim to attain $\legendre{p}{q}\legendre{q}{p}=(-1)^{\frac{p-1}{2}\frac{q-1}{2}}.$ We first invoke the fact that $q$ splits as a product of two primes in $\bbQ(\sqrt{\bar{p}p})$ if and only if $\legendre{q}{p}=1$. This holds exactly when $\legendre{\bar{p}p}{q}=1$ since we can split $\bar{p}p$ into a product of two distinct primes. 

Thus, 

\begin{align*}
	\legendre{q}{p} &= \legendre{\bar{p}p}{q}\\ &= \legendre{\bar{p}}{q} \legendre{p}{q} \\
	&=\legendre{-1}{q}^{(p-1)/2} \legendre{p}{q}\\
	&=  \legendre{p}{q} (-1)^{\frac{p-1}{2}\frac{q-1}{2}}
\end{align*} 

which is equal to the original statement as $\legendre{p}{q} = \{\pm 1.\} \qed$

\section{Characters and Gauss sums}

Gauss sums are sums involving roots of unity. We first introduce the notion of a multiplicative character, then that of a Dirichlet character. These are used a lot in analytic number theory, which is studied in Chapter 5. 

This section culminates in an proof of quadratic reciprocity using Gauss sums. In this, we note that our original formulation of $\bar{p}p=(-1)^{(p-1)/2}p$ was actually a Gauss sum in disguise.

\defn{Multiplicative character}{
	A multiplicative character is a group homomorphism $\chi: G \to K^*$, where $K^*$ is the multiplicative group of a field, usually the field of complex numbers. 
	
	In what follows, denote a character on $G$ to be a homomorphism $\chi: G \to \bbC^*$.
}

\ex{Group character}{
	Let $\chi$ be a character describing a homomorphism from $G$ onto $\bbC^*$. Then, $\sum\limits_{\alpha \in G} \chi(\alpha)=\begin{cases} 0 & \chi \neq 1 \\ |G| & \chi = 1. \end{cases}$
	
	This special case is known as a group character.
}

A Dirichlet character $\Mod q$ is a special type of character. It is an arithmetic function $\chi:\bbZ\to\bbC$ if for all integers $a$ and $b$:

\begin{enumerate}
	\item $\chi(ab)=\chi(a)(b)$
	\item $\chi(a) \begin{cases} =0 & \text{if}\ \gcd(a, m)>1 \\ \neq 0 & \text{if}\ \gcd(a, m) =1 \end{cases}$
	\item $\chi(a+q)=\chi(a)$
\end{enumerate}

Group characters give rise to Dirichlet characters $\Mod q$. Let $\rho:(\bbZ/n\bbZ)^* \to \bbC^*$ be a group character, and let $\chi:\bbZ\to\bbC$ be a Dirichlet character.

$$\chi(a)=\begin{cases} 0 & \text{if}\ [a] \not\in (\bbZ/n\bbZ)^* \\ \rho([a]) & \text{if}\ [a] \in (\bbZ/n\bbZ)^* \end{cases}$$

Let $\zeta_p$ be a primitive $p$'th root of unity. Define the Gauss sum $\tau(\chi, n)$ of a Dirichlet character $\Mod q$ as

$$\tau(\chi, n) = \sum\limits_{a \Mod q} \chi(a) \zeta^{na}.$$ Select $\zeta_q=e^{2\pi i/q}$ to be a $q$'th root of unity. Denote, $\bar{\chi}(n)=\chi(n)^{-1}=\chi(n^{-1})$ to be the inverse of $\chi(n)$, where $n^{-1}$ is the inverse of $n \Mod q$.

\prop{}{
	For a primitive character $\chi \Mod q$, we have that $$\tau(\chi, n)=\bar{\chi}(n) \tau(\chi, 1).$$ 
}

\tbf{Proof:} Let $(n, q)=1$. Then, 

\begin{align*}
	\bar{\chi}(n) \tau(\chi, 1) &= \sum\limits_x \bar{\chi}(n) \chi(x) \zeta^x \\ 
	&= \sum\limits_x \chi(xn^{-1})\zeta^x \\
\end{align*}

and by letting $y=xn^{-1}$, we have that 

\begin{align*}
	\sum\limits_x \chi(xn^{-1})\zeta^x &= \sum\limits_y \chi(y) \zeta^{ny} \\ &= \tau(\chi, n) \qed
\end{align*}

Furthermore, we argue that the absolute value of the Gauss sum is $\sqrt{q}$.

\prop{}{
	$$|\tau(\chi, n)| = \sqrt{q}.$$
}

\textbf{Proof:} Firstly, consider that $|\tau(\chi, n)|^2=\tau(\chi, n)\bar{\tau}(\chi, n)=\sum\limits_x \sum\limits_y \chi(x)\bar{\chi}(y) \zeta^{n(x-y)}.$

Then, 

\begin{enumerate}
	\item If $x \not\equiv y \Mod q$, then $\sum\limits_{n \Mod q} \zeta^{n(x-y)} = 0.$
	\item If $x \equiv y \Mod q$, then $\sum\limits_{n \Mod q} \zeta^{n(x-y)} = q.$
\end{enumerate}

We consider the following lemma:

\lemma{}{
	Let $\chi$ be primitive. If $(n, q) >1, \tau(\chi, n)=0$.
}

\tbf{Proof:} Let $q=rd$ and $n=md$. Since $r \mid q$ and $\chi$ is a primitive character, there exists $c \in (\bbZ/q\bbZ)^*$ such that $c \equiv1 \Mod r$, and that $\gcd(c, q)=1, \chi(c) \neq 1.$ 

Let $c_1 = c_0^{-1}\equiv 1 \Mod r$. It then follows that (with $\zeta_r=\zeta_q^d$) because $$\tau(\chi, n) = \sum\limits_{a \Mod q} \chi(a)\zeta_r^{mx},$$ 

\begin{align*}
	\chi(c_0)\tau(\chi, n) &= \sum\limits_{x \in \bbZ/q\bbZ} \chi(c_0 x) \zeta_r^{mx} \\ &= \sum\limits_{x \in \bbZ/q\bbZ}\chi(x) \zeta_r^{mc_1x} \\ &= \sum\limits_{x \in \bbZ/q\bbZ}\chi(x) \zeta_r^{mx} = \tau(\chi, n) \qed
\end{align*}

Coming back to the proof, note that $\sum\limits_{n \Mod q} |\tau(\chi, n)|^2 = \sum\limits_{n \Mod q} |\chi(n)|^2 |\tau(\chi, 1)|^2.$ 

We sum over $n \Mod q$; note that since $\chi(n)=0$ if $(n, q)>1$, we sum $\phi(q)$ copies of $|\tau(\chi, n)|^2$ on the left hand side, where $\phi(q)$ is Euler's totient function. Specifically,

\begin{align*}
	 \sum\limits_{n \Mod q} |\chi(n)|^2 |\tau(\chi, 1)|^2 &= \phi(q) |\tau(\chi, n)|^2 \\ &= \phi(q) |\chi(n)|^2|\tau(\chi, 1)|^2 \\ &= q\ \phi(q)
\end{align*}

Thus, $|\tau(\chi, n)| = \sqrt{q}. \qed$

We now turn to formulating an alternate proof of the Law of Quadratic Reciprocity. To do this, we invoke the following lemma.

\lemma{}{
	$$\tau(\chi, 1)^2=(-1)^{(q-1)/2}q$$
}

\tbf{Proof:} Expanding the Gauss sum, we get that \begin{align*} \tau(\chi, 1)^2 &= \sum\limits_{n=0}^{q-1} \chi(n) \zeta^n \tau(\chi, 1) \\ &= \sum\limits_{n=0}^{q-1} \zeta^n \tau(\chi, n) \end{align*}

We then consider that \begin{align*} \sum\limits_{n=0}^{q-1} \zeta^n \tau(\chi, n) &= \sum\limits_{n=0}^{q-1} \zeta^n \left(\sum\limits_{y=0}^{q-1} \chi(y) \zeta^y \right) \\ &= \sum\limits_{x=0}^{q-1}\left(\sum\limits_{y=0}^{q-1} \chi(y) \zeta^{x(y+1)} \right) \\ &= \sum\limits_{y=0}^{q-1} \chi(y) \left( \sum\limits_{x=0}^{q-1} \zeta^{x(y-1)} \right) \end{align*}

This is a product of two sums. Let the second sum be $S(y)$. It equals $q$ if $y=-1$, and $0$ otherwise. Thus, $\tau(\chi, 1)^2 = \chi(-1) \cdot q = (-1)^{(q-1)/2}q. \qed$

In what follows, we work with $\bbZ[\zeta]$ and transform it into a problem in $\bbZ$. Let $\bar{q}= (-1)^{(q-1)/2}$. 

Note that $\tau(\chi, 1)^p = \sum\limits_{n=0}^{q-1} \chi_1(x)^p \zeta^{pn} = \sum\limits_{n=0}^{q-1} \chi_1(x) \zeta^{pn} \equiv \tau(\chi, p)$. Note that $$\tau(\chi, p) \equiv \legendre{p}{q} \tau(\chi, 1) \Mod p,$$ and thus

$$\tau(\chi, 1)^{p+1} \equiv (-1)^{(q-1)/2}q \cdot \legendre{p}{q} \Mod p.$$ It then follows that 

\begin{align*}
	\tau(\chi, 1)^{p+1} &= (-1)^{(q-1)/2}q \cdot  (-1)^{(p+1)/2} q^{(p-1)/2} \\ &= -1^{(q-1)/2}q \cdot -1^{(q-1)(p-1)/4} q^{(p-1)/2}\\ &= (-1)^{(q-1)/2}q \cdot (-1)^{(p-1)(q-1)/4} \legendre{q}{p} \Mod p
\end{align*} 

We now work in $\bbZ$. Luckily, we can simply repeat what we've achieved for $\bbZ[\zeta]$. To conclude, 

$$ (-1)^{(q-1)/2}q \cdot \legendre{p}{q} \\ \equiv (-1)^{(q-1)/2}q \cdot (-1)^{(p-1)(q-1)/4} \legendre{q}{p} $$

and since $q$ is coprime with $p$, 

$$\legendre{p}{q} \\ \equiv  (-1)^{\frac{p-1}{2} \frac{(q-1)}{2}} \legendre{q}{p} \qed$$

\section{Appendix: Fermat's Last Theorem}

Fermat's Last Theorem states that:

\thrm{}{
	For all $n\in\bbZ^{+}, n>2$, there exists no solutions to the equation $a^n+b^n=c^n$, where $a,b,c \in \bbZ$.
}

This theorem was famously proved by Wiles in 1994. Kummer proved a weaker version of this theorem:

\thrm{}{
	Suppose $p$ is an odd prime which does not divide the class number of the field $\bbQ(\zeta_p)$. Then, the equation $a^p+b^p+c^p$, with $a, b, c$ coprime to $p$, has no integer solutions.
}

I'll now present a proof of Fermat's Last theorem for $n=3$:

\textbf{Propsition:} There exists no integer solutions to $a^3+b^3=c^3$.

This proposition can be transformed, through the transformation $c \rightarrow -c$, into $a^3+b^3+c^3=0$, where $a, b, c$ are pairwise coprime and not all positive. Assume, for sake of contradiction, that there exists an integer triplet that satisfies 

\begin{equation}
	a^3+b^3+c^3=0.
\end{equation}

This means that one of the $(a, b, c)$ must be even, and the rest are odd. Assume that WLOG, $c$ is even. Then $a, b$ are odd. 

\lemma{}{$a, b$ cannot be equal.

}

\emph{Proof:} If $a, b$ are equal, then $2a^3=-c^3$, which implies that $c$ is even. This is a contradiction.

Note that since $a$ and $b$ are odd, their sum are difference are both even numbers. Let $a+b=2x$, and $a-b=2y$. It is trivial to show that $x$ and $y$ have different parity, by considering $a, b =1 / 3 \Mod 4$ and taking $x, y \Mod 4$.

It then follows that $a = x+y$, and $b=x-y$.

However then notice that since $a^3+b^3+c^3=0$, then

\begin{align*}
	a^3+b^3=-c^3 & \\
	(x+y)^3 + (x-y)^3 = -c^3 & \\
	2x(x^2+3y^2) = -c^3 & \\
\end{align*}

We get that $c$ is even.

Note that $x^2+3y^2$ is always odd as $x$ and $y$ have different parity. Since $c$ is even, then $x^2$ is even ($\rightarrow x$ even), and $y$ is odd. Now we prove that the greatest common divisor of $2x$ and $x^2+3y^2$ is $1$ or $3$. We do casework on the GCD.

\textbf{Case 1:} The two common factors are coprime.

Then we can represent $2x = r^3$ and $x^2+3y^2=s^3$, where $r^3 \cdot s^3 = -c^3$. Since $x^2+3y^2$ is odd, then $s$ is also odd. Then $s$ is odd. Represent $s = e^2+3f^2$

We can also represent $x$ and $y$ as:

\begin{align*}
	x = e(e^2-9f^2) & \\
	y = 3f(e^2-f^2) & \\
\end{align*}

Since $x, y$ are coprime, then $e, f$ are also coprime.

Note that $r^3 = 2x = 2e(e+3f)(e-3f)$. They are all coprime - since if $3 \mid e$, then $x$ and $y$ would not be coprime. Hence all 3 terms on the R.H.S can be represented as cubes of smaller integers, yielding a smaller solution. By the \emph{method of infinite descent}, this is a contradiction.

\textbf{Case 2:} We can represent $x = 3w$ as $3 \mid x$. Thus, $-c^3 = 6w(9x^2+3y^2)=18w(3x^2+y^2)$. Since $x, y$ are coprime, then $18w$ and $3x^2+y^2$ are also coprime. Represent $18w$ and $3x^2+y^2$ as smaller cubes. Let

\begin{align*}
	18w = r^3 & \\
	3x^2+y^2 = s^3. & \\
\end{align*}

Then we can also represent $s$ as $3e^2+f^2$. Note that $y=e(e^2-9f^2)$, and $w = 3f(e^2-f^2)$ It then follows that $r^3 = 54f(e+f)(e-f).$ 

Note that since $54 = 3^3 \cdot 2$, we have that $3 \mid r$. Then $(\frac{r}{3})^3$ divides $-2f(e+f)(e-f)$

Thus, it can be deduced that the $-2f$, $e+f$, and $e-f$ are coprime. However, like in the previous case, they can also be represented as cubes of smaller integers. This is a contradiction by the method of infinite descent.

\chapter{Analytic number theory}
\minitoc

This chapter is rather brief. It leads up to the class number formula.

\section{Zeta functions, L-functions and Dirichlet series}

The Riemann zeta function is one of the most famous functions in mathematics. It is one example of an \emph{L-function}.

\defn{Riemann zeta function}{
	The Riemann zeta function is $$\zeta(s) = \sum\limits_{k=1}^\infty \frac{1}{k^s}$$ where $s=\sigma+it$, further satisfying $\sigma, t \in \bbR$ and $\sigma>1$.
}

It is very common in Dirichlet series to write $s=\sigma+it$, where $\sigma = \text{Re}(s)$ and $t=\text{Im}(s)$.

\prop{Euler product formula for the Riemann zeta function}{
	$$\zeta(s) = \sum\limits_{k=1}^\infty \frac{1}{k^s}= \prod\limits_p \frac{1}{1-p^{-s}}$$
}

\tbf{Proof:} Expand and divide both sides by $\frac{1}{2^s}$.

\begin{align*}
	\zeta(s) &= 1+\frac{1}{2^s}+\frac{1}{3^s}+\dots \\
	\frac{1}{2^s} \zeta(s) &= \frac{1}{2^s}+\frac{1}{3^s} + \frac{1}{4^s} + \dots
\end{align*}

Thus we have $$\left(1-\frac{1}{2^s}\right) \zeta(s) = 1+\frac{1}{3^s}+\frac{1}{5^s}+\frac{1}{7^s}+\dots.$$ Then divide by $\frac{1}{3^s}$ and subtracting the new expression from the one above, yielding us $$\left(1-\frac{1}{3^s}\right)\left(1-\frac{1}{2^s}\right) \zeta(s) = 1+\frac{1}{5^s}+\frac{1}{7^s}+\frac{1}{11^s}+\dots.$$

Repeating indefinitely, we have that $\left(1-\frac{1}{7^s}\right)\left(1-\frac{1}{5^s}\right)\left(1-\frac{1}{3^s}\right)\left(1-\frac{1}{2^s}\right)\zeta(s)=1$, or $$\zeta(s) = \prod\limits_p \frac{1}{1-p^{-s}},$$ as desired. $\qed$

\ex{Euler}{
	Euler showed that $$\zeta(2m) = (-1)^{m-1} \frac{(2\pi)^{2m}}{2(2m)!} B_{2m}$$ where $B_k$ are the Bernoulli numbers and $m\geq 1$.
	
	For the first few values of $m$, we have that $$\zeta(2) = \frac{\pi^2}{6},\ \text{and}\  \zeta(4)=\frac{\pi^4}{90}.$$
}

So what are $L$-functions? $L$-functions are meromorphic functions on the complex plane "associated with something", which could be a Dirichlet character, a number field, or a modular form. However, all $L$-functions have a Dirichlet series expansion $$L(X, s)=\sum\limits_{n \geq 1} \frac{a_n}{n^s}.$$

\ex{}{
	Let $a_n=\chi(n)$ for a Dirichlet character $\chi$. Then, $$f(s)=L(s, \chi).$$ Furthermore, $f(s)$ converges absolutely for $\sigma >1$.
}

\defn{Dirichlet series}{ 
	Series of the form $$\sum\limits_{n\geq 1} \frac{a_n}{n^s}$$ are called Dirichlet series. 
	
	Dirichlet L-functions, represented as $L(s, \chi)$, are special cases of Dirichlet series.
}

We often write Dirichlet series as $\sum a_n n^{-s}$, where by standard notation, summation starts at $n=1$. Write $s=\sigma + it$, with $\sigma$ and $t$ real. 

$L$-series may give rise to an $L$-function via \emph{analytic continuation}. An $L$-series has analytic continuation if there is a meromorphic function that coincides with the series on its domain of convergence.

\prop{}{
	Let $a_1, a_2, \dots, a_n$ be a series of complex numbers that have a sum equal to $A_n=|a_1+\dots+a_n|$, which is bounded. Then, the series $\sum\limits_n a_nn^{-s}$ converges.
}

\tbf{Proof:} Let $A_0=0$, and for each $n$, let $a_n=A_n-A_{n-1}$. By partial summation, 
\begin{align*} 
	\sum\limits_{n=1}^N \frac{a_n}{n^s} &= \sum\limits_{n=1}^N \frac{A_n}{n^s} - \frac{A_{n-1}}{n^s} \\
	&= \frac{A_N}{N^s} - \sum\limits_{n=1}^{N-1} A_n \left(\frac{1}{(n+1)^s}-\frac{1}{n^s}\right).
\end{align*}

Note that since $\int_a^b \frac{dx}{x^{s+1}}=-\frac{1}{s}(\frac{1}{b^s}-\frac{1}{a^s})$, we have that $$\sum\limits_{n=1}^N \frac{a_n}{n^s}=\sum\limits_{n=1}^{N-1} A_n(-s) \int_n^{n+1} \frac{dx}{x^{s+1}}.$$

Let $A(x) = A_{\lfloor x \rfloor}.$ We have that 

\begin{align*}
	\sum\limits_{n=1}^N \frac{a_n}{n^s}&=\sum\limits_{n=1}^{N-1} A_n(-s) \int_n^{n+1} \frac{dx}{x^{s+1}}\\
	&= \frac{A_N}{N^s}+s\int_1^N \frac{A(x)}{x^{s+1}}\ dx
\end{align*}

Letting $|A_n| \leq 0$, we have that $\left| \frac{A_N}{N^s}\right| \leq \frac{c}{N^{\sigma}}$, which $\to 0$ as $N\to \infty$. Extend $N$ in the sum above to infinity. Knowing that $\int_1^\infty \frac{A(x)}{x^{s+1}}$ converges absolutely, we have that $$\sum\limits_{n=1}^\infty = s\sum_1^\infty \frac{A(x)}{x^{s+1}}\ dx.$$

Since $A_n$ is bounded, the $a_n$ are bounded, and thus the sum $\sum\limits_n \frac{|a_n|}{n^\sigma}$ converges for $\sigma >1$. $\qed$

\coro{}{
	Let $\chi:(\bbZ/m\bbZ)^{\times} \to \bbC$ be a non-trivial Dirichlet character. Then the Dirichlet series $$L(s, \chi)$$ converges for $\sigma>0$.
}

\tbf{Proof:} We want to show that $\chi(1)+\chi(2)+\dots+\chi(n)$ are bounded, and apply the theorem above.

Let $S$ be the sum of $\chi$ over a period. Note that $\chi(a) \neq 1$ for a unit $a \Mod m$. Thus $$\chi(a)S = \sum\limits_{k \in (\bbZ/m\bbZ)^\times} \chi(ak)=\sum\limits_{k \in (\bbZ/m\bbZ)^\times} \chi(k)=0$$ and thus $S=0$. $\qed$

We can further extend the convergence condition above. If there exists a $k>0$ such that $$|A_n| < k\cdot \log(n)$$ where $|A_n|=\sum_i |a_i|$, then the series $\sum\limits_{n\geq 1} a_nn^{-s}$ converges for $\sigma >0$.

\defn{Abscissa of convergencce}{
Denote the \emph{abscissa of convergence} for a Dirichlet series $\sigma_0$ such that the series converges for all $s$ with $\text{Re}(s)=\sigma > \sigma_0$ and diverges for every $s$ with $\text{Re}(s)=\sigma < \sigma_0.$ 
}

\ex{}{
	The abscissa of convergence for $\sum\limits_n \frac{|a_n|}{n^\sigma} \leq \sigma$. 
}

\defn{Dedekind zeta function}{
	Let $K$ be a number field. Let $j_n$ denote the number of ideals with ideal norm equal to $n$. Then we define the \emph{Dedekind zeta function} of $K$ as $$\zeta_K(s) = \sum\limits_{n=1}^\infty \frac{j_n}{n^s}.$$ Or we can write the Dedekind zeta function $$\zeta_K(s)=\sum\limits_{\mfa} \frac{1}{\mfN(\mfa)^s}$$ where $\mfa \in \mcO_K.$ 
	
	In the case $K=\bbQ$, the Dedekind zeta function and the Riemann zeta function are identical. 
}

The Dedekind zeta function can be written as $$\zeta_K(s)=\prod\limits_{\mfp \in \mcO_K} \frac{1}{1-\left(\frac{1}{N(\mfp}\right)^s}$$ 

\prop{}{
	$\zeta_K(s)$ converges absolutely provided that $\sigma = \text{Re}(s) >1$.
}

\tbf{Proof:} Note that since $N(\mfp)\geq p$, we have that

\begin{align*}
	|\zeta_K(s)| &= \left| \prod\limits_\mfp 1-\left(\frac{1}{N(\mfp})\right)^s \right| \\
	&\leq \prod\limits_p \left(1-\frac{1}{p^\sigma} \right)^{-n} \\
	&= \zeta(\sigma)^n \qed
\end{align*}

\section{Class number formula}

Hecke showed that there is an analytic continuation of the Dedekind zeta function $\zeta_K(s)$ for $\sigma=1$ to all of $\bbC$. As a detour, we demonstrate how we can analytically continue the Riemann zeta function. Then we prove the class number formula, and show Hecke's analytic continuation of $\zeta_K(s)$.

\thrm{Analytic continuation of $\zeta(s)$}{
	The Riemann zeta function $\zeta(s)$ has an analytic continuation that satisfies the functional equation $$\zeta(1-s)=\frac{2}{(2\pi)^s}\Gamma(s) \cos\left(\frac{\pi s}{2}\right) \zeta(s)$$ where $\Gamma(s)=\int_0^\infty e^{-t}t^{s-1}\ dt$ for $s>0$ is the Gamma function. 
	
	Thus the Riemann zeta function has its analytic continuation defined on $\bbC \backslash \{1\}$, with a simple pole at $s=1$ with residue $1$.
}

\tbf{Proof:} This was how Riemann did it. It is very non-trivial and non-intuitive but is used in conjunction with Poisson summation and other concepts in number theory.

Firstly, he considered the contour integral $$I=\int_C \frac{(-z)^{s-1}}{e^z-1}\ dz$$ where the branch line lies along the positive real $z$ axis, and $C$ is a contour that comes from $+\infty$, goes around the branch point at $z=0$ and returns to $+\infty$ above the branch line.

Choose parts where $(-z)^{s-1}=e^{-i\pi(s-1)}x^{s-1}$ just above the branch line and $(-z)^{s-1}=e^{i\pi(s-1)}x^{s-1}$ just below. Then, we have, by the residue theorem and the fact that $$J=\int_0^\infty \frac{x^{s-1}}{e^x-1}=\Gamma(s)\zeta(s)$$

\begin{align*} 
	I &=(e^{-i\pi(s-1)}-e^{i\pi(s-1)})\cdot J \ dx  \\
	&= -2i\sin\pi(s-1) \Gamma(s)\zeta(s)
\end{align*} 

and therefore, $$\zeta(s)=\frac{1}{2i \sin{\pi s}} \cdot 
\frac{1}{\Gamma(s)} \cdot I.$$

Using the reflective property of the Gamma function $$\Gamma(s)\Gamma(1-s)=\frac{\pi}{\sin{\pi s}},$$ we have that $$\zeta(s)=\frac{\Gamma(1-s)}{2\pi i} \cdot I.$$

Consider the closed contour integral $$K=\oint_\Gamma \frac{(-z)^{s-1}}{e^z-1},$$ taking the contour in $I$ and joining it with the contour $\Gamma$ being a circle of radius $(2N+1)\pi$. 

By such, $$K=I+L$$ where $$L=i\int_0^{2\pi} Re^{i\theta} \frac{R^{s-1}e^{i(s-1)(\theta-\pi)}}{e^{R\cos\theta+i R\sin\theta}-1}\ d\theta.$$ The value of the integral tends to $0$ for sufficiently large $R$, provided that $\sigma<0$.

Note that by the residue theorem, $J$ is equal to $2\pi i$ multiplied by the sums of the residues of the poles of the integrand, which occur at $z=2\pi in$ for $n\in \bbZ \backslash \{0\} $. For such values of $z$, the poles are 

$$\begin{cases}
	(e^{-\frac{i \pi}{2}}\cdot 2\pi|n|)^{s-1} & \ \text{if}\ n\ \text{is positive} \\
	(e^{\frac{i \pi}{2}}\cdot 2\pi|n|)^{s-1} & \ \text{if}\ n\ \text{is negative}	 
\end{cases}$$

Thus we have that $$I=(2\pi i) 2\cos\frac{\pi}{2}(s-1) \sum\limits_{n\geq 1} (2\pi n)^{s-1}$$ and thus we have that, using the fact that $$\zeta(s)=\frac{\Gamma(1-s)}{2\pi i} \cdot \int_C \frac{(-z)^{s-1}}{e^z-1}\ dz,$$ we get that $$\frac{\zeta(s)}{\zeta(1-s)}=2\Gamma(1-s)(2\pi)^{s-1}\cos{\frac{\pi}{2}}(s-1)$$

Let $1-s=x$. We can thus simplify the expression, for $\text{Re}(x)>1$, as $$\zeta(1-x)=2^{1-z}\pi^{-x} \cos{\frac{\pi x}{2}}\Gamma(x)\zeta(x). \qed$$
\ex{}{
	Letting $B_k$ denote the $k$'th Bernoulli number, we have that 
	$$\zeta(2k) = \sum\limits_{n=1}^\infty \frac{1}{n^{2k}} = \frac{(-1)^{k-1} B_{2k} (2\pi)^{2k}}{2(2k)!}.$$
	
	By this, it immediately follows that $$\zeta(1-2k)=-\frac{B_{2k}}{2k}.$$
}

\tbf{Proof:} Define the \emph{Bernoulli polynomial} $B_n(X)$ as a polynomial satisfying the following properties:

\begin{enumerate}
	\item $B_0(x)=1$
	\item $B'_n(x) = B_{n-1}(x)$
	\item $\int_0^1 B_n(x)\ dx = 0$ for $n\geq 1$.
\end{enumerate}

The Bernoulli number $B_k$ is defined as $B_k=B_k(0)$.

The first Bernoulli polynomials are $B_0(x)=1$, $B_1(x)=x-\frac{1}{2}$, and $B_2(x)=x^2-x+\frac{1}{6}$. Our plan is to first consider $B_1(x)$ and use the recursive property of the Bernoulli polynomial to devise a general formula for $B_{2k}$.

It is known that the Fourier expansion $F(x)$ of a function $f(x)$ with period $T$ is $$F(x)=\frac{a_0}{2}+\sum\limits_{n=1}^\infty a_n \cos\left(\frac{2n\pi x}{T} \right)+\sum\limits_{n=1}^\infty b_n \sin\left(\frac{2n\pi x}{T} \right)$$ where $a_n=\frac{1}{T}\int_{-T/2}^{T/2} f(x) \cos\left(\frac{2n\pi x}{T} \right)$ and $b_n=\frac{1}{T}\int_{-T/2}^{T/2} f(x) \sin\left(\frac{2n\pi x}{T} \right)$.

Consider the Fourier transform of $f(x)=x$. Then $a_n=0$ except for $a_0=\frac{1}{2}$, and $$b_n=4\int_0^{\frac{1}{2}} x\sin(2\pi n x)\ dx = -\frac{(-1)^{n+1}}{n\pi}$$ and thus in $[-\frac{1}{2}, \frac{1}{2}]$, $$x=-\sum\limits_{n=1}^{\infty} \frac{(-1)^{n+1}}{n\pi} \sin(2\pi n x)+\frac{1}{2}.$$

And thus in the range $[0, 1]$ (because we're calculating $B_1(x)$ in $[-\frac{1}{2}, \frac{1}{2}]$ so we ought to "shift"), $$B_1(x)=x-\frac{1}{2}=-\sum\limits_{n=1}^{\infty} \frac{(-1)^{n+1}}{n\pi} \sin\left(2\pi n \left(x-\frac{1}{2}\right)\right)$$

From $B'_n(x) = B_{n-1}(x)$ we have that $\int_0^x B_n(x)\ dx=B_{n+1}(x)-B_{n+1}(0)$. From this we conclude that $$B_2(x)=2\sum\limits_{n=1}^\infty \frac{(-1)^{n+1}}{(2\pi n)^2} \cos\left(2\pi n \left(x-\frac{1}{2}\right)\right)$$

Repeated integration yields the general formula $$B_{2k}(x)=2(-1)^{k+1} \sum\limits_{n=1}^\infty \frac{(-1)^{n+1}}{(2\pi n)^{2k}}\cos\left(2\pi n \left(x-\frac{1}{2}\right)\right)$$

Evaluate $B_{2k}(x)$ at $x=0$. We get that $$\frac{B_{2k}}{(2k)!}=2(-1)^{k+1} \sum\limits_{n=1}^\infty \frac{(-1)^{n+1}}{(2\pi n)^{2k}}\cos(-n \pi)$$ and by isolating $\frac{1}{n^{2k}}$ we get that $$\zeta(2k)=\sum\limits_{n=1}^\infty\frac{1}{n^{2k}}=\frac{(-1)^{k+1}(2\pi)^{2k}B_{2k}}{2(2k)!}. \qed$$

The proof that $\zeta(1-2k)=-\frac{B_{2k}}{2k}$ is left as an exercise.

\thrm{Class number formula}{
	Let $K$ be a number field and let $h_K$ be its class number. Let $\omega$ be the number of roots of unity that it has into $K$, and let $r$ and $r_2$ respectively be the number of real and complex embeddings of $K$ into $\bbC$. 
	
(We used to write $r$ for the number of real embeddings and $s$ for the number of complex embeddings but it coincides with notation for the class number formula. In this case, write $r_1=r, r_2=s.$) 
	
	Then, $\zeta_K(S)$ extends to a function on $\text{Re}(z)>1-\frac{1}{n}$ which is holomorphic except for a simple pole at $s=1$ with residue $$\lim\limits_{s-1} (s-1) \zeta_K(s) = \frac{2^{r_1+r_2} \pi^{r_2}\ R_K} {\omega\sqrt{D_K}}h_K,$$ where $R_K$ and $D_K=\text{disc}(\mcO_K)$ denote respectively the regulator and the discriminant of $\mcO_K$. 
}

Like many instances in this text, we give an example of this formula before proving it. 

\ex{Class number formula for $K=\bbQ$}{
	Let $K=\bbQ$. Thus $n=1, r_1=1, r_2=0, h_K=1, \omega=2, D_K=1, R_K=1.$ The theorem states that $\zeta_{\bbQ}(s)=\zeta(s)$ is holomorphic on $\text{Re}(s) > 0$ except for a simple pole at $s=1$ with residue $$\lim\limits_{s\to 1^+} (s-1)\zeta_{\bbQ}(s) = 1.$$
	
	We already know that $\zeta(s)$ has residue 1 at $s=1$. We can evaluate the residues of many other Dedekind zeta functions. In practice this formula is often used to find the class number, which is harder to evaluate. In this case the class number $h_{\bbQ}=1$, which makes sense.
}

Now we prove the class number formula. Define a \emph{cone} to be a subset $x \subset \bbR^n$ such that if $x\in X$ and $\lambda \in \bbR_{>0}$, then $\lambda x \in X$.

\lemma{}{
	Let $F:X\to\bbR_{>0}$ satisfy $F(c x)=c^nF(x)$. Additionally let $\mcF=\{x \in X: F(x) \leq 1\}$ be bounded with volume $v>0$. Let $\Gamma \subseteq \bbR^n$ be a lattice with volume $\Delta = vol(\Gamma)$. Then, $$\zeta_{F,\Gamma}(s)= \sum\limits_{x \in \Gamma \cap X} \frac{1}{F(x)^s}$$ converges on $\text{Re}(s)>1$ and satisfies $\lim\limits_{s\to1}(s-1)\zeta_{F,\Gamma}(s)=\frac{v}{\Delta}$.
}

\tbf{Proof:} Write $$\zeta_K(s)=\sum\limits_{C \in C_K} f_C(s)$$ where the sum runs over ideal classes in the class group, and $$f_C(s) = \sum\limits_{\mfa \in C} \frac{1}{N(\mfa)^s}.$$ Then choose an integral ideal $\mfb \in C^{-1}$, the inverse class. Thus, we know that for all $\mfa \in C$, $\mfa\mfb$ is a principal ideal. Mapping $\mfa \mapsto \langle \alpha \rangle$ gives a bijection $$\{\mfa \in C\} \iff \{\langle \alpha \rangle:  \mfb \mid \langle \alpha \rangle\}.$$

This suggests that we can write $f_C(s)$ as $$f_C(s) = N(\mfb)^s \sum\limits_{\mfb \mid \langle \alpha \rangle} \frac{1}{|N(\mfa)|^s}.$$ 

For all $r \in \bbR_{>0}$ we have that $\text{vol}(\frac{1}{r}\Gamma)=\frac{\Delta}{r^n}.$ Thus, 
$$v=\lim\limits_{r\to\infty} \left(\frac{\Delta}{r^n} \cdot \#\left\{\frac{1}{r}\Gamma \cap \mcF\right\}\right)= \Delta \frac{\#\left\{\frac{1}{r}\Gamma \cap \mcF\right\}}{r^n}.$$

Label the points of $\Gamma \cap X$ such that $0\leq F(x_1) \leq F(x_2) \leq \dots$, and define $r_k = F(x_k)^{\frac{1}{n}}$. If we define $\gamma(r)=\#\{\frac{1}{r}\Gamma \cap \mcF\}$ we have that for $\varep > 0, \gamma(r_k-\varep)<k\leq \gamma(r_k)$ and thus $$\frac{\gamma(r_k-\varep)}{(r_k-\varep)^n}\left(\frac{r_k-\varep}{r_k}\right)^n<\frac{k}{r^n_k}\leq \frac{\gamma(r_k)}{r^n_k}.$$ In the limit as $r_k=\infty$, we thus have that $\lim\limits_{r_k\to\infty}\frac{k}{r^n_k}=\lim\limits_{k\to\infty}\frac{k}{F(x_k)}=\frac{v}{\Delta}$. To show that $\zeta_{F, \Gamma}$ converges is very similar to showing that $\zeta$ converges. 

The above inequality shows that given $\varep>0$, there exists a $k_0$ such that if $k\geq k_0$, then $$\left(\frac{v}{\Delta}-\varep\right)\frac{1}{k}<\frac{1}{F(x_k)}<\left(\frac{v}{\Delta}+\varep\right)\frac{1}{k}$$ and thus $$\left(\frac{v}{\Delta}-\varep\right)^s \sum\limits_{k=k_0}^\infty \frac{1}{k^s}< \sum\limits_{k=k_0}^\infty \frac{1}{F(x_k)^s}<\left(\frac{v}{\Delta}+\varep\right)^s \sum\limits_{k=k_0}^\infty\frac{1}{k^s}.$$

Knowing that $\zeta_{F, \Gamma}$ converges for $\text{Re}(s)>1$, multiply by $s-1$ and consider the limit as $s\to1^+$: we have that $$\frac{v}{\Delta}-\varep \leq \lim\limits_{s\to1}(s-1)\zeta_{F, \Gamma}\leq \frac{v}{\Delta}+\varep$$ and the result follows. $\qed$

Pick $\varep_1, \dots, \varep_{r_1+r_2-1}$ to be fundamental units. Let $\lambda=(1, \dots, 1; 2, \dots, 2)$. Then, $\{\lambda, \phi(\varep_1), \dots, \phi(\varep_{r_1+r_2-1})\}$ is a basis for $\bbR^{r_1} \times \bbC^{r_2}$. 

Let $\bbZ_K^\times = \alpha \in \bbZ_K: N(\alpha)=\pm 1$, and let $H=\{x \in \bbR^{r_1+r_2}: \text{tr}(x)=0\}.$ The statement above is true as the map $\lambda:K\to \bbR^{r_1+r_2}$ restricts to a map $\lambda: \bbZ_K^\times \to H$ with image $\Gamma \cong \bbZ^{r_1+r_2-1}$.

Thus we can write, for $l(x)\in \bbR^{r_1+r_2}$, that $$l(x)=c\lambda+c_1 \phi(\varep_1)+\dots+c_{r_1+r_2-1} \phi(\varep_{r_1+r_2-1})$$ such that $c = \frac{1}{n} \log |N(x)|$. Then consider the cone $X$ to consist of all $x$ such that \begin{enumerate}
	\item $N(x) \neq 0$
	\item The coefficients $c_i$ satisfy $0 \leq c_i<1$ for all $i$
	\item $0\leq \text{arg}(x_1) < \frac{2\pi}{\omega}$, where $x_1$ is the first component of $x$.
\end{enumerate}

Let $c \in \bbR^+$. Then, $N(cx)=c^n N(x) \neq 0$, so $l(cx)=(\log c)\lambda + l(x)$. Also, $\text{arg}(cx_1)=\text{arg}(x_1)$. Thus, if $x\in X$ and $c \in \bbR^+$, then $cx \in X$ and thus $X$ is a cone. 

\lemma{}{
	Let $\eta(\alpha) \subseteq \mcO_K$ be the set of all elements in $\mcO_K$ that are associates of $\alpha$, including $\alpha$. Then exactly one member of $\eta(\alpha)$ has an image in $X$.
}

\tbf{Proof:} Given $y\in \bbR^{r_1}\times \bbC^{r_2} \cong \bbR^n$ with nonzero norm, we can write $y= x\cdot \phi(\varep)$ where $x\in X$ and $\varep$ is a unit. Then, write $$l(y)=c\lambda+c_1 \phi(\varep_1)+\dots+c_{r_1+r_2-1} \phi(\varep_{r_1+r_2-1}).$$ Write $c_i=m_i+\mu_i$, where $m_i \in \bbZ, 0\leq \mu_i<1$, and write $u=\varep_1^{m_1}\dots \varep_{r_1+r_2-1}^{m_{r_1+r_2-1}}$. 

Then put $z =y \cdot \phi(u^{-1})$. Write $$0\leq \text{arg}(z_1)-\frac{2\pi k}{m} < \frac{2\pi}{m}$$ for some integer $k$. Choose $w \in \mu(K)$ a root of unity such that $\tau_1(\zeta)=e^{\frac{2\pi i}{m}}$. Thus, writing $r=r_1+r_2-1$, we have that $$z \cdot \phi(w^{-r})=y \cdot \phi(u^{-1})\phi(w^{-r}) \in X.$$ Let this value $z \cdot \phi(w^{-r})=x$. Then, $$y=x \cdot \phi(uw^r)$$ and  this image is unique. $\qed$

An immediately corollary of this is that we can write $$f_C(s)=N(\mfb)^s \sum\limits_{x\in \Gamma \cap X} \frac{1}{|N(x)^s|}.$$

We want to calculate $v = \text{vol}(\{x \in X: N(x)\leq 1\})$ and $\Delta=\text{vol}(\Gamma)$. To calculate $\Delta$, we have $\Gamma$ be the lattice $\subseteq \bbR^{r_1}\times \bbC^{r_2}$ such that $x =\phi(b)$ for some $b \in \mfa$. Thus we know that $\Gamma$ is generated by $\phi(a_1), \dots, \phi(a_n).$ 

Let $M$ be a matrix with entries $(\rho_i\alpha_j)$, where $\rho_i$ varies over all real and complex embeddings of $K$. Thus, $$D(\mfa)=\text{det}(M)^2 = N(\mfa)^2 D_K.$$ Letting $C$ be the matrix consisting of inner products $(\langle \phi(\alpha_i), \phi(\alpha_j)\rangle)=M^T \bar{M}$, we thus have that $|\text{det}(C)|^{1/2}=|\text{det}(M)|.$ Since $$\text{vol}(\Gamma)=|\text{det}(M)|=|\text{det}(C)|^{1/2}=D(\mfa)^{1/2}$$ we have that $\text{vol}(\Gamma)=N(\mfa) |D_K|^{1/2}$. 

\lemma{}{
Recall that $\mcF = \{x\in X:F(x)\leq 1\}.$ $\text{vol}(\mcF)$ is given by

	$$v=\frac{2^{r_1+r_2}\pi^{r_2} R_K}{\omega}.$$
}

\tbf{Proof:} Define $\mcF_k$ for $0 \leq k < \omega$ by mapping $x \mapsto e^{\frac{2\pi k}{\omega}}$. The new volume arrived from multiplication by a unit is volume-preserving. Thus we have that $\mcF_k=\mcF \cdot |N(e^{\frac{2\pi k}{\omega}})|=\mcF.$

Let $\bar{\mcF}$ to be the intersection of $$\bigcap\limits_{k=0}^{\omega} \mcF_k$$ with $\{(x_1, \dots, x_{r_1}; x_{r_1+1}, \dots, x_{r_1+r_2}): x_1, \dots, x_{r_1} >0\}.$ 
 
It follows that multiplying any point in $\bar{F}$ by $(\pm1, \dots, \pm1; 1, \dots, 1)$ yields $$\text{vol}(\mcF) = \frac{2^r}{\omega} \text{vol}(\bar{\mcF}).$$
 
How do we compute $\text{vol}(\bar{\mcF})?$ We do this through a change of variables. Consider the isomorphism $$\bbR^{r_1}\times \bbC^{r_2} \to \bbR^n$$ given by $$(x_1, \dots, x_{r_1}, z_1, \dots, z_{r_2})\mapsto (x_1, \dots, x_{r_1}, R_1, \phi_1, \dots, R_{r_2}, \phi_{r_2})$$ where $z_k = R_ke^{i\phi_k}$. Thus $$l(x_1, \dots, x_{r_1}, z_1, \dots, z_{r_2})=(\log x_1, \dots, x_{r_1}, \log R_1^2, \dots, \log R_{r_2}^2)$$ with the Jacobian of this change of variables being equal to $R_1\cdot\dots\cdot R_{r_2}$.

Thus $\bar{F}$ satisfies the conditions $x_1>0, \dots, x_{r_1}>0, R_1>0, \dots, R_{r_2} > 0$ and $x_1\dots x_{r_1} (R_1\cdots R_{r_2})^2 \leq 1$. Also, the $\phi_{r_1+1}, \dots, \phi_{r_1+r_2}$ independently take values in $[0, 2\pi])$.

Now let's replace the variables $x_1, \dots, x_{r_1}, R_1, \dots, R_{r_2}$ with the variables $c, c_1, \dots, c_{r_1+r_2-1}$ (as in the formula $l(y)=c\lambda+c_1 \phi(\varep_1)+\dots+c_{r_1+r_2-1} \phi(\varep_{r_1+r_2-1})).$ The image of $\bar{\mcF}$ is given by $0<c \leq 1$ and $0 \leq \zeta_k < 1$ for all $k\in [1, r_1+r_2-1]$. 

The $j$'th components of this Jacobian satisfy $$\log x_j = \frac{1}{n} \log c + \sum\limits_{k=1}^r c_k \phi_j(\varep_k)$$ $$\log R_j^2 = \frac{2}{n} \log c + \sum\limits_{k=1}^r c_k \phi_{r_1+j}(\varep_k).$$ Therefore we have that the transformation is given by 


$$J=\begin{pmatrix}
        \frac{x_1}{nc} & x_1\phi_1(\varep_1) & \dots & x_1\phi_1(\varep_{r_1}) \\
        \vdots & \vdots & \ddots & \vdots \\
        \frac{x_{r_1}}{nc} & x_{r_1}\phi_{r_1}(\varep_1) & \dots & x_1\phi_{r_1}(\varep_{r_1}) \\ \\
        \frac{R_1}{nc} & \frac{R_1}{2} \phi_{r_1+1}(\varep_1) & \dots & \frac{R_1}{2} \phi_{r_1+1}(\varep_{r_1}) \\
        \vdots & \vdots & \ddots & \vdots \\
        \frac{R_{r_2}}{nc} & \frac{R_s}{2} \phi_{r_1+r_2}(\varep_1) & \dots & \frac{R_s}{2} \phi_{r_1+r_2}(\varep_{r_1}) 
   	\end{pmatrix} $$
	
which by extraction,
	
$$=\frac{x_1\cdots x_{r_1} R_1 \cdots R_{r_2}}{2^{r_2} nc}\begin{pmatrix}
        1 & \phi_1(\varep_1) & \dots & \phi_1(\varep_{r_1}) \\
        \vdots & \vdots & \ddots & \vdots \\
        1 & \phi_{r_1}(\varep_1) & \dots & \phi_{r_1}(\varep_{r_1}) \\ \\
        2 &  \phi_{r_1+1}(\varep_1) & \dots & \phi_{r_1+1}(\varep_{r_1}) \\
        \vdots & \vdots & \ddots & \vdots \\
        2 & \phi_{r_1+r_2}(\varep_1) & \dots &  \phi_{r_1+r_2}(\varep_{r_1}) 
   	\end{pmatrix}$$
	
and by matrix rearrangement,

$$=\frac{x_1\cdots x_{r_1} R_1 \cdots R_{r_2}}{2^{r_2} nc}\begin{pmatrix}
        n & 0 & \dots & 0 \\
        1 & \phi_2(\varep_1) & \dots & \phi_2(\varep_{r_1}) \\
        \vdots & \vdots & \ddots & \vdots \\
        1 & \phi_{r_1}(\varep_1) & \dots & \phi_{r_1}(\varep_{r_1}) \\ \\
        2 &  \phi_{r_1+1}(\varep_1) & \dots & \phi_{r_1+1}(\varep_{r_1}) \\
        \vdots & \vdots & \ddots & \vdots \\
        2 & \phi_{r_1+r_2}(\varep_1) & \dots &  \phi_{r_1+r_2}(\varep_{r_1}) 
   	\end{pmatrix}$$
	
The determinant of this matrix is exactly $nR_K$, where $R_K$ is the regulator. It thus follows, because $c = x_1\dots x_{r_1}(R_1\dots R_{r_2})^2$, that $$|J|=\frac{R_K}{2^{r_2} R_1 \cdots R_{r_2}}.$$

With this result we can compute $\text{vol}(\bar{\mcF)}$.

\begin{align*}
	\text{vol}(\bar{\mcF)}&= 2^{r_2} \int \dots \int_{\mcF} \ dx_1\dots dx_r dy_{r_1+1}dz_{r_1+1}\dots dy_{r_1+r_2} dz_{r_1+r_2} \\
	&= 2^{r_2} \int \dots \int_{\mcF} R_1\cdot \dots\cdot R_{r_2}\ dx_1\dots dx_{r_1} dR_1\dots dR_{r_2} d\phi_1\dots d\phi_{r_2} \\
	&= 2^{r_2} (2\pi)^{r_2} \int_{0}^1 \dots \int_{0}^1 R_1\cdot \dots\cdot R_{r_2}\ dx_1\dots dx_{r_1}dR_1\dots dR_{r_2} \\
	&= 2^{r_2} (2\pi)^{r_2} \int_{0}^1 \dots \int_{0}^1 |J| R_1\cdot \dots\cdot R_{r_2}\ dc dc_1 \dots dc_{r_1} \\
	&= 2^{r_2} (2\pi)^{r_2} R_K
\end{align*}

and thus \begin{align*}
	\text{vol}(\mcF) &= \frac{2^{r_1}}{\omega} \text{vol}(\bar{\mcF}) \\
	&= \frac{2^{r_1+r_2}\pi^{r_2}R_K}{\omega},
\end{align*}

as desired. $\qed$
	
Finishing off the proof, we have that 

\begin{align*} 
	\lim\limits_{s\to 1} (s-1) f_C(s) &= N(\mfb)\frac{v}{\Delta} \\
	&= N(\mfb) \left(\frac{2^{r_1+r_2}\pi^{r_2} R_K}{\omega}\right) \left(N(\mfb)\sqrt{|D_K|}\right)^{-1}
\end{align*}

so $$\lim\limits_{s=1}(s-1)f_C(s) = \frac{2^{r_1+r_2}\pi^{r_2} R_K}{\omega \sqrt{|D_K|}}$$ and thus by summing over the ideal classes,

$$\lim\limits_{s=1}(s-1)\zeta_K(s) = \frac{2^{r_1+r_2}\pi^{r_2} R_K}{\omega \sqrt{|D_K|}} h_k$$ which is the class number formula. $\qed$

\thrm{Hecke's analytic continuation of the Dedekind zeta function}{
	Hecke showed that $\zeta_K(z)$ extends to a meromorphic function on $\bbC$ with no poles except for the simple pole at $s=1$.
	
	Define the \emph{gamma factors} $$\Gamma_{\bbR}(z) = \pi^{-z/2} \Gamma(\frac{z}{2}), \Gamma_{\bbC}(z)=(2\pi)^{-z} \Gamma(z)$$ and the \emph{completed zeta function} $$\xi_K(z) = |D_K|^{z/2} \Gamma_{\bbR}(z)^r \Gamma_{\bbC}(z)^s \zeta_K(z).$$ Then, $\xi_K(z)$ is holomorphic except for simple poles at $s=0, 1$ and that $$\xi_K(z)=\xi_K(1-z).$$
}

\ex{}{
	For $K=\bbQ$, we have that \begin{align*}
	\xi_\bbQ(z)&= \Gamma_{\bbR}(z) \zeta(z) \\
	&= \pi^{z/2} \Gamma\left(\frac{x}{2}\right)\zeta(z)
	\end{align*}
}

\chapter{Local Fields}
\minitoc

Before we introduce what a local field is, we have to introduce valuations and absolute values. The study of local fields is closely linked to that of the $p$-adic numbers, for $p$ prime. It is used to motivate many concepts in this theory.

But why do we study $p$-adic numbers? Firstly, they are a paradigm example of local fields. Secondly, they encompass many beautiful properties that extend to Galois theory and class field theory. The properties of p-adic numbers can be generalised widely. 

For now, I'll not introduce what a local field is. I'll instead present the classification of local fields. All local fields fall into one of the following categories.

\defn{Classification of local fields}{
	A local field is either $\bbR$ or $\bbC$, a finite extension of $\bbQ_p$, or the set of formal Laurent series over a finite field, denoted $\bbF_p[[X]]$.
}
 
 All global fields fall into the following categories:
 
 \defn{Classification of global fields}{
 	A global field is either a finite extension of $\bbQ$ (an algebraic number field) or a finite extension of $\bbF_q(X)$ - the function field of an algebraic variety over a finite field. 
 }
 
\section{Absolute values}

From undergraduate mathematics, you learned that real numbers could be constructed as a Cauchy sequence of rational numbers. On the real number line, the notion of distance is clear. It is also clear in $\bbR^2$. 

Let $x_1$ and $x_2 \in \bbR$ . Their distance is $|x_1-x_2|$. Moreover, the absolute value of $x_1$ is the distance between $0$ and $x_1$. 

How do we generalise the notion of distance to arbitrary fields? We introduce the notion of a generalised absolute value.

\defn{Absolute value}{
	Let $K$ be a field. An absolute value on $K$ is a homomorphism $|\cdot|:K\to\bbR$ such that for all $x, y\in K$:
	
	\begin{enumerate}
		\item $|x| \geq 0$, with equality attained in the trivial case where $x=0$.
		\item $|xy| \leq |x|\cdot |y|$
		\item (Triangle inequality) $|x+y| \leq |x|+|y|$.
	\end{enumerate}
	
	This absolute value is a non-Archimedean absolute value if it also satisfies the property $$|x+y| \leq \max\{|x|, |y|\}$$ This is known as the \emph{strong triangle inequality} or the \emph{ultrametric inequality}. 
	
	From this property, it swiftly follows that if $|x|<|y|$, then $|x\pm y|=|y|$. This is because $$|y| = |x+y-x| \leq \max\{|x+y|, |-x|\}=|x+y|. \qed$$ We follow similar logic to prove that $|y|=|x-y|$.
	
	An absolute value that does not satisfy the above property is an Archimedean absolute value. An alternate property of archimedean absolute values are that for all $n>1, n\in\bbN$, we have that $|n|>1$. This is reiterated in the definition on places.
}

\ex{}{
	Examples of absolute values include:
	\begin{enumerate}
		\item The standard absolute value on $\bbR$ and $\bbZ$.
		\item The trivial absolute value - i.e. $|x|=\begin{cases}  1 & x\neq 0 \\ 0 & x=0\end{cases}$. The trivial absolute value is the 	only absolute value that can exist on a finite field. This is because one can raise an arbitrary element to any power to get $1$.
		\item The $p$-adic absolute value on $\bbQ$.
	\end{enumerate}
}

The absolute value induces a metric, and thus a topology. This is achieved by $d(x_1, x_2)=|x_1-x_2|$. 

\prop{}{
	Let $|\cdot|_1$ and $|\cdot|_2$ be two absolute values. If one can represent $|\alpha|_1=(|\alpha|_2)^c$ for some positive real number $c>0$. Then, these two absolute values are equivalent.
}

\tbf{Proof:} We can write $|\alpha-a|_1 <r$ for some $a$. This naturally holds if and only if $|\alpha-a|_2^c<r$, or $|\alpha-a|_2<r^{1/c}.$

Thus, any open ball with respect to $|\cdot|_1$ is an open ball with respect to $|\cdot|_2$. Thus, the topologies induced by the two respective absolute values are identical. $\qed$

With the notion of absolute value defined, we can now define a local field.

\defn{Local field}{
	A local field is a field $K$ with an absolute value $|\cdot |$ such that the field is locally compact under the topology induced by $|\cdot|$.
}

\defn{Place}{
	A place is an equivalence class of absolute values. There are two types of places - archimedean and non-archimedean places. 
	
	\begin{enumerate}
		\item An absolute value $|\cdot|:K\to\bbR^+$ is an archimedean place satisfies the property that for all $n>1, n\in\bbN$, we have that $|n|>1$. In fact, we have a bijection between the archimedean places and the embeddings of a number field $K$ into $\bbC$, up to conjugation.
		\item A non-archimedean place is an absolute value $|\cdot|:K\to\bbR^+$ with the property that there exists $n>1, n\in \bbN$ such that $|n|<1$. 
	\end{enumerate}
}

Note that the archimedean places are also known as \emph{places at infinity}. By the bijection claimed above, we say that $|\cdot|$ is a real place if the embedding is real, and $|\cdot|$ is complex if the embedding is a pair of complex conjugates.

\coro{Triangle + non-archimedean implies strong triangle}{
	Let $|\cdot|$ be a non-archimedean absolute value that satisfies the triangle inequality. Then, $$|\alpha+\beta|\leq \max\{|\alpha|, |\beta|\},$$ or the ultrametric inequality is achieved.
}

\tbf{Proof:} Let $k>0$. Thus, 

\begin{align*}
	|\alpha+\beta|^k &= |(\alpha+\beta)^k| \\
	&= \left|\sum\limits_{n=0}^k \binom{k}{n} \alpha^n \beta^{k-n}\right| \\ 
	&\leq \sum\limits_{n=0}^k \left|\binom{k}{n}\right| |\alpha|^n |\beta|^{k-n}
\end{align*}

But can we bound $|\binom{k}{n}|$? Turns out that we can. We invoke the following lemma.

\lemma{}{
	Let $|\cdot|$ be a non-archimedean absolute value on $\bbQ$. We have that $|n| \leq 1$, for all $n\leq \bbZ$.
}

\tbf{Proof:} By way of contradiction, assume that there exists such an $n\in\bbZ$ such that $|n|>1$. Then, we can define $N=n^k$. We then have $|m|<1$ such that $$|N|=|n|^k > \frac{m}{1-|m|}.$$

Write $N$ in base $m$ - i.e. $$N=\sum\limits_{k=0}^r a_k m^k.$$ Then, since $|a_i|= |1+\dots+1| \leq a_i \cdot |1| < m$, we have that 

\begin{align*}
	|N| &\leq |a_0|+|a_1||m|+\dots+|a_r||m|^r \\
	&< n(1+|m|+\dots+|m|^r).
\end{align*}

But then note that since $|m|<1$, we can apply the geometric series formula to attain $$|M|<\frac{n}{1-|n|},$$ which is a contradiction. $\qed$

Returning to the corollary, note that since $|\binom{k}{n}|\leq 1$, it follows that \begin{align*}|\alpha+\beta|^k&=\sum\limits_{n=0}^k \left|\binom{k}{n}\right| |\alpha|^n |\beta|^{k-n}\\ &\leq (k+1) \max\{|\alpha|, |\beta|\}^k,\end{align*} and thus $$|\alpha+\beta| \leq \sqrt[k]{(k+1)} \max\{|\alpha|, |\beta|\}.$$ As $k\to\infty$, we get our desired result. $\qed$

A valued number field is a tuple $(K, |\cdot|)$ consisting of a number field and an absolute value defined on it. Before we move to $p$-adic numbers, we introduce the completion of a valued field $K$, denoted $\hat{K}$. 

\defn{Completion}{
	Let $K$ be a number field. Its completion $\hat{K}$ is the set of Cauchy sequences for the absolute value $|\cdot|$, modulo equivalence classes.
}

It is to be noted that $\hat{K}$ is also a valued field. For Cauchy sequences $(a_n)$ and $(b_n)$, we can define addition and multiplication in the usual sense to make $(a_n+b_n)$ and $(a_nb_n)$, thus endowing $\hat{K}$ with a ring structure.

We are left to show that $\hat{K}$ is a field. Let $a_n \to a$. There exists a sufficiently large $n$ - say $n\geq N$, such that $$a_n \geq C$$ for a large $C>0$. Then consider the sequence given by $(\frac{1}{a_n})$. We know that it is Cauchy too, for that for $n, m>0$, $$\left|\frac{1}{a_n}-\frac{1}{a_m}\right|=\left|\frac{a_m-a_n}{a_ma_n}\right|\leq C^{-2} |a_m-a_n|.$$ Since $C^{-2} |a_m-a_n| \to 0$ as $n\to \infty$, it can be concluded that $(\frac{1}{a_n})\to(\frac{1}{a})$, which is in $\hat{K}$.

This suggests that $a \in \hat{K^\times}$. We extend $|\cdot|$ to $K$ by considering $|a|=\lim\limits_{n\to\infty} |a_n|. \qed$

\section{p-adic numbers}

We now introduce the notion of a $p$-adic integer, or a $p$-adic number. We motivate our study by considering that all integers can be represented in base 10. All integers can also be represented in base $p$, for $p$ a prime.

To motivate the study of $p$-adic numbers, consider the Taylor expansion of a polynomial. Let $T(x)$ be the Taylor expansion of a polynomial, and $\alpha\in\bbC$ be where it is centered at. We write the polynomial in the form $$T(x)=\sum\limits_{i=0}^n a_i(x-\alpha)^i,$$ for $a_i \in \bbC$. The expansion in powers of $(x-\alpha)$ show to what order $T(x)$ vanishes at $\alpha$.

We can then draw parallels with the integers - we can represent all positive integers in base $p$. Let $m$ be a positive integer. We can write $$m=\sum\limits_{i=0}^n a_ip^i,$$ for $a_i\in\bbZ$. The expansion will show if $m$ is divisible by $p$; if yes, which power of $p$.

The Laurent series $$L(x)=\sum\limits_{i=k}^\infty a_i(x-\alpha)^i$$ expands a function into a series represented in terms of the primes. The power series field $\bbC((X-\alpha))$ consists of all Laurent series around the point $x-\alpha$. 

If we can exhibit a mapping between $\bbC(X)$ and $\bbC((X-\alpha))$, can we exhibit a similar mapping between $\bbZ$ and something else? It turns out that we can, and we do this using the $p$-adic numbers. (n.b. our choice of $\alpha$ in the Laurent series is analogous to choosing $p$ in our $p$-adic expansion.)

We can represent rational numbers $\frac{a}{b}$ as $\sum_{i=k}^{\infty} a_ip^i$. For every rational number, we associate with it a Laurent series in terms of $p$ and its powers. This is known as a $p$-adic expansion of a rational number. 

\defn{$p$-adic integers}{
	A $p$-adic integer is a formal series $$\alpha=a_0+a_1p+a_2p^2+\dots = \sum\limits_{n=0}^{\infty} a_n p^n.$$ They form a ring. We denote the ring of $p$-adic integers by $\bbZ_p$. (Author's note: Some texts denote the integers modulo p as $\bbZ_p$ instead of $\bbZ/p\bbZ$. This does not only lead to misleading notation, but is also sacrilegious.)
}

\prop{}{There is a natural mapping $\bbZ_p \to \bbZ/p^k\bbZ$. 
}

But what does this mean? 

Consider the power-series representation of a $p$-adic integer. We can "cut" the power-series at $p^k$. Define $$\alpha_k=a_0+a_1p+\dots+a_{k-1}p^{k-1}$$ Taking the limit as $p\to\infty$, we get the desired mapping.

\begin{itemize}
	\item Let $k=1$. Then, $\alpha_1=a_0. $
	\item Let $k=2$. Then, $\alpha_2=a_0+a_1p=\alpha_1+a_1p$. 
	\item Let $k=3$. Then, $\alpha_3=a_0+a_1p+a_2p^2=\alpha_2+a_2p^2$. Since $p$ is fixed, we need to define $a_2$ sensibly, and continue recursively. 
\end{itemize}

We conclude that by this means, a unique sequence of $\alpha_k$ defines a unique $p$-adic integer $\alpha \in \bbZ_p$. This "natural mapping" comes in the form of an inverse limit of rings: $$\bbZ_p = \lim\limits_\leftarrow\bbZ/p\bbZ$$

With any inverse limit, there is a natural projection mapping. In this case, for $i \leq j$, we have that $$\pi_{ij}: \bbZ/p^j\bbZ \to \bbZ/p^i \bbZ$$ through the natural projection map $$\pi_{ij}:  \Mod p^j \to a \Mod p^i.$$

The inverse limit $\lim\limits_\leftarrow\bbZ/p\bbZ$ is thus given by all $$\{x_i\} \in \prod\limits_i \bbZ/p^i\bbZ$$ that satisfies the projection mapping $\pi_{ij}:x_j \to x_i$, where $i\leq j$.

\defn{$p$-adic numbers}{
	The $p$-adic numbers are of the form $$a_{-n}\frac{1}{p^n}+a_{-n+1}\frac{1}{p^{n-1}}+\dots+a_{-1}\frac{1}{p}+a_0+a_1p+\dots$$ where all the $a_i \in \{0, 1, \dots, p\}$. 
	
	There is an infinite series of numbers that is to the left of $a_{-n}$. We truncate it using dots. An example of a $p$-adic number would be $\dots6666=-1$ in $\bbQ_7$. (Why?)
}

The set of all such numbers is denoted by $\bbQ_p$, and is a field. Note that the $p$-adic integers are not a field since any $p$-adic integer that ends in 0 is not invertible. 
	
	To solve this, we transform any $p$-adic integer $\frac{a}{b}$ into $\frac{a}{p^k \cdot u}$. Take $a_1=\frac{a}{p^k}$. By transforming the number $\frac{a}{b}$ into $\frac{a_1}{u}$, we construct the $p$-adic numbers, which are naturally a field.

Naturally, the rational numbers are the $p$-adic numbers that are eventually periodic. From this, we can define a $p$-adic absolute value. 

\ex{Applications of $p$-adic numbers}{
	\begin{enumerate}
		\item In $\bbQ_5$, $\frac{1}{10}=\dots222.3$.	
		\item Direct calculation shows that $-1=(p-1)+(p-1)p+(p-1)p^2+p^{n-1} \Mod {p^n}$.
		\item For any square in $\bbZ_3$, the last non-zero digit must be 1. This is done by considering $(\dots1)^2 = \dots1$, and $(\dots2)^2=\dots11.$
		\item $e=1+\frac{1}{2}+\frac{1}{6}+\frac{1}{24}+\dots$. Let's try to evaluate it in $\bbZ_2$. Note that $\frac{1}{2}=0.1.$ To evaluate $\frac{1}{6}$, consider that $\frac{1}{6}=\frac{1}{x^2+x}=\frac{1}{x(1+x)}$ evaluated at $x=2$, which is equivalent to $\frac{1}{x}\cdot \sum\limits_{n=0}^{\infty}(-x)^n$; it then follows that $\frac{1}{6}=\dots01.1.$ 
		
		In fact, $e$ does not converge in $\bbZ_2$. The function $e^x$ exists in the $\bbZ_p$ if and only if $p \mid x$, and $p\neq 2$. Its radius of convergence is on the annulus $|x|_p<p^{\frac{-1}{p-1}}$.
		\item The gamma function is a generalisation of the factorial function to $\bbR$. It is defined as $\Gamma(n)=(n-1)!$ for integers, and $$\Gamma(x)=\int\limits_0^\infty e^{-t}\ t^{x-1}\ dt.$$ The $p$-adic gamma function, denoted $\Gamma_p(x)$, equals $$(-1)^x \prod\limits_{0<i<x, p \nmid i} i$$ for positive integers $x$. This function defines a mapping $\Gamma_p:\bbZ_p\to\bbZ_p^\times$, where $\bbZ_p^\times$ is the set of invertible $p$-adics integers. 
		
		Formally, we have that $\frac{\Gamma_p(x+1)}{\Gamma_p(x)}=\begin{cases} -x & \text{if}\ x \in \bbZ_p^\times \\ -1 & \text{if}\ x \in p\bbZ_p\end{cases}$. This is also known as \emph{Morita's $p$-adic gamma function}. 
	\end{enumerate}
}

\defn{$p$-adic valuation}{
	Define the $p$-adic valuation $|x|_p = \begin{cases} p^{-e} & \text{if}\ x = p^eu\ (\neq 0) \\  0 & \text{if}\ x=0\end{cases}$
	
	where $u$ is a unit. The $p$-adic valuation is also known as the $p$-adic absolute value, or norm.
	
	Do note that like how $\bbR$ is the completion of $\bbQ$ for the norm $|\cdot|_\infty$, $\bbQ_p$ are respectively the completions of $\bbQ$ for $|\cdot|_p$. 
}

This $p$-adic valuation does not measure the size of $x$. Rather, $|x|_p$ becomes small if it is divisible by a high power of $p$. We thus conclude that with respect to $p$, the sequence $\{a_np^n\}_{n\to\infty}$ converges to $0$.

\ex{$p$-adic valuation}{
	\begin{enumerate}
		\item Consider the rational number $x=\frac{15}{7}$. Then, $|x|_2=1,  |x|_3=\frac{1}{3}, |x|_5=\frac{1}{5}, |x|_7=7$, and $|x|_p=1$ for all $p>7.$
		\item The $p$-adic valuation satisfies the \emph{product formula}, which states that when taken over all primes $p$ and the usual absolute value (denoted $|\cdot|_{\infty}$), the product $$\prod\limits_{p, \infty}|x|_p=1.$$ 
	\end{enumerate}
}

There is a way of computing $p$-adic expansions. We use the fact that $\frac{1}{1-p}=1+p+p^2+\dots$. 

\prop{Computing $p$-adic numbers}{
	Let $\frac{a}{b} \in \bbQ.$ Then, we can write $\frac{a}{b}$ as $$\frac{a}{b}=x+y\cdot\frac{1}{1-p^r},$$ for $r\geq 0, 0\leq b \leq p^{r-1}-1$, and $a\in\bbZ$.
	
	The above holds when $p \nmid b$. Let $\frac{A}{B} \in \bbQ$ such that $p\mid b$. In this case, we can factor out $p$ and write $\frac{A}{B}=p^r\frac{a}{b}$, where $p \nmid a, b$, and continue.
}

We can illustrate this with an example. Let's compute $\frac{1}{3}$ in $\bbQ_5$. We know that there exists a $|\cdot|_5$-Cauchy sequence $(a_n)\in\bbZ$ such that $\frac{1}{3}=\lim\limits_{n\to\infty}(a_n)$.

We know that $\frac{1}{3}=1-\frac{2}{3}$. Using the fact that $1-5^2=-24=-8\cdot 3$, we have that $$\frac{1}{3}=1-\frac{2}{3}=1+\frac{16}{-24}=1+16\cdot\frac{1}{1-5^2}.$$ We then expand the infinite series: 

\begin{align*}
	1+16\cdot\frac{1}{1-5^2} &= 1+16(1+5^2+5^4+5^6+\dots)\\
	&= 1+(1+3\cdot5)(1+5^2+5^4+5^6+\dots)\\
	&= 1+(1+3\cdot 5+5^2+3\cdot 5^3+5^4+3\cdot 5^5+5^6+\dots),
\end{align*}

and thus we conclude that $(a_n)$ = $\{2, 3, 5, 3, 5, \dots\}$, or $\begin{cases} 2 &\text{if}\ n=1 \\ 3 &\text{if}\ 2\mid n \\  5 &\text{if}\ 2\nmid n\ \text{and}\ n\neq1  \end{cases}$

We return to our proof. Define $r$ such that $p^r \equiv 1 \Mod n$. This is possible because $(n, p)=1$. Thus $p^r-1 = kn$ for some $k\in\bbZ$. Then, we can write $$\frac{m}{n}=\frac{km}{kn}=\frac{km}{p^r-1}=\frac{-km}{1-p^r}$$ 

But then, we can write $-km=-a(p^r-1)+b$ for integers $a, b$ (such that $0\leq b \leq p^{r-1}-1$, and $a\in\bbZ$). It then follows that \begin{align*} \frac{m}{n}=\frac{-a(p^r-1)+b}{1-p^r}&=a+b\cdot\frac{1}{1-p^r} \qed \\ &(= a+b+bp^r+bp^{2r}+\dots ) \end{align*}

Write $b=b_0+b_1p+\dots+b_{r-1}p^{r-1}$. As seen in the example above, the $p$-adic expansion has periodicity $r-1$ after the initial digit. Thus, the $p$-adic expansion has coefficients $$b_0, b_1, \dots, b_{r-1}, b_0, b_1, \dots, b_{r-1}, \dots.$$

When studying the $p$-adic absolute value, we learned that a nice property of working in $\bbQ$ is that it admits unique factorisation. In general number fields, this is not always the case. They have the property of unique ideal factorisation. 

Let $K$ be a number field. $(p)\in K$ factorises as $p\mcO_K=\prod\limits_{i} \mfp_i^{e_i}$, where $i=e(\mfp/p)$ is the ramification index $\mfp$ over $p$.

\defn{$\mfp$-adic valuation in a general number field $K$}{
	Let $\mfp \subset \mcO_K$ be a prime ideal lying above a rational prime $p$. We can factorise $\alpha\mcO_K$, where $\alpha \in K^{\times}$, as $$\alpha \mcO_K = \mfp^r \cdot \mfq_1^{r_1}\cdot\dots\cdot \mfq_n^{r_n}.$$ 
	
	Let $v_{\mfp}(\alpha)=r$. We define the $\mfp$-adic absolute value to be the function
	\begin{align*} |\cdot|_{\mfp}: &K^\times \to \bbR^+ \\ &\alpha \mapsto p^{-\frac{v_\mfp(\alpha)}{e(\mfp/p)}}.\end{align*} 
}

% after p-adic

Ostroski's theorem classifies the absolute values on $\bbQ$.

\thrm{Ostrowski's theorem}{
	Let $|\cdot|$ be an absolute value on $\bbQ$. It is either equivalent to the real absolute value $|\cdot|$, or a $p$-adic absolute value $|\cdot|_p$.
	
	It is trivial to show that $|\cdot|_\infty$ is archimedean, and $|\cdot|_p$ is non-archimedean. 
}

\tbf{Proof:} We consider two cases - when $|\cdot|$ is archimedean and when it is not.

\tbf{Case 1:} Firstly, assume that $|\cdot|$ is archimedean. Then, we want to show that there exists an equivalence of absolute values $|\cdot| \sim |\cdot|_\infty$. This means that there exists a constant $c>0$ such that $\forall a\in\bbQ$, $|a| = |a|_\infty^c$.

By writing $|a|=\frac{|b|}{|c|}$, we can thus conclude that the equivalence also holds if there exists $c>0$ such that $|a|^c_\infty = a^c$ for all $a\in\bbZ, a\geq1$. Taking logs, we have that $c$ satisfies $$c = \frac{\log|a|}{\log a};$$ we thus conclude that for all $a, b \in \bbZ_{\geq1}$, then $$\frac{\log|a|}{\log a}=\frac{\log|b|}{\log b}.$$

Let $m, n>1$ be integers. We can write $$m=a_0+a_1n+\dots+a_rn^r.$$ We then have that $r \leq \frac{\log(m)}{\log(n)}$ by taking logs on both sides of the above equation.

By the triangle inequality, $|m| \leq \sum|a_i||n|^i \leq \sum |a_i|N^r$, where $N=\max\{1, |n|.\}$ From $$|a_i| \leq |1+\dots+1| = a_i \leq n,$$ we have that $$|m| \leq (1+r)nN^r \leq \left(1+\frac{\log m}{\log n}\right)  nN^{\frac{\log m}{\log n}}.$$

We can then take map $m \mapsto m^k$, and take $k$'th roots on both sides to attain $$|m| \leq (1+\frac{k \log m}{\log n}^{\frac{1}{k}} n^{\frac{1}{k}} N^{\frac{\log m}{\log n}};$$ and thus in the limit $t\to\infty$, $$|m| \leq N^{\frac{\log m}{\log n}}.$$

We proceed by taking logs to conclude that $$\frac{\log|m|}{\log m}\leq\frac{\log N}{\log n}.$$ We have that $|N|>1$ since $n>m$. Since we take $m$ to be arbitrary, we note that $|m|>1$ for all $m \in \bbZ_{\geq 1}$, and we can make the same conclusion for $n$.

This arbitrary choice of $m$ and $n$ allows us to swap them, and proceed with the line of reasoning as above. Thus, we can write that $$\frac{\log|m|}{\log m}\geq\frac{\log N}{\log n}.$$

It only remains that $$\frac{\log|m|}{\log m}=\frac{\log N}{\log n}. \qed$$

\tbf{Case 2:} The non-Archimedean case is much easier. Consider $R$ to be the local ring associated with the field, and let $\mfm$ be its maximum ideal. Since maximum ideals are prime, we can write $\mfm \cap \bbZ = (p)$ for some $p$. 

Thus, $|m|=1$ if $m \nmid p$, and thus $|np^r|=|p^r|$ if the numerator and denominator of $n$ is not divisible by $p$. It then follows that if $|p|=(\frac{1}{p})^a$, and thus we get our desired result $|\cdot|=|\cdot|_p^a$, $\qed$.

\section{Valuations, Hensel's lemma, Witt vectors}

\defn{Valuation}{
	A valuation on a field $K$ is a surjective homomorphism $$v:K^* \to \bbR$$ that satisfies the following properties: 
	
	\begin{enumerate}
		\item $v(0)=\infty$
       		 \item $v(x+y) \geq \inf\{v(x), v(y)\}$
        		\item $v(xy) \geq v(x)+v(y)$
    	\end{enumerate}
}

\defn{Discrete valuations}{
	Use notation as defined above. We call $v$ a \emph{discrete valuation} if the image of $v$ in $\bbR$ is equal to $\bbZ$. Thus, it defines a non-zero homomorphism $$v:K^* \to \bbZ.$$
	
	Formally, we call the image of a valuation $v$ the \emph{value group}; it is notated $v(K^*) \subset R.$
	
	(n.b.) Some authors choose to differentiate between the notion of a \emph{normalised discrete valuation} and a \emph{discrete valuation}. A discrete valuation, in this case, is when the group $v(K^*)$ is discrete, and a normalised discrete valuation is when $v(K^*)=\bbZ.$ We don't use this definition. 
}

\ex{Valuation on polynomials}{
	For an irreducible polynomial in $k[X]$, we can write $f(x) \in k(X)$ in the form $$f(x)=x^r \cdot \frac{g(x)}	{h(x)},$$ where $g, h\in k[X]$, and $g(0)\neq h(0) \neq 0$. 
	
	We can also represent any $f(x)$ by writing $$f(x)=P(x)^r \cdot \frac{g(x)}{h(x)},$$ where $g, h \in k[X] \backslash P(X)k[X]$. The homomorphism $$v_P:k(X)^\times \to \bbZ,$$ given by the mapping $$P(X)^r \cdot \frac{g(x)}	{h(x)} \mapsto r,$$ is a valuation.
	
	This valuation measures the index of a polynomial in a rational function. Formally, for a polynomial $F(x)=a_0+a_1x+\dots+a_nx^n$, we can write $$v_x(F)=\min\{i:a_i \neq 0\}.$$ However, we can also evaluate the valuations for other functions.
	
	Considering the function $F(x)=\frac{x^3(x-2)}{(x-1)(x^2+5)}$, the following hold:
	
	\begin{itemize}
		\item $v_x(F)=3$
		\item $v_{x-2}(F)=1$
		\item $v_{x-1}(F)=-1$
		\item $v_{x^2+5}(F)=-1$
	\end{itemize}
	
	and for all other irreducibles $P(x)$, $v_P(F)=0$. 
}

Why were the terms $p$-adic absolute value and $p$-adic valuation interchangeable? Put simply, a valuation naturally extends itself to an absolute value. Let $v$ be a valuation on a field $K$, and let $\alpha\in(0, 1)$. Define the function $$|\cdot|:K\to\bbR_{\geq 0}$$ through the mapping $$x\mapsto \alpha^{v(x)}.$$ This function is an absolute value on $K$. 

Similarly, we can define the valuation $$v:K^\times \to \bbR$$ through the mapping $$x\mapsto \log_{\alpha}|x|,$$ where $|\cdot|$ is a non-archimedean absolute value on $K$.

We can perform a \emph{completion} of $K$ with respect to its absolute values in a way similar to how a Dedekind domain $A$ can be localised. Let $[(x_n)]$ be an equivalence class of Cauchy sequences with class representative $(x_n)$. The completion of $X$ in the metric space $\hat{X}$ is the metric space whose elements are equivalence classes of Cauchy sequences with the metric $$d([(x_n)], [(y_n)])=\lim\limits_{n\to\infty} d(x_n, y_n).$$

\defn{Completion of a field}{
	Let $|\cdot |$ arise from a discrete valuation $v$ on $K$, meaning that $|x| = \alpha^{v(x)}$ for some $\alpha \in (0,1)$. 
	
	The completion of $K$ with respect to $|\cdot|$ is an absolute-valued field $(L, |\cdot|_L)$ that is complete as a metric space with an embedding $\phi: K\to L$ such that $$|x| = |\phi(x)|_L$$ for $x\in K$.
}

There is a universal property associated with the completion of a field. Let $\hat{K}$ be the completion of $K$ with respect to $|\cdot|$. Then, every embedding of $K$ into $L$ can be uniquely extended to an embedding of $\hat{K}$ into $L$. Also, $\hat{K}$ is unique with this property up to isomorphism. 

\defn{Various terms}{
	Let $v$ be a valuation on a field $K$. The valuation ring is defined as the set $$\mcO=\{x\in K: v(x) \geq 0\}.$$ If we let $K=\bbQ$ and $v=v_p$, then $\mcO=\bbZ_(p)$ - the localisation of $\bbZ$ at the prime ideal $(p)$, which is also equal to $\{\frac{a}{b}:a, b \in \bbZ, p\nmid b \}.$ 
	
	The unit group of $\mcO$, denoted $\mcO^\times$, is the set $$\mcO=\{x\in K: v(x) = 0\}.$$ It corresponds to the group of invertible elements in $\bbZ_(p)$, or $\bbZ_(p)^\times.$
	
	The maximal ideal, denoted $\mfm$, is $p\bbZ_(p)$; the residue field, meanwhile, is $$k=\mcO/\mfm=\bbZ_(p)/p\bbZ_(p)\cong \bbZ/p\bbZ.$$
}

The valuation ring $\mcO$ is an integral domain. This is because $v(xy)=v(x)+v(y) \neq \infty$, and thus $xy\neq 0.$ In this instance, $K$ is the valuation ring's field of fractions. 

\defn{Discrete valuation ring}{
	Let $A$ be an integral domain. It is a discrete valuation ring if it is the valuation ring of its fraction field $K$ with respect to some discrete valuation $v$. For example, $v_p$ on $\bbQ$ or a number field $K$ is a discrete valuation.
	
	We denote $\pi$ to be a uniformiser on $A$ if $v(\pi)=1.$
}

In fact, the existence of the uniformiser shows that $A$ is a principal ideal domain. Fixing $\pi$, we can write every element $x\in K^*$ as $x=u\pi^n$, for $u$ a unit. The non-zero ideals of $A$ correspond exactly to $$(\pi^n)=\{x \in A:v(x)\geq n\}.$$ 

The maximal ideals are $(\pi)$. We call rings with a unique maximum ideal \emph{local rings}. 

With the concept of a uniformiser in mind, we can then extract the valuation from a discrete valuation ring. Let $x\in A$. Then, it follows that $v(x)=n$, where $n$ is the least integer for which $x \in (\pi^n)$. We know that such an $n$ exists because $(\pi^0)=(1)=A.$ Define $v(0)=\infty$, and extend $A$ to the field of fractions of $v$ through $v(\frac{a}{b})=v(a)-v(b)$. 

Then, $K$ admits a discrete valuation $v$, with discrete valuation ring $A=\{x \in A: v(x)\geq 0\}.$

\thrm{Hensel's lemma}{
	Let $k$ be the residue field of $A$, for $A$ a discrete valuation ring. Let $f(x) \in A[X]$ be monic, and write $\bar{f}(x)$ for the image of $f$ in $\mcO[X]$, meaning that $\bar{f}=f \mod \mfm$. In this case, $\mfm$ is the maximal ideal.
	
	We consider the factorisation of $\bar{f}(x)$. If it factors as $$\bar{f}=\bar{g}\bar{h},$$ with both $\bar{g}, \bar{h}$ monic and relatively prime in $k[X]$, then $f$ factors as $$f=gh$$ in $k[X]$, with $\bar{g}=g$ and $\bar{h}=h \Mod \mfm$. Furthermore, we have that $\deg(g)=\deg(\bar{g})$. The process of finding the polynomials $\bar{g}$ and $\bar{h}$ satisfying the required conditions is called \emph{lifting}.
	
	This lemma is one of the most important result in the theory of local fields.
}

\thrm{Hensel's lemma, alternatively stated}{
	We can use derivatives to perform lifting.
	
	Let $f(x)\in A[X]$ be monic, and let $a \in A$. Let $\bar{f} = f \Mod \mfm \in k[X]$, and suppose $\bar{f}$ has a root $\bar{a} \in k$
	
	If the conditions $$f(a)=0\Mod \mfm, f'(a) \not\equiv 0 \Mod \mfm$$ are satisfied, then there exists a unique $b \in A$ such that $f(b)=0, b=a\Mod \pi$. This condition, where $a$ is any lift of $\bar{a}$ to $\mcO$, makes $a$ a \emph{simple root}.
}

Before we begin with the proof of the lemma, we begin with some examples.

\ex{}{
	Let $K=\bbQ_7$ and $\mcO=\bbZ_7$. Then, consider the polynomial $$f(x)=x^2+3.$$ We know that 2 is a simple root of $\bar{f}(x)$ since $f(2)=0 \Mod 7$, and $f'(2) \neq 0 \Mod 7$.  
	
	By Hensel's lemma, it is clear that $f(x)=x^2+3$ has a root in $\bbQ_7$. 
}

\ex{}{
	Which elements in $\bbZ_2$ are squares?
	
	Let $a \in \bbZ_2$, and assume that there exists a $b\in \bbZ_2$ such that $a=b^2$. Write $a=p^nu, b=p^mv$ for $n, m \in \bbZ$ and $u, v$ units ($u, v \in \bbZ_p^*$). 
	
	Since $v$ is a unit, $2 \nmid v$. Let $v=2k+1$ for $k \in \bbZ_2$. We have that $a=b^2=p^{2m}v^2=p^{2m}(2k+1)^2=p^{2m}(4k^2+4k+1)$.
	
	This formulation imposes new restrictions on $a$, namely that $2\mid n$ and $u\equiv 1 \Mod 8$. We aim to prove these conditions. Let $u=8h+1$, where $h \in \bbZ_2.$ 
	
	Note that for any $a \in \bbZ_2$, we have that $f(2a+1)=4a^2+4a-8k$. Thus, it follows that $u$ is a square if and only if there exists such an $a$ such that $g(a)=a^2+a-2k=0$. 
	
	We apply Hensel's lemma to the polynomial $g(a).$ Since $a^2+a\equiv 0 \Mod 2$, we have that $g(m)\equiv 0 \Mod 2$. Additionally, $g(m) \not\equiv 0 \Mod 2)$, suggesting that the conditions for applying Hensel's lemma are met. The lemma states that there is an simple root $a \in \bbZ_2.$ We then conclude that both $u$ and $a=p^nu$ are squares in $\bbZ_2$, and thus the claim is satisfied. $\qed$
}

\ex{Elliptic curves}{	
	Fix an elliptic curve $E:y^2=x^3+ax+b$ over $\bbQ_p$, with $a, b \in \bbZ_p$. Also let $\text{ord}_p(\Delta(E))=0$. There exists a surjective homomorphism $$\phi:E(\bbQ_p) \to E(\bbF_p),$$ where $E(K)$, for $K$ a field, is defined as the set of tuples $(x, y)$ such that $y^2=x^3+ax+b.$ We call this the \emph{reduction} homomorphism.
	
	\tbf{Proof:} Let $(x_0:y_0:1) \in E(\bbF_p)$. Let $x_0^*, y_0^* \in \bbZ_p$ such that $x_0^*=x_0 \Mod p$, and $y_0^*=y_0 \Mod p$. 
	
	Considering the polynomial $F(x, y)=y^2-(x^3+ax+b)$, we note that $$F(x_0^*, y_0^*)\equiv 0 \Mod p.$$ But then $\del{F(x_0^*, y_0^*)}{X}$ and $\del{F(x_0^*, y_0^*)}{Y}$ both $\not\equiv 0 \Mod p$.
	
	Applying Hensel's lemma, we then conclude that there exists a tuple $(x_1, y_1)$ such that $F(x_1, y_1)=0, x_1=x_0^* \Mod p, y_1=y_0^* \Mod p $. The reduction map is explicit: it is given by $\phi:(x_1:y_1:1)\to(x_0:y_0:1)$
}

Before we present the proof, we present a stronger version of Hensel's lemma. It tells us how Hensel's lemma aims to seek approximations to roots of polynomials, just as in Newton's method. 

\thrm{Strong Hensel's lemma}{
	Let $a \in \mcO$ with $|f(a)|<|f('a)|^2$. Then, there exists a unique $b \in \mcO$ such that $$f(b)=0,$$ and $$|b-a|=|\frac{f(a)}{f'(a)}|<|f'(a)|,$$ or $|b-a| < 1$. 
}

\lemma{Strong Hensel implies Hensel}{
	\tbf{Proof:} Suppose that $f(\bar{a})=0$, but $f'(\bar{a}) \neq 0$. Lift $\bar{a}$ to $\mcO$, and denote that by $a$. 
	
	We thus get that $f(a)\equiv 0 \Mod \mfm$, and that $f'(a)\not\equiv 0 \Mod \mfm.$ This then suggests that both $|f(a)|<1$ and $|f'(a)|^2=1$, and thus $|f(a)|<|f'(a)|$. 
	
	Therefore, there exists a $b$ such that $f(b)=0$, and $|b-a|<|f'(a)|=1$, meaning that $$b \equiv a \Mod \mfm.\qed$$
}

We now prove Hensel's lemma with insights from Newton's method. Define the sequence $$a_{n+1}=a_n-\frac{f(a_n)}{f'(a_n)},$$ for all $n\geq1$. Denote $k=|\frac{f(a)}{f'(a)^2}|<1$. 

By inducting on $n$, we show that for all integers $n\geq 1$, that 

\begin{enumerate}
\item Firstly, $|a_n| \in \mcO$ (or rather $|a_n|\leq1$). 
\item Secondly, that $|f'(a_n)|=|f'(a_1)|$. 
\item Thirdly, that $f(a_n) \leq |f'(a_1)|^2 k^{2^{n-1}}$.
\end{enumerate}

Let $f(x) \in \mcO[X]$. Then, for some $g(x, y) \in \mcO[X, Y]$, we have that $$f(x+y)=f(x)+f'(x)y+g(x, y)y^2.$$ This holds because if we write $f(x)=\sum\limits_{i=0}^n a_ix^i$, we have that $$f(x+y)=\sum\limits_{i=0}^n a_i(x+y)^i = a_0+\sum\limits_{i=1}^n (c_i(x^i+ix^{i-1}y)+g_i(x, y)y^2),$$ and thus 
\begin{align*} f(x+y)&=\sum\limits_{i=0}^n a_i x^i+ \sum\limits_{i=1}^n ia_i x^{i-1} y + \sum\limits_{i=1}^n g_i (x, y) y^2 \\ &= f(x)+f'(x)y+g(x, y) y^2 \qed\end{align*} 

Denote the above by the \tbf{first polynomial identity.}

We also require the identity $f(x)-f(y)=(x-y)g(x, y)$; the proof is left as an exercise. Denote this by the \tbf{second polynomial identity.} Assume that the three propositions, as above, hold for $n$. We show that it holds for $n+1$. 

\begin{enumerate}
\item To show that $|a_n| \leq 1$, we show that $|\frac{f(a_n)}{f'(a_n)}\leq 1$. This holds true as $|\frac{f(a_n)}{f'(a_n)}=\frac{f(a_n)}{f'(a_1)}\leq|f'(a_1)|^2 k^{2^{n-1}}$, which $\leq 1. \qed$
\item To prove that $|f'(a_n)|=|f'(a_1)|$ for $n+1$, note that $f(a_n) \leq |f'(a_1)|^2 k^{2^{n-1}}$ implies that $f(a_n) \leq |f'(a_1)|$, since $k<1$, and evidently $|f'(a_1)| \neq 0$. 

From our second polynomial identity, we have that $|f(x)-f(y)| = |x-y||g(x, y)| \leq |x-y|$, and thus $|f'(a_{n+1})-f'(a_n)| \leq |a_{n+1}-a_n|=\left|\frac{f(a_n)}{f'(a_n)}\right|=\left|\frac{f(a_n)}{f'(a_1)}\right|<|f'(a_1)|=|f'(a_n)|$, which gives our desired result that $|f'(a_{n+1})|=|f'(a_n)|=|f'(a_1)|.$
\item To prove that $f(a_n) \leq |f'(a_1)|^2 k^{2^{n-1}}$, we use our first polynomial identity. Let $x=a_n$ and $y=-\frac{f(a_n)}{f'(a_n)}$. Then, $f(a_{n+1})=f(x+y)$.

And thus $f(a_{n+1})=f(a_n)+f'(a_n)\left(-\frac{f_(a_n)}{f'(a_n)}\right)+z\left(\frac{f(a_n)}{f'(a_n)}\right)^2=z\left(\frac{f_(a_n)}{f'(a_n)}\right)^2$, where $z\in\mcO$.

Thus, we can bound $|f(a_{n+1})|$ below by $\left(\frac{f(a_n)}{f'(a_n)}\right)^2.$ We conclude by saying that $\left(\frac{f_(a_n)}{f'(a_n)}\right)^2 =\left(\frac{f_(a_n)}{f'(a_1)}\right)^2 \leq \frac{|f'(a_1)|^4 k^{2^n}}{|f'(a_1)|^2}=|f'(a_1)|^2 t^{2^n}.$

Our induction is now complete. $\qed$
\end{enumerate}

The sequence $\{a_n\}$ is Cauchy in $K$. This is because $$|a_{n+1}-a_n| \leq |f'(a_1)| k^{2^{n-1}},$$ by our propositions above.

Let $a_n$ converge to $b$. Thus, $|b| \leq 1$, or $|b| \in \mcO$. In the limit as $n \to \infty$, we have that $|f'(b)|=|f'(a_1)|=|f'(a)|$, and thus $f(b)=0.$

We now show that $b$ is the only root of $f(x)$ in $\{x \in \mcO: |x-a| < |f'(a)|.\}$ To prove this, let $f(\alpha)=0$. We know that $|\alpha-a|<|f'(a)|$ since that $|b-a|<f'(a)|$. Setting $\alpha=b+k$, we have that, by using our first identity, $$0=f(\alpha)=f(b+k)=f(b)+f'(b)k+zk^2$$ for some $z \in \mcO$. 

If $k\neq 0$, then $0=f(b)+f'(b)k+zk^2=f'(b)h+zk^2=-zk$, so $|f'(\alpha)\leq |\alpha-b|<|f'(b)|$. This is a contradiction as $|f'(b)|=|f'(a)|$; therefore, $\alpha=b. \qed$ 

When one dives deeper in $p$-adic analysis, one invariably comes across the notions of Witt vectors and Teichmüller representatives. They are applications of the Hensellian lifting process. 

For motivational purposes, consider the polynomial $f(x)=x^{p-1}-1$, which has $p-1$ roots in $\bbZ/p\bbZ$, and thus $\bbZ_p$ by Hensel's lemma. In this example, the Teichmüller mapping is the mapping that sends any element of $(\bbZ/p\bbZ)^*$ to the $p-1$'th root of unity in $\bbZ_p^*$ that reduces to the element. 

\defn{Teichmüller representative and character}{
	Let $\chi: \bbZ_p \to \bbZ_p$. A Teichmüller representative is the solution to $$\omega(x)=\lim\limits_{n\to\infty} x^{p^n}$$
	
	A Teichmüller character is a Teichmüller representative with its domain resricted to $\bbF_p$. 
}

In fact, a ring that satisfies the conditions for the Teichmüller lift is known as a strict $p$-ring. More specifically, $A$ is a strict $p$-ring if it is $p$-torsion free, $p$-adically complete, and $A/pA$ is a perfect ring. (A ring of characteristic $p$ is said to be perfect if the Frobenius map $\phi: R\to R$ is an isomorphism.)

In general, for $K$ a non-archimedean local field, $\mcO$ its valuation ring and $\mfm$ its maximal ideal, let $f(x)=x^n-x$. $n$ is the order of the residue field $k$ of $K$. Note that $$f(x)=\prod\limits_k (x-\alpha) \Mod \mfm,$$ meaning that each $\alpha\in k$ is a simple root of $f \Mod \mfm$. 

By Hensel's lemma, the Teichmüller representative associated with $\alpha$, denoted $[\alpha]$, satisfies $$f([\alpha])=0,$$ and that $$[\alpha]\equiv \alpha \Mod {\mfm}, [\alpha]^n=[\alpha]$$ 

The construction for Teichmüller representatives and characters seem rather contrived. They arise naturally out of Witt vectors. Instead of representing a $p$-adic number as $$a=a_0+a_1p+a_2p^2+\dots,$$ we can represent it in terms of a Teichmüller character $\chi:\bbF_p \to \bbZ_p$. 

\defn{Witt vectors}{
	Let $R$ be a commutative ring. A Witt vector over $R$ is a sequence of elements of $R$, denoted $(X_0, X_1, X_2, \dots)$. They are elements of the product ring $A^{\bbN}$.
	
	Witt vectors generalise the $p$-adic numbers. Let $R=\bbF_p$. Any Witt vector over $\bbF_p$ gives rise to a $p$-adic number. 
}

But why do we do this? The process of $p$-adic addition and multiplication is often difficult with conventional power series notation. 

Let $a, b \in \bbZ_p$. When one starts learning $p$-adic theory, one would add these two numbers by considering their components. We start by having $$c_0\equiv a_0+b_0 \Mod p,$$ then $$c_1\equiv a_0+a_1p+b_0+b_1p \Mod {p^2}$$ $$c_2\equiv a_0+a_1p+a_2p^2+b_0+b_1p+b_2p^2 \Mod{p^3},$$ which gets tedious very quickly. 

Denoting $\chi(a_i)$ for the element-wise image of the Teichmüller character, one can identify the infinite $p$-adic sequence $X=(X_0, X_1, X_2, \dots), X_i \in \bbF_p$ with the $p$-adic number $\chi(X_0)+\chi(X_1)p+\chi(X_2)p^2+\dots$. In the motivational material below, we abuse notation by writing $\chi(c_0), \chi(b_0)$ etc. as $c_0, b_0$. 

Let $c$ be the Witt vector satisfying $$\sum\limits_{n=0}^\infty \chi(c_n)p^n= \sum\limits_{n=0}^{\infty}\chi(a_n)p^n+\sum\limits_{n=0}^{\infty}\chi(b_n)p^n.$$

Consider the equivalences $c_0\equiv a_0+b_0 \Mod p$ and $c_1\equiv a_0+a_1p+b_0+b_1p \Mod {p^2}$. By doing some algebraic manipulation, we get that $$c_1 \equiv \frac{a_0+b_0-c_0}{p}+a_1+b_1 \Mod p.$$ Even more shrewd is the observation that $\Mod {p^2}$, we have that by applying the binomial expansion and the fact that $x \equiv y \Mod p \implies x^p \equiv y^p \Mod{p^2}$ , \begin{align*} c_0&=c_0^p \equiv (a_0+b_0)^p \\ &= a_0^p+b_0^p+p(a_0^{p-1}b_0+\binom{p}{2}a_0^{p-2}b_0^2+\dots+a_0b_0^{p-1}) \\ &= a_0+b_0+p(a_0^{p-1}b_0+\dots+a_0b_0^{p-1}) \end{align*}

Indeed after some cancellations one can deduce that the value of $c_1=a_1+b_1-(a_0^{p-1}b_0+\dots+a_0b_0^{p-1}).$ I'll leave the calculations for $c_2$, etc. as an exercise.

Thus, we have that \begin{align*} c_0&\equiv a_0+b_0 \Mod p \\ c_0^p+c_1p &\equiv a_0^p+a_1p+b_0^p+b_1p \Mod{p^2} \\ c_0^{p^2}+c_1^pp+c_2p&\equiv a_0^{p^2}+a_1^pp+a_2p+b_0^{p^2}+b_1^pp+b_2p \Mod {p^3} \end{align*}

This is much more convenient. The polynomials $c_0$, $c_0^p+c_1p$, and $c_0^{p^2}+c_1^pp+c_2p$ are examples of Witt polynomials. 

\defn{Witt polynomials}{
	Let $p$ be a prime number, and let $(X_0, X_1, X_2, \dots)$ be a Witt vector. For $n\geq 0$, define the $n$-th Witt polynomial to be $$W_n=\sum\limits_{i=0}^n p^i x_i^{p^{n-i}} = x_0^{p^n}+px_1^{p^{n-1}}+\dots+p^nx_n.$$
	
	$c_0$, $c_0^p+c_1p$, and $c_0^{p^2}+c_1^pp+c_2p$ correspond to $W_0, W_1$, and $W_2$ respectively.
}

\thrm{Witt vectors form a commutative ring}{
	Let $R$ be a commutative ring. The Witt vectors over $R$ form a commutative ring. We usually denote this by $W(R)$.
}

To prove this, we must formalise addition and multiplication of Witt vectors, such that entry-wise addition and multiplication of Witt vectors are given by universal polynomials. Let $(X_0, X_1, X_2, \dots)$ and $(Y_0, Y_1, Y_2, \dots)$ be two Witt vectors with all the $X_i \in R$. Define the homomorphism $$\phi: (X_0, X_1, X_2, \dots, Y_0, Y_1, Y_2, \dots) \mapsto (X_0^p, X_1^p, X_2^p, \dots, Y_0^p, Y_1^p, Y_2^p, \dots).$$ 

When two Witt vectors are added or multiplied, we represent their respective components in terms of the additive and multiplicative polynomials $\alpha_n$ and $\pi_n$. More specifically, we have that for two Witt vectors $X=(X_0, X_1, X_2, \dots)$ and $Y=(Y_0, Y_1, Y_2, \dots)$, then $$X+Y=(\alpha_0(X_0, Y_0), \alpha_1(X_0, Y_0, X_1, Y_1), \dots)$$ $$X \cdot Y =(\pi_0(X_0, Y_0), \pi_1(X_0, Y_0, X_1, Y_1), \dots)$$

\begin{enumerate}
	\item $(X_0, X_1, X_2, \dots)+(Y_0, Y_1, Y_2, \dots)=(Z_0, Z_1, Z_2, \dots)$, where the $Z_n$ are defined by $$Z_n=\alpha(X_n, Y_n)=X_n+Y_n+\frac{\phi(X^{n-1}+Y^{n-1})-\sum\limits_{i=0}^n p^i Z_0^{p^{n-i}}}{p^n}$$ where as you can see, $\sum\limits_{i=0}^n p^i Z_0^{p^{n-i}}$ is a Witt polynomial.
	\item $(X_0, X_1, X_2, \dots)\cdot(Y_0, Y_1, Y_2, \dots)=(Z_0, Z_1, Z_2, \dots)$, where the $Z_n$ are defined by $$Z_n=\pi(X_n, Y_n)=\phi(X^{n-1})Y_n+\phi(Y^{n-1})X_n+X_nY_np^n+\frac{\phi(X^{n-1}Y^{n-1})-\sum\limits_{i=0}^n p^i Z_0^{p^{n-i}}}{p^n}$$
\end{enumerate}

Then one can simply verify that the ring operations satisfy the properties of a commutative ring. $\qed$

Having defined the notion of a Witt vector, we can redefine the Teichmüller representative. Let $\bar{a}=a \Mod \mfm$. Then the Witt vector $(\bar{a}, 0, 0, \dots) \in W(\bbF_p)$, denoted by $a^{\tau}$, is the Teichmüller representative of $a \Mod \mfm$.

\prop{}{
	Let $[\alpha]$ be the Teichmüller representative of $\alpha$. It naturally yields a multiplicative injection $$[-]:k \xhookrightarrow\ \mcO.$$ 
}

\tbf{Proof:} The injection exists because for $[\alpha]=[\alpha^{'}]$, then $\alpha \equiv [\alpha] \equiv [\alpha^{'}] \equiv \alpha^{'} \Mod\mfm$ implies $\alpha = \alpha^{'}.$

To prove that the map is multiplicative, let $[\alpha]$ and $[\alpha^{'}]$ be Teichmüller representatives. Then, $$([\alpha][\alpha^{'}])^n=[\alpha]^n[\alpha^{'}]^n=[\alpha][\alpha^{'}]$$ and that $$[\alpha][\alpha^{'}]=\alpha \alpha^{'} \Mod\mfm$$ Since the element $[\alpha \alpha^{'}]$ is unique, it then follows that $[\alpha][\alpha^{'}]=[\alpha \alpha^{'}]$, and thus the map is multiplicative. $\qed$

This also implies that $k^\times$ is a multiplicative subgroup of $\mcO^\times$. 

Fix $p$, and let it be invertible in $X$. If $X=(X_0, X_1, X_2, \dots)$ is a Witt vector, define the \emph{shift map}, or \emph{Verschiebung map} $V: W(X) \to W(X)$ by $$V(X)=(0, X_0, X_1, \dots).$$ Then the homomorphism $W_*: W(A) \to A^{\bbN}$ dictates that $V$ sends $(X_0, X_1, X_2, \dots)$ to $(0, pX_0, pX_1, \dots)$. Letting $W_n$ denote the Witt polynomial, we have the short exact sequence $$0 \to W_k(A) \to W_{k+i}(A) \to W_i(A) \to 0.$$

A short exact sequence is a series of objects with morphisms between them $0 \to A \to B \to C \to 0$ such that the image of each homomorphism is the kernel of the next one. 

Dictate the mapping $w:W(A)\to A^{\bbN}$ by $$\{x_n\}_{n\geq 1} \mapsto \{w_n\}_{n\geq 1}$$ as the mapping by \emph{ghost coordinates}, where $w_n=\sum\limits_{d \mid n} dx_d^{n/d}$. 

\defn{Big Witt ring}{
	There is another formulation of the Witt vectors - through ghost components $x^{(n)}$ and the big Witt ring $\bbW_S(A)$ for $S$ a set satisfying $$n \in S\ \text{and}\ d \mid n \implies d \in S.$$ 
	
	Formally, if one has a Witt vector $x=(X_0, X_1, X_2, \dots) \in A^{\bbN}$, then define the big Witt ring as the functor from $\mathsf{Ring} \to \mathsf{Ring}$ such that $$\bbW(A)=A^{\bbN}$$ and there is a natural transformation $w:\bbW_S(A)\to (A\mapsto A^{\bbN})$, where the functor $A\mapsto A^{\bbN}$ is given by the ghost map $\{x_n\}_{n\geq 1} \mapsto \{w_n\}_{n\geq 1}.$
}

The ghost map is a natural transformation of functors from rings to rings. 

\thrm{}{
	There exists a unique ring structure on $\bbW$ such that the ghost map $$w:\bbW_S(A)\to (A\mapsto A^{\bbN})$$ is a natural transformation of functors from $\mathsf{Ring} \to \mathsf{Ring}$.
}

\tbf{Proof:} Let $A$ be the polynomial ring $\bbZ[a_n, b_n]$, where the $a_i$ and $b_i$ belong to the components of their respective Witt vectors. Let $\phi_p:A\to A$ be the ring homomorphism that maps $a_n \mapsto a_n^p$ and $b_n \mapsto b_n^p$ which satisfies the property $\phi_p(f)=f^p \Mod{pA}$. 

Denote $a = \{a_n\}_{n\in S}$ and $b = \{b_n\}_{n\in S}.$ It is not hard to show, from the formulations of the additive and multiplicative polynomials, that $w(a)+w(b)$, $w(a)\cdot w(b)$, and $-w(a) \in A^S.$ 

Let the sequences $\alpha$, $\beta$, and $\gamma$ be defined such that $w(\alpha)=w(a)+w(b)$, $w(\beta)=w(a)\cdot w(b)$, and $w(\gamma)=-w(a)$. Since $A$ is a torsion-free ring, the ghost map is injective. (Why?) Thus, the polynomials are unique.

Let $A'$ be any ring, and let $a'=\{a'_n\}_{n\geq0}$ and $b'=\{b'_n\}_{n\geq0}$ be two sequences. Let $\bbW_S[f(a)]=a'$, and $\bbW_S[f(b)]=b'$. Then, we can define $a'+b'=\bbW_S(\alpha)$, $a'\cdot b'=\bbW_S(\beta)$, and $-a'=\bbW_S(\gamma)$. 

This is our construction of the ring - it simply remains for us to check that the ring axioms hold. Let $A''$ be non-torsion, and let $$A'' \to A'$$ define a surjective homomorphism. Then, $$\bbW_S(A'') \to \bbW_S(A')$$ is also surjective. Since $A''$ satisfies the ring axioms, $A'$ must too. $\qed$

We can then define the \emph{Verschiebung}, or the shift map, on the big Witt vector. Let $$V_n:\bbW_{S/nS}(A) \to \bbW_{S}(A)$$ through $$V_n((x_d | d \in S/nS))_m = \begin{cases} x_d\ & \text{if}\ m=nd. \\ 0 & \ \text{otherwise}.  \end{cases}$$ 

Additionally, define $$V_n^w((x_d | d \in S/nS))_m = \begin{cases} nx_d\ & \text{if}\ m=nd. \\ 0 & \ \text{otherwise}.  \end{cases}$$

The following square commutes: 

\[
\begin{tikzcd}
\bbW_{S}(A) \arrow{r}{w} \arrow[swap]{d}{V_n} & A^{S/n} \arrow{d}{V_n^w} \\
\bbW_{S/nS}(A) \arrow{r}{w} & A^{S/nS}
\end{tikzcd}
\]

which essentially shows that $V_n$ is an additive map. On the topic of category theory, the Teichmüller representative admits a map in the big Witt vector as $$[-]_S:A\to\bbW_S(A).$$ This is defined by $\{[a]_S\}_n=\begin{cases} a & \ \text{if}\ n=1. \\ 0 & \ \text{otherwise}. \end{cases}$ 

In this case, the following square commutes: 

\[
\begin{tikzcd}
A\ar[r,-,double equal sign distance,double] \arrow[swap]{d}{[-]_S} & A \arrow{d}{[-]_S^w} \\
\bbW_S(A) \arrow{r}{w} & A^S
\end{tikzcd}
\]

We end this section with some more category theory. Let $\Delta$ and $\varep$ be the homomorphisms $\Delta:\bbW(A) \to \bbW(\bbW(A))$ and $\bbW(A)\to A$ respectively. Explicitly, we have that $\varep$ takes $a = \{a_n\}_{n\geq0} \mapsto a_1$, and $\Delta:w_n(\Delta(a))=F_n(a)=a^n \Mod{p\bbW(A)}$.

\thrm{}{
	The functor $\bbW$ and the homomorphisms $\Delta$ and $\varep$ form a comonad on $\mathsf{Ring}$.
}

\prop{}{
	If $A$ is torsion-free, then $\bbW_S(A)$ is torsion-free.
}

\tbf{Proof:} This is the same as proving that any invertible integer in $A$ is invertible in $\bbW_S(A)$. $\qed$

From this lemma, we have that $w:\bbW(\bbW(A))\to\bbW(A)^{\bbN}$ is injective. Since $F_n(a)$ is in the image of the ghost map and $w_n(\Delta(a))=F_n(a)$; thus, $\Delta$ exists. 

The comonad structure is given by $$\bbW(\Delta_A) \circ \Delta_A = \Delta_{\bbW(A)} \circ \Delta_A,$$ which sends $\bbW(A) \to \bbW(\bbW(\bbW(A)))$, and $$\bbW(\varep_A)\circ \Delta_A = \varep_{\bbW(A)} \circ \Delta_A,$$ which sends $\bbW(A) \to \bbW(A). \qed$

(n.b.) A $\lambda$-ring is a ring $A$ equipped with a homomorphism $\lambda:A\to\bbW(A)$ that makes $A$ a comonad under $\bbW(-), \Delta, \varep$. The \emph{Adams operation} associated with the $\lambda$-ring is the composition of the $\lambda$ map and the ghost map, which is $$\psi^n:A\xrightarrow{\lambda} \bbW(A) \xrightarrow{w_n} A.$$

\section{Further ramification}

This section is short. It is aimed at introducing the notions of decomposition, inertia, and ramification groups so that we can use them in future chapters. In this section, we alternate between the ramification theory that uses prime ideals and a new ramification theory which uses the theory of valuations.

Recall the AKLB setup, where $A$ is a Dedekind domain, $K$ is its field of fractions, $L$ is a finite separable extension of $K$, and $B$ is the integral closure of $A$ in $L$. Here we further assume that $L/K$ is Galois and has Galois group $\text{Gal}(L/K)$. Some authors choose to name this setup the $AKLBG$ setup, where $G = \text{Gal}(L/K)$. 

\defn{Unramified extensions}{
	Let $L/K$ be a finite extension of fields. Let $\mcO_K$ and $\mcO_L$ be valuation rings, $\pi_K$ and $\pi_L$ be uniformisers of $K$ and $L$, and $k=\mcO_K/\pi_K$ and $l= \mcO_L / \pi_L$ be \emph{residue class fields}. Then, $L/K$ is unramified if $l/k$ is separable and one has $$[L:K] = [l:k].$$
	
	The extension $l/k$ is sometimes called the \emph{residual extension}.
}

\prop{}{
	Let $K$ be a local field with perfect residue field $k$. Let $L/K$ be finite and unramified with residual extension $l/k$, and let $L'/K$ be the same. There is a bijection between the set of $K$-morphisms $L \to L'$ and the set of $k$-morphisms $l \to l'$.
}

\tbf{Proof:} A stronger statement is the claim that there is an equivalence of categories between the extensions of $k$ and the unramified extensions of $K$.

Consider the functor $F(L)= l$. Let $\sigma \in \text{Hom}_K(L, L')$ be a $K$-morphism that produces a morphism $\mcO_L \to \mcO_{L'}$ by restriction to $\mcO_L$. It is thus a local map of local rings which induces a morphism $\bar{\sigma} \in \text{Hom}_k(l, l')$. Then consider the map $${\text{Hom}}_K(L, L') \to {\text{Hom}}_k(l, l')$$ which preserves identity. It remains to describe the functor inverse to $F$. If $l/k$ is an extension of $k$, we have that $l = k[\alpha]$ for some $\alpha$ that has minimum polynomial $\in k[x]$.

Let that minimum polynomial be $g$, and lift (in Hensellian fashion) $g$ to $G(x)$. Let $\beta$ be the unique root in $\bar{K}$ that reduces to $\alpha$. If we let $L=K[\beta]$, we have that $$[L:K] = \deg G = \deg g = [l:k].$$

We are left to show that a root of $\bar{g}$ in $l'$ can be lifted uniquely to a root of $\bar{g}$ in $l$. But this is just an application of Hensel's lemma. $\qed$

Consider $K_1/K, K_2/K, \dots$ to be unramified subextensions of $L/K$, and denote their compositum $K_n/K$. That compositum is the \emph{maximal unramified subextension} of \tbf{L:K}.

\defn{Tamely and wildly ramified extensions}{
	Let $l$ and $k$ be the residue class fields of $L$ and $K$ respectively. An algebraic extension $L/K$ is \emph{tamely ramified} if the extension $l/k$ is separable and one has that $$([L:K_n], p)=1$$ or the degree of each finite subextension of $L/K_n$ is coprime with $p$.
	
	Let $e$ be the ramification index and $f$ be the inertial degree. Assume the conditions $ef = [L:K]$ (proven in Chapter 1) and $l/k$ is separable. Then, the statement is equivalent to saying that if $(e, p)=1$ then the extension is \emph{tamely ramified.}
	
	If $e \mid p$, then call the extension \emph{wildly ramified}. 
}

\coro{}{
	Let $L/K$ and $K'/K$ be two extensions inside an algebraic closure $\bar{K}/K$. Define $L'$ such that $L=L'K'.$ Then, $$L/K\ \text{unramified implies}\ L'/K' \ \text{unramified}$$ and $$L/K\ \text{tamely ramified implies}\ L'/K' \ \text{tamely ramified}.$$
}

\tbf{Proof:} The unramified case is equivalent to the previous lemma. We've proven that $[L:K]=[l:k]$ so $L/K$ is unramified. Let $L = K'(\alpha)$ for some $\alpha$, and repeat the same lifting process to conclude that $[L':K']=[l':k']$, thus concluding that $L'/K'$ is unramified.

We introduce an intermediate lemma when proving the tamely ramified case.

\lemma{}{
	Let $L/K$ be a finite extension. It is tamely ramified if and only if it is generated by radicals $$L=T(\sqrt[m_1]{a_1}, \dots, \sqrt[m_r]{a_r}).$$
}

 \tbf{Proof:} Consider the group of radical elements $$R(L/K) = \{m: x^m \in K\ \text{for} \ x \in L^\times\}$$ and consider its subgroup $$P(L/K) = \{m: x^m \in K\ \text{for} \ x \in L^\times, (m, p)=1\}$$
 
Define $L'$ such that $\text{Gal}(L'/T) = P(L/K)$. Our goal here is to prove that $L=L'$, which immediately leads to the proposition. Denote the maximal ideal by $m_K$.

\prop{}{
	Consider the maps $$P(L/K) \to vL \twoheadrightarrow\ \frac{vL}{vK}$$ $$x \mapsto vx \mapsto vx \Mod{vk}.$$ This induces an isomorphism $P(L/K)/T^\times \to vL/vK$.
}

We check that the homomorphism has kernel equal to $T^\times$. Let $x$ be in the kernel. Write $x = au$ with $a \in K^\times$ and $u \in \mcO_L^\times$ such that there exists an $m \geq 1$ with $x^m=b, b\in K, (m, p)=1$.

Denote $u^m=c$ for some $c \in K$. Write $u=du'$ where $u'$ is a unit in $\mcO_L^\times$. It then follows that $\bar{u'}=\bar{1} \equiv 1\Mod{\bar{m_L}}$. We know that ${(u')}^m=\frac{c}{d^m}$ so we can assume that $u \in \mcO_L^\times$ and $\bar{u}=1$.

Consider the equation $x^m=c$. It has a solution $\alpha \in \mcO_K$ such that $\bar{\alpha}=\bar{1}$ by Hensel's lemma. Since $u\alpha^{-1}$ is a root of unity of order $m$, we have that $u\alpha^{-1} \in K$, implying that $u \in K$ and thus $x \in K$. Therefore the homomorphism has kernel $T^\times$.

We are left to check that the homomorphism $P(L/K) \to vL$ that sends $x \mapsto vx$ is indeed surjective. Let $\alpha \in vL$, and let $\alpha=v(a)$ with $a \in L$. We have that $$v(N(a)) = v\left( \prod\limits_{i=1}^m \sigma_i(a)\right)= \sum\limits_{i=1}^m v(\sigma_i(a)) = m\alpha$$ where $m = [L:T], (m, p)=1$.

Now take $b = N(a) \in K^\times$. Since $v(b) = v(a^m)$, we can write $a^m= bu$ with $u \in \mcO_L^\times$. Just like in the step above, we can factorise $u = cu'$ with $u' \in \mcO_L^\times$ and $\bar{u'}=\bar{1}$. Thus, we have that $a^m = bu = bcu' = du'$.

Consider the polynomial $f:x^m-u$, where $u \equiv 1 \Mod{m_K}$. Since $1 \in \bar{K}$ is a simple root of $\bar{f}$, there exists a unique $\omega \in \mcO_K$ such that $f(\omega)=0$ and that $\bar{\omega}=1$. It thus follows that we can write $1+m_K \subseteq (\mcO_K^\times)^m$. Therefore, by writing $(u'')^m = u'$, there exists a $d = \frac{x}{u'} \in T^\times$, finishing the proof. $\qed$

We state the following proposition without proof: If $e(L/K)=f(L/K)=1$ and $L/K$ is unramified, then $L=K$. 

Let $T(L/L')$ be maximal and unramified, and denote the maximum unramified extension of $L/L'$ by $T(L/L')$. Because $[L:T(L/L')]$ divides $[L:T]$ and $([L:T], p)=1$, it follows that $([L:T(L/L')], p)=1.$ Thus $L/L'$ is tamely ramified. The proof that $f(L/K)=1$ is immediate as we can write $\bar{L}=\bar{T}=\bar{K}$, so $\bar{L}=\bar{L'}$. 

The proof that $e(L/K)=1$ uses the lemma above. Let $z\in vL$. Since the homomorphism that sends $x \mapsto vx$ is surjective, there exists an $a \in P(L/K)$ such that $v(a) \equiv z\Mod{vK}$. Therefore we can write $v(a)=z+v(b)$ for some $b \in K$. It's clear that $\gamma = v(a)-v(b)=v(ab^{-1})$; thus, $vL \subseteq P(L/K)$, implying that $$vL' \subseteq vL\ \text{(evidently)}\ \subseteq v(P(L/K)) \subseteq vL',$$ implying that $vL'=vL$. Thus these two conditions are proven, and it follows that $L=L'. \qed$ 

We move back to the tamely ramified case. Consider the commutative diagram

\[
\begin{tikzcd}
L \arrow{r} \arrow{d} & L' \arrow{d} \\
T \arrow{r} \arrow{d} & T'  \arrow{d}\\
K \arrow{r} & K' 
\end{tikzcd}
\]

If $L/K$ is tamely ramified, then $L=T(\sqrt[m_1]{a_1}, \dots, \sqrt[m_r]{a_r})$. Thus, 

\begin{align*} L' &= LK' =LT' \\
&= T(\sqrt[m_1]{a_1}, \dots, \sqrt[m_r]{a_r}),
\end{align*}

and hence $L'/K'$ is tamely ramified. $\qed$

For $\mfq \in \text{Spec}(B)$, the ramification index $e_{\mfq}$ of $\mfq$ is the multiplicity of $\mfq$ in the factorisation of $$\mfp B = \prod\limits_{\mfq \mid \mfp} \mfq^{e_\mfq}$$ and the degree $f_\mfq = [B/\mfq:A/\mfp]$ is the inertia degree of $\mfq$. 

Furthermore, define the set of primes $\mfq$ lying above $\mfp$ as the \emph{fiber} above $\mfp$, and denote the set $\{\mfq | \mfp\}.$ Let $g_{\mfp}$ denote the cardinality of the fiber: $\#\{\mfq | \mfp\}.$ If $\mfq$ lies above $\mfp$, then $L/K$ is \begin{enumerate} 

\item \emph{ramified} at $\mfq$ if $e_{\mfq}>1$
\item \emph{totally ramified} at $\mfq$ if $e_{\mfq}>[L:K]$, or $f_\mfq=1, g_\mfp=1$.
\item \emph{unramified} at $\mfq$ if $e_{\mfq}=1$ and $B/\mfq$ is a separable extension of $A/\mfp$.

\end{enumerate}

When $L/K$ is unramified, we write that 

\begin{enumerate}
	\item $\mfp$ \emph{is inert} in $L$ if $\mfq = \mfp B$ is prime, so $e_\mfq = g_\mfp=1, f_\mfq = [L:K]$
	\item $\mfp$ \emph{is split} in $L$ if $g_\mfp = [L:K]$, so $e_\mfq=f_\mfq=1$ for all $\mfq | \mfp$.
\end{enumerate}

Note that $e_\mfq f_\mfq g_\mfp = [L:K]$, as shown in the first chapter. 

\defn{Decomposition group}{
	Assume the AKLBG setup. Let $v$ be a valuation of $K$ and $w$ be a valuation of $L$. Then the decomposition group of an extension $w|v$ is defined by $$D_w = D_w(L/K) = \{\sigma \in \text{Gal}(L/K) | \sigma(w)=w\}$$ which is the stabiliser of $w$ in $G=\text{Gal}(L/K)$.
	
	Alternatively stated, let $\mfq \in B$. Then $D_\mfq$ is the stabiliser of $\mfq$ in $G$, satisfying $\sigma(\mfq)=\mfq$.
}

From this we can define the \emph{inertia} and \emph{ramification groups} $I_w$ and $R_w$ respectively. 

\defn{Inertia and ramification group}{
	The inertia group of the extension of valuations $w|v$ is defined as $$I_w(L/K) = \{\sigma \in G_w |\ \sigma(x) \equiv x \Mod{\mfP} \ \text{for all}\ x \in \mcO_L\}.$$ 
	
Letting $\mfq \in B$, the inertia group $I_\mfq$ is the set of $\sigma \in G$ satisfying $\sigma(x) \equiv x\Mod{\mfp}$ for all $x \in \mcO_L$.
	
	The ramification group is defined as $$R_w(L/K) = \{\sigma \in G_w: \frac{\sigma(x)}{x} \equiv 1 \Mod{\mfP}\}.$$
	
The analogue for $R_{\mfq}$ is all $\sigma \in G$ satisfying $\frac{\sigma(x)}{x} \equiv 1 \Mod{\mfq}$ for all $x \in \mcO_K$.
}

\prop{}{
	Let $M$ be an intermediate field between $K$ and $L$. Let $w$ be a valuation. One has that $$D_w(L/M) = D_w(L/K) \cap \text{Gal}(L/M)$$ and $$I_w(L/M) = I_w(L/K) \cap \text{Gal}(L/M).$$
}

\tbf{Proof:} From $D_w(L/M) \subseteq \text{Gal}(L/M) \subseteq \text{Gal}(L/K)$ it swiftly follows that $\sigma \in G$ lies in $D_w(L/M)$ if and only if it fixes $M$. Thus, $\sigma$ lies in $\text{Gal}(L/M)$. Additionally $\sigma$ must also satisfy $\sigma(w)=w$ and thus it lies in $D_{w(L/K)}$. Therefore $\sigma$ lies in $D_w(L/M).$ 

\prop{}{
	Let $w'$ be a valuation of $L'$ and $v'$ be the valuation $w'$ with respect to $K'$. Let the inclusion $\tau$ map $L\to L'$ and $K \to K'$. Let $w = \tau(w')$ and $v$ be the valuation $w$ with respect to $K$. 
	
	Consider $\phi: \text{Gal}(L'/K') \to \text{Gal}(L_K)$. This mapping induces the homomorphisms $$G_{w'}(L'/K') \to G_{w}(L/K)$$ $$I_{w'}(L'/K') \to I_{w}(L/K)$$ $$R_{w'}(L'/K') \to R_{w}(L/K)$$
}

\tbf{Proof:} (Neukirch) Let $\sigma' \in D_{w'}(L'/K'), \sigma = \phi(\sigma')$. Note that $\sigma$ and $\sigma'$ are conjugate. One has that, for $x \in L$, $$\|x\|_{\sigma(w)} = \|\sigma x \|_w = \| \tau^{-1} \sigma' \tau x \|_w = \| \sigma' \tau x \|_{w'} = \| \tau x \|_{w'} = \|x \|_w.$$ Thus, $\sigma \in D_w(L/K).$

Let $\sigma \in I_w(L/K)$. Then for $x \in \mcO_L$, we have that $$w(\sigma x-x) = w(\tau^{-1})(\sigma' \tau x - \tau x) = w'(\sigma'(\tau x)-(\tau x)) > 0 $$ and thus $\sigma \in I_w(L/K).$ Similar reasoning for the ramification groups $R_w$ show that for $\sigma' \in R_{w'}(L'/K')$ and $x \in L^*$ we have that $$w\left(\frac{\sigma x}{x}-1\right) = w\left(\tau^{-1}\left(\frac{\sigma' \tau x}{\tau x}\right)\right) = w'\left(\frac{\sigma'\tau x}{\tau x}-1\right) > 0 $$ so $\sigma \in R_w(L/K)$. $\qed$

$L/K$ is an extension of fields, as above. Let $\mfp$ be the maximal ideal of $\mcO_K$ and let $\mfP$ be the maximal ideal of $\mcO_L$. Let $\alpha=\mcO_K/\mfp$ and let $\beta = \mcO_L / \mfP$.
	
The extension of fields $\beta/\alpha$ is normal. Moreover, there is a short exact sequence $$1 \to I_w \to D_w \to \text{Gal}(\beta/\alpha) \to 1.$$

\defn{Decomposition and inertia fields}{
	With the AKLBG setup, let $B/\mfq$ be a separable extension of $A/\mfp$. Define the decomposition and inertia fields $L^{D_\mfq}$ and $L^{I_\mfq}$ as the fields satisfying 
	
	\begin{align*} 
		e_\mfp& = [L:L^{I_\mfq}] \\
		f_\mfp& = [L^{I_\mfq}:L^{D_\mfq}] \\
		g_\mfp& = [L^{D_\mfq}:K] \\
	\end{align*}
}

The extension $(B/\mfq)/(A/\mfp)$ has Galois group $\text{Gal}((B/\mfq)/(A/\mfp)) \simeq D_\mfq / I_\mfq$. Also note that $[B/\mfq:A/\mfp]=f_\mfp$ so $$f_\mfp = \frac{|D_\mfq|}{|I_\mfq}$$

Note that since $L:L^{D_\mfq} = |D_\mfq| = e_\mfp f_\mfp$ and $\frac{|D_\mfq|}{|I_\mfq}|=f_{\mfp}$, then $$|I_\mfq|=[L:L^{I_\mfq}]=e_\mfp$$

\section{Adeles and ideles}

Hello from the other side!

The French terms \emph{adèle} and \emph{idèle} mean "additive ideal element" and "ideal element" respectively. 

One cannot perform much analysis on $\bbQ$. One could pass a problem in $\bbQ$ to a problem in $\bbR$ through the metric completion, though do we really need to? Ostrowski's theorem, classifying the absolute values on $\bbQ$, says that there exists an absolute value for each prime number $p\in\bbZ$. Is there a way where we can perform analysis while preserving the structure of the primes? Phrased otherwise, is there a way of taking a problem presented globally and "localising" it by considering its various completions?

Hasse's \emph{local-global principle} seems like a good place to take inspiration from. We know that if a polynomial $f \in \bbQ[X]$ has a rational solution, then we have a $p$-adic solution. This is convenient as $\bbQ$ embeds in $\bbQ_p$. We start with a global solution and obtain local solutions at each prime. 

Hasse's local-global principle asks the following question: can we glue together solutions over $\bbR$ and the $p$-adic integers to yield a solution over rationals? Note that on number fields, this is equal to asking when we can glue together complex embeddings and $\mfp$-adics, for prime ideals $\mfp$.

(n.b.) Local principles are when the local properties of a mathematical principle inform you about the global properties of that object. The Gauss-Bonnet theorem, from differential geometry, relates the Gaussian curvature of a compact two-dimensional Riemannian manifold (which is a local characteristic) with its Euler characteristic (a global property). The class number formula equates local information (residue of a function) on the left-hand side with global information on the right-hand side.

The Hasse-Minkowski theorem tells us when a quadratic form has rational solutions. 

\defn{}{
	A quadratic form (homogeneous polynomial) over a field $K$ is a polynomial with terms all of degree two. Explicitly, $$f(x_1, \dots, x_n)=\sum\limits_{1\leq i, j\leq n} \alpha_{ij}x_i x_j.$$
}

\thrm{Hasse-Minkowski}{
	Let $f$ be a quadratic form. There exists a non-trivial tuple $(a_1, \dots, a_n) \in \bbQ^n$ such that $f(a_1, \dots, a_n)=0$ if an only if there exists a tuple over $\bbR$ and over $\bbQ_p$ for all $p$ such that $f(a_1, \dots, a_n)=0$.

}

More generally, the condition above also holds for $c\in \bbQ^\times$ - meaning that $f(x_1, \dots, x_n)=c$ has a solution if and only if it has solutions in $\bbR$ and $\bbQ_p$ for all $p$. And it (pretty self-evidently) doesn't hold in $\bbZ$; meaning that solvability in $\bbR$ and $\bbZ_p$ need not guarantee solvability in $\bbZ$.

Consider the quadratic form $x^2+11y^2=3$. It evidently has solutions in $\bbR$. For $\bbZ_p$ where $p\neq 2, 11$, take $\Mod p$ on $x^2 \equiv 3-11y^2$ and use Hensel's lemma. For $p=2, 11$, consider that $\frac{3}{11}$ is a square in $\bbZ_2$. Also consider that $3\equiv 5^2 \Mod{11}$, so we can take $y=0$. However there are no integer solutions. There are rational solutions $(\frac{1}{2}, \frac{1}{2}), (\frac{1}{3}, \frac{4}{3})$, as the Hasse-Minkowski theorem outlines, since there are solutions in $\bbR$ and $\bbQ_p$.

This doesn't hold in non-quadratic forms. A famous counter-example comes from Selmer, who writes that the cubic equation $$3x^3+4y^3+5z^3$$ has no non-trivial rational zero, and thus violates the Hasse principle.

(n.b.) There is no algorithm currently that evaluates whether the cubic $ax^3+by^3+cz^3=0$ has a non-trivial rational zero. 

Back to adeles. 

It is useful to take into account the completions of the rational numbers - namely the $p$-adic numbers when solving a problem over $\bbQ$. However note that though each $\bbQ_p$ is locally compact, their product $\prod\limits_p \bbQ_p$ is not. This is not convenient. We can endow Haar measures on locally compact fields and thus perform harmonic analysis on it (which was heralded by John Tate in his 1950 thesis). With adeles, we work with the restricted product (preserving local compactness) instead of the Cartesian product. 

We know that $\hat{\bbZ}=\prod\limits_p \lim\limits_\leftarrow \bbZ/p^r\bbZ$ which is isomorphic to $\lim\limits_\leftarrow \bbZ/n\bbZ$, which gives us an injective ring homomorphism $\bbZ\to \hat{\bbZ}$ which takes each element and reduces it $\Mod n$. With the ring of adeles, we can consider the completions of a global field simultaneously while ensuring that each completion is locally compact. 

\defn{Restricted direct product}{
	Let $K$ be a number field, and let $v$ be an absolute value on $K$. Let $G_v$ be a locally compact commutative group, and let $H_v$ be a compact open subgroup for all but a finite number of $v$. 
	
	The restricted direct product of the $G_v$ with respect to the $H_v$ are $$\prod\limits_v G_v=\{(x_v) \mid x_v\in G_v\}$$ with $x_v\in H_v$ for all but finitely many $v$. Formally, we write, for $\{X_i\}_{i\in I}$ a family of topological spaces and $\{U_i\}_{i\in I}$ a family of open subsets with $U_i \subset X_i$, the \emph{restricted product} $\mathrlap{\coprod}\prod (X_i, U_i)$ is the topological space given by $$\mathrlap{\coprod}\prod (X_i, U_i)= \{(x_i): x_i \in U_i\ \text{for all but finitely many}\ i\in I\} \subset \prod_i X_i.$$
	
	A reformulation of the restricted direct product: take $S$ to include at least all $v$ wherein $H_v$ is not defined, then, define $$G_S=\prod\limits_{v\in S} G_v \times \prod\limits_{v\not\in S} H_v .$$ 
	
	The restricted direct product is the union of all such $G_S$ ($G=\bigcup_S G_S$), making $G$ a locally compact topological group.}

	
Refer to the construction of $G_S$. The restricted direct product sometimes preserves algebraic structure, including local compactness. If the individual $G_i$s are (locally compact) groups or rings and $H_i$ are respectively (locally compact) subgroups or subrings, then $G$ admits respectively a group or ring structure. 

Note that by Tychonoff's theorem, $G_S$ is locally compact in the product topology. Every subgroup is locally compact with respect to the topology of the restricted direct product since the product topology on $G_S$ is identical to the topology on the restricted direct product of $G_v$ with respect to $H_v$ (denote it by $G$). 

Since every element in this LCG belongs to some subgroup $G_S$, $G$ is locally compact.

\defn{Adeles}{
	With notation as in the definition above, take $G_{\infty}=\bbR$, $G_p=\bbQ_p$, $H_p=\bbZ_p$. The adele group $\bbA_\bbQ$ (represented explicitly as $\{a_\infty, a_2, a_3, \dots\}$) over $\bbQ$ are defined as $$\bbA_\bbQ = \bbR \times \prod\limits_p \bbQ_p$$ such that $a_p \in \bbZ_p$ for all but finitely many $p$. The restricted product is a subset of $\prod\limits_p \bbQ_p$.
	
	In general, denote the adele group of an algebraic group by $G_A$. Fix a field $K$ - if $G_K$ is the additive group of $K$, $G_A$ naturally has a ring structure which we denote by the adele ring, and notate it $\bbA_K$. Then, the adele ring $\bbA_K$ is defined as all $$(a_v) \in \prod\limits_v K_v$$ satisfying $a_v \in \mcO_v$ for all but finitely many $v$. 
	
	So, with notation as written in the definition of the restricted product, let $S$ be a finite set of places of $K$. Define $$\bbA_K^S=\prod\limits_{v \in S} K_v \prod\limits_{v \not\in S} \mcO_v.$$ Then, $$\bbA_K = \bigcup\limits_S\ \bbA_K^S.$$
	
	Formally, we write that $\bbA_K=\mathrlap{\coprod}\prod (K_v, \mcO_v)$ for all $v$ in the set of places of $K$. 
}

Take $\bbA_\bbQ$ for example. It is the restricted direct product of $(\bbR, \{0\}), (\bbQ_2, \bbZ_2), (\bbQ_3, \bbZ_3), \dots$

Just like with the integers and rationals, for $K$ a number field, there is natural isomorphism $\hat{\mfo_K} \to \prod\limits_{\mfp} \lim\limits_\leftarrow \mfo_K/\mfp^m$; this is because there is a map between $\hat{\mfo_K}$ and $\lim\limits_{\leftarrow} \mfo_K/n\mfo_K$ and we can simply map the ring to $\mfo_K/\mfp^m$.

\defn{Ring of finite adeles / ring of adeles}{

Denote the \emph{ring of finite adeles} $\bbA_K^{\text{fin}}$ as the restricted direct product of $(K_{\mfp}, \mfo_{K_\mfp})$ over primes $\mfp \in K.$ The ring of finite adeles of the rationals $\bbA_\bbQ^{\text{fin}}$ as the restricted direct product $\prod_p \bbQ_p$ with respect to $\bbZ_p$, meaning that we admit tuples ($\alpha_p$) for which $\alpha_p \in \bbZ_p$ for all except finitely many $p$. 

Note that $\bbA_\bbQ^{\text{fin}}=\{x_2, x_3, \dots\}$ with $x_p \in \bbQ_p$ for all $p < \infty$; this satisfies the additional condition that $x_p \in \bbZ_p$ for all but finitely many $p$.
}

\defn{Relationship between finite adeles and adeles}{
	Let $K$ be a global field. Its adele ring is the product of its ring of finite adeles $\bbA_K^{\text{fin}}$ with the completions of $K$ at its infinite places. Explicitly, $$\bbA_K = K_{\bbR} \times \bbA_K^{\text{fin}}.$$
	
	Taking the rationals for example, $\bbA_\bbQ^{\text{fin}}$ is identical to $\bbA_\bbQ$, omitting $\bbR=\bbQ_\infty$.
}

$\bbA_K$ is locally compact, with canonical embedding $K \xhookrightarrow\ \bbA_K$ given explicitly by $x \mapsto (x, x, \dots, x).$ Recall how there is a way to embed elements in a number field into $\bbC$. Similarly we can inject elements into adeles through the canonical embedding, which we usually call the \emph{diagonal embedding}.

\ex{}{
	Let $K=\bbQ(\sqrt{2})$. The element $\sqrt{2}$ injects into the adeles as $$(\sqrt{2}, -\sqrt{2}, \sqrt{2}, \sqrt{2}, \dots)$$
}

\defn{Principal adeles}{
	The elements in the image of the embedding $K \xhookrightarrow\ \bbA_K$ are called \emph{principal adeles}.
}

\thrm{Universal property of the adeles}{
	Recall, in our definition of restricted direct product, that $G_S=\prod\limits_{v\in S} G_v \times \prod\limits_{v\not\in S} H_v.$ Let $S$ be a finite set of places. Define the subring of $S$-adeles $$\bbA_{K,S}=\prod\limits_{v\in S} K_v \times \prod\limits_{v\not\in S} \mcO_v.$$ If we take $\bbA$ to be the union of all $S$-adeles, then there exists a map $\rho: \bbA \to X$ such that for every sufficiently large set $S$ there exists a unique $\rho_S:\bbA_{K,S} \to X$ such that $\rho_S = \rho$.
}

\defn{Ideles}{
	In the adeles, we defined $\bbA_K$ to be the union of all $S$-adeles $\bbA_K^S$, where $S$ is a finite set of places. Similarly, define the set of $S$-ideles as $$\bbI_K^S=\prod\limits_{v \in S} K_v^* \prod\limits_{v \not\in S} \mcO_v^*.$$ By such, the set of ideles is the union of the S-ideles, or $$\bbI_K = \bigcup\limits_S\ \bbI_K^S.$$
	
	The ideles are the unit group of the adeles. So, $\bbI_K = \bbA_K^*$.
}

Note that the ideles have an natural embedding $K^* \xhookrightarrow\ \bbI_K$ just like the adeles do. Additionally, define the principal ideles to be the image of the embedding $K^* \xhookrightarrow\ \bbI_K$.

\defn{Idele class group}{
	The idele class group $C_K$ is the quotient group $$C_K =\bbI_K / K^*.$$
}

We can define an absolute value on $\bbA_K$ which we extend from the absolute value $\|\cdot\|_v$ on $K_v$. Define $$\|a\| = \prod\limits_{v} \|a\|_v$$ which converges as $\|a\|_v \leq 1$ for almost all $v$. We call this the \emph{adele norm}.

We mentioned, in Chapter 1, the \emph{ideal class group}. There is a convenient relation between the ideal class group $Cl_K = \mcI_K / \mcP_K$ and the idele class group $C_K = \bbI_K / K^\times$. This is shown through the surjective homomorphism between the idele group $\bbI_K$ to the ideal group $\mcI_K$ satisfying

\begin{align*}
	\bbI_K &\to \mcI_K \\
	\alpha &\mapsto (\alpha) = \prod\limits_{\mfp} \mfp^{v_{\mfp}(\alpha)}
\end{align*}

where the $\mfp$ denotes primes of $K$ satisfying $v_\mfp(\alpha) = v_\mfp(\alpha_v)$.

This construction induces a morphism between the short exact sequences

\[
\xymatrix{
  1 \ar[r] & K^\times \ar[d] \ar[r] & \bbI_K \ar[d] \ar[r]& C_K \ar[d] \ar[r] & 1 \\
  1 \ar[r] & \mcP_K \ar[r] & \mcI_K \ar[r] & Cl_K \ar[r] & 1
}
\]

There is a similar \emph{idele norm} for the ideles. Let $x\in \bbI_K$. Then, for some set $S$, $x \in \bbI_K^S$. Thus, for every $v \not\in S$, $\|x_v\|_v=1$. Define the product $$\|x\| = \prod\limits_v \|x\|_v$$ to be the \emph{idele norm} of $x$. 

\defn{1-ideles}{
	Define the \emph{1-ideles} $\bbI_K^1$ to be the set of all ideles with idele norm 1, or $$\bbI_K^1= \{x \in \bbI_K: \|x\|=1\}.$$
	
	Similarly, define the 1-idele class group to be $C_K^1 = \bbI_K^1 / K^*.$
}

By the product formula (shown and proven in the section below), we have that $K^* \subset \bbI_K^1$. 

\section{Theorems on the adeles and ideles}

We now prove a few theorems about adeles and ideles.

\thrm{Product formula (Artin)}{
	Let $L$ be a global field, and let $M_L$ be its set of places. For all $x \in L^{\times}$, we have that $$\prod_{v \in M_L}  ||x||_v = 1,$$ where $|| \cdot ||_v$ denotes the absolute value for each place.
}

\tbf{Proof:} Let $L/K$ be a separable field extension of finite degree. If $K=\bbQ$ or $\bbF_p(t)$ we can directly verify the product formula. Let $L_w / K_v$ be a finite extension of locally compact fields. We have that $$||x||_{L_w}=|N_{L/K}(x)|_{K_v}$$ for all $x \in L_w$. Then we can take products and conclude that $$||x|| = \prod\limits_{v \in M_K}\prod\limits_{w \mid v} ||x||_w = \prod\limits_{v \in M_K}\prod\limits_{w \mid v} |N_{L_w/K_v}x|.$$

\prop{}{
	Let $L/K$ be a finite separable extension of global fields, and let $v$ be a place of $K$. There is an isomorphism of algebras $$L \otimes_K K_v \to \prod\limits_{w|v} L_w$$ defined component-wise as $l \otimes x \mapsto (lx, \dots, lx)$.
} 

\tbf{Proof:} Since $L/K$ is finite, $L \otimes_K K_v$ is also finite and is thus isomorphic to a finite product of separable extensions of $K_v$, and thus a finite product of finite local fields. It remains to show that there is a correspondence between the finite separable extensions $\{L_i\}$ and $\{L_w: w\mid v\}.$

We first want to show that there exists a map from $\phi:L_i\to L_w$ such that $\phi(L_i) \simeq L_i.$ Fix an absolute value on $K_v$ representing the place $v$. Each $L_i$ has a unique absolute value $|\cdot |_w$ that restricts to $|\cdot|_v$ as $L_i$ extends $K_v$. 

Then consider the map $L \hookrightarrow L \otimes_K K_v \simeq \prod_i L_i \twoheadrightarrow L_i$ views $L$ as a subfield of $L_i$; this suggests that $|\cdot|_w$ restricts to an absolute value on $L$ that uniquely determines $w|v$, and there exists such a map $\phi$ as aforementioned.

We now show that $\phi$ is surjective. Define the map $$\phi_w: L \otimes_K K_v \to L_w$$ given explicitly by $l \otimes x \mapsto lx$. This map is surjective as $L_w$ is the completion of $L$ and the image of $\phi_w$ contains $L$. We use the fact that for every surjective homomorphism $\phi: A\to B$ of $K$-algebras, there is a corresponding projection of $A$ on a subproduct of its factors (as the kernel of $\phi$ is a subproduct of $\prod K_i$. Take $A\simeq \text{ker}\ \phi \times \text{im}\ \phi$. In this case we can take $B = \text{im}\ \phi. \qed )$

With this we know that $\phi_w$ factors through the projection of $L \otimes_K K_v \simeq \prod_i L_i$ onto one of its factors $L_i$, thus showing that $L_i\simeq L_w$ and $\phi(L_i)=L_w$, making $\phi$ surjective. The proof of injectivity is omitted. $\qed$

From the lemma above, we know that a basis for $L$ is also a basis for $L \otimes_K K_v \simeq \prod\limits_{w|v} L_w$. Then note $$N_{L/K}(x)= N_{(L \otimes_K K_v)/K_v}(x) = \prod\limits_{w|v} N_{L_w/K_v}(x).$$ Taking normalised absolute values and products over all places $v$, we get that $$\prod\limits_{v} ||N_{L/K}(x)||_v = \prod\limits_{v} \prod\limits_{w|v} ||x||_w = \prod\limits_w ||x||_w=1.$$ 

The fact that $$\prod\limits_{v} ||N_{L/K}(x)||_v=1$$ can be verified directly with $K=\bbQ$ or $\bbF_p. \qed$ 

\coro{}{
	If $a \in K$ is a non-zero principal adele, the product formula implies that $||a||=1$.
}

Here are some topological properties of the adele ring. Like usual, let $K$ be a global field and $\bbA_K$ be its adele ring.

\prop{}{
	$\bbA_K$ is locally compact and Hausdorff.
}

\tbf{Proof:} We first invoke the fact that the by taking the restricted product of topological spaces, we preserve local compactness. 

Let $\{X_i\}_{i \in I}$ be a family of locally compact topological spaces and $\{U_i\}_{i \in I}$ be a corresponding family of open subsets $U_i \subset X_i$ with all but a finite number of subsets compact. For any set $S \subset I$, the topological space $$X_S = \prod\limits_{i \in S} X_i \times \prod\limits_{i \not\in S} U_i$$ is a finite product of locally compact spaces, with finite product compact by Tychonoff's theorem. 

Note that we can construct compact neighbourhoods as products of compact neighbourhoods as products of open sets are open and products of compact sets are compact. This makes the $X_S$ locally compact; they cover $X$. Furthermore, $X$ is locally compact as each $x\in X_S$ has a compact neighbourhood included in $X_S$ and which is also a compact neighbourhood in $X$. If we consider $x \in U \subset U' \subset X_S$, we have that the image of $U'$ is compact; furthermore, $U$ is open in $X$ as $X_S$ is open in $X. \qed$

With the adeles, since $K_v$ are compact and all but finitely many $\mcO_v$ are locally compact (note that $\mcO_v: \{x \in K_v: ||x||_v \leq 1\}$ is closed), $\bbA_K$ is locally compact.

Now we prove that the adeles are Hausdorff. Since the local fields $K_v$ are all Hausdorff, the product space $\prod_v K_v$ is Hausdorff. Since $\bbA_K$ has a finer topology than the subspace topology on $\prod_v K_v$, $\bbA_K$ is also Hausdorff. $\qed$

In the following section we lead up to the famous \emph{strong approximation theorem}, which is an adelic formulation of the Chinese remainder theorem.

\prop{}{
	Let $L$ be a finite separable extension of a global field $K$. There is a natural isomorphism: $$\bbA_L \simeq \bbA_K \otimes_K L\cong {\mathrlap{\coprod}\prod}_{v\in M_K} (K_v \otimes_K L, \mcO_v \otimes_{\mcO_K} \mcO_L).$$ 
}

\tbf{Proof:} This is equivalent to showing that the diagram below commutes. 

\[
\begin{tikzcd}
L \arrow{r}{\sim} \arrow{d} & K \otimes_K L \arrow{d} \\
\bbA_L \arrow{r}{\sim} & \bbA_K \otimes_K L
\end{tikzcd}
\]	

Now since every element of $\bbA_K \otimes_K L$ is a sum of elements $(a_v) \otimes x$, where $(a_v) \in \bbA_K$ and $x\in L$, there is a natural isomorphism between $$\bbA_K \otimes_K L\cong {\mathrlap{\coprod}\prod}_{v\in M_K} (K_v \otimes_K L, \mcO_v \otimes_{\mcO_K} \mcO_L)$$ that maps $$(a_v) \otimes x \mapsto (a_v \otimes x)$$

It was shown earlier on that $K_v \otimes_K L \simeq \prod_{w|v} L_w$. A similar analogue allows us to conclude that $\mcO_v \otimes_K \mcO_L \simeq \prod_{w|v} \mcO_w$. By this, we have that $$\bbA_K \otimes_K L\cong {\mathrlap{\coprod}\prod}_{v\in M_K} (K_v \otimes_K L, \mcO_v \otimes_{\mcO_K} \mcO_L) \simeq {\mathrlap{\coprod}\prod} (L_w, \mcO_w) = \bbA_L$$ which displays an isomorphism of topological rings. It remains only to verify that the image of $x \in L$ into $\bbA_K \otimes_K L$ is identical to its canonical embedding into the adele ring $\bbA_L$. $\qed$

\coro{}{
	Let $L$ be a finite separable extension of a global field $K$, and let $[L:K]=n$. Furthermore let $\bbA_K^+$ denote the additive structure on $\bbA_K$. Then, $$\bbA_L \simeq \bbA_K \oplus \dots \oplus \bbA_K$$ and $$\bbA_L^+ \simeq \bbA_K^+ \oplus \dots \oplus \bbA_K^+,$$ where there are $n$ summands in each R.H.S. This isomorphism also restricts to $L = K \oplus \dots\oplus K$ of the principal adeles (where $L \subset \bbA_L$). The analogue for the additive group also holds.
}

\tbf{Proof:} Let $v_1, \dots, v_n$ be a basis for $L/K$.

We have isomorphisms $\bbA_L = \bbA_K \otimes_K L \simeq w_1 \cdot \bbA_K \oplus\dots\oplus w_n \cdot \bbA_K^+ \simeq \bbA_K \oplus\dots\oplus \bbA_K.$

Consider $a\in L$. Letting $a = \sum_i b_i\omega_i$, with the $b_i \in K$, we map $$a \mapsto w_1\cdot \{b_1\}, \dots, w_n \{b_n\}$$ where $\{b_i\}$ is the principal adele defined by $b_i$. We can then map $w_1\cdot \{b_1\}, \dots, w_n \{b_n\}$ to $$(b_1, \dots, b_n) \in K\oplus\dots\oplus K \subset \bbA_K\oplus\dots\oplus \bbA_K.$$ Since the dimensions of $L$ and $K\oplus\dots\oplus K$ are the same, there is such an isomorphism. 

The proof for the additive group of the adeles is identical. $\qed$

\thrm{}{
	$\bbA_K^+/K^+$ is compact in the quotient topology. Additionally the principal adeles $K \subset \bbA_K$ form a discrete compact subgroup of $\bbA_K^+$.
}

\tbf{Proof:} Let $K=\bbQ$ or $K=\bbF_p$. Let $L/K$ be a finite separable extension. It follows from the corollary above that if we prove the theorem for $K$, then it holds for $L$ since $$\bbA_L/L \simeq \frac{\bbA_K \oplus \dots \oplus \bbA_K}{K \oplus \dots \oplus K}=\bbA_K/K \oplus \dots\oplus \bbA_K.$$ We will verify the claim for $\bbQ$.

To show that $\bbQ^+$ is discrete in $\bbA_Q^+$, our aim here is to find an open set $U$ that contains $0\in \bbA_\bbQ^+$ but contains no other elements of $\bbQ^+$. 

Consider the set $$U= \{a \in \bbA_K: ||a||_{\infty} <1\ \text{and}\ ||a||_p \leq 1\ \text{for all}\ p<\infty\}$$ where $|\cdot|_\infty$ denotes the real absolute value of $\bbQ$ (for a general global field $K$, its unique infinite absolute value) and $|\cdot|_p$ denotes the $p$-adic absolute value. 

The product formula then implies that $||a||=1$ for all non-zero $a \in \bbQ \subset \bbA_\bbQ$. Thus, $U \cap \bbQ = \{0\}$, thus proving that $\bbQ^+$ is discrete in $\bbA_\bbQ^+. \qed$ 

Next we prove that $\bbA_\bbQ^+/\bbQ^+$ is compact. Consider the set $$W=\{a \in \bbA_\bbQ: ||a||_v \leq 1\ \forall v\}.$$ Let $S=\{\infty\}$ be the place at infinity, and let $U_{\infty}=\{x \in \bbQ_{\infty}:||x||_{\infty} \leq 1\}.$ 

We now show that every adele $a=(a_v)$ is the sum of $b+c$, where $b\in\bbQ$, $c \in W$. Let $a=(a_v) \in \bbA_{\bbQ}^+$. Then for each prime $p$ there exists a rational number $k_p = \frac{l_p}{p^{n_p}}$ (with $l_p \in \bbZ$ and $n_p \in \bbZ^+$ such that $$|a_p-k_p|_p \leq 1$$) and $||k_p||_{q} \leq 1$ for $q \neq p$. (Why?)  

Let $c'=\sum\limits_{p<\infty} x_p \in \bbQ \subset \bbA_\bbQ$, and choose $x_\infty \in \bbZ$ such that $$||a_\infty-c'_\infty-k_\infty||_{\infty} \leq 1.$$ We can choose $x_{\infty} \in \bbZ$ to be the nearest integer to the rational number $a_\infty-c'_\infty$. 

Now, let $c=\sum\limits_{p\leq\infty} x_p \in \bbQ \subset \bbA_\bbQ$, and let $b=a-c$ with $c \in \bbQ$. We claim that $b\in W$. This is because for each $p<\infty$, we have that $x_v \in \bbZ_p$ for $v\neq p$ and that 

\begin{align*}
	||a-c||_p &= \left\| a_p-\sum\limits_{w\leq \infty} x_w\right\|_p \\
	&\leq \max{(\|a_p-k_p\|_p, \max(\{\|x_v\|_p: v\neq p\}))}\\
	&\leq \max(1, 1) \\
	&= 1.
\end{align*}

For the prime at infinity, we have that $|b||_{\infty}=||a_\infty-c_\infty||=||a_\infty-c'_\infty-k_\infty||_{\infty} \leq 1.$ 

Thus, $b \in W$, and $W$ surjects onto $\bbA_K/K$ under the quotient map $W \to \bbA_K/K$. Since $W$ is a compact set, the image of $\bbA_K/K$ must also be compact. $\qed$

\coro{}{
	Let $K$ be a global field. $\bbI_K$ is locally compact.
}

\tbf{Proof:} Since each $K_v^\times$ is Hausdorff, $\prod_v K_v^\times$ is Hausdorff in the product topology, suggesting that its restricted product with respect to the $\mcO_v^\times$ is Hausdorff. (It is, after all, a subset of $\prod_v K_v^\times$)

Further recall since $K_v^\times$ is locally compact and $\mcO_v^\times$ is compact (since it is closed), the restricted product $\mathrlap{\coprod}\prod (K_v^\times, \mcO_v^\times)$ is locally compact. $\qed$

\coro{}{
	The principal ideles $K^*$ are a discrete subgroup of $\bbI_K$. 
}

\tbf{Proof:} Since $K$ is discrete in $\bbA_K$, it also follows that $K \times K$ is a discrete subset of $\bbA_K \times \bbA_K$. Thus the image of $K^\times$ lies in $\bbI_K=\bbA_K^\times$ and is discrete since $\bbI_K$ injects into $\bbA_K \times \bbA_K$. $\qed$

Furthermore, if $L$ is a finite-dimensional vector space over $K$, then $L$ is discrete and compact in $\bbA_L$.

The theorem above also allows us to endow a Haar measure onto $\bbA_K$. Let $m_p$ be the Haar measure on $K_p$, and let $S$ be a finite set of primes of $K$ that contain all infinite primes. Let $E_p$ be an open subset of $K_p$ where $m_p(E_p)$ is defined, and let $E_p=\mcO_p$ for all $p \not\in S$. Letting $E \subset \bbA_K$ to be of the form $E=\prod\limits_p E_p$, we define the measure of the set $$m(E)=\prod\limits_p m(E_p)$$ which is defined as $m_p(E_p)=m_p(\mcO_p)=1$ for almost all $p$. 

\coro{}{
	There is a measure on the additive group $\bbA_K^+$ of $\bbA_K$. Take the basis for the $\sigma$-algebra all measurable sets of form $\prod_v B_v$ with $\mu_v(B_v)<\infty$ with $v \in M_K$ and $B_v=\mcO_v$ for all but finitely many $v$. Then define $$\mu\left(\prod\limits_v B_v\right)= \prod\limits_v \mu_v(B_v).$$
}

\lemma{Adelic Blichfeldt-Minkowski}{
	Let $K$ be a global field. There is a constant $C>0$ such that whenever $x \in \bbA_K$ is such that $\|x\| > C$, there exists a non-zero principal adele $a \in K \subset \bbA_K$ for which $\|a\|_v \leq \|x\|_v$ for all $v \in M_K$.
}

\tbf{Proof:} Let $c_0$ be the Haar measure of $\bbA_K^+/K^+$ induced from the normalised Haar measure on $\bbA_K^+$. So $c_0=\text{covol}(K)$ is the measure of a fundamental region of $K$ in $\bbA_K$ under the normalised Haar measure $\mu$. Let $c_1$ be the Haar measure of the set $A:\{a\}_v \in \bbA_K^+$ that satisfy 

$$\begin{cases}
	\|a\|_v \leq \frac{1}{2}\ &\text{if}\ $v$\ \text{is archimedean} \\
	\|a\|_v \leq 1 &\text{if}\ $v$\ \text{is non-archimedean}
\end{cases}$$

The $\frac{1}{2}$ above seems arbitrary but we prove here that it works. Cassels, in his proof, chooses $\frac{1}{10}$; in fact any value lesser than $\frac{1}{2}$ suffices. Here, I prove that we can take $$C=\frac{c_0}{c_1}.$$

Consider the set $B:\{b\}_v \in \bbA_K^+$ such that

$$\begin{cases}
	\|b\|_v \leq \frac{1}{2} \|x\|_v\ &\text{if}\ $v$\ \text{is real archimedean} \\
	\|b\|_v \leq \frac{1}{2} \sqrt{\|x\|_v} &\text{if}\ $v$\ \text{is complex archimedean} \\
	\|b\|_v \leq \|x\|_v\ &\text{if}\ $v$\ \text{is non-archimedean} \\
\end{cases}$$

This set has measure $$\mu(B)=c_1 \cdot \prod\limits_v \|x\|_v > c_1 \cdot C = c_0.$$ 

Since $\mu(B)>c_0$, it cannot lie in a fundamental region of $K$. Thus we can select distinct $b_1, b_2 \in B$ that have the same image in $\bbA_K/K$ such that $\alpha = b_1-b_2 \in K$ is non-zero. 

We can then take 

$$\|\alpha \|_v=\|b_1-b_2\|_v\leq\begin{cases}
	\|b_1\|+\|b_2\| \leq 2\cdot \frac{1}{2} \|x\|_v = \|x\|_v &\text{if}\ $v$\ \text{is real archimedean} \\
	\|b\|_v \leq (2\cdot \frac{1}{2} \sqrt{\|x\|_v})^2 = \|x\|_v &\text{if}\ $v$\ \text{is complex archimedean} \\
	\max{(\|b_1\|, \|b_2\|)} \leq \|x\|_v &\text{if}\ $v$\ \text{is non-archimedean} \\
\end{cases}$$

The normalised valuation $\|\cdot \|_v$ for the complex archimedean case is the square of the usual archimedean complex valuation on $\bbC$. Here our desired bound is the square of the maximum Euclidean distance between two points in the disc of radius $\frac{1}{2}\sqrt{\|x\|_v}$.

Such a construction of $\alpha$ finishes the proof. $\qed$

\thrm{}{
	The 1-idele class group $\bbI_K^1/K^\times$ is compact.
}

\tbf{Proof:} We first observe that $\bbA_K/K$ is a compact group, we can take arbitrarily large compact subsets of $\bbA_K$ (omitting proof details). Choose a compact subset $C \subset \bbA_K$ such that $m(C) > m(\bbA_K/K)$, where $m$ denotes the measure on $bba_K$. Let $C' = C-C$ and $C'' = C'\times C'$, which are continuous and thus defined. Since $K$ is discrete in $\bbA_K, K \cap C''$ is finite. Let $K \cap C'' = \{\alpha_1, \dots, \alpha_n\}.$ Then $$V = C' \cup \alpha_1^{-1} C' \cup \dots \cup \alpha_n^{-1} C'$$ is compact in $\bbA_K$. 

Let $E$ be a finite set of primes in $K$. Then the subset $\bbA_K(E) = \prod\limits_{p \in E} K_p \times \prod\limits_{p \not\in E} \mcO_p$ is open in $\bbA_K$. Let $E_1, \dots, E_m$ be sets such that the compact set $V$ is contained in $\bbA_K(E_1) \cup \dots \cup \bbA_K(E_m)$, and let $E_0$ be the union of the $E_i$ such that $$\bbA_K(E_0)=\bbA_K(E_1) \cup \dots \cup \bbA_K(E_m).$$ It swiftly follows that $V$ is contained in $\bbA_K(E_0)$. 

Let $a$ be an adele. Since $a \to |a|_p$ is a continuous map for each $p$, it then follows that $|a|_p$ is bounded on the compact set $V$. Therefore there exists a bound $\delta$ for which $|a|_p \leq \delta$ for $ a\in V, p \in E_0$. Thus, $$V \subset \prod\limits_{p \in E_0} \{\alpha \in K_p \mid |\alpha|_p \leq \delta\} \times \prod\limits_{p \not\in E_0} \mcO_p.$$

Let $c \in \bbI_K$ such that $c, c^{-1} \in V$. Then they are also in the group defined $$W= \prod\limits_{p \in E_0} \{\alpha \in K^\times \mid |\alpha|_p \leq \delta, |\alpha^{-1}|_p \leq \delta\} \times \prod\limits_{p \not\in E_0} \mcO_p^\times$$ which is compact in $\bbI_K$ as the two constituents of the product are compact.

Let $i \in \bbI_K^1$ be an idele. We aim to show that $i \in K^\times W$, thus proving the statement as $I_K^1/K^\times$ is in the image of the compact set $W$. Consider a bounded subset of $\bbA_K$ denoted $C$, as above. We have that $m(iC)=m(C)$ and $m(i^{-1}C)=m(C)$ since $|i|=1$. 

Note that there exists $ia_1, ia_2 \in iC$ such that $ia_1-ia_2 \in K^\times$. This is because for $a$ an adele we have that $$m(s) = \int_{\bbA_K} \chi(a)\ da = \int\limits_{\bbA_K/K} \sum\limits_{\alpha \in K} \chi(a+\alpha)\ d \bar{a} \leq \int_{\bbA_K/K}\ 1 d\bar{a}$$ and thus if $\sum\limits_{\alpha \in K} \chi(a+\alpha)>1$ we can set $a_1=a+\alpha_1$ and $a_2=\alpha_2$ for some $\alpha_1, \alpha_2$ such that $a_1, a_2 \in S$. 

Set $c_1 = a_1-a_2$. We know that $c_1 \in C'$ and $ic_1 \in K^\times$. It follows that there exists elements $i^{-1}b_1, i^{-1}b_2 \in i^{-1}C$ such that $i^{-1}b_1- i^{-1}b_2 \in K^\times$. Putting $c_2=b_1-b_2$, note that $c_2 \in C'$ and $i^{-1}c_2 \in K^\times$. 

The product $(ic_1)(i^{-1}c_2)=c_1c_2 \in K^\times \cap C''$ so $c_1c_2=\alpha_i$ for some $i$. Note that $c_1 \in C' \subset V$, and $c_1^{-1}=\alpha^{-1}c_2$ so is a subset of $V$. It thus follows that $c_1^{-1} \in W$ and $i = (ic_1)c_1^{-1} \in K^\times W. \qed$

The statement that $\bbI_0^K/K^\times = J_K^0$ is compact is equivalent to the Dirichlet unit theorem.

Here we present two versions of the \emph{approximation theorem}. The first, whose proof follows Neukirch, does not use the theory of adeles. The second is an adelic formulation of the theorem (the \emph{strong approximation theorem}).

\thrm{Approximation theorem}{
	Let $\|\cdot \|_1, \dots, \|\cdot \|_n$ be different valuations of the field $K$, and let $a_1, \dots, a_n \in K$. Then, for every $\varep>0$, there exists an $x \in K$ such that $$\|x-a_i\|_i < \varepsilon $$ for all $i=1, \dots, n$ and $\|x\|_v \leq 1$
}

\tbf{Proof} (Neukirch): Since the valuations are inequivalent, we can pick a value $\alpha$ such that $\|\alpha \|_1 <1$ and $\|\alpha \|_n \geq 1$. Similarly, there also exists $\beta \in K$ such that $\|\beta \|_n < 1$ and $\|\beta \|_n \geq 1.$ Take $y=\frac{\beta}{\alpha}$. We have that $\|y\|_1 > 1$ and $\|y \|_n < 1$.

We now induct on $n$ to prove that there exists a $z \in K$ such that $\| z \|_1 > 1$ and $\| z \|_j > 1$ for all $j \in [2, n]$. For $n=2$, the result follows from the fact that two valuations $\| \cdot\|_1$ and $\|\cdot \|_2$ are equivalent if there exists a real number $s>0$ such that $\|x \|_1 = \|x \|^s_2$.

Assume we've found $z \in K$ satisfying $\|z \|_1 >1$ and $\|z \|_j <1$ for $j in [2, n-1]$.

Then if $\|z\|_n \leq 1$, we take $m$ to be sufficiently large and consider $z^my$. If $\|z\|_n > 1$, however, consider the sequence $t_m = \frac{z^m}{1+z^m}.$ 

This sequence converges to $1$ with respect to $\|\cdot \|_1$ and $\|\cdot \|_n$, and to 0 with respect to all other valuations. We take $t_m y$ for sufficiently large $m$. This suffices.

In this case, for every $i$ we should construct a $z_i$ that is very close to 1 with respect to $\| \cdot \|_i$, and very closet o $0$ with respect to $\|\cdot \|_j$ for $j\neq i$. Lastly, take $$x = a_1z_1+\dots+ a_nz_n.$$ Such an $x$ suffices. $\qed$ 

\thrm{Strong approximation theorem}{
	Let $K$ be a global field and let $v_0$ be a place. Let $S$ be a finite set of places not containing $v_0$. Let $a_v \in K_v$ for $v\in S$. For each $\varep>0$, there exists an $x\in K$ such that $$\|x-a_i\|_v < \varepsilon$$ for all $i=1, \dots, n$ and $\|x\|_v \leq 1$ for $v \not\in S$ and $v \neq v_0$.
}

Firstly, convince yourself that the claim that there exists a sequence of positive numbers $\gamma_v$, with $\gamma_v = 1$ for all but finitely many $v$ such that $\bbA_K = W+K$, where $$W = \prod\limits_v \{x \in K_v: \|x\|_v \leq \gamma_v\}$$ holds. (Hint: $W$ contains a complete set of coset representatives for $K \subset \bbA_K.$)

Let $\eta$ be an idele where $0 < \| \eta\|_v < \alpha_v^{-1}\varep$ for $v \in S$ and $\| \eta\|_v < \alpha_v^{-1}$ for $v \not\in S$ and $v \neq v_0$. As long as $\| \eta_{v_0} \|_{v_0}$ is sufficiently large, we have that $\|\eta \| > \|x \|_v$, where $x$ is the bound described in the Adelic Blichfeldt-Minkowski lemma. Thus, there exists a $\lambda \in K^*$ such that $\| \lambda\|_v \leq \| \eta_v\| _v$ for all $v$. 

Let $\alpha$ be an adele where $\alpha_v = a_v$ for $v\in S$, and $\alpha_v =0$. Then, write $\alpha \lambda^{-1}=\beta+b$ for some $\beta \in W$ and $b \in K$. 

Then perform casework on whether $v \in S$. For $v \in S$, $$\|x-a_v\|_v = \|x-a_v\|_v = \|\lambda \beta_v\|_v \leq \|\lambda\|_v \gamma_v < (\gamma_v^{-1} \varep)\gamma_v = \varep$$

For $v \not\in S$, $$\|x\|_v = \|\lambda \beta_v\|_v \leq \|\lambda \|_v \gamma_v \leq \gamma_v^{-1}\gamma_v = 1. \qed$$

\coro{Chinese remainder theorem}{
	It's an elementary theorem but I'll state it here for formality's sake. Given coprime $n_i$, there is a solution to the congruence identity 
	
	$$\begin{cases}
	x &\equiv a_1 \Mod{n_1} \\
	x &\equiv a_2 \Mod{n_2} \\
	&\vdots \\
	x &\equiv a_k \Mod{n_k} \\
	\end{cases}$$
	
	and all integer solutions are congruent modulo $N=\prod\limits_i n_i$.
}

Consider the conditions of the strong approximation theorem - $\|x-a_i\|_v < \varepsilon$ and $\|x\|_v \leq 1$. When we look at the integers, we have that $$x \equiv a_i \Mod{p_i^{n_i}},$$ which is precisely the Chinese remainder theorem. 

Thus, the strong approximation theorem is an adelic setting of the Chinese remainder theorem. 

\chapter{Class Field Theory I}
\minitoc

\section{Introduction}

What is class field theory? How was class field theory developed, and who was involved in its development?

First question. Class field theory is the study of finite-dimensional abelian extensions of global and local fields. Simply put, it is the \tbf{theory of abelian extensions}. 

Second question. We'll have to begin somewhere, so let's begin with Kronecker. Kronecker, following work by Abel who constructed abelian extensions of $\bbQ(i)$ with the lemniscatic elliptic function $\text{sl}(z)$, generated abelian extensions of imaginary quadratic fields using similar functions. In 1853, Kronecker developed his Kronecker-Weber theorem, which states that every finite abelian extension of $\bbQ$ is contained within a cyclotomic field $\bbQ(\zeta_n)$ for some $n$. 

He then stated his \emph{Jugendtraum}, or \emph{dream of youth}, that one could explicitly construct all abelian extensions of $\bbQ(\sqrt{-n})$. This is solved by the theory of complex multiplication, which showed that every abelian extension of $K$ can be obtained by the roots of an elliptic curve with complex multiplication. (Bit of trivia: such theory explains why $e^{\pi\sqrt{163}}=262537412640768743.9999999999993\dots$ is so close to an integer.) Hilbert then conjectured, in his \emph{12'th problem}, whether one could explicitly characterise the abelian extensions of an arbitrary number field. 

The Kronecker-Weber theorem was first proven completely by Weber (with gaps filled by Neumann), then Hilbert. Kronecker's Jugendtraum was proven completely by Takagi.

The notion of a "class field" is attributed to Hlibert, and is the maximal unramified abelian extension of a number field. Weber, generalising Dirichlet's theorem on primes in arithmetic progressions to arbitrary number fields, developed his notion of a class field at roughly the same time. Hilbert then conjectured properties about his class field, and this work was continued by Furtwangler. Takagi took Furtwangler's work on the Hilbert class field and work on Kronecker's Jugendtraum and developed the \emph{first and second inequalities} of class field theory. 

Then Artin followed Takagi's work (written during WWI) and developed a non-abelian class field theory. Hasse came along and developed a local-global principle for class field theory, deriving the main theorems and Artin reciprocity from his local class field theory. Chevalley then used the ideles to derive global class field theory from local class field theory. It was through this method that cohomology was introduced into class field theory and is now a crucial part of its language.

Now we have a bunch of names and some unknown terms. Perhaps the two approaches to class field theory would serve more pedagogical use. These are the \emph{ideal-theoretic} and the \emph{idele-theoretic} approach. 

The ideal-theoretic (or "classical") approach, which was developed first in the 1920s and the 1930s, using analytic methods and elementary methods. This approach derives local class field theory from global class field theory. The idele-theoretic approach derives global class field theory from local class field theory via the ideles, which you saw in the previous chapter. Group cohomology is inseparable from the idele-theoretic approach. 

Since we've already spent time discussing the adeles and ideles, we'll go the idele-theoretic approach to arrive at the main theorems of class field theory in this chapter. We'll then retrace a bit and discuss how the ideal-theoretic approach to class field theory plays out in the next chapter.

\section{Kronecker-Weber theorem}

\thrm{Kronecker-Weber theorem}{
	Every finite abelian extension of $\bbQ$ lies in a cyclotomic field $\bbQ(\zeta_n)$.
}

We will soon prove that this is equal to the \emph{local Kronecker-Weber theorem}.

\thrm{Local Kronecker-Weber theorem}{
	Every finite abelian extension of $\bbQ_p$ lies in a cyclotomic field $\bbQ_p(\zeta_n)$.
}

Thus the only way to make abelian extensions are:

\begin{enumerate}
	\item $\bbQ(\zeta_n)$ for an integer $n$, which has Galois group $(\bbZ/n\bbZ)^*$
	\item Subfields of these extensions
\end{enumerate}

\ex{Quadratic fields contained in cyclotomic fields}{
	Gauss, through Gauss sums, proved that $$\bbQ(\sqrt{(-1)^{\frac{p-1}{2}}}) \subseteq \bbQ(\zeta_p)$$ Or, every quadratic field is contained in some cyclotomic field.
}

\tbf{Proof:} Let $M$ be intermediate between $\bbQ$ and some cyclotomic field $\bbQ(\zeta_p)$. Casework on $p \Mod{4}.$ One can reformulate the theorem as follows:
	
\begin{itemize}
	\item If $p \equiv 1 \Mod{4}$, then $\bbQ(\sqrt{p}) \subseteq \bbQ(\zeta_p)$
	\item If $p \equiv 3 \Mod{4}$, then $\bbQ(\sqrt{-p}) \subseteq \bbQ(\zeta_p)$
\end{itemize}


Let $G = \text{Gal}(\bbQ(\zeta_p)/\bbQ).$ We know that $\bbQ(/\zeta_p)/\bbQ= p-1$ and it has basis $\{1, \zeta_p, \dots, \zeta_p^{p-1}\}$ with the sum of the basis elements being $0$. Furthermore any $\sigma$ that sends $\zeta_p$ to $\zeta_p^r$ for $r$ a primitive root must fix any $\alpha \in \bbQ(\zeta_p)$.

Write $$\alpha = \sum\limits_{i=1}^{p-1} a_i \zeta_p^i$$ with $a_i \in \bbQ$. There must be some $a_i$, where $i \in [1, p-1]$ such that $a_i=a.$ Partition $\alpha$ (note $\alpha$ is the sum of the basis elements except for 1) into two sums $\alpha_0$ and $\alpha_1$. $\alpha_0$ contains the powers of $\zeta_p$ with exponents being quadratic residues $\Mod p$, and $\alpha_1$ contains quadratic non-residues $\Mod p$.

$$\alpha_0 = \sum\limits_{i=1}^{(p-1)/2} \zeta_p^{r^{2i-2}}$$
$$\alpha_1 = \sum\limits_{i=1}^{(p-1)/2} \zeta_p^{r^{2i-1}}$$

Note that we can send $\sigma(\alpha_0)=\alpha_1$ and $\sigma(\alpha_1)=\alpha_0$. There are $\frac{p-1}{2}$ terms in $\alpha_0$ and $\frac{p-1}{2}$ terms in $\alpha_1$, so there are $\frac{(p-1)^2}{4}$ terms in $b=\alpha_0\alpha_1.$ 

Let $b=\alpha_0\alpha_1.$ Note that $\sigma$ fixes $b$. We now compute $b$. In the following proof, let $\chi_p$ be the \emph{quadratic residue character}. 

$$\begin{cases}
	\chi_p=1\ \text{if}\ $p$\ \text{is a quadratic residue} \\
	\chi_p=1\ \text{if}\ $p$\ \text{is a quadratic non-residue} \\
\end{cases}$$

\tbf{Case 1:} Let $p\equiv 1\Mod{4}$. Then every term is of the form $\zeta_p^j \zeta_p^k$, where $j$ is a quadratic residue and $k$ is a quadratic non-residue. Note that for any $j$, $\chi(-j)=\chi(j)$ so $j$ and $-j$ are etiher both residues or both non-residues. It then follows that $\zeta_p^k$ and $\zeta_p^j$ are not inverses of each other.

It swiftly follows that $\sum\limits_i a_i = \frac{(p-1)^2}{4}$. Write $b=\sum\limits_{i=1}^{p-1} a_i \zeta_p^i$. Since there is some $i$ such that $a_i=a$, it follows that $\sum\limits_i a_i = (p-1)a=\frac{(p-1)^2}{4}$ so $a=\frac{p-1}{4}$. 

\begin{align*}
	b&=a(\zeta_p+\zeta_p^2+\dots+\zeta_p^{p-1}) \\
	&= \left(\frac{p-1}{4}\right) \sum\limits_{i=1}^{p-1} \zeta_p^i \\
	&= \frac{1-p}{4}.
\end{align*}

\tbf{Case 2:} Suppose that $p \equiv 3\Mod{4}$. Since $\chi(-j)=-\chi(j)$, it follows that one of $j$ and $-j$ is a quadratic residue, and the other is a non-residue. Therefore for each $\zeta_p^i \in \alpha_0$ there is an element $\zeta_p^j$ in $\alpha_1$ such that $\zeta_p^i \zeta_p^j=1$. 

This suggests that we can write $b = \frac{p-1}{2}+b'$ where $b'$ is the sum of the remaining $\frac{(p-1)^2}{4}-\frac{p-1}{2}=\frac{(p-1)(p-3)}{4}$ terms.

By writing $b'=\sum\limits_{i=1}^{p-1} a_i'\zeta_p^i$, we have that $\sum\limits_i a_i'= \frac{(p-1)(p-3)}{4}$. Since there exists some $a'$ such that $a'=a'_i$, we can write $\sum\limits_i a'_i=(p-1)a'$ and thus $a' = \frac{p-3}{4}.$

\begin{align*}
	b'&= \left(\frac{p-3}{4}\right) \sum\limits_{i=1}^{p-1} \zeta_p^i \\
	&= -\frac{p-3}{4}.
\end{align*}

and thus $b = \frac{p-1}{2}+b'=\frac{p+1}{4}.$ 

What do we know about $\alpha_0$ and $\alpha_1$? Since they sum up to $-1$ and have product $\frac{p+1}{4}$ or $\frac{1-p}{4}$ depending on $p \Mod{4}$,

\begin{align*}
	(x-\alpha_0)(x-\alpha_1) &= x^2+x-\frac{p-1}{4}\ \text{if}\ p\equiv 1 \Mod{4}\\
	(x-\alpha_0)(x-\alpha_1) &= x^2+x-\frac{1-p}{4}\ \text{if}\ p\equiv 1 \Mod{4}
\end{align*}

and thus $\alpha_0, \alpha_1 = $

$$\begin{cases} 
	\frac{-1\pm \sqrt{p}}{2}\ \text{if}\ p\equiv 1 \Mod{4} \\ 
	\frac{-1\pm \sqrt{-p}}{2}\ \text{if}\ p\equiv 3 \Mod{4}
\end{cases}$$

which proves the lemma. $\qed$

It thus follows that $\bbQ(\sqrt{\chi_p(-1)p})$ is contained between $\bbQ$ and $\bbQ(\zeta_p)$. This is because $\text{Gal}(\bbQ(\zeta_p)/\bbQ) \cong \bbZ/(p-1)\bbZ$ which contains a unique subgroup of index two.

Gauss' theorem on quadratic fields is a special case of the Kronecker-Weber theorem. Here we prove it in full.

\lemma{Local K-W implies Global K-W}{
	The local Kronecker-Weber theorem implies the global Kronecker-Weber theorem.
}

\tbf{Proof:} Let $K/\bbQ$ be finite and abelian. Let $p$ be a ramified prime, and pick $\mfp$ such that $\mfp \mid p$. Let $K_\mfp$ be the completion of $K$ at $\mfp$. Letting $D_\mfp$ be the decomposition group, we have that $\text{Gal}(K_\mfp / \bbQ_p) \cong D_\mfp$.

Since $\text{Gal}(K_\mfp / \bbQ_p)$ is abelian, $K_\mfp \subseteq \bbQ_p (\zeta_{m_p})$ for some $m_p \in \bbZ^+.$ Let $n_p = v_p(m_p)$. Thus we have that $$n = \prod\limits_p p^{n_p}$$ where the product ranges through all primes $p$ that ramify. 

Let $L=K(\zeta_m)$, which is a compositum of Galois extensions and thus abelian. We aim to show that $L = \bbQ(\zeta_m)$ and thus $\text{Gal}(L/\bbQ)$ is abelian. Taking $\mfq$ to be a prime that lies above $\mfp$, we have that the completion $L_\mfq$ of $L$ at $\mfq$ is a finite abelian extension of $\mfq_p$. Thus, we have that $$L_\mfq = K_\mfp(\zeta_m).$$

Let $F_\mfq$ be the maximal unramified extension of $\bbQ_p$ in $L_\mfq$. Thus, $L_\mfq / F_\mfq$ is totally ramified, and the Galois group $\text{Gal}(L_\mfq / F_\mfq) = I_\mfq \subseteq \text{Gal}(L/\bbQ)$.

This means that we can compute $I_p$ locally. Note that the extension $K(\zeta_m)/K$ is ramified if and only $p \mid m$. So since $K_\mfp \subseteq \bbQ_p(\zeta_{m_p})$ and $\bbQ_p(\zeta_{m_p / p^{n_p}})$ is ramified, then $K_\mfp \subseteq F_\mfq(\zeta_{p^{n_p}}).$ Thus we have that the intersection $F_\mfq \cap \bbQ_p(\zeta_{p^{n_p}}) = \bbQ_p$ and thus $$I_p \simeq \text{Gal}(\bbQ_p(\zeta_{p^{e_p}})/\bbQ_p)$$ which has order $\phi(p^{e_p})$ where $\phi$ denotes the Euler totient function.

Let $I=\bigcup\limits_{p \mid m} I_p $. Since $\text{Gal}(L/\bbQ)$ is abelian, we have that $I \subseteq \prod I_p$, and thus $$\|I| \leq \prod\limits_{p \mid m} |I_p| = \prod\limits_{p | m} |(\bbZ/p^{n_p} \bbZ)^\times| = \prod\limits_{p\mid m} \phi(p^{n_p}) = \phi(m) = [\bbQ(\zeta_m):\bbQ]$$

Since each inertia field $L^{I_p}$ is unramified at $p$ (exercise), it follows that $L^I \subseteq L^{I_p}$ so $L^I/\bbQ$ is unramified. By Minkowski theory there are no non-trivial unramified extensions of $\bbQ$, and thus $L^I=\bbQ$. Therefore $$[L:\bbQ] = [L:L^I] \leq [\bbQ(\zeta_m):\bbQ].$$ So $L=\bbQ(\zeta_m)$, as claimed, and it follows that $K \subset L =\bbQ(\zeta_m)$. $\qed$

In the proof of local Kronecker-Weber, we consider when $L/\bbQ_p$ is unramified, tamely ramified, and wildly ramified.

\tbf{Case 1: Extension is unramified}

Let $L/K$ be unramified and finite. Then, the inertia group is trivial as $e=1$ and therefore $[L/K] = [l:k]$ with an isomorphism of Galois groups $\text{Gal}(L/K) \simeq \text{Gal}(l/k)$.

Let $\alpha$ generate $l/k \simeq \bbF_p$. It is thus a root of unity with order coprime to $p$. Apply Hensel's lemma to $f = x^n-1$. We have a root $\beta \in \mcO_L$ such that $\beta = \alpha \Mod{\mfp}$, where $\mfp$ lies above $p$. 

Therefore we have that $$[K(\beta):K] \geq [k(\alpha):k]=[l:k]=[L:K]$$ and thus $L=K(\beta)=K(\zeta_n)$. Take $K=\bbQ_p$ and we are done. $\qed$

\tbf{Case 2: Extension is tamely ramified}

\lemma{}{
	Let $K$ and $L$ be finite extensions of $\bbQ_p$. Let $m_L$ be the maximal ideal of $L$, and assume that $L/K$ is tamely ramified with ramification index $e$ (meaning that $p\not\mid e$). 
	
	Then, there exists $\pi \in m_K / m_K^2$ and a root $\alpha$ of $x^e-\pi=0$ such that $L=K(\alpha)$. 
}

\tbf{Proof:} Let $|x|$ be the absolute value on $\bbC_p$, the completion of the algebraic closure of $\bbQ_p$. Let $\pi_0 \in m_K / m_K^2$, and let $\pi$ be a uniformising parameter such that $|\beta^e|=\pi_0.$ Then, $\beta^e = \pi_0 u$ for some $u \in \mcO_L^\times$. 

Since $f=1$, the extension of the residue fields is trivial. Thus there exists a $u_0 \in U_K$ such that $u = u_0 \Mod{m_L}$. Therefore we can represent $u=u_0+x$, with $x \in m_L$. 

Let $\pi = \pi_0 u_0$. Then, $\beta^e = \pi_0(u_0+x) = \pi + \pi_0x$ and thus $|\beta^e-\pi|<|\pi_0|=|\pi|$. Letting $\alpha_1, \alpha_2, \dots, \alpha_e$ be the roots of $f(x)=x^e-\pi$, I claim that $L=K(\alpha_i)$ for some $i$.

Note that $|\alpha_i|^e = |\pi|$ so the $\alpha_i$ differ by roots of unity. Thus, $|\alpha_i|=|\alpha_j|$ and thus $$|\alpha_i-\alpha_j \leq \max\{|\alpha_i|, |\alpha_j|\}=|\alpha_1|$$

However note $$\prod\limits_{i \neq 1} |\alpha_i - \alpha_1| = |f'(\alpha_1)| = |e\alpha_1^{e-1}| = |\alpha_1|^{e-1}$$ so $|\alpha_i-\alpha_1|=|\alpha_1|$ for all $i \neq 1.$ Since $$\prod\limits_i |\beta-\alpha_i| = |f(\beta)| < |\pi| = \prod\limits_i |\alpha_i|$$ we have that $|\beta-\alpha_i|<\alpha_1|$ for some $i$. 

(n.b.) The following method of using the Galois closure is due to Krasner.

Let $M$ be the Galois closure of $K(\alpha_1, \beta)/K(\beta)$. Letting $\sigma \in \text{Gal}(M/K(\beta))$, we have that for $i\neq 1$, $$|\beta-\sigma(\alpha_1)| = |\sigma(\beta-\alpha_1)| = |\beta-\alpha_1| < \alpha_1| = |\alpha_i-\alpha_1|$$ 

However $$|\alpha_1 - \sigma(\alpha_1)| \leq \max\{|\alpha_1-\beta|, |\beta-\sigma(\alpha_1)|\} < |\alpha_i-\alpha_1|.$$ The implication is that $\sigma(\alpha_1) \neq \alpha_i$ for $i \neq 1$, so $\sigma(\alpha_1)=\alpha_1$ - and by such a mapping, $\alpha_1 \in K(\beta)$ and thus $$K(\alpha_1) \subset K(\beta) \subset L$$ 

However by the Eisenstein criterion, $f(x)$ is irreducible, so $$[L(\alpha_1):L]=e=[L:K]$$ and thus $L=K(\alpha_1). \qed$

\prop{}{
	$$\bbQ_p((-p)^{\frac{1}{p-1}})=\bbQ_p(\zeta_p)$$
}

\tbf{Proof:} Let $\alpha = (-p)^{\frac{1}{p-1}}$. Then $\alpha$ is a root of the polynomial $x^{p-1}+p$. Thus, the extension $$\bbQ((-p)^{\frac{1}{p-1}}) = \bbQ_p(\alpha)$$ is totally ramified of degree $p-1$, and $\alpha$ is a uniformiser. 

Let $\pi = \zeta_p-1$. The minimal polynomial of $\pi$ is $$\frac{(x+1)^p-1}{x} = x^{p-1}+px^{p-2}+\dots+p$$ which is Eisenstein, so $\bbQ_p(\pi)=\bbQ_p(\zeta_p)$ is totally ramified and has degree $p-1$ with $\pi$ being a uniformiser.

Note that $\frac{-\pi^{p-1}}{p}$ is a unit in the ring of the integers of $\bbQ_p(\zeta_p)$. Let $g(x) = x^{p-1}-u$. Then, $g(1)\equiv 0 \Mod{\pi}$ and $g'(1) = p-1 \not\equiv 0 \Mod{\pi}$ so the conditions for Hensel's lemma are satisfied and we can lift $1$ to a root $\beta \in \bbQ_p(\zeta_p)$ of $g(x)$.

From $$-\pi^{p-1}=pu = p\beta^{p-1}$$ we have that $$(\frac{\pi}{\beta})^{p-1}+p=0$$ and therefore $\frac{\pi}{\beta}$ is a root of the minimal polynomial of $\alpha$, thus implying that $\alpha \in \bbQ_p(\zeta_p)$ and that $\bbQ_p(\alpha)=\bbQ_p(\zeta_p). \qed$

Let $L/\bbQ_p$ be tamely ramified, and let $K/\bbQ_p$ be the maximal unramified subextension of $L/\bbQ_p$. Then by our first lemma, $L=K(\pi^{1/e})$ for some $\pi \in m_K/m_K^2$. 

Since $K/\bbQ_p$ is unramified, we can represent $\pi = -up$ for some unit $u \in K$. Combined with the fact that $p \nmid e$, we have that the discriminant of $f(x)=x^e-u$ is not divisible by $p$. Thus, $K(u^{1/e})/K$ is unramified.

In the unramified case we showed that for some $M$, $$K(u^{1/e}) \subset K(\zeta_M) \subset \bbQ_p(\zeta_{Mn}).$$ Let $T=\bbQ_p(\zeta_{Mn})L$. Galois theory tells us that $T/\bbQ_p$ is an abelian extension so $\bbQ_p((-p)^{1/e})/\bbQ_p$ is Galois. ($\pi=-up$, and $u^{1/e}, \pi^{1/e} \in T$) Thus, $\zeta_e \in \bbQ_p((-p)^{1/e})$, which is totally ramified. Therefore $\bbQ_p(\zeta_e)/\bbQ_p$ is too. However $e \nmid p$ so $\bbQ_p(\zeta_e)/\bbQ_p$ is the trivial extension. 

Since $\bbQ_p((-p)^{\frac{1}{p-1}})=\bbQ_p(\zeta_p)$, we conclude that $\bbQ_p((-p)^{1/e}) \subset \bbQ_p(\zeta_p)$ and thus $$L=K(\pi^{1/e}) = K(u^{1/e}, (-p)^{1/e}) \subset \bbQ_p(\zeta_{Mnp}). \qed$$

\tbf{Case 3: Extension is wildly ramified}

Kummer theory studies extensions of the type $L=K(\sqrt[n]{a})$ (called \emph{Kummer extensions}). There are proofs that don't use Kummer theory but they are more long-winded. We first prove the case for $p>2$, then address $p=2$ (a much simpler case). 

\defn{Kummer pairing}{
	Let $K$ be a field with algebraic closure $\bar{K}$ and let $(n, \text{char}(K))=1.$ Also assume that $\zeta_n \in K$. Define the \emph{Kummer pairing} as the map 
	\begin{align*}
		\langle \cdot, \cdot \rangle: \text{Gal}(\bar{K}/K)\times K^\times &\to \langle \zeta_n \rangle \\
		\langle \sigma, a \rangle &\mapsto \frac{\sigma(\sqrt[n]{a})}{\sqrt[n]{a}}
	\end{align*}
}

Let $F$ be a field whose characteristic is coprime to $n$. Let $K=F(\zeta_n)$ and $L=K(\sqrt[n]{a})$ for some $a \in K^\times$. Let $A$ the subgroup of $K^\times / (K^\times)^n$ generated by $a$. The Kummer pairing induces a map $$H \times A \to \langle \zeta_n \rangle$$ that is compatible with the action on $G/H$. 

Define the homomorphism $\omega$ by 

\begin{align*} 
\omega: \text{Gal}(K/F) &\to (\bbZ/n\bbZ)^\times \\
\zeta_n^{\omega(\sigma)} &= \sigma(\zeta_n)
\end{align*}

We have that $$\langle h a^{\omega(\sigma)} \rangle=\langle h,a\rangle^{\omega(\sigma)} =\sigma(\langle h, a\rangle) = \langle h^{\sigma}, \sigma(a) \rangle = \langle h, \sigma(a) \rangle$$ for all $\sigma \in \text{Gal}(K/F)$ and $h \in H$. Since $G$ is abelian, the action on $H$ is trivial, and thus $h^\sigma = h$ and thus $$\sigma(a)/a^{\omega(\sigma)} \in (K^\times)^n$$ for all $\sigma$. We use this result in the following lemma:

\lemma{}{
	Let $p$ be an odd prime. There is no extension $K/\bbQ_p$ such that $$\text{Gal}(K/\bbQ_p) \cong (\bbZ/p\bbZ)^3.$$
}

\tbf{Proof:} Proof by contradiction. Suppose that $K$ is an extension of $\bbQ_p$ with Galois group $G=\text{Gal}(K/\bbQ_p) \cong (\bbZ/p\bbZ)^3$. Then $K/\bbQ_p$ and $\bbQ_p(\zeta_p)/\bbQ_p$ are linearly disjoint extensions as the order of $\text{Gal}(\bbQ_p(\zeta_p)/\bbQ_p) \cong (\bbZ/p\bbZ)^\times$ is not divisible by $p$ but the order of $(\bbZ/p\bbZ)^3$ is. 

By Kummer theory, we can construct a subgroup $$A \subseteq \bbQ_p(\zeta_p)^\times / (\bbQ_p(\zeta_p)^\times)^p$$ isomorphic to $(\bbZ/p\bbZ)^3$ with $$K(\zeta_p)= \bbQ_p(\zeta_p, A^{1/p})$$ with $A^{1/p}$ being the $p$'th roots of the elements $a \in \bbQ_p(\zeta_p)^\times$.

$\bbQ_p(\zeta_p, \sqrt[p]{a})/\bbQ_p$ is abelian. So by the result above we have that $$\sigma(a)/a^{\omega(\sigma)} \in (\bbQ_p(\zeta_p)^\times)^p$$ for all $\sigma \in G$. Note that $\omega$ sends 
\begin{align*}
	G &\to (\bbZ/p\bbZ)^\times \\
	\sigma(\zeta_p) &= \zeta_p^{w(\sigma)}
\end{align*}

Let $\pi = \zeta_p-1$ be a uniformiser. For each $a \in A$ we have that $$v_\pi(a) = v_\pi(\sigma(a)) \equiv \omega(\sigma) v_\pi(a) \Mod{p}$$ and thus $$(1-\omega(\sigma))v_\pi(a) \equiv 0 \Mod{p}$$ for all $\sigma \in G$, implying that for all $\omega(\sigma) \in \omega(G)$, $v_\pi(a) \equiv 0 \Mod{p}$. 

Multiplying both sides by $\pi^{-v_\pi(a)}$, we can take $v_\pi(a)=0$. Then multiply by a suitable power of $\zeta_{p-1}^p$ to conclude that $a \equiv 1 \Mod{\pi}$ since the image of $\sigma_{p-1}$ generates the multiplicative group of the residue field.

Let $U_1 = \{u \equiv 1 \Mod{\pi}\}$. Take $A \subseteq U_1/U_1^p$. Write each $u \in U_1$ as a power series in $\pi$ with constant 1 and integer coefficients in $[0, p-1]$. 

Immediately we know that $\zeta_p \in U_1$ since $\zeta_p = 1+\pi$. For all $b \in [0, p-1]$, we have that $$\zeta_p^b = 1+b\pi+O(\pi^2).$$ Therefore for each $a \in A \subseteq U_1$, choose $b$ such that for some integer $c \in [0, p-1]$ and $e \in \bbZ_{\geq 2}$ we have that $$a = \zeta_p^b (1+c\pi^e + O(\pi^{e+1})).$$

We also have $$\frac{\sigma(\pi)}{\pi} = \frac{\sigma(\zeta_p -1)}{\zeta_p -1}= \frac{\zeta_p^{\omega(\sigma)} -1}{\zeta_p -1} = \zeta_p^{\omega(\sigma)-1}+\dots+1 \equiv 1+1+\dots \Mod{\pi} \equiv \omega(\sigma) \Mod{\pi}$$ 

with $\omega(\sigma)$ an integer $\in [1, p-1]$. Therefore we have $$\sigma(a) = \zeta_p^{b\omega(\sigma)}(1+cw(\sigma)^e\pi^e + O(\pi^{e+1}));$$ however,  $$a^{\omega(\sigma)} = \zeta_p^{b\omega(\sigma)}(1+cw(\sigma)^e\pi^e + O(\pi^{e+1})).$$

Since we can write any $u \in U_1$ as $u = \zeta_p^b u_1$ with $u_1 \equiv 1\Mod{\pi^2}$, each term in the binomial expansion of $u_1^p = (1+O(\pi^2))^p$ is a multiple of $p\pi^2$ with $v_\pi(p \pi^2) = p+1.$ Thus $u^p \equiv 1\Mod{\pi^{p+1}}$.

From $\sigma(a)/a^{\omega(\sigma)} \in (\bbQ_p(\zeta_p)^\times)^p$, we have that $\sigma(a) / a^{\omega(\sigma)} \in U_1^p$. Since we proved that every element of $U_1^p \equiv 1 \Mod{\pi^{p+1}}$, it follows that $\sigma(a) = a^{\omega(\sigma)} (1+O(\pi^{p+1}))$, and therefore $$\sigma(a) \equiv a^{\omega(\sigma)}  \Mod{\pi^{p+1}}.$$

Thus we must either have $e \equiv 1 \Mod{p-1}$ (Why?) and also $e \geq p$. This shows that every $a \in A$ is represented by $\zeta_p^b (1+c\pi^p + O(\pi^{p+1})) \in U_1$ with $b, c\in \bbZ$ and therefore lies in the subgroup $U_1/U_1^p$. This subgroup satisfies $$\langle \zeta_p, 1+\pi^p \rangle \subseteq U_1/ U_1^p$$ which is isomorphic to a subgroup of $(\bbZ/p\bbZ)$. This contradicts the fact that $A \cong (\bbZ/p\bbZ)^3. \qed$

We are left to consider the case where $p=2$ since there are extensions of $\bbQ_2$ with Galois group isomorphic to $(\bbZ/2\bbZ)^3$. These include $\bbQ_2(\zeta_{24})$.

\lemma{}{
	Let $K/\bbQ_2$ be cyclic of degree $2^r$. Then $K$ lies in $\bbQ_2(\zeta_n)$.
}

\tbf{Proof:} The totally ramified extension $\bbQ_2(\zeta_{2^{m+2}})$ has Galois group $$\text{Gal}(\bbQ_2(\zeta_{2^{m+2}})/\bbQ_2) \simeq \bbZ/2\bbZ \times \bbZ/2^m \bbZ.$$ Also the cyclotomic field $\bbQ_2(\zeta_{2^{r+2}})$ has Galois group $\bbZ/2\bbZ \times \bbZ/2^r \bbZ$. Letting $m = (2^{2^{r}}-1)(2^{r+2})$, we have that if $K \not\subseteq \bbQ_2(\zeta_m)$, then $$\text{Gal}(K(\zeta_m)/\bbQ_2) \simeq \begin{cases} \bbZ/2\bbZ \times (\bbZ/2^r \bbZ)^2 \times \bbZ/2^s \bbZ \quad\ & 1 \leq s \leq r\ \text{, or} \\ (\bbZ/2^r \bbZ)^2 \times \bbZ/ 2^s \bbZ \quad \ & 2 \leq s \leq r \end{cases}$$

In each of the above cases we can extract a quotient isomorphic to $(\bbZ/2\bbZ)^4$ or $(\bbZ/4\bbZ)^3$. Proving that no extension of $\bbQ_2$ has either of these Galois group show that $K$ must lie in $\bbQ_2(\zeta_m)$, and the claim is proven. 

I'll prove the first case and leave the second as an exercise. Note that $$\bbQ_2^\times / (\bbQ_2^\times)^2 \simeq \bbZ/2\bbZ \times U_1/U_1^2 \times \{\pm 1\}$$ where $U_1$ is generated by $u \equiv 1\Mod4$ and thus $U_1^2$ is generated by $u \equiv 1\Mod4$.

Therefore $U_1/U_1^2 \simeq \bbZ/2\bbZ$, and thus $$\bbQ_2^\times / (\bbQ_2^\times)^2 \simeq (\bbZ/2\bbZ)^3$$ which shows that the Galois group cannot possibly be $(\bbZ/2\bbZ)^4. \qed$ 

In summary, we've done \emph{class field theory over $\bbQ$ by constructing all the abelian extensions over it.} 

(n.b.) The original proof (developed by Hilbert) of the Kronecker-Weber theorem utilise \emph{higher ramification groups}. 

\section{Profinite groups}

Before we state the main theorems of class field theory, we look at profinite groups and their completions, and other topological constructions. This allows us to understand $\text{Gal}(K^({\text{ab}})/K)$. 

(Remark) The name \emph{profinite group} comes from the term "\tbf{pro}jective limit of \tbf{finite} groups". 

\defn{Profinite group}{
	A profinite group is a topological group that has a basis of neigbourhoods of the identity element consisting of normal subgroups. 
	Let $G$ be a topological group. We can construct a profinite group $\hat{G}$ out of $G$ by taking the profinite completion $$\hat{G} = \lim\limits_{\leftarrow} G_N$$
	
	where the projective (inverse) limit has $N$ range over open normal subgroups of finite index. 
} 

There is categorical description of the profinite completion as a \emph{categorical limit}. Let $\mcN$ be the poset be the category whose objects are finite index normal subgroups $N \trianglelefteq G$ and morphisms $N_1 \to N_2$ if and only if $N_1 \subseteq N_2$. Then, there is a mapping $N \mapsto G/N$. The limit of the diagram 

\[
\begin{tikzcd}
N_1 \arrow{r}\arrow{d} & G/N_1 \arrow{d} \\
N_2  \arrow{r} & G/N_2
\end{tikzcd}
\]

is the profinite completion of $G$.

\ex{}{
	Consider the inverse system $\{\bbZ/\bbZ, \phi_{nm}\}$ where we have the map $\phi_{nm}: \bbZ/n\bbZ \to \bbZ/m\bbZ$ for $m \leq n$. The \emph{profinite completion} of $\bbZ$, denoted $\hat{\bbZ}$, is the inverse limit $$\hat{\bbZ} = \lim\limits_{\leftarrow} \bbZ/n\bbZ = \prod\limits_p \bbZ_p$$ where $n \in \bbN$. This is the set of equivalence classes of integer tuples $(x_1, x_2, x_3, \dots)$ with $x_n \in \bbZ$ for all $n \in \bbZ$, and $x_m = x_n \Mod{m}$ whenever $m \mid n$. 
}

Profinite groups are compact and totally disconnected as it is a closed subset of the product of finite groups. 

\defn{pro-$p$ completion}{
	$p$-groups are groups wherein the order of each element is a power of $p$. The pro-$p$ completion of $G$ is $$G_{(p)} = \lim\limits_{\leftarrow} G/N$$ where the inverse limit runs over all $N \trianglelefteq G$ and $G/N$ is a $p$-group.
}

Let $G$ be a group. We can endow $G$ with the \emph{profinite topology} to obtain a profinite group. 

\defn{}{
	Let $(G_j)_{j \in J}$ be an inverse system (an inverse system has objects $X_i$ and morphisms $X_j \to X_i$ for $i \leq j$ - more specifically we have objects $G/N_i$ and morphisms $G/N_j \to G/N_i$). We let $G_j$ have the discrete topology for each $j \in J$. Endow $\prod_j G_j$ with the product topology. Then the subspace topology on $$\lim\limits_{\leftarrow} G_j \subseteq \prod_j G_j$$ is the \emph{profinite topology}.
}

Let $H \subseteq G$. It is evident that if $G$ is a profinite group, $H$ is also one too. Moreover, they satisfy a universal property. For all $ \phi: G\to \hat{G}$ that sends $g \in G$ to the sequence of its images in the finite quotients $G/N$ and homomorphism $\rho: G\to H$, there exists a unique homomorphism such that the diagram below commutes:

\[
\begin{tikzcd}
    A \arrow{r}{\phi} \arrow[swap]{dr}{\rho} & B \arrow[dashed]{d}{\exists !} \\
     & C
\end{tikzcd}
\]

If $G$ is profinite, then $G \cong \hat{G}$. Groups that have these properties are known as \emph{strongly complete} groups.

\prop{}{
	$G \cong \hat{G}$ if and only if every finite index subgroup of $G$ is open.
}

\tbf{Proof:} ($\Rightarrow$) If every finite index subgroup of $G$ is open, then every normal subgroup with finite index is open, meaning that the topology on $G$ is finer than the profinite topology. Thus we have the same profinite completion under both topologies.

($\Leftarrow$) If $G$ has a finite index subgroup that is not open, then no subgroup of $H$ is open. Thus the intersection of all the conjugates of $H$ is not open in $G$. Any of the subgroups of $G$ are not open either. Then consider the universal property of the profinite compltion. If $G \cong \hat{G}$, the image of $N$ under $\phi: G \to \hat{G}$ is an open subgroup of $\hat{G}$ by construction. This is a contradiction. $\qed$

We mentioned two definitions of the profinite group earlier. We prove that they are equivalent.

\tbf{Proof:} Let $G$ be a profinite group. Let $\{N_i | i \in I\}$ be a set of open normal subgroups indexed by $i$. Construct the $i$ by defining $j \geq i$ if $N_j \subseteq N_i$. Thus, the groups $G_i = G/N_i$ form a disjoint open covering of $G$ which must be finite as $G$ is compact. We have the projections $$g_{ij}: G_j \to G_i$$ and thus have a projective system $\{G_i, g_{ij}\}$ of discrete, compact groups. Consider the homomorphism $$f:G \to \lim\limits_{i\in I} G_i $$ given by $$\sigma \mapsto \prod\limits_{i\in I} \sigma_i$$ where $\sigma_i = \sigma \Mod{N_i}$. I want to show that $f$ is n isomorphism and a homeomorphism.

Note that the kernel of $f$ is $\bigcap\limits_{i\in I} N_i$ which is unity as $G$ is Hausdorff and thus the $N_i$ form a basis of neighbourhoods of 1.

Let $1_{G_i}$ denote the neutral element in $G_i$. The groups $$U_S=\prod\limits_{i \not\in S} G_i \times \prod\limits_{i \in S} 1_{G_i}$$ form a basis of neighbourhoods with identity in $\prod\limits_{i \in S} G_i$. Then since $$f^{-1}(U_S \cap \lim\limits_\leftarrow G_i) = \bigcap\limits_{i \in S} N_i$$ we know that $f$ is continuous. Note that since $G$ is compact, $f(G)$ is closed in the inverse limit. It is also dense as we have that $\{x_i\}_{i \in I} \in \lim\limits_\leftarrow G_i$. Let $x=\{x_i\}_{i \in I} $. Then $x(U_S \cap \lim\limits_\leftarrow G_i)$ is a fundamental neighbourhood of $x$.

Now choose a $y \in G$ that maps to $x_k$ under $G \to G/\bigcap\limits_{i\in S} N_i$. Then $y \Mod{N_i} = x_i$ for all $i \in S$ so $f(y) \in x(U_S \cap \lim\limits_\leftarrow G_i)$. 

Thus the closed set $f(G)$ is dense in the inverse limit, so $f(g) = \lim\limits_\leftarrow G_i$. The compactness of $G$ allows us to conclude that $f: G\to \lim\limits_\leftarrow G_i$ is an isomorphism and a homeomorphism.

Note that the $G_i$ are Hausdorff and compact. Therefore the projective limit $G = \lim\limits_\leftarrow G_i$ is compact by Tychonoff's theorem. Note that the groups $$U_S = \prod\limits_{i \not\in S} G_i \times \prod\limits_{i \in S} N_i$$ make up a basis of neighbourhoods of the neutral element in $\prod\limits_{i \in I}$ consisting of normal subgroups. 

Therefore the normal subgroups $U_S \cap \lim\limits_\leftarrow G_i$ form a basis of neighbourhoods of the neutral element and hence $\lim\limits_\leftarrow G_i$ is a profinite group. $\qed$

\ex{}{
	Let $K$ be local field. Its additive group and its multiplicative group are profinite.
	
	$\hat{\bbZ}$ is a profinite group. Furthermore $\text{GL}_n(\hat{\bbZ})$ is also profinite.
}

\ex{}{
	The idele class group of the 1-ideles $C_K^1=\bbI_K^1 / K^\times$ is a profinite group as it is disconnected and a compact group.
}

The Fundamental Theorem of Galois theory tells us that there is a \emph{Galois correspondence}, or a bijection between the lattice structures of the intermediate extensions of $L/K$ and the subgroups of $\text{Gal}(L/K)$. However, when $L/K$ is infinite, the Galois correspondence fails. Thus the subgroups of $\text{Gal}(L/K)$ may not fully correspond to subextensions of $L/K$. 

\ex{$\text{Gal}(\bar{\bbF}_p/\bbF_p)$}{
	The absolute Galois group $G=\text{Gal}(\bar{\bbF})_p/\bbF_p$ contains the Frobenius automorphism given by $$\varphi: x \mapsto x^p \quad \text{for all}\ x \in \bar{\bbF}_p.$$
	
	The subgroup $(\varphi) = \{\varphi^n \mid n\in \bbZ\}$ has the same fixed field $\bbF_p$ as $G$. However note that unlike in the case where the field extension is finite, $(\varphi) \neq G.$
	
	Construct a sequence $\{ a_n\}_{n \in \bbN}$ of integers satisfying $$a_n \equiv a_m \Mod{m}$$ whenever $m \mid n$, but there is no $a \in \bbZ$ satisfying $a_n \equiv a \Mod{n}$ for all $n \in \bbN$. 
	
	Let $$\psi_n = \varphi^{a_n}/\bbF_{p^n} \in \text{Gal}(\bbF_{p^n}/\bbF_p).$$ If $\bbF_{p^m} \subseteq \bbF_{p^n}$, then $m \mid n$. Thus $a_n \equiv a_m \Mod{m}$ and thus $$\psi_n / \bbF_{p^m} = \varphi^{a_n}/\bbF_{p^m}= \varphi^{a_m}/\bbF_{p^m} = \psi_m.$$
	
	Since $\varphi_n/\bbF_{p^m}$ has order $m$, $\psi_n$ defines an automorphism of $\bar{\bbF}_p$. Then $\psi \not\in (\varphi)$ as $\psi = \varphi^a$ for all $a \in \bbZ,$ implying that $$\psi/\bbF_{p^n} = \varphi^{a_n}/\bbF_{p^n}=\varphi^a/\bbF_{p^n}$$
	
	This suggests that $a_n \equiv a\Mod{n}$, contradiction. 
}

\ex{}{
	Let $$K = \bbQ(\sqrt{2}, \sqrt{3}, \sqrt{5}, \dots)$$ so $\text{Gal}(K/\bbQ) = \prod \bbZ/2\bbZ$. Note $\text{Gal}(K/\bbQ)$ is uncountable so it has uncountably many subgroups of order 2. However it has only countably many subfields of $2^k$ over $\bbQ$ for each $k$. 
	
	Thus, the subfields of $K$ and the subgroups of $K/\bbQ$ are not in bijection.
}

Let $L/K$ be Galois. It embeds a canonical topology called the \emph{Krull topology}. We can endow infinite Galois extensions with the Krull topology to distinguish which infinite Galois groups do have a Galois correspondence. This correspondence is defined between intermediate fields and closed subgroups of the Galois group. 

\defn{Krull topology}{
	Let $\sigma \in \text{Gal}(L/K)$. Take the cosets $\sigma \text{Gal}(L/M)$ as a basis of neighbourhoods of $\sigma$, where $M/K$ is finite. A non-empty subset of $L/K$ is an \emph{open set} when each element of $U$ is contained in $$\sigma \in \sigma \text{Gal}(L/M) \subset U.$$ These open sets, alongside the empty set, define a topology on $\text{Gal}(L/K)$.
}

\ex{}{
	Let $$K = \bbQ(\sqrt{2}, \sqrt{3}, \sqrt{5}, \dots)$$ as used in the previous example. The Krull topology on this product is the product topology where each copy of $\{\pm 1\} \cong \bbZ/2\bbZ$ is given the discrete topology.
}

Here are some properties of the Krull topology.

\prop{}{
	Let $L/K$ be a Galois extension. $\text{Gal}(L/K)$ is compact and Hausdorff with respect to the Krull topology.
}

\tbf{Proof:} We first prove that it is Hausdorff. Let $\sigma, \tau$ be two elements of $\text{Gal}(L/K)$. Then, there is an $\alpha \in L$ such that $\sigma(\alpha) \neq \tau(\alpha)$. Let $\alpha$ be contained in a finite extension $M$ of $K$, and name $M=K(\alpha)$. 

Thus, $\sigma|_F \neq \tau|_F$ and thus $\sigma \text{Gal}(L/F) \neq \tau \text{Gal}(L/F)$. This means that $\sigma \text{Gal}(L/F)$ and $\tau \text{Gal}(L/F)$ are disjoint open sets around $\sigma$ and $\tau$, proving that $\text{Gal}(L/K)$ is Hausdorff. $\qed$

The compactness proof is more involved. The idea is to consider the mapping $$f:\text{Gal}(L/K) \to \prod\limits_F \text{Gal}(F/K)$$ where $$\sigma \mapsto \prod_F \sigma |_F.$$ By this we embed $\text{Gal}(L/K)$ as a closed subset of a product of compact spaces with the product topology, from which we can apply Tychonoff's theorem.  

Note that $f$ is injective. This is because $L$ is covered by finite Galois extensions, meaning that if $f(\sigma)=f(\tau)$, then $\sigma |_F = \tau |_F$ for all intermediate extensions $F/K$. Thus, $\sigma = \tau$ on all intermediate Galois extension of $K$ in $L$, meaning $\sigma=\tau$ on $L$.

Let $F_0$ be an intermediate field of $F/K$ with finite Galois group. Consider the set $$U= \prod\limits_{F\neq F_0} \text{Gal}(F/K) \times \{\sigma_0\}$$ form a subbasis of open sets of $\prod\limits_F \text{Gal}(F/K)$ where $F/K$ varies over the finite subextensions of $L/K$ and $\sigma_0 \in \text{Gal}(F_0/K)$. Let $\sigma$ be a preimage of $\bar{G}$. Then, $f^{-1}(U) = \sigma G(L/F_0)$, meaning that $f$ is continuous. Evidently $$f(\sigma G(L/F_0)) = f(G) \cap U$$ so $f$ is open and thus a homeomorphism. It remains for us to show that $f(G)$ is closed in $\prod\limits_K \text{Gal}(F/K)$. Let $M \subseteq M'$ be a pair of finite Galois subextensions of $L/K$.  Then the set $$V_{M'/M} = \left\{\prod\limits_F \sigma_F | \prod\limits_F \sigma_F \in \prod\limits_F \text{Gal}(F/K)\right\}$$ satisfying $\sigma_{L'} |_L = \sigma_L$ has the property $$f(g) = \bigcap\limits_{M \subseteq M'} V_{M'/M}.$$

So now we show that $V_{M'/M}$ is closed. Let $\{\sigma_1, \dots, \sigma_n\}$ be the set of automorphisms making up $\text{Gal}(M/K)$. Then letting $S_i \subseteq \text{Gal}(M'/K)$ be the set of extensions of $\sigma_i$ to $M'$, we have that $$V_{M'/M} = \bigcup\limits_{i=1}^n \left( \prod\limits_{F \neq M, M'} \text{Gal}(F/K) \times S_i \times \{\sigma_i \}\right)$$ which is a union of closed sets. Thus $V_{M'/M}$ is closed. $\qed$

\prop{}{
	Let $F/K$ be a normal subextension of $L/K$, which is Galois and has Galois group $G = \text{Gal}(L/K)$. Then, $H= \text{Gal}(L/F)$ is a normal subgroup of $G$ with fixed field $F$, and there is a short exact sequence of profinite groups $$1 \to \text{Gal}(L/F) \to \text{Gal}(L/K) \to \text{Gal}(F/K) \to 1.$$ Furthermore we also have that $$G/H \cong \text{Gal}(F/K).$$
}

\tbf{Proof:} Our main goal here is to exhibit the canonical homomorphism $\varphi: \text{Gal}(L/K) \to \text{Gal}(M/K)$. Given $\tau: L\to L \in \text{Gal}(L/K)$, the restriction map $\tau |_M: M \to M$ is an element of $\text{Gal}(M/K)$ as $\tau(M)=M$ is a subfield of the algebraic closure of $L$. This defines the homomorphism. It is continuous as the Krull topology is endowed with a universal property. Note that $$\text{Gal}(L/K) \times F \to F$$ given by $$(\tau, x) \mapsto \tau(x) = \varphi(\tau(x))$$ is continuous as $F \subset L$ and $\text{Gal}(L/K) \times L \to L$ is continuous. Thus, $\varphi$ is continuous. $\qed$

Furthermore this homomorphism has kernel $\text{Gal}(L/F)$. Denote it by $H$. $H$ contains $F$ and must be equal to $F$. Let $L^H$ be the fixed field defined as $$L^H = \{\alpha \in L: h(\alpha)= \alpha\}\ \text{for all}\ h \in H.$$ If we had $\alpha \in L^H-F$ we could construct $H$ that sends $\alpha$ to a distinct root $\alpha' \neq \alpha$ of its minimal polynomial $f$ over $F$.  

Note that the map is also surjective as we can extend any $\sigma \in \text{Gal}(F/K)$ to $\text{Gal}(L/K)$, and thus $$G/H \cong \text{Gal}(F/K).$$

\thrm{Fundamental theorem of Galois theory for infinite extensions}{
	Let $L/K$ be a Galois extension and let $G=\text{Gal}(L/K)$ have the Krull topology and thus is a profinite group. There is a bijection between the maps $$\{F \mapsto \text{Gal}(L/F)\} \longleftrightarrow \{H \mapsto L^H\}$$ where $F/K$ is a subextension of $L/K$ and $H$ is a closed subgroup of $G$.
	
	Additionally, $L^{\text{Gal}(L/F)}=F$, and $\text{Gal}(L/L^H)=H$ when $H$ is closed.
}

\tbf{Proof:} Let $S \subset L$ be a finite subset. There is a field extension $E/K$ such that $K(S) \subset E$ and that $E/K$ is finite Galois. Thus, $L/K$ is the union of its finite Galois subextensions. By our short exact sequence above, the map $$\text{Gal}(L/K) \to \text{Gal}(E/K)$$ is surjective and continuous. Thus, $L^G=K$ as no element of $L-K$ is fixed by $\text{Gal}(L/K)$. Note that by the Krull topology on $G$, the group $\text{Gal}(L/M)$ is closed. Thus, $$L^{\text{Gal}(L/M)} =M.$$

Let $H \subset G$ be closed. Suppose that $g \in \text{Gal}(L/L^H)$. Let $E/K$ be a finite subextension containing $K(S)$, and consider the homomorphism $$\varphi: \text{Gal}(L/K) \to \text{Gal}(E/K).$$ It is clear that $L^H \cap E = E^{\varphi(H)}$. Since $g$ fixes $L^H$, it fixes $E^{\varphi(H)}$.

Let $h \in H$ be such that $\varphi(h)=\varphi(g)$, and define the open neighbourhood $U_S(g) = \{g' \in G | g'(s) =g(s)\}$. This implies that $h \in U_S(g)$. But this also means that $g \in H$ as $U_S(g)$ meets $H$, establishing the Galois correspondence. $\qed$

\thrm{Waterhouse}{
	Let $L/K$ be Galois. Every profinite group $G$ is isomorphic to the Galois group of $L/K$.
}

This theorem involves the claim that every subgroup of automorphisms is Galois. More specifically, let $L$ be a field, let $G$ be a finite subgroup of $\text{Aut}(L)$, and let $K$ be the fixed field of $G$. Then, a result of Artin shows that $$[L:K]=|G|.$$ 

\section{Cohomology of groups}

The main theorems of class field theory can be formulated without cohomology. This was done in an ideal-theoretic formulation which culminated roughly at the development of Artin reciprocity and Takagi's main works. It was Hochschild who introduced cohomology into class field theory circa. 1950s. Local class field theory was first developed by Brauer groups which were defined through central simple algebras. This process was tedious and could be interpreted using cohomology groups - a much stronger engine. Cohomology also gives much topological insight which is convenient to have.

The goal of this section is not to provide a comprehensive introduction to the homological algebra, but rather to provide a concise treatment of the main ideas of group and Galois cohomology as it is used in class field theory.

\defn{$G$-module}{
	Let $G$ be a group. A $G$-module $A$ is an abelian group $M$ with a map $$G\times M \to M \quad (g, m) \mapsto gm$$
	
	such that for all $g, g' \in G$ and $m, m' \in M$, \begin{enumerate}
		\item $g(m+m') = gm+gm'$
		\item $(gg')(m)=g(g'm)$
		\item $1m = m$
	\end{enumerate}
}

A $G$-homomorphism is exactly what you'd think it is. It's a map $\phi: M\to N$ such that for all $g \in G, m \in M$, \begin{enumerate}
	\item $\phi(m+m') = \phi(m)+\phi(m')$
	\item $\phi(gm) = g(\phi(m))$
\end{enumerate}

Write $\text{Hom}_G(M, N)$ for the set of $G$-homomorphisms $\varphi:M\to N$. Explicitly this is a $G$-module with $$(\varphi+\varphi')(m) = \varphi(m)+\varphi'(m)$$ $$(g\varphi)(m) = g(\varphi(g^{-1}m)).$$

Furthermore, if $A$ is a $G$-module, defined the set of \emph{$G$-invariant elements} of $A$ by $$A^G: \{a \in A: ga =a\ \text{for all}\ g\in G\}.$$

There is a dual to $G$-invariant elements, namely the $G$-coinvariant elements, denoted $A_G$. It is the maximal quotient of $M$ that fixes $G$, and is defined by $M/ \langle mg-m \rangle$ for all $g\in G, m \in M$. This is equivalent to taking the quotient of $A$ by its \emph{augmentation ideal} $I_G$, generated by $\{g-1 \mid g\in G\}$. 

$G$-modules are modules over rings. 

\defn{Group ring}{
	The group ring, denoted $\bbZ[G]$, is the free abelian group consisting of the elements of $G$, namely $$\bbZ[G] = \{\sum\limits_{\sigma \in G} n_\sigma \sigma | n_\sigma \in \bbZ\}.$$
	
	Let $A$ be a $G$-module over $\bbZ[G]$. For $a \in A$, the action of $\bbZ[G]$ on $A$ is $$\left(\sum\limits_{\sigma \in G} n_\sigma \sigma \right) = \sum\limits_{\sigma \in G} n_\sigma(\sigma a).$$
}

The category $\textsf{Mod}_G$ of $G$-modules can be identified with the category of modules over $\bbZ[G]$.

\thrm{}{
	Let $A$ be a $G$-module which is composed of the direct sum of $G$-modules isomorphic to $\bbZ[G]$.
Then the short exact sequence $$1 \to X \xrightarrow{h} Y \xrightarrow{g} Z \to 1$$ where $X, Y, Z$ are $G$-modules and $h, g$ are homomorphisms induces the exact sequence $$1 \to \text{Hom}(X, A) \to \text{Hom}(Y, A)\to \text{Hom}(Z, A) \to1.$$
}

\tbf{Proof:} Let $A_i$ be a family of $G$-modules, and let $X$ be a $G$-module. There is a canonical isomorphism $$X \otimes (\bigoplus\limits_i A_i) \cong \bigoplus\limits_i (X \otimes A_i)$$ that sends $$\text{Hom}_G(\bigoplus\limits_i X_i, A) \cong \prod\limits_i \text{Hom}_G(X_i, A)$$ and by taking $X = \bigoplus\limits_i X_i$ with $X_i \cong \bbZ[G]$, we have that $$\text{Hom}_G(X, A) = \prod\limits_i \text{Hom}_G(X_i, A).$$

Set $$A_i = \text{Hom}_G(X_i, A) \cong \text{Hom}(\bbZ[G], A).$$ Note that $\text{Hom}(\bbZ[G], A) \cong A$ as for any $\theta \in \text{Hom}(\bbZ[G], A)$ we have $f(1) \in A$. 

Defining $B_i$ and $C_i$ analogously, we retrieve the exact sequence $$1 \to A_i \to B_i \to C_i \to 1. \qed$$

Furthermore, if $$1 \to A \to B \to C \to 1$$ is an exact sequence of $\bbZ$-modules and $X$ is the direct sum of $G$-modules isomorphic to $\bbZ[G]$, we have that $$1 \to X \otimes A \to X \otimes B \to X\otimes C \to 1$$ is an exact sequence. This is shown by letting $X = \bigotimes\limits_i Z_i$ with each $Z_i \cong \bbZ$. We have the canonical isomorphism $$X \otimes A \cong \bigoplus\limits_i (Z_i \otimes A) \cong \bigoplus\limits_i A_i$$ with $B_i = Z_i \otimes A \cong A$. Define similar constructions for $B$ and $C$. This implies that the exactness of $1 \to A \to B \to C \to 1$ implies the exactness of $1 \to X \otimes A \to X \otimes B \to X\otimes C \to 1. \qed$

\defn{Induced and coinduced $G$-modules}{
	Let $M$ be an $\bbZ[H]$-module. Define the \emph{induced module} $\text{Ind}_H^G(M)$ to be the set of maps $\varphi:G \to M$ such that $\varphi(hg)=h\varphi(g)$ for all $h \in H$. Then $\text{Ind}_H^G(M)$ is an induced $G$-module with the operations 
	
	\begin{align*}
		(\varphi+\varphi')(x) &= \varphi(x)+\varphi'(x) \\
		(g\varphi)(x) &= \varphi(xg)
	\end{align*}
	
	More specifically, $$\text{Ind}_H^G(M) = M\otimes_{\bbZ[H]} \bbZ[G]$$ which also defines an exact functor $$\text{Ind}_H^G(M): \textsf{Mod}_H \to \textsf{Mod}_G.$$
	
	Define the \emph{coinduced} $G$-module as being of the form $$\text{CoInd}^G_H(M) = \text{Hom}_{\bbZ[H]} (\bbZ[G], M).$$ 
}

The relation $\text{Ind}_H^G(M) = M\otimes_{\bbZ[H]} \bbZ[G]$ might seem contrived. To see this isomorphism, let $M$ be a $\bbZ[G]$-module. Define $$\phi: M\otimes_{\bbZ[H]} \bbZ[G] \to \text{Ind}_H^G(M)$$ given by the $$\phi(g \otimes a)= g \otimes g^{-1}a.$$ For $g, k \in G$, we have that $$k \cdot \phi(g \otimes a) = (kg) \otimes g^{-1} a = \phi(kg \otimes ka) = v(k \cdot(g \otimes a))$$ which shows that $\phi$ is a homomorphism with inverse map $$\phi^{-1}(g\otimes a ) = g \otimes ga. \qed$$

Exactness of the functor shows that an exact sequence $0 \to A \to B \to C \to 0$ implies the exactness of $0 \to \text{Ind}_H^G A \to \text{Ind}_H^GB \to \text{Ind}_H^GC \to 0$, which is clear. 

A $G$-module $M$ is induced if it has the form $\text{Ind}_{\{1\}}^G N$ for some abelian group $N$, where $\{1\}$ is the trivial group. This means we can write $$M = N\otimes_\bbZ \bbZ[G].$$ 

Just like many cohomological constructions, there is a universal property with the induced modules. Let $\phi$ denote the map $\text{Ind}_H^G(N)$. There is a map $\text{Hom}_G(M, \text{Ind}_H^G(N)) \simeq \text{Hom}_H(M, N)$ (check where the identity $1_G$ is mapped) which yields the following commutative diagram

\[
\begin{tikzcd}
    M \arrow[r, dashed, "\exists !", "\alpha"'] \arrow[dr, swap, "\beta"] & \text{Ind}_H^G(N) \arrow{d}{\phi} \\
     & N
\end{tikzcd}
\]

There exists a unique homomorphism $\alpha: M\to \text{Ind}_H^G(N)$ such that $\phi \circ \alpha = \beta$. 

\defn{Injective $G$-modules}{
	Let $M$ be a $G$-module. $M$ is injective if for every inclusion $A \subset B$ of $G$-modules, every homomorphism of $G$-modules $\phi: A\to M$ extends to $B$ such that there exists a unique factorisation $\eta: B\to M$. In other words, $$\text{Hom}_G(\cdot, M)$$ is an exact functor.
}

Moreover, the following diagram commutes:

\[
\begin{tikzcd}
    A \arrow[r] \arrow[dr, dashed, "\eta", "\exists!"'] & M \arrow{d} \\
     & B
\end{tikzcd}
\]

\prop{}{
	Every $G$-module is the sub-object of some injective. Thus, we can inject $G$-modules $M$ into an injective $G$-module $I$.
}

\tbf{Proof:} Let $G = \{e\}$. We first show that the abelian category $\textsf{Ab}$ has enough injectives. This is clear as for an abelian group, let the dual $M^\vee = \text{Hom}(M, \bbQ/\bbZ).$ Choose a free abelian group $F$ that surjects onto $M^\vee$. Then $M$ injects onto $F^\vee$ through $$M \xhookrightarrow\ M^{\vee\vee} \xhookrightarrow\ F^{\vee}$$ thus making $F^\vee$ injective.

Let $M$ be $G$-module, and let $M_0$ be the group structure of $M$. Embed $M_0$ into an injective abelian group through $M_0 \xhookrightarrow\ I$. 

Define $\text{Ind}^G(M)$ as the hom-set $\text{Hom}(\bbZ[G], M)$ (where we drop the subscript $H$). Then we can append the functor $\text{Ind}^G$ to obtain an inclusion $$\text{Ind}^G(M_0) \xhookrightarrow\ \text{Ind}^G(I).$$

Append this inclusion map with the inclusion $M \xhookrightarrow\ \text{Ind}^G(M_0)$, which sends $m$ to the mapping $g \mapsto gm$. The composition of both maps yields an injective homomorphism $M \xhookrightarrow\ \text{Ind}^G(I)$. But then $\text{Ind}^G(I)$ is an injective $G$-module as there is a natural isomorphism $$\text{Hom}_{\textsf{Mod}_G}(M, \text{Ind}^G(I)) \cong \text{Hom}_{\textsf{Ab}}(M, I)$$ which shows that the hom-set $$\text{Hom}_{\textsf{Mod}_G}(\cdot, \text{Ind}^G(I))$$ is exact, and every $M \xhookrightarrow\ I$ ($\textsf{Mod}_G$ has enough injectives). $\qed$

Let $\mcC$ be an abelian category with enough injectives. Let $F: \mcC \to \mcD$ be a left exact functor between abelian categories. A short exact sequence in $\mcC$: $$0 \to M' \to M \to M'' \to 0$$ gives rise to an exact sequence 

\defn{Resolution}{
	A \emph{resolution} is an exact sequence of $R$-modules $$0 \to M \to R_0 \xrightarrow{d_0}\ R_1\to \dots \to R_r \xrightarrow{d_r}\ R_{r+1} \to \dots $$ with dual $$\dots \to R_{r+1} \xrightarrow{d_r}\ R_r \to \dots  \to R_1 \xrightarrow{d_0}\ R_0 \to M \to 0.$$ 
}

If the $R_i$ are injective objectives of $\mcC$, then the resolution is injective. Denote the injective resolution by $M \to I^\bullet$. 

\prop{}{
	Let $M \to I^\bullet$ and $M \to J^\bullet$ be injective resolutions of $M$. There exists a homomorphism from $M \to I^\bullet$ to $M \to J^\bullet$. 
}

\tbf{Proof:} Consider the commutative diagram 

\[
\begin{tikzcd}
0 \arrow{r} &M \arrow{r}& I^0 \arrow{r} \arrow{d} &I^1 \arrow{r} \arrow{d} &\dots \\
0 \arrow{r} &M \arrow{r} &J^0 \arrow{r} &J^1 \arrow{r} &\dots
\end{tikzcd}
\]

There exists an injective morphism $$0 \to M \to I^0$$ with $I^0$ being injective. Let $B^1$ be the cokernel of $M \to I^0$. Thus there exists a map $0 \to B^1 \to I^1$ with $I^1$ injective.

Continue by defining cokernels $$B^{n+1} = \text{Coker}(B^1 \to I^1)$$ and we are done. $\qed$

\defn{Cochains, cocycles, and coboundaries}{
	Let $M$ be a $G$-module. The group of \emph{$i$-cochains} of $G$ with coefficients in $A$ is the set of \emph{$i$th cocyle maps} $$C^i(G, M) = \{f: G^i \to M\}$$ which yields the \emph{$i$'th differential map}, defined by $$d_M^i: C^i(G, M) \to C^{i+1}(G, M).$$ Explicitly we have that for a tuple $(g_0, \dots, g_r) \in G$, that \begin{align*} d^i(f)(g_0, \dots, g_i) = &g_0\cdot f(g_1, \dots, g_i)\\  &+ \sum\limits_{j=1}^i (-1)^jf(g_0, \dots, g_{j-2}, g_{j-1}g_j, g_{j+1}, \dots, g_i)+(-1)^{i+1}f(g_0, \dots, g_{i-1}). \end{align*}
	
	Set $Z^i(G, M) = \text{ker}(d^i)$ to be the group of \emph{$i$-cocycles} of $G$ with coefficients in $M$.
	Set $B^i(G, M) = \text{Im}(d^{i-1})$ to be the group of \emph{$i$-coboundaries} of $G$ with coefficients in $M$.
}

Note that $B^0(G, M)=0$. Furthermore we have that $(C^i(G, M), d^i)$ is a cochain complex as the composition of any two adjacent $d^i$ is 0. Since $d^i \circ d^{i-1} =0$ for all $i\geq 1$, we have that $$B^i(G, M) \subseteq Z^i(G, M)$$ for all $i \geq 0$. 

\defn{Cohomology group}{
	Let $M^G$ be the set $$ M^G: \{m \in M \mid gm = m \ \text{for all} g \in G\}.$$ The functor $M \mapsto M^G$ is left exact (but not right exact). If we have an exact sequence $$0 \to M' \to M \to M'' \to 0$$ of $G$-modules, we obtain that $$0 \to (M')^G \to M^G \to (M'')^G$$ is exact. Note that the map $M^G \to (M'')^G$ may not be surjective so $0 \to (M')^G \to M^G \to (M'')^G \to 0$ need not be a short exact sequence. 
	
	Construct an injective resolution $$0 \to M \to I^0 \xrightarrow{d^0}\ I^1 \xrightarrow{d^1}\ I^2 \xrightarrow{d^2}\ \dots$$ which yields the complex $$0 \xrightarrow{d^{-1}} (I^0)^G \xrightarrow{d^0}\ (I^1)^G \to \dots \xrightarrow{d^{i-1}}\ (I^i)^G \xrightarrow{d^i}\ (I^{i+1})^G  \to \dots$$
	
	The \emph{$r$'th cohomology group} of $G$ is $$H^i(G, M) = \frac{\text{Ker}(d^i)}{\text{Im}(d^{i-1})}.$$
}

\ex{Alternate formulation of $H^1(G, M)$}{
	$Z^1(G, M)$ denotes the set of all $f:G\to M$ satisfying $f(gh) = gf(h)+f(g)$ for all $g, h \in G$. We call this the \emph{crossed homomorphism}.
	Let $m \in M$. The map $\sigma \mapsto \sigma m - m$ is the \emph{principal crossed homomorphism}, and is equivalent to $B^1(G, M).$
	
	Thus, $$H^1(G, M)= \frac{\{\text{crossed homomorphisms} G\to M\}}{\{\text{principal crossed homomorphisms}\}}.$$
}

\ex{$H^i$ for small $i$}{
	$H^0(G, M)$ is equal to $M^G$. Furthermore if $A$ is a trivial $G$-module, then $H^1(G, M) = \text{Hom}(G, M)$.
}

The maps $d^i$ are equivalent to the differential maps above. Moreover, set $$Z^i(G, M) = \text{Ker}(d^i)$$ to be the group of \emph{$i$-cocyles}, and $$B^i(G, M) = \text{Im}(d^{i-1})$$ to be the set of \emph{$i$-coboundaries} of $G$. Then, the $i$th cohomology group is $$H^i(G, M) = \frac{Z^i(G, M)}{B^i(G, M)}.$$

We say that $M$ is \emph{acyclic} if $H^i(G, M)=0$ for all $i>0$. Thus, injective $G$-modules are acyclic.

\lemma{Shapiro}{
	Let $H \leq G$ and $N$ be an $H$-module. There is a canonical isomorphism $$H^i(G, \text{Ind}_H^G(N)) \to H^i(H, N)$$ for all $i \geq 0$.
}

\tbf{Proof:} Any homomorphism $\alpha: \bbZ \to M$ is uniquely determined by $\alpha(1)$. Thus $m$ is the image of $1$ under $\alpha$ if and only if it is fixed by $G$. Thus, $\text{Hom}_G(\bbZ, M) \cong M^G$. 

Consider the case $i=0$. We want to show that $$\text{Ind}_H^G(N)^G \cong N^H.$$ But this is clear as by the properties of $\text{Ind}_H^G(M)$, we have that $$\text{Ind}_H^G(N)^G = \text{Hom}_{\bbZ[G]}(\bbZ, \text{Ind}_H^G(M)) \cong \text{Hom}_{\bbZ[H]}(\bbZ, N) = N^H.$$

Consider the general case. Let $0\to N \to I^\bullet$ be an injective resolution of $N$. Applying the functor $\text{Ind}_H^G$, we claim that $0 \to \text{Ind}_H^G(N) \to \text{Ind}_H^G(I^\bullet)$ is an injective resolution of $\text{Ind}_H^G(N)$.

Recall that the functor $\text{Ind}_H^G: \textsf{Mod}_H \to \textsf{Mod}_G$ is exact. Furthermore, we have that for each $j \geq 0$, $$\text{Hom}_{\bbZ[G]}(\cdot, \text{Ind}_H^G(I^j)) \cong \text{Hom}_{\bbZ[H]}(\cdot, I^j)$$ which accompanied with the fact that $I^j$ is injective, shows that $\text{Hom}_{\bbZ[G]}(\cdot, \text{Ind}_H^G(I^j))$ is exact. This implies that $$H^i(H, N)  = H^i((I^\bullet)^H) \cong H^i(\text{Ind}_H^G(I^\bullet)^H)=H^i(G, \text{Ind}_H^G(N)),$$ as desired. $\qed$

\coro{}{
	$H^i(G,M)$ is acyclic, meaning that $H^i(G, M)=0$ for $i>0$.
}

\tbf{Proof:} Let $M=\text{Ind}^G(M_0)$. Then, $$H^i(G, M) \cong H^i(\{1\}, M_0) = \text{Ext}_{\bbZ[\{1\}]}^i(\bbZ, M_0) = \text{Ext}_\bbZ^i(\bbZ, M)=0. \qed$$ 

In our definition of the induced module, we looked at both induced and co-induced modules. These definitions are equivalent for finite groups. We claim that there exists a canonical isomorphism of $G$-modules $$\phi: \text{CoInd}_H^G(M) \cong \text{Ind}_H^G(M)$$ given by $$\phi(\varphi) = \sum\limits_{g' \in H \backslash G} g^{-1} \otimes \varphi(g)$$ where $g \in G$ is a representative of $g'$.

To prove this, first note that $\phi$ is well-defined. Note that for $\varphi \in \text{CoInd}_H^G$, we have that $$(hg)^{-1} \otimes \varphi(hg) = g^{-1} h^{-1} \otimes \varphi(hg) = g \otimes \varphi(g).$$

Let $h \in H$. For $g'' \in G$ we have that $$\phi(g'\varphi) =\sum\limits_{g' \in H \backslash G} g^{-1} \otimes \varphi(gg'')= g'' \sum\limits_{g' \in H \backslash G} (gg'')^{-1} \otimes \varphi(gg'') = g''\phi(\varphi)$$

These coset representatives form a basis for $\bbZ[G]$ as a free $\bbZ[H]$-module. Thus the inverse mapping $\phi^{-1}$ sends $$\sum\limits_{g' \in H \backslash G} g^{-1} \otimes b_g$$ to the unique map $\varphi$ that sends $g'$ to $b_g. \qed$

\thrm{}{
	Let $L/K$ be Galois. For $i>0$, $$H^i(\text{Gal}(L/K), L) = 0.$$
}

\tbf{Proof:} Let $\alpha \in L$. The \emph{normal basis theorem} states that $\{\sigma \alpha \mid \sigma \in G\}$ is a basis for $L$ as a $K$-vector space. Lett $(\sigma\alpha)_{\sigma \in G}$ be a normal basis, and define an isomorphism of $G$-modules $K[G] \to L$ given explicitly by $$\sum\limits_{\sigma \in G} a_\sigma \sigma \mapsto \sum\limits_{\sigma \in G} a_\sigma \sigma \alpha.$$

Note that $K[G] = \text{Ind}^G(K)$ so apply Shapiro's lemma to conclude that for all $i>0$, $$H^i(G, L) \cong H^i(\{1\}, K)=0. \qed$$

If $A$ is a trivial $G$-module, the principal crossed homomorphism is the trivial map as $mg-m = m-m=0$. In this case we have the following corollary:

\coro{}{
	Let $G$ be a group and $M$ be a trivial $G$-module. Then, $$H^1(G, M) \cong \text{Hom}(G, M).$$
}

Let the norm map $$N_G: A\to A$$ send $a \mapsto \sum\limits_{g \in G} ga$. By Cayley's theorem, the norm map sends $A$ into $A^G$, thus turning a general element of $A$ into a $G$-invariant element. 

\thrm{Alternate description of $H^1(G, M)$}{
	Let $\sigma \in G$. Then, $$H^1(G, M) = \frac{\text{Ker}(N_G)}{(1-\sigma)M}$$
}

\tbf{Proof:} By the properties of crossed homomorphisms, 

\begin{align*} 
	\varphi(\sigma^2) &= \sigma \varphi(\sigma)+\varphi(\sigma) \\
	\varphi(\sigma^3) &= \varphi(\sigma \cdot \sigma^2)= \sigma^2 \varphi(\sigma)+\sigma\varphi(\sigma)+\varphi(\sigma) 
\end{align*}

and thus $$\varphi(\sigma^r) = \sigma^{r-1} \varphi(\sigma)+\dots+\sigma\varphi(\sigma)+\varphi(\sigma) = (\sigma^{k-1}+\dots+ \sigma^0)\varphi(\sigma).$$ This yields the map $$\phi: \{\text{crossed homomorphisms}\ G \to A\} \to A$$ given by mapping $\varphi$ to its identity. 

Thus, $$N_G(\varphi(1)) = (\sigma^{k-1}+\dots+ \sigma^0)\varphi(1) = \varphi(\sigma^k) = \varphi(1)=0.$$ Therefore the image of $\phi$ lies inside of $\text{Ker}(N_G)$. If $a \in N_G$, then the mapping $$\sigma^k \mapsto (\sigma^{k-1}+\dots+ \sigma^0)a$$ defines a crossed homomorphism that maps to $a$. This demonstrates that $\phi$ is an isomorphism; moreover the mapping $\phi \mapsto \phi(\sigma)$ induces an isomorphism $$H^1(G, M) \cong \frac{\text{Ker}(N_G)}{(\sigma-1)M}$$ with the denominator being as such as principal homomorphism map to $\sigma a-\sigma. \qed$

If $G$ is cyclic of order $r$ and is generated by $\sigma$ and the crossed homomorphism $\varphi$ is defined by its value on $\sigma$, $m$ satisfies the equation $$\sigma^{f-1}m+\dots + \sigma m + m =0.$$ 

Let $M$ and $M'$ be $G$ and $G'$-modules, and let $\alpha: G\to G$ and $\beta: M \to M'$ be homomorphisms (note that they go in opposite directions). $\alpha$ and $\beta$ are compatible if $$\beta(\alpha(g(m)))=g(\beta(m)).$$

Before we present the following example, we present a homological fact. If $F$ is a left exact functor, then we can begin with a short exact sequence $$0 \to M' \to M \to M'' \to 0$$ and extract a long exact sequence 

\begin{align*}
	0 \to H^0(G, M') \to \dots &\to H^i(G, M') \to H^i(G, M) \to H^i(G, M'')  \\
	&\xrightarrow{\delta_i} H^{i+1}(G, M') \to H^{i+1}(G, M) \to H^{i+1}(G, M'') 
\end{align*}

where the $\delta_i$ are \emph{connecting homomorphisms}. This mapping between short exact sequences and long exact sequences is functorial. To see this we introduce the notion of a \emph{derived functor}. 

Let $F: \mcA \to \mcB$ be covariant and left exact. If $$0 \to A \to B \to C\to 0$$ is a short exact sequence in $A$, applying the functor $F$ yields $0 \to F(A) \to F(B) \to F(C)\to 0$ which is not necessarily exact as the map $F(B) \to F(C)$ may not be surjective. For every $i \geq 1$ there is a functor $R^iF: \mcA \to \mcB$ that continues the sequence (and preserves exactness): $$0 \to F(A) \to F(B) \to F(C) \to R^1F(A) \to R^1F(B) \to R^1F(C) \to R^2F(A) \to \dots$$

But how do we actually construct the $R^i$? 

\defn{Derived functor}{
	Let $A \in \mcA$, and $A \to I^\bullet$ be an injective resolution. Let $F: \mcA\to\mcB$ be a left exact functor between two abelian categories, and let $\mcA$ have enough injectives. The right exact functors $R^iF$ of $F$ satisfy $$R^iF(A) = H^i(F(I))$$ where $H^i$ is the $i$'th cohomology group. These functors of $F$ are the \emph{right derived functors}.
}

Let $F$ be a left exact functor, and $R^i$ denote a derived functor. If we have an injective resolution $M \to I^\bullet$, relate the cohomology groups and the right exact functors by $$(R^i(F))(M) = H^i(F(I^\bullet)).$$ 

The most crucial part of this demonstration is that a morphism between short exact sequences 

% define derived functors

\[
\begin{tikzcd}
0 \arrow{r} &A \arrow{r}\arrow{d}& B \arrow{r} \arrow{d} &C \arrow{r} \arrow{d} &0 \\
0 \arrow{r} &A' \arrow{r} &B' \arrow{r} &C' \arrow{r} &0
\end{tikzcd}
\]

yields a commutative diagram 

\[
\begin{tikzcd}
&\dots \arrow{r} &R^{i-1}F(C) \arrow{r}& R^iF(A) \arrow{r} \arrow{d} &R^iF(B) \arrow{r} \arrow{d} &R^iF(C) \arrow{r} &\dots \\
&\dots \arrow{r} &R^{i-1}F(C') \arrow{r}& R^iF(A') \arrow{r} &R^iF(B') \arrow{r} &R^iF(C') \arrow{r} &\dots
\end{tikzcd}
\]

If we have a long exact sequence and apply $F$: $$0 \to F(I^0) \to F(I^1) \to F(I^2) \to \dots$$ we have that $$R^iF(X) = \frac{\text{Ker}(F(I^i))}{\text{Im}(F(I^i))}.$$

Let's return to our long exact sequence

\begin{align*}
	0 \to H^0(G, M') \to \dots &\to H^i(G, M') \to H^i(G, M) \to H^i(G, M'')  \\
	&\xrightarrow{\delta_i} H^{i+1}(G, M') \to H^{i+1}(G, M) \to H^{i+1}(G, M'') .
\end{align*}

If $H^i(G, M)=0$ then there exists an isomorphism $$H^i(G, M'') \cong H^{i+1}(G, M').$$

\ex{}{
	Let $g' \in G$. Let the homomorphisms $\alpha: G\to G$ that sends $\sigma \mapsto g'\sigma g'^{-1}$ and $\beta: M\to M$ that sends $m \mapsto g'^{-1}m$ be compatible. The homomorphisms $$H^i(G, M) \to H^i(G, M)$$ with regards to $\alpha$ and $\beta$ are the identity map.
}

Consider the map $\phi: H^r(G, N) \to H^r(G, N)$. Let $M_0$ be the abelian group structure of $M$. Define the groups $M_*$and $M^*$ such that $M_* = \text{Ind}^G(M_0)$, and $M^* = M_* / M$. With regards to the sequence $$0 \to M \to M_* \to M^* \to 0$$ there is a diagram with exact rows

\[
\begin{tikzcd}
H^{i-1}(G, M_*) \arrow{r}\arrow{d}& H^{i-1}(G, M^*) \arrow{r} \arrow{d} &H^{i-1}(G, M) \arrow{r} \arrow{d} &0 \\
H^{i-1}(G, M_*) \arrow{r}& H^{i-1}(G, M^*) \arrow{r}  &H^{i-1}(G, M) \arrow{r}  &0 
\end{tikzcd}
\]

in which since the middle vertical map is the identity, the third vertical map is the identity, and the result follows. The process of suitably establishing isomorphisms between $i$-cohomology and $i+1$-cohomology groups is known as \emph{dimension shifting}.

\ex{Cocycle maps}{
	Let $C^i(G, M)$ denote the $i$th cocycle map. Furthermore suppose that $\alpha: G' \to G$ and $\beta: M \to M'$ are compatible. Then, the map $$C^i(G, M) \to C^i(G', M')$$ which sends $f \mapsto \beta(f(\alpha \times \dots \times \alpha))$ induce maps on the cohomological groups $H^i(G, M) \to H^i(G', M')$. 
	
	One can show this by checking compatibility with the differential maps $d^i$. Let $f'$ be the image of $f$. But from $$d^i f'(g_0', \dots, g_i') = \beta(d^i f(\alpha(g_0'), \dots, \alpha(g_i'))),$$ it follows that $$g_0'(f'(g_1', \dots, g_i')) = g_0' \beta(f(\alpha(g_1'), \dots, \alpha(g_i'))) = \beta(\alpha(g_0')f(\alpha(g_1'), \dots, \alpha(g_i'))). \qed$$ 
}

The process of constructing maps $H^i(G, M) \to H^i(G', M')$ is known as \emph{functoriality}. 

\defn{Restriction and inflation homomorphisms}{
	Let $H \leq G$ and $\alpha$ be the natural inclusion map $H \xhookrightarrow\ G.$ Let $M$ be a $G$-module.  Then there is a compatible pair $(\alpha, \text{id}_M)$ which yields the \emph{restriction map} $$\text{Res}: H^i(G, M) \to H^i(H, M).$$
	
	Furthermore let $H \trianglelefteq G$, let $\beta: G \to G/H$ be the quotient map, and let $\gamma: M^H \xhookrightarrow\ M$ be the inclusion map. Then, $(\beta, \gamma)$ is a compatible pair that yields the \emph{inflation map} $$\text{Inf}: H^i(G/H, M^H) \to H^i(G, M).$$
}

The restriction homomorphism is equivalent to sending $M \to \text{Ind}_H^GM$ by sending $$m \mapsto \sum\limits_i m^{g_i} \otimes [g_i^{-1}]$$ where $g_i$ runs through coset representatives of $H$ in $G$. Then apply Shapiro's lemma to obtain $$H^i(G, M) \to H^i(G, \text{Ind}_H^G M) \cong H^i(H, M).$$

\ex{}{
	Let $i=0$. In this case, the restriction map $$\text{Res}: M^G \to M^H$$ is the inclusion map, and the inflation map $$\text{Inf}:(A^H)^{G/H} \to A^G$$ is the identity.
}

\defn{Corestriction homomorphism}{
	Consider the map $\text{Ind}_H^G M \to M$ which takes $$m \otimes [g] \to m \cdot g$$ where $[g]$ represents a set of coset representatives for $H$ in $G$. This implies that $$H^i(G, \text{Ind}_H^GM) \to H^i(G, M).$$ Then, there is a corestriction homomorphism $$\text{Cor}: H^i(H, M) \cong H^i(G, \text{Ind}_H^G M) \to H^i(G, M)$$ where the isomorphism is given by Shapiro's lemma.
}

The composition $$\text{Cor} \circ \text{Res}: H^i(G, M) \to H^i(G, M)$$ is induced by the homomorphism $M \to \text{Ind}_H^G M \to M$, which is given by $$m \mapsto \sum\limits_i (m \cdot g_i) \otimes [g_i^{-1}] \mapsto \sum\limits_i m = [G:H]=m$$ which suggests that the composition is multiplication by $[G:H]$.

\coro{}{
	Let $|G|=n$. Then, $H^i(G, M)$ is $n$-torsion for all $G$-modules $M$ and $i\geq 0$. Furthermore, if $M$ is finitely generated, then $H^i(G, M)$ is finite. 
}

% perhaps give proof?
There is an exact sequence with inflation and reflation. We'll give it without proof. Let $H \leq G$ and $M$ be a $G$-module. If $H^j(H, M)=0$ for all $0 < j < r$, then the sequence $$0 \to H^i(G/H, M^H) \xrightarrow{\text{Inf}}\ H^i(G, M) \xrightarrow{\text{Res}}\ H^i(H, M)$$ is exact.

\defn{Cup product}{
	Let $A, B$ be $G$-modules. The cup-product $\cup$ underscores uniquely determined bilinear mappings $$\cup: H^i(G, A) \times H^j(G, B) \to H^{i+j}(G, A\otimes B)$$ for $i, j \in \bbZ$ 
}

The cup product also satisfies the following conditions:
\begin{enumerate}
	\item For $i=j=0$, the pairing is $$(a, b) \mapsto a \cup b = a \otimes b$$ with $a \in H^0(G, A)$ and $b \in H^0(G, B)$. 
	\item If $$0 \to A \to A' \to A'' \to 0$$ and $$0 \to A\otimes B \to A'\otimes B \to A''\otimes B \to 0$$ are exact, then 
	\[
\begin{tikzcd}
H^i(G, A'') \times H^j(G, B) \arrow{r}{\cup}\arrow{d}{\delta} &H^{i+j}(G, A'' \otimes B) \arrow{d}{\delta}\\
H^{i+1}(G, A) \times H^j(G, B) \arrow{r}{\cup} &H^{i+j+1}(G, A \otimes B)
\end{tikzcd}
\]

such that $\delta(a'' \cup b) = \delta a'' \cup b$, where $a'' \in H^i(G, A'')$, $b \in H^j(G, B)$, and $\delta$ denotes the connecting homomorphism.
	\item If $$0 \to B \to B' \to B'' \to 0$$ and $$0 \to A\otimes B \to A'\otimes B \to A''\otimes B \to 0$$ are exact, then 
	\[
\begin{tikzcd}
H^i(G, A) \times H^j(G, B'') \arrow{r}{\cup}\arrow{d}{\delta} &H^{i+j}(G, A \otimes B'') \arrow{d}{(-1)^i \delta}\\
H^{i+1}(G, A) \times H^j(G, B) \arrow{r}{\cup} &H^{i+j+1}(G, A \otimes B)
\end{tikzcd}
\]

such that $\delta(a \cup b'') = \delta a \cup b''$, where $a \in H^i(G, A)$, $b'' \in H^j(G, B'')$, and $\delta$ denotes the connecting homomorphism. The $(-1)^i \delta$ results from the fact that the connecting homomorphism is anti-commutative.

\end{enumerate}
% cup products?

Let $G$ be a profinite group. All $G$-modules used in this section is such that the underlying group $M$ is a topological group. Additionally, the map $$G\times M \to M$$ is continuous. When $M$ is endowed with the discrete topology, we say that the $G$-module is \emph{discrete}. 

The cohomology groups are very similar in the profinite case. Let $C^i_{\text{cts}}(G, M)$ be the group of continuous maps $G^i \to M$, and define $d^i: C^i_{\text{cts}}(G, M) \to C^{i+1}_{\text{cts}}(G, M)$ to be the differential map. Then, as before, we have that $$H^i_{\text{cts}}(G, M) = \frac{Z^i_{\text{cts}}(G, M)}{B^i_{\text{cts}}(G, M)}$$ where $Z^i_{\text{cts}}(G, M) = \text{Ker}(d^i)$ and $B^i_{\text{cts}}(G, M) = \text{Im}(d^i)$.

There is a categorical construction with our profinite groups. Let $G$ be a profinite groups, $\mcM_G$ be the category of all $G$-modules and $\mcC_G$ be the category of discrete $G$-modules. Then, $\mcC_G$ is a full subcategory of $\mcM_G$. Additionally, for discrete $G$-modules $M$ and $M'$, there exists a functor $\mcM_G \to mcC_G$ that sends $$M \mapsto M^* = \bigcup\limits_{H\ \text{open in}\ G} M^H.$$

Most of the theory regarding cohomology groups hold in the profinite case. The inflation, restriction, and corestriction maps do. In particular, let $G$ be a profinite group, and let $M$ be a discrete $G$-module. If $M=\lim\limits_{\rightarrow} M_k$ where $M_k \subset M$, then $$H^i(G, M) = \lim\limits_{\rightarrow} H^i(G, M_k).$$

I'll sketch the proof. This follows from the fact that every finite subset of $M$ is contained in $M_k$ for a certain $k$ since the $M_k$ form a directed system of submodules. Then note that applying direct limits to an exact sequence preserves exactness. Apply direct limits to $C^i(G, M) = \lim\limits_{\rightarrow} C^i(G, M_k)$ and we are done. $\qed$

There is a dual theory to group cohomology, namely (you've guessed it) group homology. Here we demonstrate a way to fit the homology and cohomology groups together using \emph{Tate groups}.

\defn{$M_G$}{
Let $M$ be a $G$-module and $M_G$ be the dual to $M^G$. More explicitly, let $M_G$ be the largest quotient of $M$ on which $G$ acts trivially, meaning that $$M_G = M / \langle gm-m \mid g \in G, m \in m \rangle.$$ 

Analogous to the cohomological case, if we have a short exact sequence $$0 \to M' \to M \to M'' \to 0$$ we can apply $M_G$ through the mapping $M\mapsto G$ to obtain $$M_G' \to M_G \to M_G'' \to 0.$$ Note that the map $0 \to M'_G$ may not be surjective so adding $0 \to M_G'$ won't make the sequence exact.
}

In other words, $M_G = M/MI_G$, where $I_G$ is the \emph{augmentation ideal} of the group algebra $\bbZ[G]$, satisfying $$I_G = \left\{ \sum\limits_{g \in G} z_g [g]: \sum\limits_g z_g = 0 \right\}$$ where $[g]$ runs through a set of coset representatives like in the cohomological case.

\defn{Homology group}{
Let $M$ be a $G$-module. Let $P \in \textsf{Ab}$ be \emph{projective}, meaning that for $M=P/N$, any morphism $P \to M$ lifts to a morphism $P \to N.$ Choose a projective resolution $$\dots \to P_2 \xrightarrow{d_2}\ P_1\xrightarrow{d_1}\ P_0 \xrightarrow{d_0}\ M \to 0.$$ We can then apply $G$ onto each object and shave off the $M$ to obtain $$\dots \to (P_2)_G \xrightarrow{d_2} (P_1)_G \xrightarrow{d_1}\ (P_0)_G \to 0.$$ Then the $i$'th homology group is $$H_i(G, M) = \frac{\text{Ker}(d_i)}{\text{Im}(d_{i+1})}.$$
}

Like in the cohomological case, any short exact sequence $$0 \to M' \to M \to M'' \to 0$$ gives rise to a long exact sequence functorially: $$\dots \to H_i(G, M) \to H_i(G, M'') \xrightarrow{\delta_r}\ H_{i-1}(G, M) \to \dots \to H_0(G, M') \to H_0(G, M) \to H_0(G, M') \to 0.$$

\ex{Augmentation sequence}{
	Consider the exact sequence $$0 \to I_G \to \bbZ[G] \to \bbZ \to 0.$$ Note that $H_1(G, \bbZ[G])=0$. Since we have a long exact sequence $$H_1(G, \bbZ[G]) \to H_1(G, \bbZ) \to H_0(G, I_G) \to H_0(G, \bbZ[G])$$ it follows that the sequence $$0 \to H_1(G, \bbZ) \to I_G / I_G^2 \to \bbZ[G]/I_G \to \bbZ \to 0$$ is exact. 
	
	The middle map is inclusion: $I_G \xhookrightarrow\ \bbZ[G]$, and thus is zero. Hence we have an isomorphism $$H_1(G, \bbZ) \cong I_G/I_G^2.$$ Furthermore we have that $$\bbZ[G]_G \cong \bbZ,$$  meaning that $\bbZ$ is the largest quotient of $\bbZ[G]$ on which $G$ acts trivially.
}

\prop{}{
	There is an isomorphism $$H_1(G, \bbZ) \cong G^{\text{ab}}.$$
}

\tbf{Proof:} Let $G^c = [G, G]$ be the commutator subgroup of $G$ and thus $G^{\text{ab}}=G/G^c$ is the largest abelian quotient of $G$. I want to show that there is an isomorphism $$G^{\text{ab}} \cong I_G/I_G^2$$ through the map $g \mapsto [g]-1$. The map $$G \to I_G/I_G^2$$ that sends $$g \mapsto (g-1) \Mod{I_G^2}$$ is a homomorphism as we have 
\begin{align*} gg'-1 &= (g-1)(g'-1)+(g-1)+(g'-1)\\
&= (g-1)+(g'-1) \Mod{I_G^2)}
\end{align*}

thus showing that the map factors through $G^{\text{ab}}$. We need to show that the inverse map holds. 

Construct an inverse map through the map $I_G\to G^{\text{ab}}$ which maps $$g-1 \mapsto [g].$$ Note that $(g-1)(g'-1)$ maps to 1 since $(g-1)(g'-1)=(gg'-1)-(g-1)-(g'-1)$. Thus, the map factors through $I_G/I_G^2$, and the two maps are inverse.

Therefore $H_1(G, \bbZ) \cong I_G/I_G^2 \cong G^{\text{ab}}$, and we are done. $\qed$

Let $M$ be a $G$-module. Let the \emph{norm map} $N_G: M\to M$ be $$m \mapsto \sum\limits_{g \in G} g\cdot m.$$ Let $M_G = M/ I_GM$. The norm map induces a homomorphism $$H_0(G, M) = M_G \xrightarrow{N_G}\ M^G \to H^0(G, M).$$

Since $g$ runs through the elements of $G$, so does $g'g$ ad $gg'$. Thus, $g'(N_G(m)) = N_G(g'm)$ which yields $\text{Im}(N_G) \subset M^G$ and $I_GM \subset \text{Ker}(N_G)$. Thus there is a short exact sequence $$0 \to \text{Ker}(N_G)/I_GM \to M_G \xrightarrow{N_G}\ M^G \to M^G/N_G(M) \to 0.$$

\defn{Tate groups}{
	Define the \emph{Tate groups} $$H_T^i(G, M) = 
	\begin{cases} H^i(G, M)\ &i>0 \\ 
	M^G/N_G(M) \ &i =0 \\
	\text{Ker}(N_G)/I_GM \ &i=-1 \\
	H_{-i-1}(G, M) \ & i<-1
	\end{cases}$$
}

which conveniently converts our exact sequence above to $$0 \to H_T^{-1}(G, M) \to H_0(G, M) \xrightarrow{N_G}\ H^0(G, M) \to H_T^0(G, M) \to 0$$

\ex{}{
	Let $L/K$ be an unramified extension with Galois group $G$. Letting $U_L$ be the group of units in $L$, we have that $$H_T^i(G, U_L)=0.$$
	
	If $\pi \in L$ is a uniformiser, we can write every element of $L^\times$ as $\alpha = u\pi^m$ with $u\in U_L, M \in \bbZ$. Thus we can represent $L^\times \cong U_L \times \bbZ$. 
	
	Since $L/K$ is unramified, we can choose $\pi \in K$. Then, for $\sigma \in \text{Gal}(L/K)$ we can write $\sigma \alpha= \sigma (u\pi^m)=(\tau u)\pi^m$ and we can decompose $L^\times$ into $G$-modules. Therefore, by observing the isomorphism $$H^i(G, \prod\limits_i M_i) \cong \prod\limits_i H^i(G, M_i)$$ for $M_i$ being $G$-modules, we can write $H^i(G, U_L)$ as a direct summand of $H^i(G, L^\times)$. 
	
	Hilbert's Theorem 90 writes that $H^1(G, L^\times)=0$. It swiftly follows that $H^1(G, U_L)=0$. Since $U_L$ surjects onto $U_K$, we can just apply the isomorphism $H_T^i(G, M)$ to prove that $H^0(G, U_L)=0$. Then since $G$ is cyclic, the proposition is proven. $\qed$
}

\prop{}{
	Let $0 \to M' \to M \to M'' \to 0$ be an exact sequence of $G$-modules. This extends to a long exact sequence $$\dots \to H_T^{i-1}(G, M'') \to H_T^i(G, M')\to H_T^i(G, M)\to H_T^i(G, M'')\to H_T^{i+1}(G, M') \to \dots.$$ 
}

\tbf{Proof:} Consider the commutative diagram (there are arrows going into $H_1(G, M'')$ and arrows coming out of $H^1(G, M')$ but they are omitted due to formatting issues)

\[
\begin{tikzcd}
& H_T^{-1}(G, M') \arrow[r, draw=red]\arrow{d} & H_T^{-1}(G, M)\arrow[r, draw=red]\arrow{d}  & \mbf{H_T^{-1}(G, M'')}\arrow{d}\arrow[dddll, draw=red]  & \\
H_1(G, M'') \arrow{r}\arrow{d}\arrow[ur, draw=red]& H_0(G, M') \arrow{r} \arrow{d}{N_G} &H_0(G, M) \arrow{r} \arrow{d}{N_G} &H_0(G, M'') \arrow{r} \arrow{d}{N_G}&0 \\
0 \arrow{r}& H^0(G, M') \arrow{r} \arrow{d} &H^0(G, M) \arrow{r} \arrow{d} &H^0(G, M'') \arrow{r} \arrow{d} &H^1(G, M') \\
&\mbf{H_T^0(G, M')} \arrow[r, draw=red] &H_T^0(G, M) \arrow[r, draw=red] &H_T^0(G, M'') \arrow[ur, draw=red] & & 
\end{tikzcd}
\]

in which the red arrows display the morphisms in the short exact sequence. Apply the snake lemma and diagram chase and we are done (the purpose of this section is not on pure homological algebra so the details of the diagram chase are omitted).

\thrm{}{
	Let $M$ be an induced $G$-module. Then, $H_T^i(G, M) =0 $.  
}

\tbf{Proof:} We've proven the case $i>1$. Now we prove the theorem for $i<-1$, which correspond to the homology groups.

Let $M = \bbZ[G] \otimes_\bbZ M'$ for some abelian group $A$. Then there exist a free abelian group $A_0$ that surjects onto $A$ with kernel free and abelian. Let $A_1 = \text{Ker}(A_0 \xhookrightarrow\ M).$ Thus we have a short exact sequence $$0 \to A_1 \to A_0 \to A \to 0.$$ Tensor each element with $\bbZ[G]$ and thus we obtain $$0 \to M_1 \to M_0 \to M \to 0.$$ We also know that $H_T^i(G, M_0)=0 = H_T^i(G, M_1)$ for $i<-1$ because $M_0, M_1$ are free and also for $i \geq 1$ as $M_0, M_1$ are induced. Apply the long exact sequence above to conclude that $H_T^i(G, M)=0$ for all $i$. 

This reduces to verifying the cases $i=0, i=-1$ which is easy. $\qed$

The Tate groups share many similarities with the cohomology groups. Let $H \leq G$. There are canonical homomorphisms $$\text{Res}: H_T^i(G, M) \to H_T^i(H, M)$$ $$\text{Cor}: H_T^i(H, M) \to H_T^i(G, M)$$

with the composite map $\text{Res} \circ \text{Cor}$ still being multiplication by $[G:H]$. However, the canonical isomorphism $H^i(G, M) \to H^i(G', M')$ only holds if the homomorphism $G' \to G$ is injective. 
% Weiss, 1969

The Tate groups satisfy a periodicity condition. 

\ex{}{
	To illustrate this, let $G$ be a finite cyclic group of order $m$, and let $G$ have generator $\sigma$. Then $$H_T^{-1}(G, M) = \frac{\text{Ker}(N_G)}{\text{Im}(\sigma-1)}$$ which we showed is isomorphic to $H^1_T(G, M)$ by our alternate description of $H^1(G, M)$.
}

We'll prove the equality later.

\thrm{Periodicity condition}{
	The demonstration that $H_T^{-1}(G, M) \cong H_T^1(G, M)$ can be generalised. For a suitably chosen generator, it holds that $$H_T^i(G, M) \cong H_T^{i+2}(G, M).$$
}

\tbf{Proof:} Let $\sigma \in G$ be a generator. There is an exact sequence of $G$-modules $$0 \to \bbZ \to \bbZ[G] \to \bbZ[G] \to \bbZ \to 0$$ in which the first map maps $1 \mapsto \sum\limits_{g \in G} [g]$, the second map maps $[h] \mapsto [\sigma h]-[h]$, and the third map is $[h] \mapsto 1$. 

Because the groups in the sequence are free $\bbZ$-modules and the kernel of the third map ($I_G$) is free, we can tensor over $\bbZ$ with $M$ and thus obtain $$0 \to M \to M \otimes_\bbZ \bbZ[G].$$ Since the induced modules are acyclic for Tate groups, we have that $$H_T^i(G, \bbZ[G] \otimes_\bbZ M) = 0.$$ Recall from cohomology that from ${M'}^G \to M^G \to {M''}^G = H^0(G, M'') \to H^1(G, M')$ we obtain isomorphisms $H^i(G, M'') \cong H^{i+1} (G, M').$ Thus here we obtain isomorphisms $$H^i_T(G, M) \cong H_T^{i+2}(G, M). \qed$$

Let $0 \to M' \to M \to M'' \to 0$ be a short exact sequence of $G$-modules. Then, we have an exact hexagon 

\[
\begin{tikzcd}
& &H_T^{-1}(G, M) \arrow{r} &H_T^{-1}(G, M'') \arrow{dr} & \\
&H_T^{-1}(G, M') \arrow{ur} & & &H_T^0(G, M') \arrow{dl} \\
& &H_T^0(G, M'') \arrow{ul} & H_T^0(G, M'') \arrow{l} &
\end{tikzcd}
\]

The exactness of the hexagon follows from the following lemma.

\lemma{}{
	Let $M$ be a $G$-module, where $G$ is a finite cyclic group generated by a generator $\sigma$. We have that $$H_T^{2n}(G, M) \cong H_T^{-2n}(G, M) \cong H^0_T(G, M)$$ and $$H_T^{2n-1}(G, M) \cong H_T^{-2n-1}(G, M) \cong H^{-1}_T(G, M)$$
}

\tbf{Proof:} Let $M$ be a $G$-module. There exists a free resolution $$\dots \to \bbZ[G] \xrightarrow{N_G}\ \bbZ[G] \xrightarrow{g-1} \bbZ[G] \to \bbZ \to 0$$ from which we can compute the cohomology groups $H^i(G, M)$ using the complex $$0 \to A \xrightarrow{g-1}\ A \xrightarrow{N_G} \to \dots$$ and the homology groups $H_i(G, M)$ using $$\dots \to  A \xrightarrow{g-1}\ A \xrightarrow{N_G} \to 0.$$

Note that $A^G = \text{Ker}(g-1)$ so we have that $$H^{2n}_T(G, M) = H^{-2n}_T(G, M) = H^0(G, M).$$ Since $\text{Im}(g-1) = I_GM$ we also have that $$H^{1-2n}_T(G, M) = H^{2n-1}_T(G, M) = \frac{\text{Ker}(N_G)}{\text{Im}(g-1)}=H^{-1}_T(G, M)$$ and thus we are done by applying this result to the long exact sequence of the cohomology groups. $\qed$

This lemma demonstrates that $H^0_T(G, M) = \frac{\text{Ker}(N_G)}{\text{Im}(g-1)}$  and $H_T^{-1}(G, M)=\frac{\text{Ker}(g-1)}{\text{Im}(N_G)}$, which motivates the Herbrand quotient.

\defn{Herbrand quotient}{
	Let $G$ be a finite cyclic group, and let $M$ be a $G$-module. Assume that the Tate groups $H_T^i(G, M)$ are finite. Then, define the \emph{Herbrand quotient} as $$h(G, M) = \frac{\# H^0(G, M)}{\# H^{-1}(G, M)}$$
	
	Another construction is as follows: $f, g$ be endomorphisms such that their composition is identity. We have inclusions $\text{im}(g) \subseteq \text{ker}(f)$, and $\text{im}(f) \subseteq \text{ker}(g)$. With this construction, define the Herbrand quotient to be $$q_{f, g}(A) = \frac{[\text{ker}(f) : \text{im}(g)]}{[\text{ker}(g) : \text{im}(f)]}.$$
}

\prop{}{
	The Herbrand quotient is multiplicative. This means that if the sequence $0 \to M' \to M \to M'' \to 0$ is exact, then $$h(G, M) = h(G, M')h(G, M'').$$
}

\tbf{Proof:} This follows from exactness of the hexagon above. $\qed$

It immediately follows that if $G$ is finite cyclic, and $M$ and $M'$ are $G$-modules, then applying the above lemma to the short exact sequence $$0\to A \to A\oplus B \to B \to 0$$ yields us $$h(M \oplus M') = h(M)h(M').$$ 

It's rather arduous to compute the Tate group. If we know the Herbrand quotient (an easier computation due to its properties) and the order of one Tate group then we'll know the order of the other Tate group. We'll see the Herbrand quotient in action when we look at the main isomorphism of local class field theory.

Some of the properties of the Herbrand quotient are as follows:

\prop{}{
	If $M$ is finite, then $h(M)=1$.
}

\tbf{Proof:} Let $M$ be a $G$-module. There is an exact sequence $$0 \to M^G \to M \xrightarrow{g-1}\ M \to M_G \to 0$$ which demonstrates that $|M^G| = |M_G|$. Then note the sequence $$0 \to H_T^{-1}M \to M_G \xrightarrow{N_G}\ M^G \to H_T^0 M\to 0$$ which shows that $H_T^{-1} M$ and $H_T^0 M$ have the same order., and thus $h(M)=1. \qed$

\prop{}{
	Let $M=\bbZ$ with the trivial action. Then, $h(A)=|G|$ denotes multiplication by $[H:G]$.
}

\ex{}{
	Let $l^\times$ and $k^\times$ be the residue fields of $L$ ad $K$ respectively. Since $l^\times$ is finite, $l^\times$ has Herbrand quotient 1. By Hilbert's Theorem 90, that $H^1(G, l^\times)$. It then follows that $H^2(G, l^\times)=0$ and we can use the isomorphism between $$H_T^i(G, M)\cong H_T^{i+2}(G, M)$$ to conclude that $H_T^i(G, l^\times)=0$ for all $i$.
}


\thrm{}{
	Let $A$ be a finitely generated $G$-module where $G$ is a cyclic group of order $p$. Then, the Herbrand quotient $h(A)$ satisfies $$h(A) = p^{\frac{p\beta-\alpha}{p-1}}.$$
}

\tbf{Proof:} Note that $A$ is finitely generated. Then we can choose a torsion-free submodule $A_1 \subset A$ of finite index. Write $A_1=nA$ for some $n$. It thus follows that $\text{rank}(A_1)=\text{rank}(A)=\alpha$, and $\text{rank}(A_1^G)=\text{rank}(A^G)=\beta$. It thus follows that 

\begin{align*}
	h(A)^{p-1} &= h(A_1)^{p-1} \\
	&= \frac{q_{0, p}(A_1^G)^p}{q_{0, p}(A_1)} \\
	&= \frac{[A_1^G: pA_1^G]}{[A_1: pA_1]} \\
	&= p^{p\beta-\alpha}
\end{align*}

and the result follows. $\qed$

\thrm{Tate's theorem}{
	Let $G$ be a finite group, and $M$ be a $G$-module. Then, for all subgroups $H \leq G$, we have that $$H^1_T(H, M) = 0$$ and $$\# H^2_T(H, M) = H.$$
	
	This implies that the periodicity isomorphism $H_T^i(G, M) \cong H_T^{i+2}(G, M)$ only depends on the generator for $H^2_T(G, M)$.
}

\ex{Tate's theorem in class field theory}{
	Let $L/K$ be finite Galois with Galois group $G$. Then there exists a canonical isomorphism $$H_T^i(G, \bbZ) \to H_T^{i+2}(G, L^\times).$$
}

\tbf{Proof:} (Milne) To abuse notation, we drop the subscript $T$ in this proof as all groups are Tate groups.  

\lemma{}{
	Let $M$ be a $G$-module. If $H^1(H, M) = 0 = H^2(H, M)$ for all subgroups $H \leq G$, then $$H_T^i(G, M)=0.$$
}

\tbf{Proof:} This is easy if $G$ is cyclic as the indices are periodic. Let $G$ be solvable. Then there exists a subgroup $H$ such that $G/H$ is cyclic. Then we have $H^i(H, M)=0$. There are exact sequences % why 
$$0 \to H^i(G/H, M^H) \to H^i(G, M) \to H^i(H, M)$$ where the first map is inflation and the second map is restriction. We know that $H^i(G, M)$ for all $i>0$ - now we show that $H^0(G, M)=0$.

Let $x \in M^G$. Since $H^1(G, M)=H^2(G, M)=0$ and $G/H$ is cyclic, then $H^1(G/H, M^H)=H^2(G/H, M^H)=0$. Thus, $H^i(G, M)=0$ for all $i \in \bbZ$. Since $H^0(G/H, M^H)=0$, there exists a $y \in M^H$ such that $N_{G/H}(y)=x$ and since $H^0(H, M)=0$, there exists a $z \in M$ such that $N_H(z)=y$. By composing these two equalities we get that $$N_G(z)=(N_{G/H} \circ N_H)(z)=x$$ and thus $H^i(G, M)=0$ for all $i \geq 0$.

Consider the exact sequence $$0 \to M' \to \bbZ[G] \otimes_\bbZ M\to M \to 0.$$ Since $\bbZ[G] \otimes_\bbZ M$ is induced, $H^i(H, \bbZ[G] \otimes_\bbZ M)=0$ for all $i \in \bbZ$ and $H \leq G$. Thus $H^i(H, M) \cong H^{i+!}(H, M')$ and we can iterate down to show that $H^{-1}(G, M), H^{-2}(G, M)=0$, etc, which proves the theorem for $G$ solvable.

If $G_p$ is a Sylow $p$-subgroup, then if $(G, M)$ satisfy the lemma it swiftly follows that $(G_p, M)$ do too. Thus, $H^i(G_p, M)=0$ and thus the $p$-primary component of $H^i(G, M)$ is 0. Hence, $H^i(G, M)=0$ for all $i \in \bbZ. \qed$
 
Consider a generator $\sigma \in H^2(G, M).$ Since $\text{Cor} \circ \text{Res}= [G:H]$, $\text{Res}(\gamma)$ generates $H^2(H, C)$. Let $\varphi$ be a cocyle representing $\gamma$, and let $M(\varphi)$ be the direct sum of $M$ with the abelian group that has basis $x_\sigma$ for each $\sigma \neq 1 \in G$. Extend the action on $M$ to an action on $M(\varphi)$ by setting $$\sigma x_\tau = x_{\sigma \tau}-x_\sigma - \varphi(\sigma, \tau).$$ This induces an action because $\rho \sigma x_\tau = x_{\rho\sigma\tau}-x_{\rho\sigma}+\varphi(\rho\sigma, \tau)$ where $$\rho(\sigma x_\tau) = x_{\rho\sigma\tau}-x_\rho+\varphi(\rho, \sigma\tau)-x_{\rho\tau}+x_\rho-\varphi(\rho, \sigma) + \rho\varphi(\sigma, \tau)$$

which makes sense as $\varphi$ satisfies $$\rho \varphi(\sigma\tau) + \varphi(\rho, \sigma\tau)= \varphi(\rho\sigma, \tau)+\rho(\sigma, \tau)$$ and thus $\sigma$ maps to $0$ in $H^2(G, M(\varphi)).$

I want to show that $H^1(H, M(\varphi)) = H^2(H, M(\varphi))=0$. To show this, consider the exact sequence $$0 \to I_G \to \bbZ[G] \to \bbZ \to 0.$$ Note that $H^i(H, \bbZ[G])$ are acyclic, and thus $$H^1(H, I_G) \cong \bbZ/H\bbZ$$ and $$H^2(H, I_G) \cong H^1(H, \bbZ) =0.$$

If we define the map $H(\varphi) \to \bbZ[G]$ that sends $\alpha: c \mapsto 0$ and $\alpha: x_\sigma \mapsto \sigma-1$ we get a short exact sequence $$0 \to M \to M(\varphi) \to I_G \to 0$$ with cohomology sequence $$0 \to H^1(H, M(\varphi)) \to H^1(H, I_G) \to H^2(H, M) \to H^2(H, M(\varphi)) \to 0$$ which suggests that $H^1(H, C)=0$ and $H^2(H, I_G)=0$. Thus the map $H^2(H, M) \to H^2(H, M(\varphi))$ is zero as $H^2(H, M)$ is generated by $\text{Res}(\sigma)$ which maps to the image of $\sigma$ in $H^2(H, M(\varphi))$. 

Thus, the homomorphism $H^1(H, I_G) \to H^2(H, M)$ is surjective. It is also injective as the two groups have the same order. Therefore, $H^1(H, I_G) \cong H^2(H, M).$

Combine short exact sequences $0 \to I_G \to \bbZ[G] \to \bbZ \to 0$ and  $0 \to M \to M(\varphi) \to I_G \to 0$ to attain $$0 \to M \to M(\varphi) \to \bbZ[G] \to \bbZ \to 0.$$ This sequence satisfies $H^i(G, M(\varphi)) = 0 = H^i(G, \bbZ[G])$ for all $i$. Thus there is an isomorphism $$H^i(G, \bbZ)\to H^{i+2}(G, M). \qed$$

\section{Local Class Field Theory}

Let $K$ be a local field. Local class field theory aims to characterise the finite abelian extensions of $K$. Recall that if $L/K$ is finite unramified, and that the respective residue fields of $L$ and $K$ are $l$ and $k$, then there exists a isomorphism $$\text{Gal}(L/K) \cong \text{Gal}(l/k).$$ Thus, $\text{Gal}(L/K)$ is cyclic and generated by $\sigma$ that sends $$\sigma: \alpha \to \alpha^q \Mod{\mfm_L}.$$

This is called the \emph{Frobenius mapping}; $\sigma$ is commonly known as the \emph{Frobenius element}, and is denoted $\text{Frob}_{L/K}$.

Let $L/K$ be a Galois extension. In this section, we abuse notation by writing $$H^i(L/K) = H^i(\text{Gal}(L/K), L^\times)$$ where $H^i(L/K)$ refers to the cohomology groups. 

This section is aimed at proving the main theorems of local class field theory. We'll revisit local class field theory in Chapter 8.

\thrm{Hilbert's Theorem 90}{
	Let $L/K$ be cyclic, and let $\sigma$ generate $\text{Gal}(L/K)$. Then if $N_{L/K}(a)=1$, then we can write $a=\frac{\sigma b}{b}$.
	
	This is equivalent to the following proposition (it's actually stronger as $L/K$ is taken to be finite Galois instead of just cyclic):
	
	Let $L/K$ be finite Galois, and let $G= \text{Gal}(L/K)$. Then, $$H^1(G, L^\times)=\frac{\text{Ker}(N_G)}{(\sigma-1)L^\times}=0.$$ 	
}

\tbf{Proof:} We'll prove that $H^1(G, L^\times)=0$. First we show that if $K$ is a field and $\sigma_1, \dots, \sigma_n$ are distinct automorphisms of $K$, then they are linearly independent.

Assume that they are not linearly independent, and let $c_1, \dots, c_k$ be non-zero coefficients such that $k$ is taken to be minimal. Furthermore arrange the $c_i$ such that $c_1, \dots, c_k \neq 0$ and $c_{k+1}=\dots = c_n = 0$. Since $r>1$, the fact that $c_1 \sigma_1= 0$ implies that $c_1=0$. Choose $a \in K$ such that $\sigma_1(a) \neq \sigma_r(a).$ Then we can write $$c_1\sigma_1(ax)+c_2\sigma_2(ax)+\dots+c_k\sigma_k(ax)=0$$ which suggests that $$c_1\sigma_k(a)\sigma_1(x) +c_2\sigma_k(a)\sigma_2(x) +\dots+c_k\sigma_r(a)\sigma_r(x)=0$$ and therefore $$c_1(\sigma_1(a)-\sigma_k(a))\sigma_1(x) + \dots + c_{k-1}(\sigma_{k-1}(a)-\sigma_k(a))\sigma_{k-1}(x)=0$$ but this means that $c_1(\sigma_1(a)-\sigma_k(a)) \neq 0$, thus contradicting the minimality of $k$ and suggesting that the automorphisms are linearly independent.

Let $f: G\to L^\times$ be a crossed homomorphism (a 1-cocyle). Let $c \in L^\times$. By such, we can write $$\alpha = \sum\limits_{\sigma \in \text{Gal}(L/K)} f(\sigma)c^{\sigma}.$$ By linear independence of the automorphisms, we can choose some $c \in L^\times$ such that $\alpha \neq 0$. Let $\tau \in \text{Gal}(L/K)$. We can write that $$\alpha^\tau = \sum\limits_\sigma f(\sigma)^\tau c^{\sigma\tau} = \sum\limits_\sigma f(\tau)^{-1}f(\sigma\tau)c^{\sigma\tau} = f(\tau)^{-1} \alpha$$ and thus we obtain $f(\tau)=\beta^{1-\tau}, \qed$. 

\ex{Hilbert 90 on $\bbQ(i)$}{
	Let $(a, b)$ be such that $a^2+b^2=1$. Then, $(a, b)$ are of the form $$(a, b) = \left(\frac{c^2-d^2}{c^2+d^2}, \frac{2cd}{c^2+d^2} \right).$$
	
	Let $\alpha = a+bi$. It has norm 1 in $\bbQ(i)$. Thus, by Hilbert's Theorem 90, there exists an element $c+di \in \bbQ(i)$ such that $$\alpha = \frac{c+di}{\sigma(c+di)} = \frac{c+di}{c-di} = \frac{c^2-d^2}{c^2+d^2}+ \frac{2cd}{c^2+d^2},$$ thus verifying the claim.
}

For general imaginary quadratic extensions $\bbQ(\sqrt{-D})/\bbQ$, we have that $$(a, b) = \left( \frac{c^2-Dd^2}{c^2+Dd^2}, \frac{2cd}{c^2+Dd^2} \right)$$ where the derivation is analogous to above.

\thrm{Local invariant map}{
	Let $L/K$ be unramified with Galois group $G=\text{Gal}(L/K).$ Let $$\text{inv}_{L/K}:H^2(G, L^\times) \to \bbQ/\bbZ$$ be the invariant map. Explicitly, it is defined by the composition $$H^2(G, L^\times) \to H^2(G, \bbZ) \to H^1(G, \bbQ/\bbZ) \to \bbQ/\bbZ$$ where the first map is isomorphism, the second map is the differential map $\delta^{-1}$, and the last map denotes, for each homomorphism $f \in H^1(G, \bbQ/\bbZ)$, $f \mapsto f(\sigma)$. 
	
	Furthermore, there is a unique isomorphism $$\text{inv}_K: H^2(\text{Gal}(K^{\text{unr}}/K, (K^{\text{unr}})^\times)) \cong \bbQ/\bbZ$$ wherein upon composition with the inflation map $$\text{Inf}: H^2(\text{Gal}(L/K), L^\times) \to H^2(\text{Gal}(K^{\text{unr}}/K), L^\times)$$, yields the isomorphism $$\text{inv}_{L/K}:H^2(G, L^\times) \to \frac{1}{[L:K]}\bbZ/\bbZ.$$
}

\tbf{Proof:} Note that the invariant map is injective as the maps $v$ and $\delta^{-1}$ are isomorphisms, and that $f \mapsto f(\sigma)$ is injective as each homomorphism $f:G\to\bbQ/\bbZ$ is determined by the value of $f(\sigma)$.

If $L/K$ is finite unramified, then $G$ is generated by the Frobenius element $\sigma = \text{Frob}_{L/K}$, which has order $[L:K]$. Then note that each $f \in H^1(G, \bbQ/\bbZ)$ is defined by $f(\sigma)=\frac{1}{L:K}$ so the image of the invariant map contains $\frac{1}{[L:K]}\bbZ/\bbZ$. This image has order $m=[L:K]$. In fact this is not only a containment but an equality as we have $$H^1(G, \bbQ//\bbZ) \cong H^2(G, \bbZ) \cong H^0_T(G/\bbZ) \cong \bbZ/m\bbZ.$$ It thus follows, since $K$ has unramified extensions of degree $n$ for any $n$, that $\text{inv}_{L/K}:H^2(G, L^\times) \to \frac{1}{[L:K]}\bbZ/\bbZ$ is an isomorphism.

Now we prove that the isomorphism is unique. Note a functoriality condition for the invariant map. We have that 

\[
\begin{tikzcd}
H^2(\text{Gal}(L/K), L^\times) \arrow{r}{\text{inv}_{L/K}}\arrow{d}{\text{Inf}} &\bbQ/\bbZ \arrow[d, equal] \ \\
H^2(\text{Gal}(M/K), M^\times) \arrow{r}{\text{inv}_{M/K}} & \bbQ/\bbZ
\end{tikzcd}
\]

The map $\text{inv}_K = \text{inv}_{K^{\text{unr}}/K}$ induces the invariant maps $\text{inv}_{L/K}$ by the functoriality above. Thus, $\text{inv}_K$ satisfies the isomorphisms as above.
% bit of elaboration maybe?

Consider $G=\text{Gal}(K^{\text{unr}}/K)$. We have that $$H^2(G, (K^{\text{unr}})^{\times} \cong \lim\limits_{\to} H^2(G/H, (K^{\text{unr}})^\times)^H)$$ where $H$ ranges over the open normal subgroups of $G$. Therefore, $\text{inv}_K$ is uniquely determined by $\text{inv}_{L/K}$, where we take $L=(K^{\text{unr}})^H$. It ranges over all finite unramified extensions as $H$ ranges over the open normal subgroups of $G$. $\qed$

The \emph{local fundamental class} $u_{L/K}$ is the element of $H^2(L/K)$ mapped onto the generator of $\frac{1}{[L:K]}\bbZ/\bbZ$ by the invariant map.

We can extend the invariant map to arbitrary separable extensions of $K$. See Neukirch, \emph{Algebraic Number Theory}, Chapter 5, (1.1) for a proof of the following theorem:

\thrm{}{
	Let $L/K$ be cyclic with Galois group $G=\text{Gal}(L/K)$. Let the order of $G$ be $n$. Then, we have that $$\# H_T^i(G, L^\times) = \begin{cases} n &\text{if}\ k \equiv 0 \Mod2 \\ 1 &\text{if}\ k \equiv 1\Mod 2\end{cases}$$
	
	Neukirch calls this the \emph{class field axiom}.
}

\thrm{Local Artin reciprocity}{
	Let $K$ be a nonarchimedean local field. Then there exists a unique homomorphism $$\phi_K: K^\times \to \text{Gal}(K^{\text{ab}}/K)$$ with kernel $N(L^\times) = N_{L/K}(L^\times)$ such that for any uniformiser $\pi \in K$ and every finite unramified extension $L/K$, $\phi_K(\pi)$ acts on $L$ as the Frobenius $\text{Frob}_{L/K}$. 
	
	Additionally, for every finite abelian extension $L$ of $K$, the group of norms $N_{L/K}(L^\times)$ is in the kernel of $\phi_K$. $\phi_{L/K}$ also induces an isomorphism $$\phi_{L/K}: K^\times / N_{L/K}(L^\times) \cong \text{Gal}(L/K)^{\text{ab}}.$$
}

Thus, for every finite extension $L/K$, we have the following diagram:

\[
\begin{tikzcd}
K^\times  \arrow{r}\arrow{d}{\phi_K} &\text{Gal}(K^{\text{ab}}/K) \arrow{d} \\
K^\times / N(L^\times) \arrow{r}{\phi_{L/K}} &\text{Gal}(L/K)
\end{tikzcd}
\]

where we denote $\phi_K$ and $\phi_{L/K}$ as the \emph{local Artin maps} for $K$ and $L/K$ respectively. The map $\phi_{L/K}$ is called the \emph{norm residue symbol}.

\defn{Norm group}{
	Let $L/K$ be finite abelian. The subgroups of $K^\times$ of the form $N_{L/K}(L^\times)$ are the \emph{norm groups} in $K^\times$.
}

\defn{Artin map}{
	The Artin map, denoted $\phi_{L/K}$, is defined as $$\phi_{L/K}: K^\times / N_{L/K}(L^\times) \to \text{Gal}(L/K)^{\text{ab}}.$$ Note that the local Artin homomorphism is not an isomorphism as $\text{Gal}(L/K)^{\text{ab}}$ is compact and $K^\times$ is not. However if we take profinite completions, the local Artin homomorphism becomes an isomorphism. This follows from the local existence theorem, given below.
}

\tbf{Proof:} Let $M$ be a $G$-module, where $G$ is finite. Recall that by Tate's theorem, there are isomorphisms $H_T^i(G, \bbZ) \to H_T^{i+2}(G, M)$ which are canonical up to a choice of generator $\sigma$ of $H^2(G, M)$. 

Let $G = \text{Gal}(L/K)$ and $M=L^\times$. We thus show that for a Galois extension $L/K$, there exists a canonical isomorphism $$H_T^i(G, \bbZ) \cong H_T^{i+2}(G, L^\times).$$ Comparing the $-2$'th and $0$'th Tate groups yields us the equality $$\text{Gal}(L/K)^{\text{ab}} = H_T^{-2}(\text{Gal}(L/K), \bbZ) \cong H_T^0(\text{Gal}(L/K), \bbZ) = H_T^0(L/K) = K^\times / N_{L/K} L^\times$$ thus achieving the isomorphism $$K^\times / N_{L/K} L^\times \to \text{Gal}(L/K). \qed$$

The isomorphism induces an exact sequence $$1 \to N_{L/K}L^\times \to K^\times \to \text{Gal}(L/K)^{\text{ab}} \to 1.$$ As you would've noticed, the map $K^\times \to \text{Gal}(L/K)^\text{ab}$ is equivalent to $\phi_{L/K}$, the norm residue symbol.

The norm residue symbol is denoted $(\cdot, \cdot)$ in many texts. Here we denote it $[\cdot, L/K]$ to avoid clashing with the Hilbert symbol, as below. Explicitly, $\phi_{L/K}$ gives us the mapping $$(\cdot, L/K): K^\times \to \text{Gal}(L/K)^{\text{ab}}$$ which has kernel $N_{L/K} L^\times$. 

\ex{}{
	Let $\zeta_m$ be a primitive $m$'th root of unity with $(m, p)=1$. Then $\bbQ_p(\zeta_m)/\bbQ_p$ is unramiifed, and the norm residue symbol over $\bbQ_p$ is given by $$[a, \bbQ_p(\zeta)/\bbQ_p](\zeta) = \zeta^{p^{v_p(a)}}.$$
	
	Let $\zeta$ be a primitive $p^n$'th root of unity. In this case, we have that $$[a, \bbQ_p(\zeta)/\bbQ_p](\zeta) = \zeta^r$$ where $r \equiv u^{-1} \Mod{p^n}$. 
}

The norm residue symbol is synonymous with the Hilbert symbol. We will explore this similarity in the end of this section.

Let's further explore the Artin map. Let $\pi$ be a uniformiser. Then we have, by the local reciprocity theorem, a subfield of $K^{\text{ab}}$ fixed by $\phi_K(\pi)$. Denote that group by $K_\pi$. Denote the maximal unramified extension $K^{\text{unr}}$ as the subfield of $K^{\text{ab}}$ fixed by $\phi_K(U_K)$, where $U_K$ is the unit group. 

\ex{Local reciprocity for $\bbQ_p$}{
	Let $K=\bbQ_p$, and let the uniformiser $\pi = p.$ Then we have the decomposition $\bbQ_p^{\text{ab}} = K_1K_2$ where $$K_1 = \bigcup\limits_n \bbQ_p(\zeta_{p^n}) \quad K_2 = \bigcup\limits_n \bbQ_p(\zeta_{p^n-1})$$ and that $$\text{Gal}(\bbQ_p^{\text{ab}}/\bbQ_p) \cong \text{Gal}(K_1/K) \times \text{Gal}(K_2/K).$$ Since $p$ is totally ramified, we have that $\text{Gal}(K_1/K) \cong \bbZ_p^\times$, and that $\text{Gal}(K_2/K)=\hat{\bbZ}$. Therefore there is a restriction of $\phi_K$ to the isomorphism $$U_K \cong \text{Gal}(K_\pi / K).$$
}

\thrm{Norm limitation theorem}{
	Let $L/K$ be a finite extension and $M/K$ be the maximal abelian subextension. Then, $N(L^\times)=N(M^\times)$.
}

\tbf{Proof:} Since the norm map is transitive, $N_{L/K}(L^\times) \subseteq N_{L/K}(M^\times)$. Therefore if $M/K$ is Galois, then $\text{Gal}(M/K) = \text{Gal}(M/K)^{\text{ab}}$ and thus by Tate's theorem, both norm groups have index $[M:K]$ in $K^\times$.. Thus, the norm groups are equal.

Let $L'/K$ contain $L/K$. Let $G=\text{Gal}(L'/K)$ and $H=\text{Gal}(L/K)^{\text{ab}}$. Therefore, $M$ is the largest subfield of $L'$ that is abelian over $K$ and contained in $L$. Thus the subgroup of $G$ fixing $M$ is $G_d \cdot H$, where $G_d$ is the derived group of $G$. By the commutative diagram

\[
\begin{tikzcd}
L^\times  \arrow{r}\arrow{d}{\phi_{L'/L}} &H/H' \arrow[d, hook] \\
K^\times  \arrow{r}{\phi_{L/K}} \arrow{d}{\text{id}} &\text{Gal}(L/K) \arrow[d, two heads]\\
K^\times \arrow{r}{\phi_{M/K}} &G/G_dH
\end{tikzcd}
\]

Let $a \in N(M^\times)$. The element $\phi_{L'/K}(a) \in G/G_d$ maps to $1$ in $G/G_dH$. The surjectivity of $\phi_{L'/L}$ implies that there exists a $b \in L^\times$ such that $$\phi_{L'/K}(a) = \phi_{L'/K}(N(b))$$ and thus $a/N(b) \in N((L')^\times)$. Let $a/N(b) = N(c)$. Then, $$a = N_{L/K}(b \cdot N_{L'/L}(c)) \in N_{L/K}(L^\times). \qed$$

\thrm{Local existence theorem}{
	\tbf{Proposition 1}: Let $L/K$ be finite. Then, $N_{L/K} (L^\times)$ is an open subgroup of $K^\times$ of finite index, where $N(L^\times)$ is the norm group. 
	
	\tbf{Proposition 2}: For every open subgroup $U \subseteq K^\times$ of finite index, there also exists a finite abelian extension $L/K$ such that $U = N_{L/K} L^\times$. 
}

\tbf{Proof (Proposition 1):} Recall that the local reciprocity law follows directly from Tate's theorem as the isomorphism $$\text{Gal}_{L/K}^{\text{ab}} \cong K^\times / N(L^\times)$$ directly translates to the equality $$H^{-2}(\text{Gal}(L/K), \bbZ) \cong H^0(L/K).$$ 

By the local reciprocity law, every norm group $I_L=N_{L/K}(L^\times)$ has finite index in $K^\times$. This is given by the correspondence $$L \mapsto N_{L/K}(L^\times) \subseteq K^\times.$$ Moreover, we can deduce from this correspondence that every group containing a norm group is a norm group.

Let $[K^\times: I_L]=m$. Note that $(K^\times)^m$ is open since for any $x^m \in (K^\times)^m$, we can construct $x^m \cdot U^{n+v(m)} \subseteq (K^\times)^m$ to be an open neighbourhood of $x^m$, and take $n$ to be sufficiently large. Then we can show that $I_L$ is the union of open cosets of ${(K^\times)}^m$ in $I_L$. Let $I \subseteq K^\times$ be of index $m$. Thus, ${(K^\times)}^m \subset I_L$. Note that if $I$ is a norm group, then $(K^\times)^m$ is also a norm group. Let $a \in K^\times$. 

Firstly, consider the case where $K$ contains the $m$'th roots of unity. Let $\alpha \in K^\times$ and $L_\alpha = K(\sqrt[m]{\alpha})$, and set $$L = \bigcup\limits_{\alpha \in K^\times} L_\alpha.$$ Then $L/K$ is finite abelian as $K^\times/{(K^\times)}^m$ is finite (use the Herbrand quotient). Thus, there are only finitely many distinct fields among the $L_\alpha$. 

Since $[L_a:K] = [K(\sqrt[m]{\alpha}):K]=d$ is a divisor of $m$, the inclusion $(K^\times)^d \subseteq I_{L_\alpha}$ implies that $(K^\times)^m \subseteq I_{L_\alpha}$ and thus $(K^\times)^m \subseteq I_L$. 

% details in later versions? neukirch bonn
Kummer theory shows us that $$K^\times / (K^\times)^m \cong H^1(G, \mu_m) \cong \chi(\text{Gal}(L/K))$$ where $\mu_m$ is the group of units and $\chi$ denotes the character group of the Galois group. Thus, $$[K^\times: (K^\times)^m] = |\text{Gal}(L/K)| = [K^\times:I_L].$$ By this, $(K^\times)^m = I_L$ and thus $(K^\times)^m$ is a norm group.  

Then, consider the case where $K$ does not contain the $m$'th roots of unity. Then it must be contained in a field $K_1$ which contains the $m$'th roots of unity. In this case, $(K_1^\times)^m$ is the norm group of the extension $L/K_1 = N_{L/K_1} L^\times$. If $\mcL$ is the smallest normal extension of $K$ containing $L$, then 

\begin{align*}
	N_{\mcL/K}(\mcL^\times) &= N_{K_1/K}(N_{\mcL/K_1}(\mcL^\times)) \\
	&\subseteq N_{K_1/K}(N_{L/K_1}(\mcL^\times)) \\
	&= N_{K_1/K}((K_1^\times)^m)\\
	&= (N_{K_1/K}(K_1^\times))^m \subseteq (K^\times)^m
\end{align*}

and thus $N_{\mcL/K} \mcL^\times \subseteq (K^\times)^m$. Therefore, $(K^\times)^m$ is a norm group. $\qed$

\tbf{Proof (Proposition 2):} Let $U_K$ be the unit group of $K$. Note that it suffices to construct $L$ such that $N_{L/K}(L^\times) \subseteq U$. We have that $$\text{Gal}(L/K) \cong K^\times / N_{L/K} (L^\times)$$ (by local reciprocity); thus, $U/N_{L/K}(L^\times)$ corresponds to $\text{Gal}(L/M)$ for some intermediate extension $M/K$. Then since the norm limitation theorem says that $N_{L/K}(L^\times) = N_{L/K}(M^\times)$ for $M$ the maximal abelian subextension of $L/K$, it then follows that we can produce any finite extension $L/K$ such that $N_{L/K}(L^\times) \subseteq U$.

Let $m\bbZ$ be the image of $U$ in $K^\times / U_K^\times$. Choosing $L$ to contain the unramified extension in $K$ of degree $m$, it follows that the image of $N_{L/K}^L\times$ in $K^\times / U_K^\times$ is contained in $m\bbZ$. Since $U_K^\times$ is compact, then the open subgroups $N_{L/K}(L^\times) \cap U_K^\times$ is closed and thus compact. Since $L/K$ runs over the finite extensions of $K$, the intersection $$N_{L/K}(L^\times) \cap U_K^\times$$ is trivial and thus $$(N_{L/K}(L^\times) \cap U_K^\times) \cap (U_K^\times - U)$$ is empty. Note that both sets are open in $U_K^\times$. Thus, there exists a single $L/K$ such that $$N_{L/K}(L^\times) \cap U_K^\times \subseteq U \cap U_K^\times$$ and the claim is satisfied. $\qed$

We'll end this section off with a study of the Hilbert symbol. We'll first study it for $p$-adic numbers, then for general $K$ where $K$ is profinite.

\defn{Hilbert symbol for $p$-adic numbers}{
	Let $a$ and $b$ be $p$-adic numbers. The Hilbert symbol $(\cdot, \cdot)$ is defined as: $$(a, b) = \begin{cases} 1 & \text{if}\ z^2=ax^2+by^2\ \text{has a nonzero solution} \\ -1 & \text{otherwise} \end{cases}$$
}

It's clear that this is a pairing from $\bbQ_p / (\bbQ_p)^2 \times \bbQ_p / (\bbQ_p)^2 \to \{\pm 1\}$. Evidently, the Hilbert symbol is only well-behaved for local fields.

\defn{Hilbert symbol}{
	Let $K$ be profinite. Local class field theory tells us that there exists an isomorphism $$K^\times \cong \text{Gal}(K)^{\text{ab}}.$$ Let $L= K(\sqrt[n](K^\times))$ be the maximal abelian extension of $K$. Then there is a canonical isomorphism $$\text{Gal}(L/K) \cong K^\times / (K^\times)^n.$$
	
	From this there exists a pairing $$(\cdot, \cdot): K^\times/(K^\times)^n \times K^\times/(K^\times)^n \to \mu_n.$$
	
	We denote $(a, b)$ as the Hilbert symbol.
}

The Hilbert symbol for $p$-adic numbers is the Hilbert symbol for $K=\bbQ_p$, $n=2$. 

\defn{Relation between Hilbert symbol and norm residue symbol}{
	Let $[\cdot, \cdot]$ denote the norm residue symbol, and let $(\cdot, \cdot)$ denote the Hilbert symbol. For a field extension $L/K$, we have that $$[a, K(\sqrt[n]{b})/K] = (a, b) \sqrt[n]{b}.$$
}

The image of $a$ under the local reciprocity isomorphism is is $[a, L/K]$. Meanwhile, the image of $b$ under the isomorphism $K^\times / (K^\times)^n \cong \text{Hom}(\text{Gal}(L/K), \mu_n)$ is given by the character $\chi_b: \text{Gal}(L/K) \to \mu_n$ explicitly given by $$\chi_b(\tau) = \tau (\sqrt[n]{b}) / \sqrt[n]{b}.$$ Thus we have that $$(a, b)= \chi_b(\phi_K) = \phi(\sqrt[n]{b})/\sqrt[n]{b}$$ and thus we obtain the equivalence.

\prop{Properties of Hilbert symbols}{
	The Hilbert symbol satisfies the following properties:
	
	\begin{enumerate}
		\item $(a, bb') = (a, b)(a, b')$ and $(aa', b) = (a, b)(a', b)$.
		\item $(a, b)=1$ if and only if $b$ is a norm from $K(\sqrt[n]{a})$.
		\item $(a, 1-a)=1, (a, -a)=1$
		\item $(a, b) = (b, a)^{-1}$
	\end{enumerate}
}

\tbf{Proof:} \begin{enumerate}
	\item Follows from the definition of the Hilbert symbol. $\qed$
	\item Restrict the Artin map from $K^{\text{ab}}/K$ to $K(\sqrt[n]{b})$. This is equivalent to the Artin map for $K(\sqrt[n]{b})/K$. Thus it is clear that $(a, b)=1$ if and only if $(a, K(\sqrt[n]{b})/K)$ fixes $\sqrt[n]{b}$ (and thus if and only if it fixes $K(\sqrt[n]{b})$.) Thus $a$ is in the kernel of the Artin map of $K(\sqrt[n]{b})/K$, meaning that $a$ is a norm from $K(\sqrt[n]{b}). \qed$
	\item $0^n -(-\alpha)=\alpha$ is a norm from $K(\sqrt[n]{-\alpha})$ so $(\alpha, -\alpha)=1$. Similarly, $1^n-\alpha = 1-\alpha$ is a norm from $K(\sqrt[n]{-\alpha})$ so $(\alpha, 1-\alpha)=1$. (explicitly, for $\zeta$ being a primitive $n$'th root of unity, we have that $1-a = \prod\limits_{i=1}^n (1-\zeta^i \sqrt[n]{a})$.) $\qed$
	\item $1 = (ab, -ab) = (a, -a)(a, b)(b, a)(b, -b) = (a, b)(b, a)$ so the result follows. $\qed$
\end{enumerate}

It follows from the second bullet that if $(a, b) = 1$ for all $b \in K^\times$, then $a \in (K^\times)^n$.

The content below makes sense after studying Artin reciprocity. Let $K$ be a number field, and let $V_K$ be the set of all (archimedean, non-archimedean) valuations. By Artin reciprocity, we have that 

\begin{align*}
	\prod\limits_v (a, b)&= \prod\limits_v \chi_b (a, K_v(\sqrt[n]{b}))  \\
	&= \left( \frac{\prod_v (a, K_v(\sqrt[n]{b}))(\sqrt[n]{b})}{\sqrt[n]{b}} \right) \\
	&= \left(\frac{\text{id}(\sqrt[n]{b})}{\sqrt[n]{b}} \right) \\
	&= 1.
\end{align*}

Let $U$ be the group of units of a local field $K$. Then from the restricted Hilbert symbol $(\cdot, \cdot): U \times U \to \bbC^\times$, find a finite index subgroup $N \leq U$ such that whenever $x \in U$ or $y \in U$, then $(x, y) = 1$. The pairing $$U/N \times U/N \to \bbC^\times$$ then follows. Thus, for any $x, y \in U\times U$, the symbol $(x, y)$ is dependent on the coset representatives of $x$ and $y$ in $U/N$. 

We can extend the Hilbert symbol to the archimedean completions. Note that in the case $K=\bbR$, it holds that $n=1$ or $n=2$. Evidently, $(a, b)=1$ when $n=1$. For $n=2$, we have that $$(a, b) = (-1)^{\frac{\alpha-1}{2} \cdot \frac{\beta-1}{2}}$$ where $\alpha = \text{sgn}(a)$ and $\beta = \text{sgn}(b)$. This underscores a map $\bbR^\times \times \bbR^\times \to \{-1, 1\}$.

Let $N = (0, \infty)$ and let $(x, y)$ be a Hilbert symbol. When $y \in N$, then $\bbR(\sqrt{y})=\bbR$ so the Artin map is trivial. If $x \in N$, then $x$ is a norm from $\bbR(y)$ no matter if $\bbR(y) = \bbR$ or $\bbC$, or $y \in N$. It swiftly follows that $x \in \left(\frac{\bbR(y)}{\bbR}\right)$. Now we compute the Hilbert symbol at different coset representatives. Now since $\bbR(-1)=\bbC$ and $-1$ is not a norm from $\bbC$, it then follows that $(-1, -1)=-1$.

A brief survey of the Hilbert symbol is seen in Serre's \emph{A Course In Arithmetic}. 

\chapter{Class Field Theory II}
\minitoc


This chapter aims to introduce and prove the main theorems of class field theory and explore the notion of "reciprocity". We began studying reciprocity with quadratic reciprocity and continued with local Artin reciprocity in the last chapter. Notably, we take an idelic perspective (a la Chevalley) to derive global class field theory from local class field theory. 

We'll quickly state the main statements of class field theory in its original ideal-theoretic language. The proofs will be idele-theoretic instead of ideal-theoretic, as divulged in subsequent sections, with jargon and terminology being entirely consistent. We use the idele-theoretic way as it allows us to describe the infinite abelian extensions of $K$. With the ideal-theoretic language, we have to fix a modulus $\mfm$ and only consider the finite extensions of $K$.

\section{Preliminaries}

Let $C_K$ be a class group. Just like in local class field theory, a subgroup of $C_K$ is a norm group if it is of the form $N(C_L)$ for some extension $L/K$.

\defn{Modulus}{
	Let $K$ be a number field. A modulus $\mfm$ over $K$ is a formal product $$\mfm = \prod\limits_v v^{\mfm(v)}$$ such that $n_\mfp \geq 0$ and $n_\mfp =0$ for all but finitely many primes.
	
	Some texts choose to call the modulus a \emph{cycle}. We can view $\mfm$ as a formal product over its set of places, which we can factor as $$\mfm=\mfm_0\mfm_\infty$$ where $\mfm_0 = \prod\limits_{p \not\mid \infty} \mfp^{\mfm(\mfp)}$ and $\mfm_\infty = \prod\limits_{v \mid \infty} v^{\mfm(v)}$
}

Let $\mfa \in \bbI_K$ be an idele of a number field $K$, and let $\mfm$ be a modulus. Let $C_K$ represent the idele class group. Set the idele group $$\bbI_K^m = \{\mfa \in \bbI_K | \mfa \equiv 1\Mod\mfm \}= \prod\limits_\mfp U_\mfp^{n_\mfp} \subseteq \bbI_K.$$

Then the group $$C_K^\mfm = (I_K^\mfm \times K^\times) / K^\times$$ is the \emph{congruence subgroup} $\Mod\mfm$, and the quotient group $$C_K/C_K^\mfm$$ is the \emph{ray class group} $\Mod\mfm$.

This idele-theoretic treatment of the ray class group has its ideal-theoretic analogue.

\defn{Ray class group}{
	Let $J_K^\mfm \subseteq \mcI_K^\mfm$ be the subgroup of fractional ideals coprime to $\mfm$, and let $P_K^\mfm \subseteq \mcI_K^\mfm$ be the subgroup of principal fractional ideals in $\mcI_K^\mfm$. Then the \emph{ray class group} is the factor group $J_K^\mfm / P_K^\mfm$.
	
	(c.f.) The ideal class group $C_K$ is defined as $J_K/P_K$. Note that the ray class group modulo 1 is isomorphic to the ideal class group, with its order being equal to the class number of $K$. This is because $$C_K/C_K^1 = (\bbI_K / K^\times)/ (\bbI_K^1 \cdot K^\times  / K^\times) \cong \bbI_K / (\bbI_K^{S_\infty}\cdot K^\times).$$
}

When $\mfm=1$, then $$\mcI_K^1 = \prod\limits_{p \mid \infty} K_\mfp^\times \times \prod\limits_{p\not\mid \infty} U_p.$$ The ray class group is the same as the usual idele class group. 

\ex{}{
	Let $K=\bbQ$ and $\mfm=8$. Then, \begin{align*} J_\bbQ^8 &= \{\frac{a}{b}\bbZ | \frac{a}{b} \equiv 1, 3, 5, 7 \Mod8\} \\
	P_\bbQ^8 &= \{\frac{a}{b}\bbZ | \frac{a}{b} \equiv 1, 7 \Mod8\}
	\end{align*}
	
	since $7\bbZ = -7\bbZ = 1 \Mod8$.
	
	Therefore, $C_\bbQ^8 = J_\bbQ^8/P_\bbQ^8 = (\bbZ/m\bbZ)^\times / \{\pm1\}.$
}

\prop{}{
	The norm groups of $C_K$ form a bijection with those containing the congruence subgroups $C_K^\mfm$. 
}

\tbf{Proof:} First note that $C_K^\mfm$ is open in $C_K$ as the idele class group $\bbI_K^\mfm$ is open in $\bbI_K$. Also note that $\bbI_K^\mfm$ is contained in the group $I_K^{S_\infty}.$ Since $[C_K: I_K^{S_\infty}(K^\times)/ K^\times] = |C_K| = h$ (which is finite), then the index 

\begin{align*}
	[C_K:C_K^\mfm] &= [C_K:C_K^1] \cdot [C_K^1:C_K^\mfm] \\
	&= h \cdot [C_K^1:C_K^\mfm] \\
	&= h \cdot [\bbI_K^1 \cdot K^\times:\bbI_K^\mfm \cdot K^\times]\\
	&\leq h \cdot [\bbI_K^1:\bbI_K^\mfm] \\
	&= h \cdot \prod\limits_p [U_\mfp:U_\mfp^{n_\mfp}]
\end{align*}

is also finite. Consequently, every group containing $C_K^\mfm$ is also closed and of finite index as it is the union of a finite number of cosets in $C_K^\mfm$.

Let $H$ be a closed subgroup with preimage $J$. Note that $H$ is the complement of a finite number of closed cosets, and is thus open. The preimage $J$ is also open and contains a subset of form $$\prod\limits_{\mfp \in S} W_\mfp \times \prod\limits_{\mfp \not\in S} U_\mfp$$ where $S$ is a finite set of places and $W_\mfp$ is an open neighbourhood of unity. 

If $\mfp \in S$ is finite, choose $W_\mfp.= U_\mfp^{n_\mfp}$ as those groups form a system of neighbourhoods of unity in $K_\mfp^\times$. Else choose $W_\mfp \subset \bbR^\times_+$ as $\mfp$ is real. Then the open set will generate the group $\bbR_+^\times$ when $\mfp$ is complex. Thus, the subgroup of $J$ generated by $W$ is of the form $\bbI_K^\mfm$ so $N$ contains the subgroup $C_K^\mfm$. $\qed$

The ray class fields correspond to the different moduli of $K$, with larger moduli corresponding to smaller congruence subgroups. 

\defn{Ray class field}{
	Note that a modulus has finite support. A finite abelian extension $L/K$ that is unramified at all places not in the support of $\mfm$ wherein the kernel of the Artin map $$\phi_{L/K}: \mcI_K^\mfm \to \text{Gal}(L/K)$$ is the \emph{ray class field} of $\mfm$ (we will further elaborate on this map in this chapter). 
	
	Put otherwise, the ray class field of $K$ belonging to a norm group $N(C_L)$ is the extension $L/K$ associated with said norm group.
}

By Artin reciprocity (as we will show later in this chapter), the Galois group $\text{Gal}(L/K)$ is isomorphic to the ray class group $C_K/C_K^\mfm$. Consider when $K=\bbQ$. I want to show that the ray class fields over $\bbQ$ are the fields corresponding to the field containing the roots of unity $\bbQ(\zeta_m)$. This involves the \emph{existence theorem}. We'll learn its proof after studying the reciprocity law. 

\thrm{(Global) Existence theorem}{
	Let $m$ be a positive integer and $p_\infty$ the infinite prime of $\bbQ$. Let $\mfm = m \cdot p_\infty$. Then, the ray class field modulo $\mfm$ is $\bbQ(\zeta_m)$.
}

For now, the main idea is that due to the correspondence given by the existence theorem, every abelian extension of $K$ is a subfield of a ray class field. Furthermore, the ray class fields correspond to the different moduli of $K$, where the larger modulus corresponds to the smaller congruence subgroup. If $\mfm \mid \mfm'$, where $\mfm$ and $\mfm'$ are two moduli over $K$, then $$C_K/C_K^\mfm \subseteq C_K/C_K^{\mfm'}.$$

However there is one ray class field of importance. This is the Hilbert class field, or the ray class field with modulus $\mfm=1$ and thus associated with the congruence group $C_K^1$. 

\defn{Hilbert class field}{
	Let $K$ be a global field. The Hilbert class field of $K$ is the maximal unramified abelian extension of $K$. Moreover, as shown by the global existence theorem, it is the ray class field with trivial modulus ($\mfm=1$), and is associated with the congruence group $C_K^1.$
}

Note that the Hilbert class field must be a finite extension. In particular, infinite unramified extensions of number fields exist. An easy way to construct them is by constructing a nested sequence of Hilbert class fields. Start with $K_0$ and let $K_{n+1}$ be the Hilbert class field of $K_n$. This yields an infinite sequence of finite abelian extensions, or a class field tower: $$K_0 \subseteq K_1 \subseteq K_2 \subseteq \dots.$$ Let $L = \bigcup\limits_n K_n$. Then we can either reach a field of class number 1. Denote such a field by $K_n$. Then, for all $m\geq n$ we have that $K_n = K_m$. Or we have the case where we can construct an infinite class field tower.

Our idea here is to find a number field with sufficiently many prime factors in its discriminant to produce an infinite class field tower. Golod and Shafarevich found that $$K_0 = \bbQ(\sqrt{-30030}) = \bbQ(\sqrt{-2\cdot3\cdot5\cdot7\cdot11\cdot13})$$ is the base of an infinite class field tower.

Let $\mfm$ be divisible by all primes of $K$ that ramify in $L$. Then, there exists an \emph{global Artin map} that sends $$J_K^\mfm \to \text{Gal}(L/K)$$ through the Frobenius mapping $\mfp \to \text{Frob}(\mfp)$. 

\ex{}{
	Let $K =\bbQ(\sqrt{d})$ where $d = \prod\limits_i p_i^*$ for $p$ prime. Define $2^*=8, (-1)^*=-4, p^*=p$ for $p \equiv 1\Mod{4}$ and $p^*=-p$ for $p \equiv 3\Mod{4}.$ 
	
	Let $H(K)$ be the Hilbert class field of $K$. Then, $$K=\bbQ(\sqrt{p_1^*}, \sqrt{p_2^*}, \dots, \sqrt{p_r^*}) \subset H(K).$$
}

A key result of class field theory is that the map $$\bbI_K = J_K^\mfm / P_K^\mfm \to \text{Gal}(L/K)$$ is surjective.

\defn{Conductor}{
	Let $L/K$ be the ray class field that corresponds to a modulus $\mfm$. The conductor $\mff$ is the smallest modulus $\mfm$ corresponding to the ray class field such that $\mff \mid \mfm$ and the map $$J_K / J_K^\mfm \to \text{Gal}(L/K)$$ is an isomorphism.
	
	Let $L/K$ be an abelian extension with norm group $N(C_L)$. Similarly, the conductor $\mff$ of $L/K$ is the greatest common divisor of all moduli $\mfm$ such that $C_K^\mfm \subseteq N(C_L)$.
}

\section{Global class field theory}
% Takagi 1920
To prove most of the theorems in class field theory, we must introduce the Artin reciprocity law - the poster-child of global class field theory. This is done by producing an adelic formulation of Artin reciprocity.

Consider the ideal-theoretic language used in the last chapter. Artin reciprocity states that there exists a modulus $\mfm$ over finite places of $K$ which include all places where $L$ ramifies such that the subgroup of principal fractional ideals $P_K^\mfm$ is the kernel of the Artin map.

Recall from local class field theory that we have the local Artin homomorphism $$\phi_K: K^\times \to \text{Gal}(K^{\text{ab}}/K)$$ such that for every finite extension $L/K\in K^{\text{ab}}$ there exists a homomorphism $$\phi_{L/K}: K^\times \to \text{Gal}(L/K)$$ (produced by composing $\phi_K$ with $\text{Gal}(K^{\text{ab}}/K) \to \text{Gal}(L/K)$) such that for any uniformiser $\pi$, the reciprocity map sends $$\phi_{L/K}: \pi \mapsto \text{Frob}_{L/K}(\pi).$$

Let $v$ and $w$ be places of $K$ and $L$ respectively. It's clear that there is a surjection $$\phi_{L_w/K_v}: K_v^\times \to \text{Gal}(L_w/K_v)$$ with kernel being the norm group $N_{L_w/K_v}(L_w^\times)$. Furthermore the Frobenius map also holds, for that we have $$\phi_{L_w/K_v}: \pi_v \mapsto \text{Frob}_{L_w/K_v}(\pi_v).$$

There is a convenient embedding of Galois groups $\varphi_w: \text{Gal}(L_w/K_v) \xhookrightarrow\ \text{Gal}(L/K)$ by sending $\sigma\mapsto \sigma_L$. Note that we can write each element of $L_w$ as $lx$ for some $l\in L$ and $x \in K_v$. Thus, each $\sigma$ is determined by its action on $L$ (which fixes $K$). 

Recall that the decomposition group $D(w)$ is defined as $$D(w) = \{ \sigma \in \text{Gal}(L/K) | \sigma w = w\}$$ For a place $v \in K$, the Galois group $\text{Gal}(L/K)$ acts on the set of places $w$ that divide $v$. Thus the decomposition group is isomorphic to $\varphi_w(\text{Gal}(L_w/K_v))$ of $w$.

The local Artin map tells us that there is a homomorphism $\phi_v: K_v^\times \to D(w) \cong \text{Gal}(L_w/K_v) \subset G$ created by composing $\varphi_w$ with the local Artin map. Note also that the choice of the map $\phi_v$ is independent of the prime $w|v$. We see that when $v$ is an unramified place then the composition map $\varphi_w(\phi_{L_w/K_v}(\pi_v)) = \text{Frob}_v$ for each uniformiser $\pi_v$ which generate $K_v^\times$, and thus determine the composition map.

Let $v$ be a place of $K$. We can embed $K_v^\times$ into $\bbI_K$ by sending 
\begin{align*}
	K_v^\times &\xhookrightarrow\ \bbI_K \\
	\alpha &\mapsto (1, 1, \dots, 1, \alpha, 1, \dots)
\end{align*}

which is a reformulation of the idele norm. (recall the commutativity condition associated with the idele norm)

The following diagram exhibits the link between local and global class field theory, and is arguably the most informative diagram of this text. Let $\phi_K$ be the Artin homomorphism that sends $\bbI_K \to \text{Gal}(K^{\text{ab}}/K)$. It should be clear that the local Artin map $\phi_{L/K}$ is the homomorphism making the diagram commute, where $\phi_{L/K}(x) = \phi_K(x) |_L$. 

\[
\begin{tikzcd}
K_v^\times \arrow{r}{\phi_{L_w/K_v}} \arrow[d, hook] &\text{Gal}(L_w/K_v) \arrow[d, hook] \\
\bbI_K \arrow{r}{\phi_{L/K}} & \text{Gal}(L/K)
\end{tikzcd}
\]

We now show that the diagram above commutes. Let $i \in \bbI$, and let $L \subset K^{\text{ab}}$ be finite over $K$. Let $a_v \in U_v$ and let $L_w/K_v$ be unramified. Then, $\phi_v(a_v)=1$. Thus we have that $\phi_v(a_v)=1$ for all but finitely many $v$. Define $$\phi_{L/K}(a) = \prod\limits_v \phi_v(a)v.$$ 

If $L \subset L'$, then the local Artin map shows that there exists a unique homomorphism $$\phi: \bbI \to \text{Gal}(K^{\text{ab}}/K)$$ that sends $\phi(a)|_L = \phi_{L/K}(a)$ for all $L \subset \text{Gal}(K^{\text{ab}})$ such that the following diagram commutes:

\[
\begin{tikzcd}
\bbI_{K'}^S \arrow{r}{\phi_{L/K'}} \arrow[d, hook] &\text{Gal}(L/K') \arrow[d, hook] \\
\bbI_K^S \arrow{r}{\phi_{L/K'}} & \text{Gal}(L/K)
\end{tikzcd}
\]

Let $K'=L$. Then, $N_{L/K}(\bbI_L^S)$ is contained in the kernel of $\phi_{L/K}$. Actually the kernel contains an open subgroup of $\bbI_K^S$ (why?) and thus $\phi_K$ is continuous. 

We can now conjure a proof to the adelic version of Artin reciprocity, as well as the adelic existence theorem. This process roughly follows the work of Artin and Tate.

\thrm{Adelic reciprocity}{
	The homomorphism $\phi_K: \bbI_K \to \text{Gal}(K^{\text{ab}}/K)$ is such that $\phi_K(K^\times)=1$. It also defines an isomorphism $$\phi_{L/K}: C_K / N(C_L) \cong \text{Gal}(L/K).$$
}

\thrm{Adelic existence theorem}{
	Let $K^{\text{al}}$ be an algebraic closure of $K$, and let $N \subset C_K$ be an open subgroup of finite index. Then, there exists a unique abelian extension $L/K \subseteq K^\text{al}$ such that $N_{L/K}(C_L)=N$.
}

Our first step is to prove that the idele class group satisfies the \emph{class field axiom}, as termed by Neukirch. Let $L/K$ be finite Galois with Galois group $G$. Then, 

\begin{enumerate}
	\item $H^1(G, C_L)=0$
	\item $H^2(G, C_L) = [L:K]$.
\end{enumerate}

and we can apply Tate's theorem to obtain an isomorphism $H_T^{-2}(G, \bbZ) \to H_T^0(G, C_L)$. 

Let $L/K$ be finite and Galois, and let $\mfp$ be a place of $K$. Furthermore let $\mfp$ decompose into a product of primes $\mfP$. Then the $S$-ideles satisfy $$\bbI_L^S = \prod\limits_{\mfP \mid \mfp \in S} L_\mfP^\times \times \prod\limits_{\mfP \mid \mfp \not\in S} U_\mfP = \prod\limits_{p \in S} \prod\limits_{\mfP \mid \mfp} L_\mfP^\times \times \prod\limits_{p \in S} \prod\limits_{\mfP \mid \mfp} U_\mfP$$ wherein the products $\bbI_L^\mfp=\prod\limits_{\mfP \mid \mfp} L_\mfP^\times$ and $U_L^\mfp = \prod\limits_{\mfP \mid \mfp} U_\mfP$ are subgroups of $\bbI_L^S$. Here the elements in $\bbI_L^\mfp$ have component 1 at all the primes of $L$ not lying above $\mfp$. We can thus represent $\bbI_L^\mfp$ and $U_L^\mfp$ as $G$-modules as the automorphisms $\sigma$ only permute the primes above $\mfp$. Therefore, we have that $$\bbI_L^S = \prod\limits_{\mfp \in S} \bbI_L^\mfp \times \prod\limits_{\mfp \not\in S} U_L^\mfp.$$

A simple result is that $\bbI_K^S = H^0(G, \bbI_L^S)$. This is because given ay place $v \in K$ and any finite separable extension $L/K$, we can decompose $K_v \otimes K = K_{\alpha_1} \times \dots\times K_{\alpha_r}$, where $\alpha_1, \dots, \alpha_r$ are the extensions of $v$ to $K$. From this we can freely inject $\bbI_K^S$ into $\bbI_L^S$ via the decomposition map. 

Let $\sigma \in G$, and let $w \in K$ be a place above $v \in K$. Considering components, $(\sigma\alpha)_{\sigma\omega}$ evaluates to $(\sigma\alpha)_{\sigma\omega}=\alpha_\omega=\alpha_{\sigma\omega}$. Thus, $\sigma\alpha = \alpha$. 

Conversely, if $\alpha = (\alpha_\omega) \in H^0(G, \bbI_L^S)$, then we can write $(\sigma\alpha)_{\sigma\omega}=\sigma\alpha_\omega=\alpha_{\sigma\omega}$

Then we invoke the fact that $G$ acts transitively on the places $w$ above $v$. Thus, the $\alpha_w$ that produce $w|v$ come from the same $\alpha_v \in K_v^\times$ through the embedding $K_v^\times \to L_w^\times$. Thus we have that $\alpha \in \bbI_K. \qed$

\prop{}{
	The composition map $$H^i(G, \bbI_L^\mfp) \to H^i(G_\mfP, \bbI_L^\mfp) \to H^i(G_\mfP, L_\mfP^\times)$$ that takes each idele in $\bbI_L^\mfp$ to its component in $\mfP$ induces the isomorphism $H^i(G, \bbI_L^\mfp) \cong H^i(G_\mfP, L_\mfP^\times)$ where $G_\mfP = \text{Gal}(L_\mfP/K_\mfp)$. 
}

\tbf{Proof:} Let $\sigma \in G/G_\mfP$ run through the coset representatives of the cosets $G/G_\mfP$. Then, $\sigma \mfP$ runs through all distinct primes of $L$ above $\mfp$. We thus have that $$\bbI_L^\mfp = \prod\limits_{\sigma \in G/G_\mfP} L_{\sigma\mfP}^\times = \prod\limits_{\sigma \in G/G_\mfP} \sigma L_\mfP^\times$$ and $$U_L^\mfp = \prod\limits_{\sigma \in G/G_\mfP} U_{\sigma\mfP} = \prod\limits_{\sigma \in G/G_\mfP} \sigma U_\mfP.$$

Then we can apply Shapiro's lemma to show that $H^i(G, \bbI_L^\mfp) \cong H^i(G_\mfP, L_\mfP^\times)$. We also get that $H^i(G, \bbI_L^\mfp) \cong H^i(G_\mfP, U_\mfP). \qed$

\thrm{}{
	$$H^i(G, \bbI_L^S) \cong \prod\limits_{\mfp \in S} H^i(G_\mfP, L_\mfP^\times)$$	
}

\tbf{Proof:} Let $\mfP$ be a prime above $\mfp$. Recall from cohomology that $H^i(G, \prod_i A_i) \cong \prod_i H^i(G, A_i)$ for $A_i$ a family of $G$-modules. Note the decomposition of $\bbI_L^S$ as $\bbI_L^S=\prod\limits_{\mfp \in S} \bbI_L^\mfp \times \prod\limits_{\mfp \not\in S} U_L^\mfp$. By cohomology, we have that $$H^i(G, \bbI_L^S) \cong \prod\limits_{p \in S} H^i(G, \bbI_L^\mfp)\times \prod\limits_{p \not\in S}H^i(G, U_L^\mfp).$$

We then use the lemma above to replace $H^i(G, \bbI_L\mfp)$ with $H^i(G_\mfP, L_\mfP^\times)$. If $\mfp$ is unramified in $L$, then $L_\mfP/K_\mfp$ is unramified and thus $H^i(G, \bbI_L^\mfp) \cong H^i(G_\mfP, U_\mfP) = 1$. It then follows that $$H^i(G, \bbI_L^S) \cong \prod\limits_{\mfp \in S}H^i(G_\mfP, L_\mfP^\times). \qed$$ This theorem acts as a means of localisation, allowing us to swap the cohomology groups $H^i(G, \bbI_L)$ with the localised cohomology groups $H^i(G_\mfP, L_\mfP^\times)$.

In similar fashion, taking $\bbI_L = \cup_S \bbI_L^S$, we can also show that $$H^i(G, \bbI_L)= \bigoplus\limits_{\mfp} H^i(G_\mfP, K_\mfP^\times).$$

\coro{}{
	$$H^1(G, \bbI_L)=H^3(G, \bbI_L)=0.$$
}

\tbf{Proof:} It remains for us to prove that $H^1(G_v, K_v^\times) = H^3(G_v, K_v^\times)=0$. The case for $H^1=H^3$ is shown by Hilbert's Theorem 90. By Tate's theorem, there is an isomorphism between $H^1(G_v, \bbZ)$ and $H^3(G_v, K_v^\times)$, which both equal zero. $\qed$

Here we set out to prove the first and second inequalities of class field theory. Letting $L/K$ have cyclic Galois group $G$ of prime order $p$, we aim to show that:

\thrm{First inequality}{
	$$C_K:N(C_L) \geq p=[L:K].$$
}

\thrm{Second inequality}{
	$$C_K:N(C_L) \leq p=[L:K].$$
}

Using these two inequalities, we show that there is a reciprocity isomorphism between $G^{\text{ab}}$ (where $G=\text{Gal}(L/K)$) and the norm residue group $C_K/N(C_L)$.

\prop{}{
	Let $C_L$ denote the idele class group. It has Herbrand quotient equal to $$h(C_L) = \frac{\#(H^0(G, C_L))}{\#(H^1(G, C_L))}=p.$$
}

Note how the first inequality is immediate from this lemma. Since $\bbI_K=H^0(G, \bbI_L)$, we have that $H^0(G, C_L)=C_K$ and thus $H^0(G, C_L) = C_K / N(C_L)$. Then since $n=h(C_L) = \frac{\#(H^0(G, C_L))}{\#(H^1(G, C_L))}=\frac{|C_K / N(C_L)|}{\#(H^1(G, C_L))}$, the statement follows.

\tbf{Proof:} Let $S$ be a finite set of primes in $K$ such that $\bbI_L = \bbI_L^S \cdot L^\times$, $\bbI_K = \bbI_K^S \cdot K^\times$, and that $S$ contains all infinite primes and all primes ramified in $L$. This set exists as when we take $S$ to be sufficiently large, we can conveniently take $\bbI_K = \bbI_K^S \cdot K^\times$ and thus $C_K = \bbI_K^S \cdot K^\times / K^\times \cong \bbI_L^S / L^S$. Note (why?) that we can split $$h(C_L) = h(\bbI_L^S) \cdot h(L^S)^{-1}$$ meaning it remains for us to compute $h(\bbI_L^S)$ and $h(L^S)$ independently. 

\emph{Case 1}: $h(\bbI_L^S)$. Let $n$ be the number of primes in $S$, $n_2$ the number of primes that are inert in $L$, and $N$ the number of primes of $L$ that lie above $S$. Clearly, since $[L:K]$ is prime, we have that $N= n_2+p\cdot(n_1-n_2)$. It then remains for us to calculate the orders of the cohomology groups $H^0(G, \bbI_L^S)$ and $H^1(G, \bbI_L^S)$. 

We make use of the localisation lemma above: $H^i(G, \bbI_L^S) \cong \prod\limits_{\mfp \in S} H^i(G_\mfP, L_\mfP^\times)$. If $i=1$, we immediately have that since $H^1(G_\mfP, L_\mfP^\times)=1$, then $H^1(G, \bbI_L^S)=1$. 

If $i =0$, we have that $$H^0(G, \bbI_L^S) \cong \prod\limits_{\mfp \in S} H^0(G_\mfP, L_\mfP^\times) \cong G_\mfP$$ and thus $H^0(G, \bbI_L^S) = p^{n_2}$. (we have that $H^0(G_\mfP, L^\times_\mfP) = 1$ if $\mfp$ lying under $\mfP$ splits, and $H^0(G_\mfP, L^\times_\mfP) = p$ if $\mfp$ is inert.)

Finishing off, we have that $h(\bbI_L^S)=p^{n_2}$ as $H^1(G, \bbI_L^S)=1$. Since $h(A) = p^{\frac{p\beta-\alpha}{p-1}}$, it swiftly follows that $$h(L^S) = p^{\frac{p(n_1-1)-N+1}{p-1}}=p^{n_2-1},$$ which yields the statement that $h(C_L)=p. \qed$

\coro{}{
	Let $L/K$ be a cyclic extension of degree equal to $p^r$. Then, $K$ has infinitely many primes which are inert in $L$.
}

\tbf{Proof:} We first treat the case $[L:K]=p$, then the case $[L:K]=p^r$.

\emph{Case 1}: $[L:K]=p$. Let $S$ be the set of primes in $K$ that remain inert in $L$. We show that this number is finite. Let $\bar{\mfa} \in C_K$, and let $\mfa \in I_K$ be a representative idele of $\bar{\mfa}$ with components in $K_\mfp^\times$. For each $\mfp \in S$, we have that $\mfa_\mfp \cdot (K_\mfp)^\times)^p$ is an open neighbourhood of $\mfa_\mfp$ such that there exists an $x_p \in K^\times$ which lies in said neighbourhood.

The approximation theorem implies that there exists an $x \in K$ that approximates the $x_\mfp$ arbitrarily closely with respect to $\mfp$, for all $\mfp \in S$. We additionally have that $\mfa_\mfp \cdot x^{-1} \in (K_\mfp^\times)^p$ for all $\mfp \in S$.

I want to show that $\mfa \cdot x^{-1}$ is the norm of an idele $mfb \in I_L$. Evidently, this is the case if and only if every component $\mfa_\mfp; \in K_\mfp^\times$ is the norm of an element $b_\mfP \in L_\mfP^\times$, with $\mfP \mid \mfp$. 

Note that this holds if and only if every component $\mfa_\mfp' \in K_\mfp^\times$ is the norm of an element $b_\mfP \in L_\mfP^\times$. We now casework on whether $\mfp \in S$. If $\mfp \in S$ then this is true as $[L_\mfP:K_\mfp] = p$ ad $\mfa_\mfp' = \mfa_\mfp\cdot x^{-1} \in (K_\mfp^\times)^p$. Now assume that $\mfp \not\in S$. Then, $\mfp$ splits and thus $L-\mfP=K_\mfp$; therefore, $$\bar{\mfa} = \mfa \cdot K^\times = \mfa' \cdot K^\times = N(\mfb) \cdot K^\times = N(\mfb \cdot L^\times).$$

It then follows that $C_K = N_{\text{Gal}(L/K)}(C_L)$, which contradicts the first inequality. 

\emph{Case 2:} $[L:K] = p^r$. Assume that almost all primes of $K$ split in $L$. Then, the decomposition fields $Z_\mfP$ are proper extensions of $K$ for almost all $\mfP \in L$. These fields also contain an intermediate field $L_0$ of degree $p$. Though within $L/K$, there is a unique field of degree $p$ contained in almost all decomposition fields $Z_\mfP$, meaning that almost all primes $\mfp \in K$ decompose in $L_0/K$. Contradiction. $\qed$

We now prove the second inequality. The proof employs Kummer theory. Note that we cannot deduce, in the case of idele class groups, which idele classes are represented by a norm idele and thus lie in $N(C_L)$. This difficulty was avoided in the first inequality.

In the following, let $\bbI_K^S$ be the group of $S$-ideles and $U_K^S$ denote the set of $S$-units. Let $s = \#S$, and furthermore let $L/K$ be Galois such that $\text{Gal}(L/K) \cong (\bbZ/n\bbZ)^r$.

\thrm{}{
	Firstly, $s \geq r$. Secondly, there exists a set $T$ of places of $k$ which are disjoint from $S$ and of cardinality $s-r$ such that $$L = K(\sqrt[n]{\Delta})$$ where $\Delta$ is the kernel of the map $$U_K^S \to \prod\limits_{\mfp \in T} K_\mfp^\times / (K_\mfp^\times)^n.$$
}

\tbf{Proof:} We first present an intermediate lemma. 

\lemma{}{
	If $L/K$ is finite abelian with Galois group $G$ of order $n$, then we can write $L=K(\sqrt[n]{\Delta})$ with $\Delta = (L^\times)^n \cup K^\times$. We also have an isomorphism $$\theta: \Delta / (K^\times)^n \to \text{Hom}(G, \bbZ/n\bbZ).$$
}

\tbf{Proof:} This is an corollary of Hilbert's Theorem 90. $\qed$

By the lemma it is clear that we can construct $L= K(\sqrt[n]{D})$ with $D = (L^\times)^n \cup K$. Let $x \in D$. Then, $K_\mfp(\sqrt[n]{x})/K_\mfp$ is unramified for all $\mfp \not\in S$ as $S$ contains the places that are ramified in $L$. By such, we can write $x = u_\mfp y_\mfp^n$ with $u_\mfp \in U_\mfp$ and $y_\mfp \in K_\mfp^\times$. 

Construct the idele $y=(y_\mfp)$ by putting $y_\mfp=1$ for all $\mfp \in S$. We can then write $y=\alpha z$ with $\alpha \in \bbI_K^S$ and $z \in K^\times$. Thus, $xz^{-n} = u_\mfp \alpha_\mfp^n \in U_\mfp$, or $xz^{-n} \in \bbI_K^S \cup K^\times = U_K^S.$ It then follows that $xz^{-n} \in \Delta$, thereby showing that $D = \Delta (K^\times)^n$.

Set $N=K(\sqrt[n]{U_K^S})$. We have that $K \subset N$ as $\Delta \subset U_K^S$. Then, use the lemma above and the fact that $U_K^S\cdot (K^\times)^n / (K^\times)^n = U_K^S / (U_K^S)^n$ to arrive at $$\text{Gal}(N/K) \cong \text{Hom}(U_K^S/(U_K^S)^n, \bbZ/n\bbZ).$$

Recall that by Dirichlet's unit theorem $U_K^S$ is the product of a free group of rank $s-1$ and the unit group $\mu(K)$ with order divisible by $n$. Thus we can write $$U_K^S = \bbZ^{s-1} \times \mu(K).$$ This implies that $U_K^S / (U_K^S)^n$ is a free $\bbZ/n\bbZ$ module of rank $s$ (and thus $\text{Gal}(N/K)$ is too). We also have that $\text{Gal}(N/K)/\text{Gal}(N/L) \cong \text{Gal}(L/K) \cong (\bbZ/n\bbZ)^r$ is a free $\bbZ/n\bbZ$-module of rank $r$; thus, $\text{Gal}(N/L)$ is a free $\bbZ/n\bbZ$-module of rank $s-r$.

Let $\sigma_1, \dots, \sigma_{s-r}$ be a basis of $\text{Gal}(N/L)$, and denote the fixed field of each $\sigma_i$ by $N_i$ for all $i$. Then, $,L = \bigcap\limits_{i=1}^{s-r}N_i.$ 

For each $i$, choose a prime $\mfP_i \in N_i$ which is non-split in $N$ such that the primes $\mfp_1, \dots, \mfp_{s-r}$ lying below $\mfP_1, \dots, \mfP_{s-r}$ are distinct and do not belong to $S$. We aim to show that by choosing the set $T = \{\mfp_1, \dots, \mfp_{s-r}\}$, we have that the group $\Delta = (L^\times)^n \cap U_K^S$ is the kernel of the map $$U_K^S \to \prod\limits_{\mfp \in T} K_\mfp^\times / (K_\mfp^\times)^n.$$

We aim to show that $\text{Gal}(N/N_i)$ is the decomposition group $D_i$ of $\mfp_i$ in $N/K$. Note firstly that $\mfp_i$ is unramified in the extension $N/K$; in particular, $D_i$ is cyclic and generated by the Frobenius $F_{N/K}(v_i)$. Furthermore we have that $\text{Gal}(N/N_i) \subset D_i$ as every element of $\text{Gal}(N/K)$ which fixes the identity on $N_i$ fixes $\mfp_i$. Letting $\mfp_i'$ be the unique prime of $\mfp_i$ above $N_i$, it swiftly follows that $\mfp_i'$ is fixed too.

As $|\text{Gal}(N/N_i)|=n$ and $D_i$ has order $\leq n$, it then follows that $D_i = \text{Gal}(N/N_i)$. Finishing off the proof, note that since $L= \bigcup\limits_{i=1}^{s-r} N_i$, it follows that $L/K$ is the maximal subextension of $N/K$ where the primes $\mfp_1, \dots, \mfp_{s-r}$ split completely. Let $x \in U_K^S.$ We have that $x \in \Delta$ if and only if $K(\sqrt[n]{x}) \subseteq L$ and thus $$x \in (K_{\mfp_i}^\times)^n$$ for all $i$. This shows that $\Delta$ is the kernel of the map $U_K^S \to \prod\limits_{\mfp \in T} K_\mfp^\times / (K_\mfp^\times)^n. \qed$

\thrm{}{
	Set $$\bbI_K(S, T) = \prod\limits_{\mfp \in S} (K_\mfp^\times)^n \times \prod\limits_{\mfp \in T} K_\mfp^\times \times \prod\limits_{\mfp \not\in S \cup T} U_\mfp$$ and let $C_K(S, T) = \bbI_K(S, T) \cdot K^\times / K^\times.$  
	
	Then, $C_K(S, T) \subset N(C_K)$ and $[C_K:C_K(S, T)] = [L:K]$. 
}

The theorem writes that $C_K(S, T) \subset N(C_K)$, which constructs the condition for the Second inequality. Equality is attained when $L/K$ is cyclic, wherein $[C_K: N(C_L)] = [L:K]$. First we present a lemma.

\lemma{}{
	$$\bbI_K(S, T) \cup K^\times = (U_K^{S \cup T})^n.$$
}

\tbf{Proof:} It's easy to see that $(U_K^{S \cup T})^n \subseteq \bbI_K(S, T) \cup K^\times$. Let $y \in \bbI_K(S, T) \cup K^\times$ and let $M = K(\sqrt[n]{y})$. I want to show that $N(C_M)=C_K$ and thus $M=K$.

Let $[\alpha]$ be a class of $C_K = \bbI_K^S \cdot K^\times / K^\times$, and let $\alpha \in \bbI_K^S$ be a representative of said class. It's clear that the map $$U_K^S \to \prod\limits_{\mfp \in T} U_\mfp / U_\mfp^n$$ is surjective. To see this, let $\Delta$ be the kernel of the map. Then, $(K^\times)^n \cup \Delta = (K^S)^n$ and $\Delta (K^\times)^n / (K^\times)^n = \Delta / (K^S)^n.$

Thus we have that $$\#(U_K^S/ \Delta) = \frac{\# (U_K^S / (U_K^S)^n)}{\# (\Delta / (K^S)^n) } = \frac{n^s}{\#(\text{Gal}(L/K))}=n^{s-r}$$ which is the order of the product $\prod\limits_{\mfp \in T} U_\mfp / U_\mfp^n$ as we have $\#(U_\mfp / U_\mfp^n) = n$. By such, there exists an element $x \in U_K^S$ such that $\alpha_\mfp =xu_\mfp^n$ for some $u_\mfp \in U_\mfp$ and $\mfp\in T.$ The idele $\alpha'=\alpha x^{-1}$ belongs to the same class as $\alpha$. We show that every component of the idele is a norm from $M_\mfP/K_\mfp$. 

For $\mfp \in S$, we have that $y \in (K_\mfp^\times)^n$ and thus $M_\mfP = K_\mfp$ for $\mfp \in T$. For $\mfp \not\in S \cup T$, we have that $\alpha_\mfp'$ is a unit and $M_\mfP/K_\mfp$ is unramified. Thus, $\alpha'_\mfp \in N(I_M)$ and thus $[\alpha] \in N(C_M). \qed$

We now end off the proof. Consider the short exact sequence $$1 \to \bbI_K^{S \cup T} \cap K^\times / \bbI_K(S, T) \cap K^\times \to \bbI_K^{S \cup T} / \bbI_K(S, T) \to \bbI_K^{S \cup T}  K^\times / \bbI_K(S, T) K^\times \to 1.$$

We now observe the orders of the groups in the sequence. The order of the group on the right is $$[\bbI_K^{S \cup T}  K^\times : \bbI_K(S, T) K^\times] = [C_K: C_K(S, T)].$$ The order of the group on the left is $$[\bbI_K^{S \cup T} \cap K^\times : \bbI_K(S, T) \cap K^\times] = [U_K^{S \cup T} :(U_K^{S \cup T})^n] = n^{2s-r}.$$ The order of the group in the middle is $$[\bbI_K^{S \cup T} : \bbI_K(S, T)] = \prod\limits_{\mfp \in S} [K_\mfp^\times:(K_\mfp^\times)^n] = \prod\limits_{\mfp \in S} \frac{n^2}{|n|_\mfp} = n^{2s} \prod\limits_{\mfp} |n|_\mfp^{-1} = n^{2s}$$ and thus $$[C_K:C_K(S, T))] = \frac{n^{2s}}{n^{2s-r}}=n^r = [L:K].$$

By now, we are left to show that $C_K(S, T) \subseteq N(C_L)$. Let $\alpha \in \bbI_K(S, T).$ We just need to check, akin to the process above, that every component $\alpha_\mfp$ is a norm from $L_\mfP/K_\mfp$. 

Let $\mfp \in S$.  We note that since $\alpha_\mfp \in (K_\mfp^\times)^n$ is an $n$-th power, it is also a norm from $K_\mfp(\sqrt[n]{K_\mfp^\times})$ and thus it is a norm from $L_\mfP/K_\mfp$. Now let $\mfp \in T$. Our corollary of Hilbert's Theorem 90 shows that $\Delta \subseteq (K_\mfp^\times)^n$, and thus $L_\mfP = K_\mfp$. Finally, let $\mfp \not\in S \cup T$. The inclusion also holds as $\alpha_\mfp$ is a unit and $L_\mfP/K_\mfp$ is unramified. 

Therefore we have proven that $\bbI_K(S, T) \subseteq N(\bbI_L)$ and thus $$C_K(S, T) \subset N(C_L). \qed$$

(Remark) When $L/K$ is cyclic, $r=1$. Thus, $$[L:K] \leq [C_K: N(C_L)] \leq [C_K : C_K(S, T)] = [L:K],$$ thus obtaining that $N(C_L)=C_K(S, T).$ 

With the first and second inequalities proven, we have that $C_K:N(C_L) =[L:K].$ We are now in a position to prove the global Artin reciprocity law. For completeness' sake, let us first show that the idele class group satisfies the \emph{class field axiom}.

\thrm{Class field axiom for the idele class group}{
	Let $L/K$ be normal with Galois group $\text{Gal}(L/K)$. Then, $$|H^0(\text{Gal}(L/K), C_L)| = |H^2(\text{Gal}(L/K), C_L)|=[L:K]$$ and $$|H^{1}(\text{Gal}(L/K), C_L)| = 1.$$
}

\tbf{Proof:} Note that $h(\text{Gal}(L/K), C_L)=[L:K]$. Therefore it remains for us to show that $|H^0(\text{Gal}(L/K), C_L)| \mid [L:K].$ Let $M/K$ be a subextension of $L/K$ of degree $p$. Now consider the exact sequence $$H^0(L/M) \to H^0(L/K) \to H^0(M/K) \to 1,$$ given explicitly by $$C_M / N(C_L) \xrightarrow{\text{Nm}} C_K/N(C_L) \to C_K / N(C_M) \to 1.$$

Let $p < n$. Then, $\#(H^0(L/M)) \mid [L:M]$ and $\#(H^0(M/K)) \mid [M:K]$ and thus $$\#(H^0(L/K)) \mid [L:M][M:K] = [L:K].$$

Now assume that $p=n$. Put $K' = K(\mu_p)$ and $L' = L(\mu_p)$. Since $d = [K':K] \mid p-1$, we have that $\text{Gal}(L/K) \cong \text{Gal}(L'/K')$. Note that $L'/K'$ is a cyclic Kummer extension so the cohomology group $H^0(L'/K')$ has order $[L:K]=p$.

It remains for us to show that $H^0(L/K) \to H^0(L'/K')$ through the inclusion $C_L \to C_{L'}$ is injective. This is a lifting argument and is standard, so the proof is omitted. $\qed$

Let $L/K$ be finite abelian with Galois group $G$. We have shown that there exists a homomorphism $\phi_{L/K}: \bbI_K \to G$ such that $\phi_{L/K}(i) = \prod\limits_v \phi_v(i_v)$.

\defn{Invariant map}{
	Let $c_\mfp \in H^2(G_{L_\mfP / K_\mfp})$ be the local components of $c \in H^2(\text{Gal}(L/K), \bbI_L)$. Then, set $$\text{inv}_{L/K}(c) = \sum\limits_\mfp \text{inv}_{L_\mfP / K_\mfp} c_\mfp \in \frac{1}{[L:K]}\bbZ/\bbZ.$$ The \emph{idelic invariant maps} are the isomorphisms $$\text{inv}_{L/K}: H^2(\text{Gal}(L/K), \bbI_L) \to \frac{1}{[L:K]}\bbZ/\bbZ.$$ 
}

\thrm{}{
	The map $\text{inv}_{L/K}$ is surjective in the cyclic case.
}

\tbf{Proof:} Let $[L:K] = p^r$. Then note that $\frac{1}{[L:K]}+\bbZ$ generates $\frac{1}{[L:K]}\bbZ/\bbZ$ and thus it remains for us to find an element $c \in H^2(\text{Gal}(L/K), \bbI_L)$ satisfying $\text{inv}_{L/K}(c) = \frac{1}{[L:K]}+\bbZ$.

Decompose $$H^2(\text{Gal}(L/K), \bbI_L) \cong \bigoplus\limits_\mfp H^2(\text{Gal}(L_\mfP/K_\mfp), L_\mfP)$$ where $c$ can be decomposed into its components $c_\mfp \in H^2(\text{Gal}(L_\mfP/K_\mfp), L_\mfP).$

Since $L/K$ is cyclic and of degree $p$, there are infinitely many primes of $K$ that are inert in $L$. Thus, $K$ contains a prime $\mfp_0$ that is inert in $L$. Since $\mfp_0$ is inert, it follows that for $\mfP_0 \mid \mfp_0$, we have that $[L_{\mfP_0}: K_{\mfP_0}] = [L:K].$ Thus, we have an element $c_{\mfp_0} \in H^2(\text{Gal}(L_{\mfP_0}/K_{\mfp_0}), L^\times_{\mfP_0})$ satisfying $$\text{inv}_{L_{\mfP_0}/K_{\mfp_0}} c_{\mfp_0} = \frac{1}{[L_{\mfP_0}:K_{\mfp_0}]}+\bbZ = \frac{1}{[L:K]}+\bbZ.$$

If $c$ is the element in $H^2(\text{Gal}(L/K), \bbI_L)$ with local components $(\dots, 1, 1, c_{\mfp_0}, 1, 1, \dots)$, then we have that $$\text{inv}_{L/K}(c) = \sum\limits_{\mfp} \text{inv}_{L_\mfP/K_\mfp} c_\mfp = \text{inv}_{L_{\mfP_0}/K_{\mfp_0}} c_{\mfp_0} = \frac{1}{[L:K]}+\bbZ.$$

Now we consider the general case. Let $[L:K]=n=p_1^{r_1}\dots p_s^{r_s}$. Then we can extract an intermediate field $L_i$ of degree $[L_i:K]=p_i^{r_i}$ such that $[L_i:K]$ is cyclic. Decomposing $\frac{1}{n}=\frac{n_1}{p_1^{r_1}}+\dots+\frac{n_s}{p_s^{r_s}}$, it swiftly follows that there exists a $c_i$ such that $\text{inv}_{L/K} c_i = \frac{n_i}{p_i^{r_i}}+\bbZ$.

Set $$c = c_1 \dots c_s \in H^2(\text{Gal}(L/K), \bbI_L).$$ Doing so, we have that $$\text{inv}_{L/K}c=\sum\limits_{i} \text{inv}_{L/K} c_i = \sum\limits_i \frac{n_i}{p_i^{r_i}}+\bbZ = \frac{1}{n}+\bbZ,$$ hence proving the statement. $\qed$

\thrm{}{
	Let $L/K$ be finite abelian. Then $\phi_{L/K}: \bbI_K \to \text{Gal}(L/K)$ contains $K^\times$ in its kernel. This means that on the principal ideles $K^\times \subset \bbI_K$, the map $\phi_{L/K}$ takes on the value $1$.
	
	 (Remark) Local Artin reciprocity implies that $\phi_{L/K}$ contains $N_{L/K}(\bbI_L)$ in its kernel. Using the second inequality, it follows that the homomorphism $$\bbI_K / K^\times \cdot \text{Nm}_{L/K} \bbI_L \to \text{Gal}(L/K)$$ is an isomorphism.
	
	Thus, this theorem immediately implies the proof of global Artin reciprocity.
}

\tbf{Proof:} We first verify the theorem for cyclotomic extensions. Let $\zeta_m$ be a primitive $m$'th root of unity. Note that $\text{Gal}(\bbQ(\zeta_m)/\bbQ) = (\bbZ/m\bbZ)^\times$ so the class $[n]$ denotes the automorphism of $\bbQ[\zeta_m]$ that sends $\zeta_m \mapsto \zeta_m^n.$ We want to show that for $l \mid m$, the image of $\bbQ(\zeta_{l^r})$ under $\phi(a)$ is $1$. Let $m=l^r$ for $m \neq 2.$

Then the homomorphism $\phi_\infty: \bbR^\times / N(\bbC^\times) \to \text{Gal}(\bbQ(\zeta_m)/\bbQ)$ sends any negative real number to its complex conjugate, meaning that $\phi_\infty(a) = [\text{sign}(a)]$. If we let $a = up^s \in \bbQ_p^\times$, we have that $\phi_p(a) = [p^s]$ and $\phi_l(a)=[u^{-1}]$ by the local Artin map. It remains for us to show that $\phi(a)=1$ when $a=-1, a=l$, and $a = q \neq l$ where $q$ is a prime. Details omitted but the main point is that $\prod \phi_p(a)=1$.

Note that if the statement holds for $L/K$, one can verify that it holds for any subextension of $L/K$. So it holds for subextensions of $\bbQ(\zeta_m)$. 

\coro{}{
	If the theorem above holds for $L/K$ a cyclic extension, then the following result holds:
	
	Let $L/K$ be finite Galois. Then, $\sum \text{inv}_v(\alpha)=0$ for all $\alpha \in H^2(L/K).$ The converse claim also holds true; if this result holds for $L/K$, then the theorem holds.
}

\tbf{Proof:} Let $\chi \in \text{Hom}(G, \bbQ/\bbZ).$ We have that $\chi$ is an element of $H^1(G, \bbZ/\bbZ)$. Consider the diagram 

\[
\begin{tikzcd}
K^\times \arrow{r}\arrow{d}{\cup \delta_\chi} &\bbI_K \arrow{r}{\phi_{L/K}} \arrow{d}{\cup \delta_\chi} &G \arrow{d} \\
H^2(G, L^\times) \arrow{r} &H^2(G, \bbI_L) \arrow{r} &\bbQ/\bbZ
\end{tikzcd}
\]

where the vertical arrows $\cup \delta_\alpha$ represent cup-product. We assume that the following statement holds true, and thus the right-hand square commutes: for $\alpha \in \text{Hom}(G, \bbQ/\bbZ)$, $a \in K^\times$, it holds that $$\chi(\phi_{L/K}(a)) = \text{inv}_K(a \cup \delta_\chi).$$

Now assume that the corollary is true. Then, $\chi(\phi_{L/K}(a)) = 0$ for all characters $\chi \in G$. It swiftly follows that $\phi_{L/K}(a)=0$, meaning that the theorem holds. Now assume that $G$ is cyclic. We can choose $\chi$ to be injective, meaning that the theorem implies the corollary as the cup-product $\cup \delta_\chi$ is an isomorphism.

 \lemma{}{
	If we know that the corollary above holds for cyclic extensions, then it holds for all Galois extensions. Thus, the theorem above holds, and Artin reciprocity follows.
}

Let $\beta \in H^2(\text{Gal}(K^{\text{al}}/K), (K^{\text{al}})^\times).$ Note that if $\beta \in H^2(L/K)$ for some $L/K$ cyclic, then $\sum \text{inv}_v(\beta)=0$, where $\beta_v$ is the image of $\beta$ in $H^2(K_v)$. However, for cyclotomic extensions we have that $\beta \in H^2(K)$ lies in $H^2(L/K). \qed$

The seemingly arbitrary construction of the $H^2(\text{Gal}(K^{\text{al}}/K)), (K^{\text{al}})^\times)$ is actually the \emph{Brauer group}. 

\section{Beyond Artin reciprocity: notable results in global class field theory}

In this section, we finish off the proof of the existence theorem. We also give some results that follow from the second inequality.

\thrm{Existence theorem}{
	The norm groups of $C_K$ are the closed subgroups of finite index.
}

\tbf{Proof:} We first present a lemma.

\lemma{}{
	Let $K$ be a field that contains all the $n$-th roots of unity. Let $U_K^S \subseteq I_K^S$ be the idele group, and let $\bar{U}_K^S = U_K^S \cdot K^\times / K^\times$. 
	
	Then, if $S$ contains all the infinite primes and primes lying above the prime numbers dividing $n$, and $I_K = I_K^S \cdot K^\times$, then $C^n_K \cdot \bar{U}_K^S$ is the norm group of the Kummer extension $K(\sqrt[n]{K^S})/K$.
}

\tbf{Proof:} Let $T =K(\sqrt[n]{K^S})/K$. Then, $$\chi(\text{Gal}(T/K)) \cong K^S \cdot (K^\times)^n/(K^\times)^n \cong K^S / (K^S)^n.$$ By Dirichlet's Unit Theorem, we have that $K^S$ is finitely generated of rank $|S|-1$. Thus, $K^S / (K^S)^n$ is the direct product of $|S|$ cyclic groups of order $n$. 

Now, let $\bar{\mfa} \in N(C_T).$ Thus, $C_K^n \subseteq N(C_T)$. We want to show that every idele $\mfa \in U_K^S$ is a norm idele of the extension $T/K$. If $\mfp\in S$, we know it holds as $a_\mfp = 1$. Now let $\mfp \not\in S$. Now, $a_\mfp \in U_\mfp$ and is a norm if $K_\mfp(\sqrt[n]{K^S})/K_\mfp$ is unramified. 

This is clear as every $a \in K^S$ is a unit for $\mfp \in S$ and that $n$ is relatively prime to the characteristic of $\bar{K}_\mfp$ if $\mfp \not\in S$. Thus, for $a \in K^S$, we have that $K_\mfp(\sqrt[n]{a})/K_\mfp$ is unramified. Therefore, $\bar{U}_K^S \subseteq N(C_T)$ and thus $$C_K^n \cdot \bar{U}_K^S \subseteq N(C_T).$$

By observing the index of $[C_K: N_{T/K}(C_T)]$, we show that this inclusion is an equality. Note that by Artin reciprocity, we have that $$[C_K: N_{T/K}(C_T)]= |\text{Gal}(T/K)| = [K^S: (K^S)^n] = n^{|S|}.$$ Recall that in idelic theory, if we let $$F = \prod\limits_{\mfp \in S} (K^\times_\mfp)^p \times \prod\limits_{i=1}^m K_{\mfq_i}^\times \times \prod\limits_{\mfp \not\in S} U_\mfp,$$ recall that we can decompose
 \begin{align*}[C_K:\bar{F}] &= [I_K^{S} \cdot K^\times / K^\times : F\cdot K^\times / K^\times] \\ 
 	&=  [I_K^{S} \cdot K^\times  : F\cdot K^\times]  \\
	&= [I_K^{S}:F] / [(I_K^{S} \cap K^\times):(F \cap K^\times)].
 \end{align*}

where the last equivalence follows from the fact that $A/B \to A\cdot C / B \cdot C$ has kernel $A \cap B\cdot C / B \cong A \cap C / B \cap C$. Analogously, we can decompose 

 \begin{align*}[C_K:C_K^n \cdot \bar{U}_K^S] &= [I_K^{S} \cdot K^\times / K^\times : (I_K^S)^n \cdot U_K^S\cdot K^\times / K^\times] \\ 
 	&=  [I_K^{S} \cdot K^\times  : (I_K^S)^n \cdot U_K^S\cdot K^\times]  \\
	&= [I_K^{S}:(I_K^S)^n \cdot U_K^S] / [(I_K^{S} \cap K^\times):((I_K^S)^n \cdot U_K^S \cap K^\times)].
 \end{align*}

and thus the index in the numerator is $$I_K^{S}:(I_K^S)^n \cdot U_K^S = \prod\limits_{\mfp \in S} [K_\mfp^\times : (K_\mfp^\times)^n]. $$ Using the fact that $|n|_\mfp =1$ for $\mfp \in S$, we have that 

\begin{align*}
	[I_K^{S^*}:(I_K^S)^n \cdot U_K^S] &= \prod\limits_{\mfp \in S} [K_\mfp^\times : (K_\mfp^\times)^n] \\
	&= \prod\limits_{\mfp \in S} n^2 \cdot |n|_\mfp \\
	&= \prod\limits_{\mfp} |n|_\mfp \\
	&= n^{2|S|}.
\end{align*}

and thus we conclude that 
\begin{align*}
	[C_K:C_K^n \cdot \bar{U}_K^S] &= \frac{n^{2|S|}}{[K^S: (K^S)^n]}\\
	&= \frac{n^{2|S|}}{n^{|S|}} \\
	&= n^{|S|} \\
	&= [C_K: N_{T/K} C_T]
\end{align*}

therefore implying that $N_{T/K} C_T = C_K^n \cdot \bar{U}_K^S. \qed$

(Remark): The fact that $\#((I_K^S)^n \cdot U_K^S \cap K^\times)= (K^S)^n$ might be unobvious. We know that $(K^S)^n \subseteq (I_K^S)^n \cdot U_K^S \cap K^\times$ though. To show that the opposite inclusion holds, let $x \in (I_K^S)^n \cdot U_K^S \cap K^\times.$ Then, $x = \mfa^n \cdot u$ for some $\mfa \in I_K^S$ and $u \in U_K^S$. I want to show that $K = K(\sqrt[n]{x})$. 

Take $\mbf \in I_K^S.$ Naturally, $\mfb$ is a norm-idele of $K(\sqrt[n]{x})/K$. If $\mfp \in S$, then $\mfb_\mfp \in K_\mfp^\times$ is a norm as $K_\mfp(\sqrt[n]{x}) = K_\mfp$. If $p \not\in S$, then $\mfb_\mfp \in U_\mfp$, and thus $K_\mfp(\sqrt[n]{x}) = K_\mfp(\sqrt[n]{u_\mfp})/K_\mfp$ as $\mfp \nmid n$ is unramified. Thus, we have that $$N_{K(\sqrt[n]{x})/K} C_{K(\sqrt[n]{x})} = C_K$$ and therefore $K=K(\sqrt[n]{x})$ by Artin reciprocity. Therefore, $\sqrt[n]{x} = y \in K^\times$ and thus $x = y^n \in (K^\times)^n \cap K^S = (K^S)^n$.

Back to the proof. Let $\mcN_L = N_{L/K}(C_L)$ be the norm group of an extension $L/K$. By Artin reciprocity, we know that $[C_K: \mcN_L]$ is finite. Now construct $C_K = C_K^0 \times \Gamma_K$ and $C_L = C_L^0 \times \Gamma_L$, where $\Gamma_K, \Gamma_L \cong \bbR_+^\times$. 

Recall that we can do this as we just need to consider the extension $$1 \to C_K^0 \to C_K \to \bbR^\times_+ \to 1$$ wherein the map $C_K \to \bbR^\times_+$ is the absolute value map, and it only suffices for us to find an injection $\bbR^\times_+ \to C_K$ which yields the identity on $\bbR^\times_+$ when composed with $C_K \to \bbR^\times_+$. Said injection also yields a group of representatives for $C_L/C_L^0$, and thus we can assume that $\Gamma_K = \Gamma_L$. Therefore, $$N_{L/K}(C_L) = N_{L/K}(C_L^0) \times N_{L/K} \Gamma_K = N_{L/K} (C_L^0) \times \Gamma_K.$$

Now let $N_{L/K}C_L \subseteq C_K$ be a closed subgroup of index $n$. Then, $C_K^n \subseteq N_{L/K}C_L$. By the lemma, note that $C_K^n \cdot \bar{U}_K^S$ is a norm group for sufficiently large $S$, thereby showing that $\mcN$ is also a norm group (as it contains a norm group). $\qed$

There are many consequences of the Second Inequality of class field theory. Many of them are local-global principles.

\thrm{Hasse norm theorem}{
	Let $L/K$ be a cyclic extension, and let $x \in K^\times$. $x$ is a global norm if and only if it is a local norm in every completion $$L_\mfP / K_\mfp,$$ where $\mfP \mid \mfp$.
} 

(Remark) There are counter-examples for non-cyclic extensions. A classic one is provided by Tate and Serre: the biquadratic extension $\bbQ(\sqrt{13}, \sqrt{17})/\bbQ$ has the property that $5^2$ is not a global norm but every rational square is a local norm.

\tbf{Proof:} Consider the exact sequence of $G$-modules $$1 \to L^\times \to \bbI_L \to C_L \to 1.$$ This yields an exact sequence $$H^{-1}(G, C_L) \to H^0(G, L^\times) \to H^0(G, \bbI_L).$$

We know that $H^{-1}(G, C_L)=1$ by the class field axiom. Also note that $$H^0(G, I_L) = \bigoplus\limits_\mfp H^0(G_\mfP, L_\mfP^\times)$$ since we can always decompose $$\bbI_L^S = \bigoplus\limits_{\mfp \in S}  L_\mfP^\times \oplus \prod\limits_{\mfp \not\in S} \prod\limits_{\mfP \mid \mfp} U_\mfP$$ and obtain $$H^i(G, \bbI_L^S) = \bigoplus_{\mfp \in S} H^i(G, L_\mfp^\times) \oplus H^i(G, V)$$ where the group $H^i(G, V)$ injects into $\prod\limits_{\mfp \not\in S} H^i(G, \prod\limits_{\mfP \mid \mfp} U_\mfP)$.

By such, the homomorphism $$K^\times / N(L^\times) \to\bigoplus\limits_\mfp K^\times_\mfp / N_{L_\mfP/K_\mfp} L_\mfP^\times$$ is injective, and the theorem holds. $\qed$

We now build up towards a proof of the Hasse-Minkowski theorem, which we used to motivate the adeles back in Chapter 6. Our first step is the Chebotarev density theorem. To introduce this theorem, we need to return to $L$-series and introduce the notion of Dirichlet density. 

Let $\chi$ be a non-trivial character of $I_\mfp / P_\mfp$. Let $h_c = [I_\mfp : P_\mfp$ and let $p_0$ be a fixed ideal class of $I_\mfp$ mod $P_\mfp$. Denote $G_\mfp = I_\mfp/P_\mfp.$ Then, we can write $$\log L(s, \chi) ~ \sum\limits_{p \in G_\mfp} \chi(p) \sum\limits_{\mfp \in p} \frac{1}{N(\mfp^s)}.$$ Then we multiply by $\chi(p_0^{-1})$ and sum over all $\chi$. This yields us $$\log \zeta_k(s) ~ \sum\limits_p \sum\limits_\chi \chi(p\cdot p_0^{-1}) \sum\limits_{\mfp \in p} \frac{1}{N(\mfp^s)}$$ and thus $$\log \frac{1}{s-1} ~ h_c \sum\limits_{\mfp \in p} \frac{1}{N(\mfp^s)}.$$

For a set of primes $P$ in $K$, the limit $$\lim\limits_{s \to 1^+} \dfrac{\sum\limits_{\mfp \in p} \frac{1}{N(\mfp^s)}}{\log \frac{1}{s-1}}$$ is known as the Dirichlet density of $P$. As shown, said density is $\frac{1}{h_c}$.

\thrm{Chebotarev density theorem}{
	Let $L/K$ be Galois with Galois group $G$, and let $\sigma \in G$. Let $[L:K] = N$, and let $c$ be the number of elements in the conjugacy class of $\sigma$ in $G$. Then, the primes $\mfp \in K$ which are unramified in $K$ and that there exists $\mfP \mid \mfp$ such that $$\sigma = (\mfP, L/K)$$ have Dirichlet density equal to $\frac{c}{N}$.
	
	Here, $\sigma = (\mfP, L/K)$ means that $\sigma$ is the Frobenius automorphism of $\mfP$ in $L$.
}

\tbf{Proof:} Let $\sigma$ have order $f$, and let $F$ be the fixed field of $\sigma$. Then, $K/F$ is cyclic of degree $f$. Let $I_\mfp$ denote the fractional ideals relatively prime to $\mfp$. If we let $P_\mfp$ denote the set of principal fractional ideals coprime to $\mfp$ and $P_\mfp \subseteq H \subseteq I_\mfp$, then we that, by the Artin isomorphism, $$I_\mfp/H \to \text{Gal}(K/F).$$ 

Now let $S_{L, \sigma}$ be the set of prime ideals $\mfP \in L$ such that $\mfP \mid \mfp$ for $\mfp \in S$, and $\sigma = (\mfP, L/K)$. Let $\mfP \mid \mfq$ for some $\mfq \in F$. Then, $S_{L, \sigma}$ is in bijection with the set $S_F$ of ideals $\mfq \in F$ which lie in a given class mod $H$ and divide $\mfp$ that split in $F$. However, the density only depends on primes of degree 1 over $\bbQ$. Therefore, $S_F$ has density $\frac{1}{f}$. (why?) 

Fix $\mfp$. Note that the number of $\mfP$ in $K$ lying above $\mfp$ and satisfy $\sigma = (\mfP, L/K)$ is equal to $$\frac{[G_\sigma:1]}{[G_\mfP:1]},$$ where $G_\sigma$ is the subgroup of elements of $G$ that commute with $\sigma$, and $G_\mfP$ is the decomposition group of $\mfP.$ 

We know that $[G:G_\sigma] = c$ so $\frac{[G_\sigma:1]}{[G_\mfP:1]} = \frac{N}{cf}$. Thus, the density of $S$ is $\frac{1}{f} \cdot \frac{cf}{N} = \frac{c}{N}. \qed$

In Chapter 6, we formulated the Hasse-Minkowski theorem in the case of $\bbQ$. We state and prove it for a general global field $K$.

\thrm{Hasse-Minkowski theorem}{
	Let $K$ be a global field. If $f$ is an non-degenerate quadratic form in $n$ variables over $K$ which equals 0 in $K_v$ for each prime $p \in K$, then $f$ equals 0 in $K$.
}

\tbf{Proof:} We split the proof into 5 cases - where $n=1, n=2, n=3, n=4$, and $n \geq 5$.

\emph{Case 1}: $n=1$. Trivial.

\emph{Case 2}: $n=2$. Let $f = x^2-by^2.$ Suppose that $f$ represents $f$ in $0$ in $K_v$ for all $v$. Let $L = K(\sqrt{b})$. By the Chebotarev density theorem, it swiftly follows that the density of primes that split completely in $L$ is $\frac{1}{\#(\text{Gal}(L/K))}$. However we have that $K_v(\sqrt{b})=K_v$ for all $v$. 

Thus, $\#(\text{Gal}(L/K))=1$ and $L=K$. This means that $b$ is a square in $K$ and $f$ thus represents 0.

\emph{Case 3}: $n=3$. Let $f = x^2-by^2-cz^2$, and again let $L=K(\sqrt{b})$. By the Hasse norm theorem, since $L$ is cyclic, we can use the Hasse norm theorem. 

$f$ represents 0 in $K$ if and only if $c \in N_{L/K}(L^\times)$. But this then suggests that $c \in N_{L^p/K_v}((K_v)^\times)$ for all $p$, meaning that $f$ represents 0 in $K_v$ for all $p$.

\emph{Case 4}: $n=4$. To do this we present the following lemma:

\lemma{}{
	The following are equivalent:
	
	\begin{enumerate}
		\item The form $f = x^2-by^2-cz^2+act^2$ represents 0 in $K$.
		\item $c$ is a product of a norm from $K(\sqrt{a})$ and a norm from $K(\sqrt{b})$.
		\item If $c \in K(\sqrt{ab})$, then $c$ is a norm from the field $L=K(\sqrt{a}, \sqrt{b}).$
		\item The form $g = x^2-by^2-cz^2$ represents 0 in $K(\sqrt{ab})$.
	\end{enumerate}
	
	With this lemma, we can reduce the $n=4$ case to the $n=3$ case.
}

\tbf{Proof:} It is clear that 1) implies 2) as the reciprocal of a norm is a norm. Now we prove that 2) implies 3). Assume that $ab \not\in (K^\times)^2$. Then $\text{Gal}(L/K)$ consists of elements $1, \rho, \sigma, \tau$ such that $\rho, \sigma, \tau$ fixes the elements $\sqrt{ab}, \sqrt{a}$, and $\sqrt{b}$. 

Note that statement 2 implies that there exists $x, y \in L$ that $x^\sigma = x, y^\tau=y$, and that $x^{1+\rho}y^{1+\rho}=c$. Moreover, statement 3 implies the equivalent statement that there exists a $z \in L$ such that $z^{1+\rho}=c$, and thus 2) implies 3). 

To prove that 3) implies 4), we casework on whether $b$ is a square. If $b$ is a square, then $f$ represents 0 and $c$ is a norm from $K(\sqrt{b})=K$. If $b$ is not a square, then $f(X, Y, 0)$ does not represent 0. If $f(x, y, z)=0$, then $z \neq 0$, and thus $c = \left(\frac{x}{z}\right)^2 - b\left(\frac{y}{z}\right)^2$ is a norm from $K(\sqrt{b}). \qed$

\emph{Case 5}: $n\geq5$. Let $f = ax_1^2+bx_2^2+g(x_3, \dots, x_n)$, and suppose that $f$ represents 0 in $K_v$ for all $p$. We present the following lemma:

\lemma{}{
	$f=x^2-by^2-cz^2$ represents 0 in $K_v$ if and only if the quadratic norm residue symbol $(b, c)_v=1$. Furthermore, $f$ represents 0 in $K_v$ for all but a finite \emph{even} number of $v$.
}

\tbf{Proof:} We invoke the fact that $(a, b)_v = 1$ if $b$ is a norm for the extension $K_v(\sqrt[m]{a})/K_v$. From the previous lemma, it is clear that $f$ represents 0 in $K_v$ if and only if $c$ is a norm from $K_v(\sqrt{b})$, meaning that $(b, c)_v=1$. However, $(b, c)_v=1$ for almost all $v$. Thus, by the product formula $$\prod\limits_v (a, b)_v = 1$$ and the fact that $(b, c)_v \in \{\pm 1\}$, it follows that the number of $v$ such that $f$ does not represent 0 in $K_v$ (meaning that $(a, b)_v=-1$) is even. $\qed$

Since $g(x_3, \dots, x_n)$ is a form in $n-2 \geq 3$ variables, we can apply the above lemma. It follows that $g$ represents 0 in $K_v$ for almost all $v$. Call $S$ the set of all such $v$ such that $g$ does not represent 0. 

For any $v \in S$, it holds that there is $(x_{1, v}, \dots, x_{n, v}) \in (K^\times)^n$ such that $$g(x_{3, v}, \dots, x_{n, v}) = ax^2_{1, v}+bx^2_{2, v}.$$ Since $(K_v^\times)^2$ is open in $K_v^\times$, it follows that $g(K_v^\times, \dots, K_v^\times)$ is open. 

Then, use the weak approximation theorem to show that there exists $x_1, x_2 \in (K^\times)^2$ such that $ax_1^2+ax_2^2$ is sufficiently close to $ax^2_{1, v}+bx^2_{2, v}$ for all $v \in S$. Thus, we can construct $x_1$ and $x_2$ such that $x_1, x_2 \in g(K_v^\times, \dots, K_v^\times)$ for all $v \in S$. 

Thus, if $c=ax_1^2+ax_2^2$, then $c$ is represented by $g$ in $K_v$ for all $v \in S$. If $v \not\in S$, then $g$ represents 0 and thus represents $c$. Perform induction on $n$ to show that the form $cY^2-g(x_3, \dots, x_n)$ in $n-1$ variables represents 0 in $K$. Thus, $f$ represents 0 in $K. \qed$

\thrm{Grunwald-Wang theorem}{
	Let $K$ be a number field and $n$ a postive integer. $x\in K^\times$ is an $n$-th power if and only if it is an $n$-th power in $K_v$ for all but finitely many places $v \in K$.
}

\tbf{Proof:} Let $K$ be a number field, $m$ an integer, and $S$ a finite set of primes. Let $P(m, S)$ be the group of elements $\alpha \in K^\times$ which are in $(K_\mfp^\times)^m$ for all $\mfp \not\in S$. Note that $(K^\times)^m$ is a subgroup of $P(m, S)$. Let $\zeta_r$ be a primitive $2^r$-th root of unity where $\zeta_{r+1}^2=\zeta_r$

Thus, $\eta_r = \zeta_r+\zeta_r^{-1}$. We now obtain that $$\eta_{r+1}^2 = 2+\eta_r$$ which suggests that a field containing $\eta_r$ will contain all $\eta_\mu$ where $\mu \leq r$. Additionally, $$\zeta_{r+1}\eta_{r+1}=\zeta_r+1,$$ so for $r \geq 2$, we have that a field that contains $\eta_{r+1}$ and $\zeta_r$ will invariably contain $\zeta_{r+1}$. This means that if $\zeta_2=i$ and $\eta_r$ are in a field (for $r>2$), then $\zeta_r$ is also in this field. Thus, $K(\zeta_r)$ is the compositum $K(\zeta_r) = K(i) \cdot K(\eta_r)$ for $r>2$.

Note that the extensions $K(\eta_r)/K$ are cyclic. To see this, note that $\bbQ(\eta_r)/\bbQ$ is cyclic since the Galois group of $K(\eta_r)/K$ is a subgroup of $\bbQ(\eta_r)/\bbQ$. We see that every automorphism of $\bbQ(\zeta_r)$ is induced by an automorphism $\sigma_\mu(\zeta_r) = \zeta_r^\mu$ of $\bbQ(\eta_r)$. Furthermore, by the definition of $\eta_r$, we know that $\sigma_\mu$ and $\sigma_{-\mu}$ will induce the same automorphism of $\bbQ(\eta_r)$. 

If $K$ contains all the $\eta_r$, then $K(i)$ would contain all $\zeta_r$ which is impossible. Hence, let $s \geq 2$ be the integer such that $\eta_s \in K$ but $\eta_{s+1} \not\in K$. Since $\eta_{r+1}^2=2+\eta_r$, we know that $K(\eta_{s+1})/K$ is quadratic. Now suppose that all the $K(\zeta_r)/K$ are cyclic. Then, the only quadratic subfield that it can admit is $K(\eta_{s+1})$. Thus, $i \in K(\eta_{s+1})$. Conversely, if $i \in K(\eta_{s+1}) \subset K(\eta_r)$ for some $r>s+1$, then $K(\zeta_r)=K(\eta_r)$ and is cyclic. 

Consider the case when any of the $K(\zeta_r)/K$ are not cyclic. In this case, then $F = K(i, \eta_{s+1})=K(\zeta_{s+1})$ has degree 4 over $K$. This means that $F$ contains $K(i)$, $K(\eta_{s+1})$, and $K(i, \eta_{s+1})$ - this corresponds to the fact that $-1$, $2+\eta_s$, and $-(2+\eta_s)$ are non-squares in $K$. Since $\eta_s \in K$, $K(i)K(\zeta_s)$. Therefore, $K(\zeta_t)$ is cyclic for $t \leq s$. However, we have that $F=K(\zeta_{s+1})$ and all $K(\zeta_t)$ with $t>s$ are non-cyclic. Now let $\sigma$ be the non-trivial automorphism of $K(i)/K$ Since $\zeta_s+\zeta_s^{-1}=\eta_s \in K$ and $\zeta_s\zeta_S^{-1} = 1$, we have that $\zeta_s$ and $\zeta_s^{-1}$ are conjugate. Thus, $\zeta_s^\sigma = \zeta_s^{-1}$.

Now suppose that there exists an $\alpha \in P(m, s)$ not in $(K^\times)^m$. Then, $\alpha \not\in P(2^t, S)$ but is in $(K^\times)^{2^t}$ (since $\alpha \in (K^\times)^{m'}$). Therefore, we must consider the non-cyclic case where $t>s$.

Let $K(i)$ be our ground field. We can write $\alpha = A^{2^t}$ since the $K(\zeta_r)/K$ are cyclic. Therefore we have that $(A^{1-\sigma})^{2^t}=1$, meaning that $A^{1-\sigma}$ is a $2^t$-th root of unity in $K(i) = K(\zeta_s)$. Furthermore, $\zeta_{s+1} \not\in K(i)$ and thus $A^{1-\sigma} = \zeta_s^\mu$. With $\alpha = A^{2^t}$, replace $A$ by $A_1$, where $A_1 = A\zeta_s^\lambda$. For this value of $\lambda$, we have that $A_1^{1-\sigma} = \zeta_s^{\mu+2\lambda}$. 

If $\mu$ were even, we have that $A^{1-\sigma}=1$ or $A_1 \in K$, which contradicts the fact that $\alpha \not\in K^{(2^t)}.$ Thus, $\mu$ is odd, meaning that $$A^{1-\sigma} =\zeta_s^{m'} = \left(\frac{1+\zeta_s}{1+\zeta_s^{-1}}\right)^{m'} = (1+\zeta_S)^{(1-\sigma)m'}.$$ Therefore, we have that $\beta = A(1+\zeta_s)^{-m'}$ satisfies $\beta^{1-\sigma}=1$, meaning that $\beta = K$. 

Since $\beta = A(1+\zeta_s)^{-m'}$, we have that $A=\beta(1+\zeta_s)^{m'}$ or $\alpha = \alpha_0 \beta^{2^t}$, where $$\alpha_0 = (1+\zeta_s)^m = \eta_{s+1}^m = (i\eta_{s+1})^m = (2+\eta_s)^{\frac{m}{2}} = (-2-\eta_s)^{\frac{m}{2}} = ((2+\eta_S)^{2^{t-1}})^{m'}$$ which suggests that $\alpha_0 \in (K^\times)^{m'}.$ Furthermore, combine $\alpha \alpha_0^{-1} \in (K^\times)^{m'}$ and $\alpha \alpha_0^{-1} = \beta^{2^t}$ to obtain $\alpha \alpha_0^{-1} \in (K^\times)^m$, Thus, $\alpha \in \alpha_0 (K^\times)^m.$

Now we have that $P(m, S) \subset K^m \cup \alpha_0K^m$. We want to see whether $\alpha_0 \in P(m, S)$, and whether the two cosets are different. Let $S_0$ be the set of all primes $\mfp$ where $F_\mfp/K_\mfp$ is of degree four (being inspired by the fact that $[F:K]=4$ and furthermore its Galois group is the Klein four-group) Since unramified fields are cyclic and $K(\zeta_{s+1})$ is only ramified for primes dividing 2, we have that $S_0$ only consists of specific divisors of 2.

Assume either that $\alpha_0 \in K^m$ or $\alpha_0 \in K_\mfp^m$ for some $\mfp \in S_0$. Then, $\alpha_0 \in K^{(2^t)}$ or $K_\mfp^{(2^t)}$. Since $m'$ is odd, we have that $(2+\eta_s)^{2^t-1} \in K^{(2^t)}$ or $K_\mfp^{(2^t)}$. Taking $2^{t-1}$'th roots, $(2+\eta_s)\zeta \in K^2$ or $K^2_\mfp$, where $\zeta$ is a $2^{t-1}$'th root of unity with respect to $K$ or $K_\mfp$ - $\zeta$ can only be $\pm 1$. Since $F/K$ or $F_\mfp/K_\mfp$ has Galois group equal to the Klein four-group, $\pm(2+\eta_s)$ is not a square, which is a contradiction.

Therefore, $\alpha_0 K^m \neq K^m$; furthermore, $\alpha \in P(m, S)$ if and only if $S_0 \subset S$. Now assume that $S_0 \subset S$ and let $\mfp \in S$. In this case, $F_\mfp/K_\mfp$ collapses so $K(i), K(\eta_{s+1})$ or $K(i\eta_{s+1})$ must collapse. This suggests that either $\zeta_s$, $\eta_{s+1}$, or $i\eta_{s+1} \in K_\mfp$ and $\alpha_0 \in K_\mfp^m$ with $\alpha_0 \in P(m, S). \qed$

\thrm{Albert-Brauer-Hasse-Noether theorem}{
	Let $K$ be a number field. The map $$H^2(\text{Gal}(\bar{K}/K), \bar{K}^\times) \to \bigoplus\limits_v H^2(\text{Gal}(\bar{K}_v/K_v), (\bar{K}_v)^\times)$$ is injective, where the direct sum runs over all places $v\in K$. 
}

\tbf{Proof:} Direct application of the class field axiom. Consider the exact sequence $$1= H^1(\text{Gal}(L/K), C_L) \to H^2(\text{Gal}(L/K), L^\times) \to \bigoplus\limits_v H^2(\text{Gal}(L_2/K_v)), L_w^\times,$$ and apply the class field axiom. $\qed$

\section{Recap}
% pontryagin duality, poisson summation, tate's thesis
% CM, eichler-shimura
After the main theorems of class field theory were proven, many number theorists tried to generalise said theory to the non-abelian case. The main reason why the abelian case works (we're skipping a lot of the details here) is due to the fact that the dual groups of certain groups are isomorphic. 

The notion of an automorphic form (a function $G \to \bbC$ where $G$ is a topological group) has become increasingly common over the past 60-70 years, with the Langlands program being viewed as class field theory for $GL_n$. Class field theory aims to find correspondences between representations of $\text{Gal}(\bar{K}/K)$ and $\text{GL}_1(\bbA_K)$, where $\bbA_K$ denotes the Adele ring. Take $n=2$ for instance. The Langlands program aims to produce a correspondence between the 2-dimensional Galois representations that arise from elliptic curves, and the representations of $\text{GL}_2(\bbA_K)$ that correspond to modular forms. This is done by finding a meromorphic continuation for the $L$-function attributed to automorphic representations of $\text{GL}_n$. 

Apart from jumping straight into the Langlands programme, it's also beneficial for us to acquaint ourselves with integration and Fourier analysis on the adeles as per Tate's thesis. Tate showed that by using abelian harmonic analysis on $\bbA_K$, we can exhibit functional equations for $L$-functions attached to Hecke characters, or one-dimensional automorphic representations. This includes the Dedekind zeta function. 

Nonetheless, the key idea with the Langlands programme is to note that arithmetic $L$-functions arise from automorphic representations. It's the notion that the zeta-functions in number theory are realisations of the $L$-functions $L(s, \pi)$. Those interested can read Knapp's \emph{Introduction to the Langlands Program}.

Here's a recap of the major results in this text in order of their proofs' appearance:

\begin{itemize}
	\item Minkowski's Bound (Chapter 2)
	\item Minkowski's Theorem (Chapter 2)
	\item Dirichlet's unit theorem (Chapter 3)
	\item Quadratic reciprocity (Chapter 4)
	\item Gauss' Lemma (Chapter 4)
	\item Class number formula (Chapter 5)
	\item Ostrowski's theorem (Chapter 6)
	\item Hensel's lemma (Chapter 6)
	\item Product formula (Chapter 6)
	\item Approximation theorem (Chapter 6)
	\item Kronecker-Weber theorem (Chapter 7)
	\item Shapiro's lemma (Chapter 7) 
	\item Tate's theorem (Chapter 7)
	\item Hilbert's Theorem 90 (Chapter 7)
	\item Local Artin reciprocity, norm limitation theorem, existence theorem (Chapter 7)
	\item Adelic existence theorem (Chapter 8)
	\item First and second inequalities (Chapter 8)
	\item Class field axiom (Chapter 8)
	\item Artin reciprocity (Chapter 8)
	\item Chebotarev density theorem (Chapter 8)
	\item Hasse norm theorem (Chapter 8)
	\item Grunwald-Wang theorem (Chapter 8)
\end{itemize}


%Let $L/K$ be an abelian extension, and let $\mfp$ be an unramified prime ideal. We can thus split $\mfp = \mfP_1\dots \mfP_r$. Now let $x_\mfp \in N_\mfP L_\mfP^\times$. Then, the idele $(\dots, 1, 1, x_\mfp, 1, 1, \dots)$ is thus a norm idele of $L$. Thus, we conclude that $$N_\mfP L_\mfP^\times \subseteq N(C_L) \cap K_\mfp^\times.$$ (neukirch bonn chapter 8)

\end{document}

%Kummer theory describes the theory regarding Kummer extensions. This was used extensively in his work on Fermat's Last Theorem. He used it to prove Fermat's Last Theorem for a specific case. The case was described in the appendix of Chapter 3.

%We aim to characterise abelian extensions using Kummer theory. Evidently, the cyclotomic extension $K \subset K(\zeta_n)$ is abelian, as $\Gal(K(\zeta_n)/K)=\bbZ/n\bbZ.$ 

%\defn{Kummer extension}{
	%A Kummer extension is an extension of the type $L=K(\sqrt[n]{a})$, where $K$ is a field that contains an $n$'th root of unity. Naturally, $L/K$ is of degree $n$.
%}

%\lemma{}{
	%Let $K$ be a subfield of $\bbC$ which contains an $n$'th root of unity $\zeta_n$.  Let
%}
